\begin{enumerate}[label=\thesubsection.\arabic*,ref=\thesubsection.\theenumi]
\item Find the angle between two vectors $\overrightarrow{a}$ and $\overrightarrow {b} $ with magnitudes $\sqrt{3}$ and 2 respectively having $\overrightarrow {a}\cdot\overrightarrow {b}=\sqrt{6}$.
		\label{prob:12/10/3/1/inner}
	\\
	\solution
		\solution 
See Fig. \ref{fig:10/7/4/8Fig3}. From 
  \eqref{eq:10/7/4/8det2f}, $PQRS$ is a parallelogram.
\begin{align}
  %\label{eq:10/7/4/8det2f}
  \vec{P}  = 
 \frac{3}{2},\, 
 \vec{Q}  = \myvec{
 2 \\
 4 \\
 } ,\,
 \vec{R}  = \myvec{
 5 \\
 \frac{3}{2}
 }   
  ,\,
 \vec{S}  = \myvec{
 2\\
 -1 \\
 }   
 \\
	\implies 
 \brak{\vec{Q}-\vec{P}}^\top\brak{\vec{R}-\vec{Q}}  \neq 0
 \\
 \brak{\vec{R}-\vec{P}}^\top\brak{\vec{S}-\vec{Q}}  = 0
\end{align}
Therefore $PQRS$ is a rhombus.
\begin{figure}[H]
	\begin{center}
		\includegraphics[width=0.75\columnwidth]{chapters/10/7/4/8/figs/problem1.pdf}
	\end{center}
\caption{}
\label{fig:10/7/4/8Fig3}
\end{figure}


\item Find the angle between the the vectors $\hat{i}-2\hat{j}+3\hat{k}$ and $3\hat{i}-2\hat{j}+\hat{k}$.
	\\
	\solution
		\solution 
See Fig. \ref{fig:10/7/4/8Fig3}. From 
  \eqref{eq:10/7/4/8det2f}, $PQRS$ is a parallelogram.
\begin{align}
  %\label{eq:10/7/4/8det2f}
  \vec{P}  = 
 \frac{3}{2},\, 
 \vec{Q}  = \myvec{
 2 \\
 4 \\
 } ,\,
 \vec{R}  = \myvec{
 5 \\
 \frac{3}{2}
 }   
  ,\,
 \vec{S}  = \myvec{
 2\\
 -1 \\
 }   
 \\
	\implies 
 \brak{\vec{Q}-\vec{P}}^\top\brak{\vec{R}-\vec{Q}}  \neq 0
 \\
 \brak{\vec{R}-\vec{P}}^\top\brak{\vec{S}-\vec{Q}}  = 0
\end{align}
Therefore $PQRS$ is a rhombus.
\begin{figure}[H]
	\begin{center}
		\includegraphics[width=0.75\columnwidth]{chapters/10/7/4/8/figs/problem1.pdf}
	\end{center}
\caption{}
\label{fig:10/7/4/8Fig3}
\end{figure}


\item Evaluate the product $(3\overrightarrow {a}-5\overrightarrow {b})\cdot (2\overrightarrow {a}+7\overrightarrow {b})$.
	\\
	\solution
		\solution 
See Fig. \ref{fig:10/7/4/8Fig3}. From 
  \eqref{eq:10/7/4/8det2f}, $PQRS$ is a parallelogram.
\begin{align}
  %\label{eq:10/7/4/8det2f}
  \vec{P}  = 
 \frac{3}{2},\, 
 \vec{Q}  = \myvec{
 2 \\
 4 \\
 } ,\,
 \vec{R}  = \myvec{
 5 \\
 \frac{3}{2}
 }   
  ,\,
 \vec{S}  = \myvec{
 2\\
 -1 \\
 }   
 \\
	\implies 
 \brak{\vec{Q}-\vec{P}}^\top\brak{\vec{R}-\vec{Q}}  \neq 0
 \\
 \brak{\vec{R}-\vec{P}}^\top\brak{\vec{S}-\vec{Q}}  = 0
\end{align}
Therefore $PQRS$ is a rhombus.
\begin{figure}[H]
	\begin{center}
		\includegraphics[width=0.75\columnwidth]{chapters/10/7/4/8/figs/problem1.pdf}
	\end{center}
\caption{}
\label{fig:10/7/4/8Fig3}
\end{figure}


\item If the vertices $A,B,C$ of a triangle $ABC$ are (1,2,3), (-1,0,0), (0,1,2), respectively, then find  $\angle{ABC}$.
	\\
	\solution
		\solution 
See Fig. \ref{fig:10/7/4/8Fig3}. From 
  \eqref{eq:10/7/4/8det2f}, $PQRS$ is a parallelogram.
\begin{align}
  %\label{eq:10/7/4/8det2f}
  \vec{P}  = 
 \frac{3}{2},\, 
 \vec{Q}  = \myvec{
 2 \\
 4 \\
 } ,\,
 \vec{R}  = \myvec{
 5 \\
 \frac{3}{2}
 }   
  ,\,
 \vec{S}  = \myvec{
 2\\
 -1 \\
 }   
 \\
	\implies 
 \brak{\vec{Q}-\vec{P}}^\top\brak{\vec{R}-\vec{Q}}  \neq 0
 \\
 \brak{\vec{R}-\vec{P}}^\top\brak{\vec{S}-\vec{Q}}  = 0
\end{align}
Therefore $PQRS$ is a rhombus.
\begin{figure}[H]
	\begin{center}
		\includegraphics[width=0.75\columnwidth]{chapters/10/7/4/8/figs/problem1.pdf}
	\end{center}
\caption{}
\label{fig:10/7/4/8Fig3}
\end{figure}


	\item The slope of a line is double of the slope of another line. If tangent of the angle between them is 1/3, find the slopes of the lines.
\label{chapters/11/10/1/11}
\\
\solution 
\iffalse
\documentclass[10pt, a4paper]{article}
\usepackage[a4paper,outer=1.5cm,inner=1.5cm,top=1.75cm,bottom=1.5cm]{geometry}

\twocolumn
\usepackage{graphicx}

\usepackage{hyperref}
\usepackage[utf8]{inputenc}
\usepackage{amsmath}
\usepackage{physics}
\usepackage{amssymb}
\begin{document}
\title{Assignment-4}
\author{Name:A.SUSI\and Email :  \url{susireddy9121@gmail.com}}
%\{ Wireless Communication (FWC)}
\date{30-sep-2022}
\maketitle



\section{Problem}
\fi
\solution 
\iffalse
\section{Solution}
\begin{center}
The input given 
\boldmath
\fi 
Let
\begin{align} 
\vec{A}=\myvec{ h\\ 0 },
\vec{B}=\myvec{ a\\ b },
\vec{C}=\myvec{ 0\\ k }
\end{align}
Forming the matrix in 
	\eqref{eq:normal_line-collinear}, we obtain, upon row reduction
	\iffalse
\begin{align}
\myvec{ h-a & -b\\ h & -k  } 
\end{align}
Using row reduction, 


In the problem they have given that three points lie on a line, thats means these three points are collinear.\\
If  points on a line  are  collinear, rank of matrix is "1"then the vectors are in linearlydependent.\\
For 2 × 2 matrix Rank =1 means Determinant is 0.\\
Through pivoting,we obtain\\
\fi
\begin{align}\label{eq:}
\myvec{ h-a & -b\\ h & -k  }  
	\xleftrightarrow[]{{\frac{R_1}{h-a}}}\myvec{
1 &\frac{-b}{h-a} \\ 
 h& -k
}
	\\
	\xleftrightarrow[]{R_2\rightarrow R_2-hR_1}
\myvec{
1 &\frac{-b}{h-a} \\ 
 0&-k+\frac{bh}{h-a} 
}
\end{align} 
For obtaining a rank 1 matrix, 
\iffalse

if the rank of the matrix is 1 means any one of the row must be zero.So, making the last element in the matrix to 0.\\
\fi
\begin{align}
	-k+\frac{bh}{h-a}&=0
	\\
	\implies \frac{a}{h}+\frac{b}{k}&=1 
\end{align} 
upon simplification.
\iffalse

Hence proved.\\
\section{Construction}
 \begin{figure}[H]
\centering
\includegraphics[width=0.75\columnwidth]{fig.png} 
\caption{}
\end{figure}
\section{Code}
*Verify the above proofs in the following code.\\
\framebox{
\url{https://github.com/Susi9121/FWC/tree/main/matrix/line}}	
\bibliographystyle{ieeetr}
\end{document}
\fi

\item    Find angle between the lines, $\sqrt{3}x+y=1$ and $x+\sqrt{3}y$=1.
\label{chapters/11/10/3/9}
\\
   \solution 
Given
\begin{align}
	c_1 = \frac{7}{3},\,
c_2 = -6.
\end{align}
	From \eqref{eq:parallel_lines},
we need to find $c$ such that,
\begin{align}
	\abs{c-c_1} = \abs{c-c_2} \implies c = \frac{c_1+c_2}{2}
	 = -\frac{11}{6}.
\end{align}
Hence, the desired equation is
\begin{align}
	\myvec{3 & 2}\vec{x} &= -\frac{11}{6}
\end{align}
	See \figref{fig:chapters/11/10/4/21/1}.
\begin{figure}[H]
	\centering
	\includegraphics[width=0.75\columnwidth]{chapters/11/10/4/21/figs/line_plot.jpg}
	\caption{}
	\label{fig:chapters/11/10/4/21/1}
\end{figure}

\item Find the angle between the vectors $2\hat{i}-\hat{j}+\hat{k}$ and $3\hat{i}+4\hat{j}-\hat{k}$.
\item The angles between two vectors $\vec{a}, \vec{b}$ with magnitude $\sqrt{3}, 4$ respectively, and $\vec{a} \cdot \vec{b}= 2\sqrt{3}$ is
	\begin{enumerate}
\item $\frac{\pi}{6}$
\item $\frac{\pi}{3}$
\item $\frac{\pi}{2}$ 
\item $\frac{5\pi}{2}$
\end{enumerate}
\item Find the angle between the lines 
\begin{align}
	\overrightarrow{r}&=3\hat{i}-2\hat{j}+6\hat{k}+\lambda(2\hat{i}+\hat{j}+2\hat{k})
	\text{ and}
	\\
	\overrightarrow{r}&=(2\hat{j}-5\hat{k})+\mu(6\hat{i}+3\hat{j}+2\hat{k})
\end{align}
%
\solution  The given lines can be expressed  in the form 
of 
	\eqref{eq:param-form}
	as
\begin{align}
	\vec{x} = \myvec{3 \\ -2 \\ 6} + \kappa_1 \myvec{2 \\ 1 \\ 2}
	\\
	\vec{x} = \myvec{0 \\ 2 \\ -5 } + \kappa_2 \myvec{6 \\ 3 \\ 2}
\end{align}
From the above, it is obvious that the direction vectors of the two lines are
\begin{align}
\vec{m}_1 =\myvec{2 \\ 1 \\ 2},\
	\vec{m}_2=\myvec{6 \\ 3 \\ 2}
\end{align}
	From \eqref{eq:angle-inner}, the angle between the two lines is  obtained as
\begin{align}
	\cos \theta = \frac{19}{21}
\end{align}
\item The vectors $\vec{a}=3\hat{i}-2\hat{j}+2\hat{k}$ $\text{ and }$ $\vec{b}=\hat{i}-2\hat{k}$ are the adjancent sides of a parallelogram. The acute angle between its diagonals is \rule{1cm}{0.15mm}.
\item The sine of the angle between the straight line 
\begin{align}
	\frac{x-2}{3}=\frac{y-3}{4}=\frac{z-4}{5} 
\end{align}
and the plane  
\begin{align}
2x-2y+z=5
\end{align}
is
\begin{enumerate}
	\item $\frac{10}{6\sqrt{5}}$ 
	\item $\frac{4}{5\sqrt{2}}$
	\item $\frac{2\sqrt{3}}{5}$
	\item $\frac{\sqrt{2}}{10}$
\end{enumerate}
\solution The given line can be expressed in the form 
	\eqref{eq:param-form}
	as
\begin{align}
	\vec{x} = \myvec{2 \\ 3 \\ 4} + \kappa_1 \myvec{3 \\ 4 \\ 5}
\end{align}
Hence the direction vector of this line is 
\begin{align}
\myvec{3 \\ 4 \\ 5}
\end{align}
	From \eqref{eq:normal-form}, the normal vector of the given plane is 
\begin{align}
\myvec{2 \\ -2 \\ 1}
\end{align}
Thus, the cosine of the angle between the two is 
obtained from \eqref{eq:angle-inner} as
\begin{align}
	\frac{\sqrt{2}}{10},
\end{align}
which is sine of the angle between the plane and the line.
\item The plane $2x-3y+6z-11=0$ makes an angle $\sin^{-1}(\alpha)$ with x-axis. The value of $\alpha$ is equal to 
\begin{enumerate}
	\item  $\frac{\sqrt{3}}{2}$
	\item  $\frac{\sqrt{2}}{3}$
	\item  $\frac{2}{7}$
	\item  $\frac{3}{7}$
\end{enumerate}
\item Find the angle between the vectors $2\hat{i}-\hat{j}+\hat{k}$ $\text{and}$ $3\hat{i}+4\hat{j}-\hat{k}$.
\item The angles between two vectors $\vec{a}$ $\text{and}$ $\vec{b}$ with magnitude $\sqrt{3}$ $\text{ and }$ 4, respectively, and $\vec{a}$, $\vec{b}$= $2\sqrt{3}$ is
	\begin{enumerate}
\item $\frac{\pi}{6}$
\item $\frac{\pi}{3}$
\item $\frac{\pi}{2}$ 
\item $\frac{5\pi}{2}$
\end{enumerate}

\item The angle between the line 
\begin{align}
	\overrightarrow{r}=(5\hat{i}-\hat{j}-4\hat{k})+\lambda(2\hat{i}-\hat{j}+\hat{k})
\end{align}
	and the plane 
\begin{align}
	\overrightarrow{r} \cdot (3\hat{i}-4\hat{j}-\hat{k})+5=0
\end{align}
	is $\sin^{-1}\brak{\frac{5}{2\sqrt{91}}}$.
\item The angle between the planes 
\begin{align}
	\overrightarrow{r} \cdot (2\hat{i}-3\hat{j}+\hat{k})&=1 
	\text{ and }
	\\
	\overrightarrow{r} \cdot (\hat{i}-\hat{j})&=4  
\end{align}
is
	$\cos^{-1} \brak{\frac{-5}{\sqrt{58}}}$.
\item Find the angle between the lines 
\begin{align}
	y&=(2-\sqrt{3})(x+5)\text{ and }
	\\
	y&=(2+\sqrt{3})(x-7).
\end{align}
\item The unit vector normal to the plane $x+2y+3z-6=0$ is $\frac{1}{\sqrt{14}}\hat{i} + \frac{2}{\sqrt{14}}\hat{j} + \frac{3}{\sqrt{14}}\hat{k}$.
\item The scalar product of the vector $\hat{i}+\hat{j}+\hat{k}$ with a unit vector along the sum of vectors $2\hat{i}+4\hat{j}-5\hat{k}$ and $\lambda\hat{i}+2\hat{j}+3\hat{k}$ is equal to one. Find the value of $\lambda$.
\end{enumerate}
