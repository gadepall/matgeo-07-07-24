\begin{enumerate}[label=\thesubsection.\arabic*,ref=\thesubsection.\theenumi]
		\item Find $\abs{\overrightarrow{a}\times\overrightarrow{b}},\text{ if }\overrightarrow{a}=\hat{i}-7\hat{j}+7\hat{k}\text{ and } \overrightarrow{b}=3\hat{i}-2\hat{j}+2\hat{k}$.
	\\
		\solution
		The area of the rhombus is
\begin{align}
                \norm{\myvec{\vec{A-D}}\times \myvec{\vec{B-A}}}=\mydet{5 & 1\\1 & 5} = 24
\end{align}
See 
\figref{fig:chapters/10/7/2/10/gFig1}.
\begin{figure}[H]
 \begin{center}
  \includegraphics[width=0.75\columnwidth]{chapters/10/7/2/10/figs/fig.pdf}
 \end{center}
\caption{}
\label{fig:chapters/10/7/2/10/gFig1}
\end{figure}

\item Find $\lambda$ and $\mu$ if $(2\hat{i}+6\hat{j}+27\hat{k})\times(\hat{i}+\lambda \hat{j} + \mu \hat{k})=\overrightarrow{0}$.
	\\
		\solution
		The area of the rhombus is
\begin{align}
                \norm{\myvec{\vec{A-D}}\times \myvec{\vec{B-A}}}=\mydet{5 & 1\\1 & 5} = 24
\end{align}
See 
\figref{fig:chapters/10/7/2/10/gFig1}.
\begin{figure}[H]
 \begin{center}
  \includegraphics[width=0.75\columnwidth]{chapters/10/7/2/10/figs/fig.pdf}
 \end{center}
\caption{}
\label{fig:chapters/10/7/2/10/gFig1}
\end{figure}

\item Find the area of the triangle with vertices $A(1, 1, 2), B(2, 3, 5)$ and $C(1, 5, 5)$.
	\\
		\solution
		The area of the rhombus is
\begin{align}
                \norm{\myvec{\vec{A-D}}\times \myvec{\vec{B-A}}}=\mydet{5 & 1\\1 & 5} = 24
\end{align}
See 
\figref{fig:chapters/10/7/2/10/gFig1}.
\begin{figure}[H]
 \begin{center}
  \includegraphics[width=0.75\columnwidth]{chapters/10/7/2/10/figs/fig.pdf}
 \end{center}
\caption{}
\label{fig:chapters/10/7/2/10/gFig1}
\end{figure}

\item Find the area of the parallelogram whose adjacent sides are determined by the vectors $\overrightarrow{a}=\hat{i}-\hat{j}+3\hat{k}$ and $\overrightarrow{b}=2\hat{i}-7\hat{j}+\hat{k}$.
	\\
		\solution
		The area of the rhombus is
\begin{align}
                \norm{\myvec{\vec{A-D}}\times \myvec{\vec{B-A}}}=\mydet{5 & 1\\1 & 5} = 24
\end{align}
See 
\figref{fig:chapters/10/7/2/10/gFig1}.
\begin{figure}[H]
 \begin{center}
  \includegraphics[width=0.75\columnwidth]{chapters/10/7/2/10/figs/fig.pdf}
 \end{center}
\caption{}
\label{fig:chapters/10/7/2/10/gFig1}
\end{figure}

\item Find the area of a rhombus if its vertices are $A(3,0), B(4,5), C(-1,4)$  and  $D(-2,-1)$ taken in order. 
	\\
		\solution
	The area of the rhombus is
\begin{align}
                \norm{\myvec{\vec{A-D}}\times \myvec{\vec{B-A}}}=\mydet{5 & 1\\1 & 5} = 24
\end{align}
See 
\figref{fig:chapters/10/7/2/10/gFig1}.
\begin{figure}[H]
 \begin{center}
  \includegraphics[width=0.75\columnwidth]{chapters/10/7/2/10/figs/fig.pdf}
 \end{center}
\caption{}
\label{fig:chapters/10/7/2/10/gFig1}
\end{figure}

\item Let the vectors $\overrightarrow{a}$ and $\overrightarrow{b}$ be such that $|\overrightarrow{a}| = 3$ and $|\overrightarrow{b}| = \dfrac{\sqrt{2}}{3}$, then $\overrightarrow{a} \times \overrightarrow{b}$ is a unit vector, if the angle between $\overrightarrow{a}$ and $\overrightarrow{b}$ is
\begin{enumerate}
\item $\frac{\pi}{6}$
\item $\frac{\pi}{4}$
\item $\frac{\pi}{3}$
\item $\frac{\pi}{2}$
\end{enumerate}
		\solution
		The area of the rhombus is
\begin{align}
                \norm{\myvec{\vec{A-D}}\times \myvec{\vec{B-A}}}=\mydet{5 & 1\\1 & 5} = 24
\end{align}
See 
\figref{fig:chapters/10/7/2/10/gFig1}.
\begin{figure}[H]
 \begin{center}
  \includegraphics[width=0.75\columnwidth]{chapters/10/7/2/10/figs/fig.pdf}
 \end{center}
\caption{}
\label{fig:chapters/10/7/2/10/gFig1}
\end{figure}

\item Area of a rectangle having vertices A, B, C and D with position vectors $ -\hat{i}+ \frac{1}{2} \hat{j}+4\hat{k}, \hat{i}+ \frac{1}{2} \hat{j}+4\hat{k}, \hat{i}-\frac{1}{2} \hat{j}+4\hat{k}$ and $-\hat{i}- \frac{1}{2} \hat{j}+4\hat{k}$, respectively is
\begin{enumerate}
\item $\frac{1}{2}$
\item 1
\item 2
\item 4
\end{enumerate}
		\solution
		\iffalse
\documentclass[journal,12pt,twocolumn]{IEEEtran}
%
\usepackage{setspace}
\usepackage{gensymb}
%\doublespacing
\singlespacing

%\usepackage{graphicx}
%\usepackage{amssymb}
%\usepackage{relsize}
\usepackage[cmex10]{amsmath}
%\usepackage{amsthm}
%\interdisplaylinepenalty=2500
%\savesymbol{iint}
%\usepackage{txfonts}
%\restoresymbol{TXF}{iint}
%\usepackage{wasysym}
\usepackage{amsthm}
%\usepackage{iithtlc}
\usepackage{mathrsfs}
\usepackage{txfonts}
\usepackage{stfloats}
\usepackage{bm}
\usepackage{cite}
\usepackage{cases}
\usepackage{subfig}
%\usepackage{xtab}
\usepackage{longtable}
\usepackage{multirow}
%\usepackage{algorithm}
%\usepackage{algpseudocode}
\usepackage{enumitem}
\usepackage{mathtools}
\usepackage{steinmetz}
\usepackage{tikz}
\usepackage{circuitikz}
\usepackage{verbatim}
\usepackage{tfrupee}
\usepackage[breaklinks=true]{hyperref}
%\usepackage{stmaryrd}
\usepackage{tkz-euclide} % loads  TikZ and tkz-base
%\usetkzobj{all}
\usetikzlibrary{calc,math}
\usepackage{listings}
    \usepackage{color}                                            %%
    \usepackage{array}                                            %%
    \usepackage{longtable}                                        %%
    \usepackage{calc}                                             %%
    \usepackage{multirow}                                         %%
    \usepackage{hhline}                                           %%
    \usepackage{ifthen}                                           %%
  %optionally (for landscape tables embedded in another document): %%
    \usepackage{lscape}     
\usepackage{multicol}
\usepackage{chngcntr}
%\usepackage{enumerate}

%\usepackage{wasysym}
%\newcounter{MYtempeqncnt}
\DeclareMathOperator*{\Res}{Res}
%\renewcommand{\baselinestretch}{2}
\renewcommand\thesection{\arabic{section}}
\renewcommand\thesubsection{\thesection.\arabic{subsection}}
\renewcommand\thesubsubsection{\thesubsection.\arabic{subsubsection}}

\renewcommand\thesectiondis{\arabic{section}}
\renewcommand\thesubsectiondis{\thesectiondis.\arabic{subsection}}
\renewcommand\thesubsubsectiondis{\thesubsectiondis.\arabic{subsubsection}}

% correct bad hyphenation here
\hyphenation{op-tical net-works semi-conduc-tor}
\def\inputGnumericTable{}                                 %%

\lstset{
%language=C,
frame=single, 
breaklines=true,
columns=fullflexible
}
%\lstset{
%language=tex,
%frame=single, 
%breaklines=true
%}

\begin{document}
%


\newtheorem{theorem}{Theorem}[section]
\newtheorem{problem}{Problem}
\newtheorem{proposition}{Proposition}[section]
\newtheorem{lemma}{Lemma}[section]
\newtheorem{corollary}[theorem]{Corollary}
\newtheorem{example}{Example}[section]
\newtheorem{definition}[problem]{Definition}
%\newtheorem{thm}{Theorem}[section] 
%\newtheorem{defn}[thm]{Definition}
%\newtheorem{algorithm}{Algorithm}[section]
%\newtheorem{cor}{Corollary}
\newcommand{\BEQA}{\begin{eqnarray}}
\newcommand{\EEQA}{\end{eqnarray}}
\newcommand{\define}{\stackrel{\triangle}{=}}

\bibliographystyle{IEEEtran}
%\bibliographystyle{ieeetr}


\providecommand{\mbf}{\mathbf}
\providecommand{\pr}[1]{\ensuremath{\Pr\left(#1\right)}}
\providecommand{\qfunc}[1]{\ensuremath{Q\left(#1\right)}}
\providecommand{\sbrak}[1]{\ensuremath{{}\left[#1\right]}}
\providecommand{\lsbrak}[1]{\ensuremath{{}\left[#1\right.}}
\providecommand{\rsbrak}[1]{\ensuremath{{}\left.#1\right]}}
\providecommand{\brak}[1]{\ensuremath{\left(#1\right)}}
\providecommand{\lbrak}[1]{\ensuremath{\left(#1\right.}}
\providecommand{\rbrak}[1]{\ensuremath{\left.#1\right)}}
\providecommand{\cbrak}[1]{\ensuremath{\left\{#1\right\}}}
\providecommand{\lcbrak}[1]{\ensuremath{\left\{#1\right.}}
\providecommand{\rcbrak}[1]{\ensuremath{\left.#1\right\}}}
\theoremstyle{remark}
\newtheorem{rem}{Remark}
\newcommand{\sgn}{\mathop{\mathrm{sgn}}}
\providecommand{\abs}[1]{\left\vert#1\right\vert}
\providecommand{\res}[1]{\Res\displaylimits_{#1}} 
\providecommand{\norm}[1]{\left\lVert#1\right\rVert}
%\providecommand{\norm}[1]{\lVert#1\rVert}
\providecommand{\mtx}[1]{\mathbf{#1}}
\providecommand{\mean}[1]{E\left[ #1 \right]}
\providecommand{\fourier}{\overset{\mathcal{F}}{ \rightleftharpoons}}
%\providecommand{\hilbert}{\overset{\mathcal{H}}{ \rightleftharpoons}}
\providecommand{\system}{\overset{\mathcal{H}}{ \longleftrightarrow}}
	%\newcommand{\solution}[2]{\textbf{Solution:}{#1}}
\newcommand{\solution}{\noindent \textbf{Solution: }}
\newcommand{\cosec}{\,\text{cosec}\,}
\providecommand{\dec}[2]{\ensuremath{\overset{#1}{\underset{#2}{\gtrless}}}}
\newcommand{\myvec}[1]{\ensuremath{\begin{pmatrix}#1\end{pmatrix}}}
\newcommand{\mydet}[1]{\ensuremath{\begin{vmatrix}#1\end{vmatrix}}}
%\numberwithin{equation}{section}
\numberwithin{equation}{subsection}
%\numberwithin{problem}{section}
%\numberwithin{definition}{section}
\makeatletter
\@addtoreset{figure}{problem}
\makeatother

\let\StandardTheFigure\thefigure
\let\vec\mathbf
%\renewcommand{\thefigure}{\theproblem.\arabic{figure}}
\renewcommand{\thefigure}{\theproblem}
%\setlist[enumerate,1]{before=\renewcommand\theequation{\theenumi.\arabic{equation}}
%\counterwithin{equation}{enumi}


%\renewcommand{\theequation}{\arabic{subsection}.\arabic{equation}}

\def\putbox#1#2#3{\makebox[0in][l]{\makebox[#1][l]{}\raisebox{\baselineskip}[0in][0in]{\raisebox{#2}[0in][0in]{#3}}}}
     \def\rightbox#1{\makebox[0in][r]{#1}}
     \def\centbox#1{\makebox[0in]{#1}}
     \def\topbox#1{\raisebox{-\baselineskip}[0in][0in]{#1}}
     \def\midbox#1{\raisebox{-0.5\baselineskip}[0in][0in]{#1}}

\vspace{3cm}


\title{Question: 12.10.4.12}
\author{Nikam Pratik Balasaheb (EE21BTECH11037)}





% make the title area
\maketitle

\newpage

%\tableofcontents

\bigskip

\renewcommand{\thefigure}{\theenumi}
\renewcommand{\thetable}{\theenumi}
%\renewcommand{\theequation}{\theenumi}

\section{Problem}
Find the area of rectangle having A,B,C,D with position vectors $\myvec{-1\\[1pt]\frac{1}{2} \\[1pt] 4}$ ,$\myvec{1\\[1pt]\frac{1}{2} \\[1pt] 4}$, $\myvec{1\\[1pt]\frac{-1}{2} \\[1pt] 4}$ and $\myvec{-1\\[1pt]\frac{-1}{2} \\[1pt] 4}$ respectively.  
\section{Solution}
\fi
Since
\begin{align}
\vec{A} - \vec{B} &= \myvec{-2\\0\\0}\\
\vec{C} -\vec{B} &= \myvec{0\\-1\\0}
\end{align}
area of the rectangle is
\begin{align}
 \norm{\brak{\vec{A} -\vec{B}} \times \brak{\vec{C}-\vec{D}}}
= 2
\end{align} 
See Fig. 
   \ref{fig:chapters/12/10/4/12Rect_ABCD}
\begin{figure}[H]
  \centering
   \includegraphics[width=0.75\columnwidth]{chapters/12/10/4/12/figs/Figure_1.png}
   \caption{}
   \label{fig:chapters/12/10/4/12Rect_ABCD}
\end{figure}





\item Find the area of the triangle whose vertices are 
\begin{enumerate}
\item $(2, 3), (–1, 0), (2, – 4)$
\item $(–5, –1), (3, –5), (5, 2)$ 
\end{enumerate}
		\label{10/7/3/1}
\solution
		    See \tabref{eq:10/7/3/1/area}.
\begin{table}[H]
    \centering
    \caption{}
    \label{eq:10/7/3/1/area}
    \begin{tabular}{|c|c|c|c|}
        \hline
	     & $\vec{A}-\vec{B}$  & $\vec{A}-\vec{C}$  & $\frac{1}{2}\|\brak{\vec{A}-\vec{B}} \times \brak{\vec{A}-\vec{C}}\|$ \\
        \hline
         a)& $\myvec{ 3 \\3 }$ & $\myvec{ 0 \\ 7 }$ & $\frac{21}{2}$ \\
        \hline
	    b)& $\myvec{
 -8 \\
 4 
 }$
         &$\myvec{
 -10 \\
 -3 
 }$
  &  $32$   \\
        \hline
    \end{tabular}
\end{table}


\item Find the area of the triangle formed by joining the mid-points of the sides of the triangle whose vertices are $A(0, –1), B(2, 1)$  and  $C(0, 3)$. Find the ratio of this area to the area of the given triangle.
	\\
\solution
		Using 
	  \eqref{eq:section_formula},
the mid point coordinates are given by
	\begin{align}
		\vec{P} = \frac{1}{2}\vec(\vec{A}+\vec{B})  = \myvec{1\\0}\\
		\vec{Q} = \frac{1}{2}\vec(\vec{B}+\vec{C}) = \myvec{1\\2}\\
		\vec{R} = \frac{1}{2}\vec(\vec{A}+\vec{C}) = \myvec{0\\1}
	\end{align}
	\begin{align}
\because		\vec{P}-\vec{Q} =  \myvec{
 0 \\
 -2 
 },\,
		\vec{Q}-\vec{R} =   \myvec{
 1 \\
 1 
 }
 \\
		ar(PQR)=\frac{1}{2}{\norm{\vec(\vec{P}-\vec{Q})\times\vec(\vec{Q}-\vec{R})}}
		=1
	\end{align}
	Similarly, 
	\begin{align}
		\vec{A}-\vec{B} = \myvec{
 -2 \\
 -2 
 }
 ,\,
		\vec{A}-\vec{C} =  \myvec{
 0 \\
 -4 
 }
 \\
 \implies
		ar(ABC)=\frac{1}{2}{\norm{\vec(\vec{A}-\vec{B})\times\vec(\vec{A}-\vec{C})}}
=4
\\
		\implies \frac{ar\brak{PQR}}{ar\brak{ABC}} = \frac{1}{4}
	\end{align}
	See 
\figref{fig:10/7/3/3Fig}
\begin{figure}[H]
	\begin{center} 
	    \includegraphics[width=0.75\columnwidth]{chapters/10/7/3/3/figs/trigraph.png}
	\end{center}
\caption{}
\label{fig:10/7/3/3Fig}
\end{figure}


\item Find the area of the quadrilateral whose vertices, taken in order, are $A(– 4, – 2), B(– 3, – 5), C(3, – 2)$  and $ D(2, 3)$.
	\\
\solution
		See 
\figref{fig:chapters/10/7/3/4/Fig1}
\begin{figure}[H]
 \begin{center}
  \includegraphics[width=0.75\columnwidth]{chapters/10/7/3/4/figs/fig.pdf}
 \end{center}
\caption{}
\label{fig:chapters/10/7/3/4/Fig1}
\end{figure}
\begin{align}
\because	\vec{A}- \vec{B} =\myvec{-1\\3},\,
	  \vec{A}- \vec{D} =\myvec{-6\\-5},
	  \\
	\vec{B}- \vec{C} =\myvec{-6\\-5},\,
	  \vec{B}- \vec{D} =\myvec{-3\\-8},
	  \\
	  ar(ABD)=\frac{1}{2} \norm{\brak{\vec{A}-\vec{B}}  \times 
   \brak{\vec{A}- \vec{D}}} 
	=	\frac{23}{2}
	\\
	  ar(BCD)=\frac{1}{2} \norm{\brak{\vec{B}-\vec{C}}  \times 
   \brak{\vec{B}- \vec{D}}} 
	=	\frac{33}{2}
	\\
\implies	ar(ABCD)=  ar(ABD) +  ar(BCD)
	= 28
\end{align}


\item Verify that a median of a triangle divides it into two triangles of equal areas for $\triangle ABC$ whose vertices are $\vec{A}(4, -6), \vec{B}(3, 2), \text{ and } \vec{C}(5, 2)$. 
		\label{10/7/3/5}
		\\
\solution
		\begin{align}
\vec{D}=\frac{\vec{B}+\vec{C}}{2}
=\myvec{4\\ 0},
\\
	\vec{A}- \vec{B} =\myvec{1\\ -4},\,
	  \vec{A}- \vec{D} =\myvec{0\\ -6}
	  \\
	  \implies
  ar(ABD)=\frac{1}{2} \norm{\brak{\vec{A}-\vec{B}}  \times 
   \brak{\vec{A}- \vec{D}}} 
	       =3	
	       \\
	\vec{A}- \vec{C} =\myvec{-1\\ -8},\,
	  \vec{A}- \vec{D} =\myvec{0\\ -6}
	  \\
	  \implies
  ar(ACD)=\frac{1}{2} \norm{\brak{\vec{A}-\vec{C}}  \times 
   \brak{\vec{A}- \vec{D}}} 
   \\
	= 3 =
ar(ABD)
\end{align}
See  
\figref{fig:10/7/3/5/}.
\begin{figure}[H]
\centering
\includegraphics[width=0.75\columnwidth]{chapters/10/7/3/5/figs/fig.pdf}
\caption{}
\label{fig:10/7/3/5/}
\end{figure} 

\item The two adjacent sides of a parallelogram are 
$\vec{a}= 2\hat{i}-4\hat{j}+5\hat{k}$  and  $\vec{b} =\hat{i}-2\hat{j}-3\hat{k}$.
Find the unit vector parallel to its diagonal. Also, find its area.\\
	\solution
		The area of the rhombus is
\begin{align}
                \norm{\myvec{\vec{A-D}}\times \myvec{\vec{B-A}}}=\mydet{5 & 1\\1 & 5} = 24
\end{align}
See 
\figref{fig:chapters/10/7/2/10/gFig1}.
\begin{figure}[H]
 \begin{center}
  \includegraphics[width=0.75\columnwidth]{chapters/10/7/2/10/figs/fig.pdf}
 \end{center}
\caption{}
\label{fig:chapters/10/7/2/10/gFig1}
\end{figure}

\item The vertices of a $\triangle ABC$ are $\vec{A}(4,6), \vec{B}(1,5)$ and  $\vec{C}(7,2)$. A line is drawn to intersect sides $AB$ and $AC$ at $\vec{D}$ and $\vec{E}$ respectively, such that $\frac{AD}{AB} = \frac{AE}{AC} = \frac{1}{4}$. Calculate the area of $\triangle ADE$ and compare it with the area of the $\triangle ABC$.
\\
\solution
	See  
\figref{fig:chapters/10/7/4/6Fig1}.
\begin{figure}[H]
 \begin{center}
 \includegraphics[width=0.75\columnwidth]{chapters/10/7/4/6/figs/fig.png}
 \end{center}
\caption{}
\label{fig:chapters/10/7/4/6Fig1}
\end{figure}
	Using section formula
	  \eqref{eq:section_formula},
\begin{align}
\vec{D} =\frac{3\vec{A}+\vec{B}}{4}
	=\frac{1}{4}\myvec{13\\ 23}
	\\
\vec{E} =\frac{3\vec{A}+\vec{C}}{4}
	=\frac{1}{4}\myvec{19\\ 20}
	\\
	\vec{A}- \vec{D} 
	=\frac{1}{4}\myvec{3\\ 1},\,
	  \vec{A}- \vec{E}  
	=\frac{1}{4}\myvec{-3\\ 1}
	\\
	\vec{A}- \vec{B} =\myvec{3\\1},
	  \vec{B}-\vec{C} =\myvec{-6\\3}
\end{align}
yielding
\begin{align}
ar(ABD) =\frac{1}{2} \norm{\brak{\vec{A}-\vec{D}}  \times 
   \brak{\vec{A}- \vec{E}}} 
	=	\frac{15}{32}
	\\
	  ar(ABC) =\frac{1}{2} \norm{\brak{\vec{A}-\vec{B}}  \times 
   \brak{\vec{B}- \vec{C}}} 
	=	\frac{15}{2}
	\\
	\implies \frac{ar\brak{ADE}}{ar\brak{ABC}}=\frac{1}{16}
\end{align}

    \item Draw a quadrilateral in the Cartesian plane, whose vertices are 
    \begin{align}
        \vec{A} = \myvec{-4\\5},\, \vec{B} = \myvec{0\\7},\, 
        \vec{C} = \myvec{5\\-5},\, \vec{D} = \myvec{-4\\-2}.
    \end{align}
    Also, find its area.
\label{chapters/11/10/1/1}
   \\ 
    \solution 
See \figref{fig:11/10/1/1quad}.
    From 
        \eqref{eq:11/10/1/1area-diag},
    \begin{align}
ar\brak{ABCD}
	       = \frac{121}{2}
        \label{eq:11/10/1/1ans}
    \end{align}
    \begin{figure}[H]
        \centering
        \includegraphics[width=0.75\columnwidth]{chapters/11/10/1/1/figs/quad.png}
        \caption{Plot of quadrilateral $ABCD$}
        \label{fig:11/10/1/1quad}
    \end{figure}

\item Find the area of region bounded by the triangle whose
	vertices are $(1, 0), (2, 2)$ and $(3, 1)$. 
\item Find the area of region bounded by the triangle whose vertices
	are $(– 1, 0), (1, 3)$  and  $(3, 2)$. 
\item Find the area of the $\triangle ABC$, coordinates of whose vertices are $\vec{A}(2, 0), \vec{B}(4, 5)$ and $\vec{C}(6, 3)$.
\item The area of a triangle with vertices $\vec{A}(3, 0), \vec{B}(7, 0)$ and  $\vec{C}(8, 4)$ is
\begin{enumerate}
\item 14
\item 28
\item 8
\item 6
\end{enumerate}
\item Find the area of the triangle whose vertices are $(-8,4),(-6,6)$ and $(-3,9)$.
\item If $\vec{D}\brak{\frac{-1}{2},\frac{5}{2}},\vec{E}(7,3)$ and $\vec{F}\brak{\frac{7}{2},\frac{7}{2}}$ are the midpoints of sides of $\triangle ABC$, find the area of the $\triangle ABC$.
\item Find the sine of the angle between the vectors $\vec{a}=3\hat{i}+\hat{j}+2\hat{k}$ $\text{ and }$ $\vec{b}=2\hat{i}-2\hat{j}+4\hat{k}$.
\item Using vectors, find the area of $\triangle{ABC}$ with vertices A(1,2,3), B(2,-1,4) and C(4,5,-1).
\item Find the area of the parallelogram whose diagonals are $2\hat{i}-\hat{j}+\hat{k}$ and $\hat{i}+3\hat{j}-\hat{k}$.

\item The vector from origin to the points A and B are $\vec{a}$ = $2\hat{i}-3\hat{j}+2\hat{k}$ and  $\vec{b}$ = $2\hat{i}+3\hat{j}+\hat{k}$, respectively, then the area of $\triangle {OAB}$ is
	\begin{enumerate}
\item 340 
\item $\sqrt{25}$
\item $\sqrt{229}$
\item $\frac{1}{2}\sqrt{229}$
\end{enumerate}
\item If $\vec{a} = \hat{i}+\hat{j}+\hat{k}$ and $\vec{b} = \hat{j}-\hat{k}$, find a vector $\vec{c}$ such that $\vec{a}\times\vec{c} = \vec{b}$ and $\vec{a}\cdot \vec{c}$ = 3.
%
\item The area of the quadrilateral ABCD, where A$(0,4,1)$, B$(2,3,-1)$, C$(4,5,0)$ and D$(2,6,2)$, is equal to 
\begin{enumerate}
	\item 9 sq. units
	\item 18 sq. units 
	\item 27 sq. units 
	\item 81 sq. units
\end{enumerate}
\item Find the area of region bounded by the triangle whose vertices are $(-1, 1), (0, 5)$ and $(3, 2)$.
\item The value of $\hat{i}\cdot(\hat{j}\times\hat{k})+\hat{j}\cdot(\hat{i}\times\hat{k})+\hat{k}\cdot(\hat{i}\times\hat{j})$ is
\begin{enumerate}
\item 0
\item -1
\item 1
\item 3
\end{enumerate}
\item The value of $\hat{i}\cdot (\hat{j}\times\hat{k})+\hat{j}\cdot (\hat{i}\times\hat{k})+\hat{k}\cdot (\hat{i}\times\hat{j})$ is
\begin{enumerate}
\item 0
\item -1
\item 1
\item 3
\end{enumerate}
\end{enumerate}

