Find the equation of line 
\begin{enumerate}[label=\thesubsection.\arabic*,ref=\thesubsection.\theenumi]
	\item passing through the point $\vec{P} = (– 4, 3)$ with slope $\frac{1}{2}$.
\label{chapters/11/10/2/2}
\\
\solution
			From \eqref{eq:line-school-normal},
\begin{align}
\vec{n}\equiv \myvec{\frac{1}{2}\\ -1}
\implies \myvec{\frac{1}{2}&-1}{\vec{x}}&=-5
\end{align}
using \eqref{eq:geo-normal}.
See 
		\figref{fig:chapters/11/10/2/2/Figure}.
\begin{figure}[H]
\centering
\includegraphics[width=0.75\columnwidth]{chapters/11/10/2/2/figs/fig.pdf}
\caption{}
		\label{fig:chapters/11/10/2/2/Figure}
\end{figure}

	\item passing through $\myvec{0\\0}$ with slope $m$.\\
\label{chapters/11/10/2/3}
\solution
Given
\begin{align}
	c_1 = \frac{7}{3},\,
c_2 = -6.
\end{align}
	From \eqref{eq:parallel_lines},
we need to find $c$ such that,
\begin{align}
	\abs{c-c_1} = \abs{c-c_2} \implies c = \frac{c_1+c_2}{2}
	 = -\frac{11}{6}.
\end{align}
Hence, the desired equation is
\begin{align}
	\myvec{3 & 2}\vec{x} &= -\frac{11}{6}
\end{align}
	See \figref{fig:chapters/11/10/4/21/1}.
\begin{figure}[H]
	\centering
	\includegraphics[width=0.75\columnwidth]{chapters/11/10/4/21/figs/line_plot.jpg}
	\caption{}
	\label{fig:chapters/11/10/4/21/1}
\end{figure}

    \item passing through 
    $\vec{A} = \myvec{2\\2\sqrt{3}}$ and inclined with the x-axis at an angle 
    of 75\textdegree.
\label{chapters/11/10/2/4}
\\
    \solution 
    \begin{align}
	    \vec{n} &= \myvec{-1\\2+\sqrt{3}}
        \label{eq:11/10/2/4normal-vec}
	\\
        \implies \myvec{-1&2+\sqrt{3}}\vec{x} &=\myvec{-1&2+\sqrt{3}}\myvec{2\\2\sqrt{3}}  
	    \\
	    &= 4\brak{\sqrt{3}+1}
        \label{eq:11/10/2/4line}
    \end{align}
is the desired equation.  See \figref{fig:11/10/2/4line}.
    \begin{figure}[H]
        \centering
        \includegraphics[width=0.75\columnwidth]{chapters/11/10/2/4/figs/line.png}
        \caption{}
        \label{fig:11/10/2/4line}
    \end{figure}

\item intersecting the x-axis at a distance of 3 units to the left of origin with slope of -2.
\label{chapters/11/10/2/5}
\\
\solution 
		From the given information,
\begin{align}		
	\vec{A}=\myvec{-3\\0},\,
\vec{n} = \myvec{2 \\1}.
\end{align}
The desired equation of the line is
\begin{align}
\implies	\myvec { 2 & 1 } \brak{ \vec{x} - \myvec{ -3 \\ 0}} &= 0  \\
	\text{or, }	\myvec{ 2 & 1} \vec{x}  &= -6
        \label{eq:chapters/11/10/2/5/1}
\end{align}
See \figref{fig:chapters/11/10/2/5/Fig1}.
\begin{figure}[H]
	\begin{center}
		\includegraphics[width=0.75\columnwidth]{chapters/11/10/2/5/figs/line1.pdf}
	\end{center}
\caption{}
\label{fig:chapters/11/10/2/5/Fig1}
\end{figure}


\item intersecting the y-axis at a distance of 2 units above the origin and making an
angle of $30\degree$ with positive direction of the x-axis.
\\
\solution 
\begin{align}
    \vec{n} =  \myvec{-\frac{1}{\sqrt{3}} \\ 1},
    \vec{A} = \myvec{0 \\ 2}.
\end{align}
Hence, 
the equation of the line is given by
\begin{align}
\myvec{-\frac{1}{\sqrt{3}}&1}\brak{ \vec{x} - \myvec{0 \\ 2}} &= 0  \\
    \text{or, }	\myvec{-\frac{1}{\sqrt{3}}&1} \vec{x}  &= 2
\end{align}
%
See
    \figref{fig:chapters/11/10/2/6/line}.
\begin{figure}[H]
    \centering
    \includegraphics[width=0.75\columnwidth]{chapters/11/10/2/6/figs/line.png}
    \caption{}
    \label{fig:chapters/11/10/2/6/line}
\end{figure}


\item passing through (1,2) and making angle $30\degree$ with $y$-axis.
\item passing through the points $\vec{A}\myvec{-1\\1}$ and $\vec{B}\myvec{2\\-4}$.
\label{chapters/11/10/2/7}
\\
\solution 
		From \eqref{prop:lin-eq-unit-mat},
\begin{align}
	\myvec{ -1 & 1\\  2 & -4 }\vec{n} = \myvec{1 \\ 1 }
	\\
	\implies 
	\augvec{2}{1}{ 
	-1 & 1 & 1
	\\  
	2 & -4 & 1
	}
     \xleftrightarrow[]{R_2 \leftarrow R_2+2R_1}
	\augvec{2}{1}{ 
	-1 & 1 & 1
	\\ 
	0 & -2 & 3 
	}
	\\
     \xleftrightarrow[]{R_1 \leftarrow 2R_1+R_2}
	\augvec{2}{1}{ 
	-2 & 0 &5 
	\\ 
	0 & -2 & 3 
	}
	\implies \vec{n} = -\frac{1}{2}\myvec{ 5 \\ 3}
\end{align}
Thus, from
		\eqref{prop:lin-eq-unit},
the equation of the line is
\begin{align}
 \myvec{ 5 & 3}\vec{x}  &= -2
\end{align}
See 
   \figref{fig:chapters/11/10/2/7/Line_AB}.
\begin{figure}[H]
  \centering
   \includegraphics[width=0.75\columnwidth]{chapters/11/10/2/7/figs/Figure_1.png}
   \caption{}
   \label{fig:chapters/11/10/2/7/Line_AB}
\end{figure}





\item passing through the points $(3,4,-7)$ and $(1,-1,6)$. 
\item The vector equation of the line 
\begin{align*}
	\frac{x-5}{3}=\frac{y+4}{7}=\frac{z-6}{2} 
\end{align*}
is \noindent\rule{2cm}{0.4pt}. 
\item The vector equation of the line 
\begin{align*}
	\frac{x-5}{3}=\frac{y+4}{7}=\frac{z-6}{2}
\end{align*}
 is \noindent\rule{2cm}{0.4pt}.
\item 
The vertices of triangle $PQR$ are $\vec{P}(2,1), \vec{Q}(-2,3), \vec{R}(4,5)$. Find the equation of the median through $\vec{R}$.
\label{chapters/11/10/2/9}
\\
\solution
	\begin{figure}[H]
		\centering
 \includegraphics[width=0.75\columnwidth]{chapters/11/10/2/9/figs/line.png}
		\caption{}
		\label{fig:11/10/2/9}
  	\end{figure}
	See Fig. 
		\ref{fig:11/10/2/9}.
Using section formula, the mid point of $PQ$ is
\begin{align}
\vec{A} = \frac{\vec{P} +\vec{Q} }{2}
	= {\myvec{0\\2}}
\end{align} 
Following the approach in \probref{chapters/11/10/2/7},
\begin{align*}
	\augvec{2}{1}{ 
	4 & 5 & 1
	\\  
	0 & 2 & 1
	}
	\xleftrightarrow[R_2 \leftarrow 4R_2 ]{R_1 \leftarrow 2R_1 -5R_2}
	\augvec{2}{1}{ 
	8 & 0 & -3 
	\\ 
	0 & 8 & 4 
	}
	\implies \vec{n} = \frac{1}{8}\myvec{ -3 \\ 4}
\end{align*}
Thus,
the equation of the line is 
\begin{align}
	\myvec{-3 & 4}\vec{x} =8 
\end{align}

	\item Find the equations of the planes that pass through the points
\begin{enumerate}
\item $\vec{A}= \myvec{1\\1\\– 1}, \vec{B}=\myvec{6\\4\\– 5},\vec{C}= \myvec{– 4\\– 2\\3}$
\item $\vec{A}= \myvec{1\\1\\0}, \vec{B}= \myvec{1\\2\\1}, \vec{C}= \myvec{– 2\\2\\-1}$
\end{enumerate}
    \solution
		\begin{enumerate}
	\item From 
		\eqref{prop:lin-eq-unit-mat},
\begin{align}
\myvec{1&1&-1\\ 6&4&-5\\ -4&-2&3} \vec{n} = \myvec{1\\1\\1}
\end{align}
\begin{align*}
	\implies \myvec{1&1&-1&\vrule&1\\6&4&-5&\vrule&1\\-4&-2&3&\vrule&1}
	\\
\xleftrightarrow[R_3 \leftarrow R_3 + 4R_1]{R_2 \leftarrow R_2 - 6R_1}
\myvec{1&1&-1&\vrule&1\\0&-2&1&\vrule&-5\\0&2&-1&\vrule&5}\\ 
\xleftrightarrow[{R_1 \leftarrow 2R_1 + R_2}] {R_3 \leftarrow R_3 + R_2}
\myvec{2&0&-1&\vrule&-3\\0&2&-1&\vrule&5\\0&0&0&\vrule&0}
\end{align*}
Since we obtain a 0 row, 
the given points are collinear.
The direction vector of the line is
\begin{align}
\vec{m}=\vec{B}-\vec{C} \equiv \myvec{5\\3\\-4}
\end{align}
and the equation of a line is given by,
\begin{align}
	\vec{x}&=\vec{A}+  \kappa\vec{m}\\
&= \myvec{1\\1\\– 1} + \kappa \myvec{5\\3\\-4}
\end{align}
See 
     \figref{fig:chapters/12/11/3/6/1}.
\begin{figure}[H]
  \centering
   \includegraphics[width=0.75\columnwidth]{chapters/12/11/3/6/figs/collinear_points.png}
    \caption{}
     \label{fig:chapters/12/11/3/6/1}
     \end{figure}
     \item  In this case, 
\begin{align}
\myvec{1&1&0 \\ 1&2&1 \\ -2&2&-1} \vec{n}=\vec{1}
\end{align}
\begin{align*}
\implies
\myvec{1&1&0&\vrule&1\\1&2&1&\vrule&1\\-2&2&-1&\vrule&1}
\\
\xleftrightarrow[R_3 \leftarrow R_3 + 2R_1]{R_2 \leftarrow R_2 - R_1}
\myvec{1&1&0&\vrule&1\\0&1&1&\vrule&0\\0&4&-1&\vrule&3}
\\
	\xleftrightarrow[R_3 \leftarrow R_3 - 4R_2]{R_1 \leftarrow R_1- R_2}
\myvec{1&0&-1&\vrule&1\\0&1&1&\vrule&0\\0&0&-5&\vrule&3}\\
	\xleftrightarrow[R_2 \leftarrow 5R_2 + R_3]{R_1 \leftarrow 5R_1- R_3}
\myvec{5&0&0&\vrule&2\\0&5&0&\vrule&3\\0&0&5&\vrule&-3}
\end{align*}
Hence, the equation of the plane is
\begin{align}
\myvec{2 & 3 & -3} \vec{x} = 5
\end{align}
\end{enumerate}

\item Find the equation of the plane through the points $(2,1,0)$, $(3,-2,-2)$ and $(3,1,7)$.
\item A plane passes through the points $(2,0,0) (0,3,0)$ and $(0,0,4)$. The equation of the plane is \noindent\rule{2cm}{0.4pt}.
\item If the intercept of a line between the coordinate axes is divided by the point (-5,4) in the ratio 1:2 then find the equation of the line.
\item Find the equation of a line that cuts off equal intercepts on the coordinate axes and passes through the point $(2,3)$.  
	\\
\solution 
\label{chapters/11/10/2/12}
Let $(a,0)$  and  $(0,a)$ be the intercept points. 
\begin{align}
\vec{m} 
        &=   \myvec{
		a \\
		0 
		} - \myvec{
		   0 \\
		   a
		}  
        		  \equiv \myvec{
                           1 \\
			   -1 
		         } 
			 \\
			 \implies
\vec{n} &=  \myvec{
		     1 \\
		     1
	     } 
\end{align}
and 
the equation of the  line is
\begin{align}
	\myvec { 1 & 1 } \brak{ \vec{ x  - \myvec{ 2 \\
                                   3
			     }
		}}  &= 0  \\
\implies		\myvec{ 1 & 1} \vec{x}  &= 5 
        \label{eq:11/10/2/12/1}
\end{align}
See  \figref{fig:11/10/2/12/Fig1}.
\begin{figure}[H]
	\begin{center}
		\includegraphics[width=0.75\columnwidth]{chapters/11/10/2/12/figs/problem12.pdf}
	\end{center}
\caption{}
\label{fig:11/10/2/12/Fig1}
\end{figure}


\item 
Find the equation of a line passing through a point (2,2) and cutting off intercepts on the axes whose sum is 9.
\label{chapters/11/10/2/13}
	\\
	\solution 
Let  the intercept points be
\begin{align}
{\vec{P}}=\myvec{
  a\\
  0}
 , {\vec{Q}}=\myvec{
  0\\
  b}
  \text{ and }
   {\vec{R}}=\myvec{
  2\\
  2}
\end{align}
be the given point.  
Forming the collinearity matrix from 
		\eqref{prop:lin-dep-rank},
\begin{align}
	\myvec{ \vec{P}-\vec{Q} &\vec{P}-\vec{R}} 
	=
	 \myvec{
  a & a-2\\
  -b & -2
 }
\end{align}
which is singular if 
\begin{align}
 ab -2\brak{a+b} = 0
 \implies ab = 18
		\label{eq:11/10/2/13-a+b}
		\\
\because  a + b = 9.
\end{align}
$\therefore a,b$
are the roots of
\begin{align}
	x^2 -9x +18 = 0.
\end{align}
yielding
\begin{align}
	\myvec{a \\ b} = \myvec{6 \\ 3}, \myvec{3\\6}
\end{align}
Since 
\begin{align}
	\vec{m} = \myvec{a \\ -b},
	\vec{n} = \myvec{b \\ a} \equiv \myvec{1 \\ 2}, \myvec{2\\1}
\end{align}
Thus, the possible equations of the line are 
\begin{align}
\myvec{1 & 2}\vec{x} = 6
	\\
	\myvec{2&1}\vec{x} = 6
\end{align}
		See \figref{fig:11/10/2/13}.
	\begin{figure}[H]
		\centering
 \includegraphics[width=0.75\columnwidth]{chapters/11/10/2/13/figs/assign4.png}
		\caption{}
		\label{fig:11/10/2/13}
  	\end{figure}

\item Find the equation of the lines which passes the point (3,4) and cuts off intercepts from the coordinate axes such that their sum is 14.
\item Find the equation of the straight line which passes through the point (1, -2) and cuts off equal intercepts from axes.
\item Find the equation of the line which passes through the point (-4,3) and the portion of the line intercepted between the axes is divided internally in ratio 5:3 by this point.
\item Consider the following population and year graph. Find the slope of the line AB and using it, find what will be the population in the year 2010.
\\
\begin{figure}[H]
\centering
\includegraphics[width=0.75\columnwidth]{chapters/11/10/1/14/figs/fig.png}
\caption{}
\label{fig:chapters/11/10/1/14/1}
\end{figure}
\solution
Given
\begin{align}
	c_1 = \frac{7}{3},\,
c_2 = -6.
\end{align}
	From \eqref{eq:parallel_lines},
we need to find $c$ such that,
\begin{align}
	\abs{c-c_1} = \abs{c-c_2} \implies c = \frac{c_1+c_2}{2}
	 = -\frac{11}{6}.
\end{align}
Hence, the desired equation is
\begin{align}
	\myvec{3 & 2}\vec{x} &= -\frac{11}{6}
\end{align}
	See \figref{fig:chapters/11/10/4/21/1}.
\begin{figure}[H]
	\centering
	\includegraphics[width=0.75\columnwidth]{chapters/11/10/4/21/figs/line_plot.jpg}
	\caption{}
	\label{fig:chapters/11/10/4/21/1}
\end{figure}

\item Slope of a line which cuts off intercepts of equal length on the axes is 
\begin{enumerate}
\item -1
\item -0
\item 2
\item $\sqrt{3}$
\end{enumerate}
\item If the coordinates of middle point of the portion of a line intercepted between the coordinate axes is (3,2),then the equation of the line will be
\begin{enumerate}
\item $2x+3y=12$
\item $3x+2y=12$
\item $4x-3y=6$
\item $5x-2y=10$
\end{enumerate}
\item If the line $\frac{x}{a}+\frac{y}{b}=1$ passes the points (2,-3) and (4,-5), then $(a,b)$ is 
\begin{enumerate}
\item (1,1)
\item (-1,1)
\item (1,-1)
\item (-1,-1)
\end{enumerate}
\item The intercepts made by the plane $2x-3y+5z+4=0$ on the co-ordinate axis are $\brak{-2,\frac{4}{3},-\frac{4}{5}}$.
\item The line $\overrightarrow{r}=2\hat{i}-3\hat{j}-\hat{k}+\lambda(\hat{i}-\hat{j}+2\hat{k})$ lies in the plane $\overrightarrow{r} \cdot (3\hat{i}+\hat{j}-\hat{k})+2=0$.
\end{enumerate}
