\iffalse
\documentclass[10pt]{article}
\usepackage{graphicx}
\def\inputGnumericTable{}
\usepackage[latin1]{inputenc}
\usepackage{fullpage}
\usepackage{color}
\usepackage{array}
\usepackage{longtable}
\usepackage{calc}
\usepackage{multirow}
\usepackage{hhline}
\usepackage{ifthen}
\usepackage{amsmath}
\usepackage[none]{hyphenat}
\usepackage{listings}
\usepackage[english]{babel}
\usepackage{siunitx}
\usepackage{caption}
\usepackage{booktabs}
\usepackage{array}
\usepackage{extarrows}
\usepackage{enumerate}
\usepackage{enumitem}
\usepackage{amsmath}
\usepackage{commath}
\usepackage{gensymb}
\usepackage{amssymb}
\usepackage{multicol}
%\usepackage[utf8]{inputenc}
\lstset{
 frame=single,
 breaklines=true
}
\usepackage{hyperref}
\usepackage[margin=0.65in]{geometry}	 
%\usepackage{exsheets}% also loads the `tasks' package
\usepackage{atbegshi}
\AtBeginDocument{\AtBeginShipoutNext{\AtBeginShipoutDiscard}}

%new macro definitions
\renewcommand{\labelenumi}{(\roman{enumi})}
\newcommand{\mydet}[1]{\ensuremath{\begin{vmatrix}#1\end{vmatrix}}}
\providecommand{\brak}[1]{\ensuremath{\left(#1\right)}}
\newcommand{\solution}{\noindent \textbf{Solution: }}
\newcommand{\myvec}[1]{\ensuremath{\begin{pmatrix}#1\end{pmatrix}}}
\newenvironment{amatrix}[1]{%
	\left(\begin{array}{@{}*{#1}{c}|c@{}}
}{%
	\end{array}\right)
}

\newcommand{\myaugvec}[2]{\ensuremath{\begin{amatrix}{#1}#2\end{amatrix}}}
\providecommand{\norm}[1]{\left\1Vert#1\right\rVert}
\let\vec\mathbf{}


%\SetEnumitemKey{twocol}{
% before=\raggedcolumns\begin{multicols}{2},
% after=\end{multicols}}
%\SetEnumitemKey{fourcol}{
% before=\raggedcolumns\begin{multicols}{4},
% after=\end{multicols}} 


\begin{document}
\begin{center}
\title{\textbf{CIRCLES}}
\date{\vspace{-5ex}}
\maketitle
\end{center}
\section*{9$^{th}$Math - Chapter 10}
This is Problem-7 from Exercise 10.5\\\\
\fi
\begin{figure}[H]
	\begin{center}
		\includegraphics[width=0.75\columnwidth]{./chapters/9/10/5/7/figs/fig.pdf}
	\end{center}
\caption{}
\label{fig:chapters/9/10/5/7/1}
\end{figure}
The input parameters for construction
are available in Table
	\ref{tab:chapters/9/10/5/7/1}.
\begin{table}[H]
	\centering
	%\subimport{../chapters/9/10/5/7/tables/}{table.tex}
     \input{chapters/9/10/5/7/tables/table.tex}
%	\caption{}
	\label{tab:chapters/9/10/5/7/1}
\end{table}
From the given information,
\begin{align}
	\vec{A}&=r\myvec{\cos0\\ \sin0}=\myvec{2\\0}\\
	\vec{B}&=r\myvec{\cos\frac{\pi}{3}\\ \sin\frac{\pi}{3}}=\myvec{1\\\sqrt{3}}\\
	\vec{C}&=2\vec{O}-\vec{A}=\myvec{-2\\0}\\
	\vec{D}&=2\vec{O}-\vec{B}=\myvec{-1\\-\sqrt{3}}
\end{align}
Consider a circle of radius 2 units. Let $AC$ and $DB$ be diameters of circle which are diagonals of cyclic quadrilateral.
Then, from the above equations,
\begin{align}
	\vec{A}-\vec{C} &= \vec{D}-\vec{B}
\end{align}
\begin{enumerate}
\item \label{itm:chapters/9/10/5/7/1} $AB$ and $DC$ are parellel to each other
\begin{align}
	\vec{A}-\vec{B} &= \myvec{2\\0} - \myvec{1\\\sqrt{3}}\\
	&=\myvec{1\\-\sqrt{3}}\\
	\vec{D}-\vec{C} &= \myvec{-1\\-\sqrt{3}} - \myvec{-2\\0}\\
	&=\myvec{1\\-\sqrt{3}}
\end{align}
	Thus,  $ABCD$ is parallelogram.
\item \label{itm:chapters/9/10/5/7/2} Let's check the angle between adjacent sides of this quadrilateral,$AB$ and $BC$
\begin{align}
	\brak{\vec{A}-\vec{B}}^{\top}\brak{\vec{B}-\vec{C}} &=\myvec{1 & -\sqrt{3}}\myvec{3\\\sqrt{3}}\\	
	&= 0\\
	\implies \angle ABC &= 90\degree
\end{align}
from \ref{itm:chapters/9/10/5/7/1} and \ref{itm:chapters/9/10/5/7/2} , Hence the quadrilateral $ABCD$ is rectangle.
\end{enumerate}
See Fig. 
\ref{fig:chapters/9/10/5/7/1}.
