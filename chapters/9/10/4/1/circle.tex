\iffalse
\documentclass[journal,12pt,twocolumn]{article}
\usepackage{graphicx}
\usepackage[none]{hyphenat}
\usepackage[margin=0.5in]{geometry}
\usepackage[cmex10]{amsmath}
\usepackage{array}
\usepackage{booktabs}
\usepackage{gensymb}
\usepackage{textcomp}
\title{\textbf{circle Assignment}}
\author{Dulla Srinivas - FWC22041}
\date{\today}
\providecommand{\norm}[1]{\left\lVert#1\right\rVert}
\providecommand{\abs}[1]{\left\vert#1\right\vert}
\let\vec\mathbf
\newcommand{\myvec}[1]{\ensuremath{\begin{pmatrix}#1\end{pmatrix}}}
\newcommand{\mydet}[1]{\ensuremath{\begin{vmatrix}#1\end{vmatrix}}}
\providecommand{\brak}[1]{\ensuremath{\left(#1\right)}}

\begin{document}

\maketitle
\section*{Problem Statement:}
\fi
Two circles of radii 5cm and 3cm intersect at two points and the distance between their center is 4cm. Find the length of the common chord.
\\
\solution 
See Fig. 
		\ref{fig:9/10/4/1}.
		and
	\begin{figure}[H]
		\centering
 \includegraphics[width=0.75\columnwidth]{chapters/9/10/4/1/figs/cccc.png}
		\caption{}
		\label{fig:9/10/4/1}
  	\end{figure}

	\begin{table}[H]
		\centering
 \input{chapters/9/10/4/1/tables/table-9-10-4-1.tex}
		\caption{}
		\label{tab:9/10/4/1}
  	\end{table}
	From 
		Table 
		\ref{tab:9/10/4/1},
	\eqref{eq:circ-eq}
	and
	\eqref{eq:circ-cr},
	the equations of the two circles are
\begin{align}
	\begin{split}
	\norm{\vec{x}}^2 -25 &= 0
	\\
	\norm{\vec{x}}^2 - 8 \vec{e}_1^{\top}\vec{x} +7 &= 0
	\end{split}
		\label{eq:9/10/4/1/circs}
\end{align}
From 
		\eqref{eq:9/10/4/1/circs}
and
	\eqref{eq:circ-chord}
the equation of the common chord is 
\begin{align}
	   \vec{e}_1^{\top}\vec{x} 
	   &= 4
		\label{eq:9/10/4/1/chord}
\end{align}
%
It is easy to verify that 
\begin{align}
	\vec{q} = 4\vec{e}_1
\end{align}
is a point on 
		\eqref{eq:9/10/4/1/chord}.
		Substituting
\begin{align}
\vec{m} = \vec{e}_2, \vec{q} = 4\vec{e}_1, 
\vec{V}=\vec{I}, \vec{u} = \vec{0}, f = -25
\end{align}
in 
\eqref{eq:chord-len},
		the length of the chord in 
\eqref{eq:conic_tangent}
is given by 
\begin{align}
 \frac{2\sqrt{
\sbrak{
\vec{e}_2^{\top}\brak{4\vec{e}_1}
}^2
-
\brak
{
16\vec{e}_1^{\top}\vec{e}_1 -25
}
\brak{\vec{e}_2^{\top}\vec{e}_2}
}
}
{
\vec{e}_2^{\top}\vec{e}_2
}\norm{\vec{e}_2}
= 6
  \end{align}
	

\iffalse

/sdcard/github/matrix-analysis/chapters/9/10/4/1/tables/table-9-10-4-1.tex
\section*{Solution:}

\begin{figure}[H]
\centering
\includegraphics[width=0.75\columnwidth]{cccc.png}
\caption{Diagram generated using python}
\label{fig:square}
\end{figure}
\subsection{Theory:}
{They given two circles radius first circle radii is 5cm(Q1) and  second circle radii is 3cm(Q2) distance between circle 1 and circle 2 is 4cm. we have find the length of the chord.  }

\subsection{Mathematical Calculation:}
$\vec{O} = \myvec{0\\0}$ 
\end{center}
\subsection{Deriving equation for Circle in matrix form}
\vspace{0.2cm}
\begin{flushleft}
The equation of circle in matrix form is,\\
\vspace{0.25cm}
\end{flushleft}
\vspace{0.25cm}
\begin{equation}
 \vec{x}^T \vec{V} \vec{x} + 2 \vec{u}^T \vec{x} + f = 0
\end{equation}
\begin{flushleft}
Where\\
\end{flushleft}
\center
$\vec{V} = \vec{I}= \myvec{ 1 & 0\\ 0 & 1} , \vec{u} = \myvec{0 \\ 0}, \vec{f}=-25$\\
\endcenter
\center
  $\implies$  $ \vec{x}^T$$\vec{I}$ $\vec{x}$  + 2 $ \myvec{0\\0}^T \vec{x} -25 = 0$
\endcenter
\begin{flushleft}
\vspace{0.23cm}
Therefore, the circle equation can be written as
\end{flushleft}
\begin{equation}
    \vec{x}^T \vec{x} + 2 \myvec{0\\0}^T \vec{x} -25= 0
\end{equation}
\endcenter
\begin{flushleft}
The equation of circle in matrix form is,\\
\vspace{0.25cm}
\end{flushleft}
\vspace{0.25cm}
\begin{equation}
 \vec{x}^T \vec{V} \vec{x} + 2 \vec{u}^T \vec{x} + f = 0
\end{equation}
\begin{flushleft}
Where\\
\end{flushleft}
\center
$\vec{V} = \vec{I}= \myvec{ 1 & 0\\ 0 & 1} , \vec{u} = \myvec{-4 \\ 0}, \vec{f}=7$\\  \vspace{10mm}
\endcenter
\center
  $\implies$  $ \vec{x}^T$$\vec{I}$ $\vec{x}$  + 2 $ \myvec{-4\\0}^T \vec{x} +7= 0$
\endcenter
\begin{flushleft}
\vspace{0.23cm}
Therefore, the circle equation can be written as
\begin{equation}
    \vec{x}^T \vec{x} + 2 \myvec{-4\\0}^T \vec{x} +7= 0
\end{equation}

  Here we have to Find the Intersection of Two conics
\begin{equation}
 \vec{x}^T \vec{V_1} \vec{x} + 2 \vec{u_1}^T \vec{x} + \vec f_1 = 0
\end{equation}




\begin{equation} 
\vec{x}^T \vec{V_2} \vec{x} + 2 \vec{u_2}^T \vec{x} + \vec f_2 = 0
\end{equation}
The locus of their pair of straight lines\\

\begin{equation}                       
	\vec{x}^T \vec{(V_1+\mu V_2)x}+2\vec{(u_1+\mu u_2)x}^T \vec x + \vec f_1+ \vec f_2  = 0              
\end{equation}

  \begin{align}
\mydet{
\vec{V}_1+ \mu \vec{V}_2&\vec{u}_1+\mu\vec{u}_2
\\
	\brak{\vec{u}_1+\mu\vec{u}_2}^{\top}&f_1 +f_2
}
	= 0, 
%    \label{eq:conic_quad_form_int-mat}
\end{align}

 

\begin{align}
	\vec {V_1} = \myvec{  1 \hspace{10mm} 0  \vspace{4mm} \hspace{10mm }\\ 0  \hspace{10mm} 1 \\ } \\  \vspace{5mm}
\vec {V_2} = \myvec{  1 \hspace{10mm} 0  \vspace{4mm} \hspace{10mm }\\ 0  \hspace{10mm} 1\\ } \\ \vspace{5mm}
\vec {u_1} = \myvec{  0    \vspace{4mm} \hspace{10mm }\\ 0 \hspace{10mm} \\ } \\ \vspace{5mm}
\vec {u_2} = \myvec{  -4   \vspace{4mm} \hspace{10mm }\\ 0 \hspace{10mm} \\ } \\ \vspace{5mm}
\vec f_1=-25  \hspace{10mm}
\vec f_2=7
\vspace{5mm}
 = \myvec{I +\mu I\hspace{10mm} 0- \myvec{4 \\ 0},  \vspace{4mm}\\ 0-\mu (4,0) \hspace{10mm} 0\\      }\\
\vspace{10mm}
= \myvec{1 +\mu \hspace{10mm} 0 \hspace{10mm}  \myvec{-4\mu},  \vspace{4mm}\\ 0  \hspace{10mm} 1+\mu \hspace{10mm} 0  ,\vspace{4mm}\\ -4 \mu  \hspace{10mm} 0  \hspace{10mm} 7\mu - 25 \\      }\\
\vspace{10mm}
= \myvec{-25 \mu + 7\mu +\mu \hspace{10mm} -16\mu^2\hspace{10mm}  ,  \vspace{4mm}\\ 0  \hspace{20mm} -25\mu + 7\mu+\mu  ,\vspace{4mm}\\ }\\
\vspace{10mm}
\mu = -1
\vspace{10mm}
\end{align}

 
 
 \subsection{According to the equation 7}
 2\myvec{-1 \myvec{-4\\0} x }-25-7=0\\
 \vspace{5mm}
 
\vspace{5mm}
  8x = 32\\
  \vspace{5mm}
   x=4\\
    \vspace{5mm}
    
    So,  = \myvec{4\\ 3} - \myvec{4 \\ -3}
    \hspace{5mm}
 $   p_1=(4,-3)$
\subsection{ So The length of the common chord is 6cm}
   $$= || p - p_1|| $$
        = \myvec{4\\ 3} - \myvec{4 \\ -3} \\
        \vspace{5mm}
        =6
  \section*{\large Construction}
{
\setlength\extrarowheight{5pt}
\begin{tabular}{|c|c|c|}
  \hline
  \textbf{Symbol}&\textbf{Value}&\textbf{Description}\\
  \hline
	$r_1$&5&Radius \\
  \hline
	$r_2$&3&Radius\\
  \hline
  O&$(0,0)$&Center\\
  \hline
  O_1& $(4,0)$ &Center\\
  \hline
  P&$(4,3)$&Point Of intersection\\
  \hline
  P_1&$(4,-3)$&Point Of intersection\\
  \hline
  P-P_1& $6$ &Length of the common chord\\
  \hline
  
  
\end{tabular}
}
\end{document}
\fi
