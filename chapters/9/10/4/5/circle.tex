\iffalse
\documentclass[journal,12pt,twocolumn]{IEEEtran}
%
\usepackage{setspace}
\usepackage{gensymb}
%\doublespacing
\singlespacing

%\usepackage{graphicx}
%\usepackage{amssymb}
%\usepackage{relsize}
\usepackage[cmex10]{amsmath}
%\usepackage{amsthm}
%\interdisplaylinepenalty=2500
%\savesymbol{iint}
%\usepackage{txfonts}
%\restoresymbol{TXF}{iint}
%\usepackage{wasysym}
\usepackage{amsthm}
%\usepackage{iithtlc}
\usepackage{mathrsfs}
\usepackage{txfonts}
\usepackage{stfloats}
\usepackage{bm}
\usepackage{cite}
\usepackage{cases}
\usepackage{subfig}
%\usepackage{xtab}
\usepackage{longtable}
\usepackage{multirow}
%\usepackage{algorithm}
%\usepackage{algpseudocode}
\usepackage{enumitem}
\usepackage{mathtools}
\usepackage{steinmetz}
\usepackage{tikz}
\usepackage{circuitikz}
\usepackage{verbatim}
\usepackage{tfrupee}
\usepackage[breaklinks=true]{hyperref}
%\usepackage{stmaryrd}
\usepackage{tkz-euclide} % loads  TikZ and tkz-base
%\usetkzobj{all}
\usetikzlibrary{calc,math}
\usepackage{listings}
    \usepackage{color}                                            %%
    \usepackage{array}                                            %%
    \usepackage{longtable}                                        %%
    \usepackage{calc}                                             %%
    \usepackage{multirow}                                         %%
    \usepackage{hhline}                                           %%
    \usepackage{ifthen}                                           %%
  %optionally (for landscape tables embedded in another document): %%
    \usepackage{lscape}     
\usepackage{multicol}
\usepackage{chngcntr}
%\usepackage{enumerate}

%\usepackage{wasysym}
%\newcounter{MYtempeqncnt}
\DeclareMathOperator*{\Res}{Res}
%\renewcommand{\baselinestretch}{2}
\renewcommand\thesection{\arabic{section}}
\renewcommand\thesubsection{\thesection.\arabic{subsection}}
\renewcommand\thesubsubsection{\thesubsection.\arabic{subsubsection}}

\renewcommand\thesectiondis{\arabic{section}}
\renewcommand\thesubsectiondis{\thesectiondis.\arabic{subsection}}
\renewcommand\thesubsubsectiondis{\thesubsectiondis.\arabic{subsubsection}}

% correct bad hyphenation here
\hyphenation{op-tical net-works semi-conduc-tor}
\def\inputGnumericTable{}                                 %%

\lstset{
%language=C,
frame=single, 
breaklines=true,
columns=fullflexible
}
%\lstset{
%language=tex,
%frame=single, 
%breaklines=true
%}

\begin{document}
%


\newtheorem{theorem}{Theorem}[section]
\newtheorem{problem}{Problem}
\newtheorem{proposition}{Proposition}[section]
\newtheorem{lemma}{Lemma}[section]
\newtheorem{corollary}[theorem]{Corollary}
\newtheorem{example}{Example}[section]
\newtheorem{definition}[problem]{Definition}
%\newtheorem{thm}{Theorem}[section] 
%\newtheorem{defn}[thm]{Definition}
%\newtheorem{algorithm}{Algorithm}[section]
%\newtheorem{cor}{Corollary}
\newcommand{\BEQA}{\begin{eqnarray}}
\newcommand{\EEQA}{\end{eqnarray}}
\newcommand{\define}{\stackrel{\triangle}{=}}

\bibliographystyle{IEEEtran}
%\bibliographystyle{ieeetr}


\providecommand{\mbf}{\mathbf}
\providecommand{\pr}[1]{\ensuremath{\Pr\left(#1\right)}}
\providecommand{\qfunc}[1]{\ensuremath{Q\left(#1\right)}}
\providecommand{\sbrak}[1]{\ensuremath{{}\left[#1\right]}}
\providecommand{\lsbrak}[1]{\ensuremath{{}\left[#1\right.}}
\providecommand{\rsbrak}[1]{\ensuremath{{}\left.#1\right]}}
\providecommand{\brak}[1]{\ensuremath{\left(#1\right)}}
\providecommand{\lbrak}[1]{\ensuremath{\left(#1\right.}}
\providecommand{\rbrak}[1]{\ensuremath{\left.#1\right)}}
\providecommand{\cbrak}[1]{\ensuremath{\left\{#1\right\}}}
\providecommand{\lcbrak}[1]{\ensuremath{\left\{#1\right.}}
\providecommand{\rcbrak}[1]{\ensuremath{\left.#1\right\}}}
\theoremstyle{remark}
\newtheorem{rem}{Remark}
\newcommand{\sgn}{\mathop{\mathrm{sgn}}}
\providecommand{\abs}[1]{\left\vert#1\right\vert}
\providecommand{\res}[1]{\Res\displaylimits_{#1}} 
\providecommand{\norm}[1]{\left\lVert#1\right\rVert}
%\providecommand{\norm}[1]{\lVert#1\rVert}
\providecommand{\mtx}[1]{\mathbf{#1}}
\providecommand{\mean}[1]{E\left[ #1 \right]}
\providecommand{\fourier}{\overset{\mathcal{F}}{ \rightleftharpoons}}
%\providecommand{\hilbert}{\overset{\mathcal{H}}{ \rightleftharpoons}}
\providecommand{\system}{\overset{\mathcal{H}}{ \longleftrightarrow}}
	%\newcommand{\solution}[2]{\textbf{Solution:}{#1}}
\newcommand{\solution}{\noindent \textbf{Solution: }}
\newcommand{\cosec}{\,\text{cosec}\,}
\providecommand{\dec}[2]{\ensuremath{\overset{#1}{\underset{#2}{\gtrless}}}}
\newcommand{\myvec}[1]{\ensuremath{\begin{pmatrix}#1\end{pmatrix}}}
\newcommand{\mydet}[1]{\ensuremath{\begin{vmatrix}#1\end{vmatrix}}}
%\numberwithin{equation}{section}
\numberwithin{equation}{subsection}
%\numberwithin{problem}{section}
%\numberwithin{definition}{section}
\makeatletter
\@addtoreset{figure}{problem}
\makeatother

\let\StandardTheFigure\thefigure
\let\vec\mathbf
%\renewcommand{\thefigure}{\theproblem.\arabic{figure}}
\renewcommand{\thefigure}{\theproblem}
%\setlist[enumerate,1]{before=\renewcommand\theequation{\theenumi.\arabic{equation}}
%\counterwithin{equation}{enumi}


%\renewcommand{\theequation}{\arabic{subsection}.\arabic{equation}}

\def\putbox#1#2#3{\makebox[0in][l]{\makebox[#1][l]{}\raisebox{\baselineskip}[0in][0in]{\raisebox{#2}[0in][0in]{#3}}}}
     \def\rightbox#1{\makebox[0in][r]{#1}}
     \def\centbox#1{\makebox[0in]{#1}}
     \def\topbox#1{\raisebox{-\baselineskip}[0in][0in]{#1}}
     \def\midbox#1{\raisebox{-0.5\baselineskip}[0in][0in]{#1}}

\vspace{3cm}


\title{Question: 9.10.4.5}
\author{Nikam Pratik Balasaheb (EE21BTECH11037)}





% make the title area
\maketitle

\newpage

%\tableofcontents

\bigskip

\renewcommand{\thefigure}{\theenumi}
\renewcommand{\thetable}{\theenumi}
%\renewcommand{\theequation}{\theenumi}

\section{Problem}
Three girls Reshma, Salma and Mandip are playing a game by standing on a circle of radius 5m drawn in a park. Reshma throws a ball to Salma, Salma to Mandip, Mandip to Reshma. If the distance between Reshma and Salma and
between Salma and Mandip is 6m each, what is the distance between Reshma and Mandip?

\section{Solution}
\fi
Consider Reshma, Salma and Mandip be standing at $\vec{A}$, $\vec{B}$ and $\vec{C}$ respectively, and the center the of the circle $\vec{O}$.
The input parameters are listed in Table 
\ref{tab:chapters/9/10/4/5/}.
Let 
\begin{align}
	\vec{B} = \myvec{0\\0},\,
	\vec{O} = \myvec{5\\0}
\end{align}
Therefore, the equation of the cicle is given by 
\begin{align}
	\norm{\vec{x}-\vec{O}}^2 &= 25\\
\implies 	\norm{\vec{x}}^2 - 2 \vec{O}^{\top}\vec{x} + \norm{\vec{O}}^2 - 25 &= 0\\
\implies	\norm{\vec{x}}^2 - 2 \myvec{5 & 0}\vec{x} &= 0
	\label{eq:chapters/9/10/4/5/1}
\end{align}
Also, $\vec{A}$ and $\vec{C}$ are equidistant (6m) from $\vec{B}$, we can say that they lie on the circle having $\vec{B}$ as center and radius 6m. Equation of this circle is given by
\begin{align}
	\norm{\vec{x}}^2 - 2\vec{B}^{\top} \vec{x} + \norm{\vec{B}}^2 - 36 &= 0\\
	\norm{\vec{x}}^2 &= 36\\
	i.e., \vec{u} = \myvec{0\\0}\;, \; f = -36
	\label{eq:chapters/9/10/4/5/2}
\end{align}
From \eqref{eq:chapters/9/10/4/5/1} and \eqref{eq:chapters/9/10/4/5/2}, the line passing through $\vec{A}$ and $\vec{C}$  is
\begin{align}
	\myvec{5&0} \vec{x} &= 18\\
\implies 	\vec{x} &= \myvec{\frac{18}{5}\\[1pt] 0} + \mu \myvec{0\\1}\\
	i.e., \; \vec{h} = \myvec{\frac{18}{5}\\[1pt] 0} \; , \; \vec{m} = \myvec{0\\1}
\end{align}
For the circle,$\vec{V} = \vec{I}$
\begin{multline}
	\mu_i = \frac{1}{\vec{m}^{\top}\vec{V}\vec{m}} \brak{-m^{\top}\brak{\vec{V}\vec{h}+\vec{u}} \pm \sqrt{\brak{\vec{m}^{\top}\brak{\vec{V}\vec{h}+\vec{u}}}^2 - \text{g}\brak{\vec{h}}\brak{\vec{m}^{\top}\vec{V}\vec{m}}}}
\end{multline}
where,
\begin{align}
	\text{g}\brak{\vec{h}} = \vec{h}^{\top}\vec{V}\vec{h} + 2\vec{u}^{\top}\vec{h} +f
\end{align}
yielding
\begin{align}
	\mu_i = \pm \frac{24}{5}
\end{align}
Therefore,
\begin{align}
	\vec{A} = \myvec{\frac{18}{5}\\[1pt] \frac{24}{5}},\,
	\vec{C} = \myvec{\frac{18}{5}\\[1pt] -\frac{24}{5}}
\end{align}
and the 
distance between Reshma and Mandip is
\begin{align}
	\norm{\vec{A} - \vec{C}} = \norm{\myvec{0\\[1pt] \frac{48}{5}}}
	= \frac{48}{5}
\end{align}
\begin{table}[H]
\begin{center}
	\input{chapters/9/10/4/5/tables/table_1.tex}
\end{center}
\caption{}
\label{tab:chapters/9/10/4/5/}
\end{table}
See Fig. 
    \ref{fig:chapters/9/10/4/5/}.
\begin{figure}[H]
  \centering
    \includegraphics[width=0.75\columnwidth]{chapters/9/10/4/5/figs/Figure_1.png}
    \caption{}
    \label{fig:chapters/9/10/4/5/}
\end{figure}




