\documentclass[journal]{IEEEtran}
\usepackage[a5paper, margin=10mm]{geometry}
\usepackage{float} % for [H] placement option

%\usepackage[a5paper, top=10mm, bottom=10mm, left=10mm, right=10mm]{geometry}

%
\usepackage{gvv-book}
\usepackage{gvv}

\makeindex

\begin{document}
\bibliographystyle{IEEEtran}
\onecolumn


\title{
	\begin{flushleft}
	MATRICES \\ In Geometry
	\\
\rule{0.4\columnwidth}{0.4pt}
\end{flushleft}
}
\author{
\vspace{7cm}
	\begin{flushleft}
\includegraphics[width=0.2\columnwidth]{figs/logo.jpg}
\\
		{	\huge G. V. V. Sharma}
	\end{flushleft}
%\IEEEpubid{\makebox[\columnwidth]{978-1-7281-5966-1/20/\$31.00 ©2020 IEEE \hfill} \hspace{\columnsep}\makebox[\columnwidth]{ }}
}
\maketitle

\newpage
\section*{About this Book}

This book attempts to introduce matrices through high school coordinate geometry. All problems in the book are from NCERT mathematics textbooks from Class 9-12.   
The content is sufficient for industry jobs.
There is no copyright, so readers are free to print and share.  
\begin{flushright}
\today
\end{flushright}
Github: https://github.com/gadepall/matgeo
		\\
License: https://creativecommons.org/licenses/by-sa/3.0/
\\
and
\\
https://www.gnu.org/licenses/fdl-1.3.en.html

\newpage


\tableofcontents

\newpage
%\twocolumn
\onecolumn


%\renewcommand{\theequation}{\theenumi}
\numberwithin{equation}{enumi}
\numberwithin{figure}{enumi}
%\renewcommand{\thefigure}{\theenumi}
\renewcommand{\thetable}{\theenumi}


\section{Vector Arithmetic}
\subsection{Formulae}
\input{chapters/formulae/vectors/two.tex}
\subsection{Direction}
\input{chapters/vectors/direction.tex}
\subsection{Section Formula}
\begin{enumerate}[label=\thesubsection.\arabic*,ref=\thesubsection.\theenumi]

\item Find the coordinates of the point which divides the join of $(-1,7) \text{ and } (4,-3)$ in the ratio 2:3.
	\\
		\solution
	Using section formula \eqref{eq:section_formula}, the desired point is
\begin{align}
\frac{1}{1+\frac{3}{2}}  \myvec{\myvec{
4\\
-3
}
  +
   \frac{3}{2}\myvec{
-1\\
7
}}
=\myvec{
1\\
3
}
\end{align}
See 
\figref{fig:chapters/10/7/2/1/Fig}
\begin{figure}[H]
\begin{center}
   \includegraphics[width=0.75\columnwidth]{chapters/10/7/2/1/figs/linefig.png}
\end{center}
\caption{}
\label{fig:chapters/10/7/2/1/Fig}
\end{figure}


\item Find the coordinates of the points of trisection of the line segment joining $(4,-1)$  and  $(-2,3)$.
	\\
		\solution
	Using section formula,
\begin{align}
\vec{R}=\frac{1}{1+\frac{1}{2}}\brak{\myvec{4\\-1}+\frac{1}{2}\myvec{-2\\-3}}
=\myvec{2\\ \frac{-5}{3}}\\
\vec{S}=\frac{1}{1+\frac{2}{1}}\brak{\myvec{4\\-1}+\frac{2}{1}\myvec{-2\\-3}}
=\myvec{0\\ \frac{-7}{3}}
\end{align}
which are the desired points of trisection.  See
		\figref{fig:chapters/10/7/2/2/Figure}
\begin{figure}[H]
\centering
\includegraphics[width=0.75\columnwidth]{chapters/10/7/2/2/figs/dj.pdf}
\caption{}
		\label{fig:chapters/10/7/2/2/Figure}
\end{figure}

\item Find the ratio in which the line segment joining the points $(-3,10) \text{ and } (6,-8)$ $\text{ is divided by } (-1,6)$.
	\\
		\solution
	Using section formula,
\begin{align}
         \myvec{-1\\6} &=\frac{{\myvec{-3\\10}+k\myvec{6\\-8}}}{1+n}\\
	 \implies 7k\myvec{1 \\ -2} &= 2\myvec{1 \\ -2}
	 \\
	 \text{or, } k &= \frac{2}{7}
\end{align}
See \figref{fig:10/7/2/4Fig1}.
\begin{figure}[H]
 \begin{center}
  \includegraphics[width=0.75\columnwidth]{chapters/10/7/2/4/figs/fig.png}
 \end{center}
\caption{}
\label{fig:10/7/2/4Fig1}
\end{figure}

\item Find the coordinates of a point $A$, where $AB$ is the diameter of a circle whose centre is $ C(2,-3)$  and  $B$ is $(1,4)$.
	\\
		\solution
		\begin{align}
	\vec{C} = \frac{\vec{A+B}}{2} 
	\implies 	\vec{A} = 2\vec{C}-\vec{B} 
	 = \myvec{3\\-10\\}	
	\end{align}       
	See 
\figref{fig:chapters/10/7/2/7Fig}.
\begin{figure}[H]
\begin{center}	
	\includegraphics[width=0.75\columnwidth]{chapters/10/7/2/7/figs/Vector1.png}
\end{center}
\caption{}
\label{fig:chapters/10/7/2/7Fig}
\end{figure}
	

\item If $A$ and  $B$ are $(-2,-2)$ and  $(2,-4)$, respectively, find the coordinates of $P$ such that $AP= \frac {3}{7}AB$  and $ P$ lies on the line segment $AB$.
	\\
		\solution
	Using section formula, 
\begin{align}
\vec{P}&=\frac{1}{1+\frac{3}{4}}\brak{\myvec{-2\\-2}+\frac{3}{4}\myvec{2\\-4}}
=\myvec{\frac{-2}{7}\\[1pt] \frac{-20}{7}}
\end{align}
See 
   \figref{fig:chapters/10/7/2/8/vec.png}.
\begin{figure}
   \centering 
 \includegraphics[width=0.75\columnwidth]{chapters/10/7/2/8/figs/vec.png}
   \caption{}
   \label{fig:chapters/10/7/2/8/vec.png}
   \end{figure}

\item Find the coordinates of the points which divide the line segment joining $A(-2,2)$  and  $B(2,8)$ into four equal parts.
	\\
		\solution
	Using section formula,
\begin{align}
\vec{R}_k=\frac{\vec{B}+k\vec{A}}{1+k}
\end{align}
See 
\tabref{tab:10/7/2/9}
and 
\figref{fig:chapters/10/7/2/9/Fig}
\begin{table}[H]
\centering
\caption{}
\label{tab:10/7/2/9}
\begin{tabular}{|c|c|}
\hline
	$k$ & $\vec{R}_k$ \\
\hline
3 & 
\myvec{
-1\\
\\
\frac{7}{2}
}\\
\hline
1 & \myvec{
0\\
5
}
\\
\hline
	$\frac{1}{3}$ &\myvec{
1
\\
\frac{13}{2}
}
 \\
\hline
\end{tabular}
\end{table}
\begin{figure}[H]
\begin{center}
   \includegraphics[width=0.75\columnwidth]{chapters/10/7/2/9/figs/10.7.2.9.png}
\end{center}
\caption{}
\label{fig:chapters/10/7/2/9/Fig}
\end{figure}


\item Find the position vector of a point $\vec{R}$ which divides the line joining two points $\vec{P}$
and $\vec{Q}$ whose position vectors are $\hat{i}+2\hat{j}-\hat{k}$ and $-\hat{i}+\hat{j}+\hat{k}$ respectively, in the
ratio 2 : 1
\begin{enumerate}
    \item  internally
    \item  externally
\end{enumerate}
\solution
		See 
\tabref{tab:chapters/12/10/2/15/}.
\begin{table}[H]
\centering
\caption{}
\label{tab:chapters/12/10/2/15/}
\begin{tabular}{|c|c|}
\hline
$k$ & $R_k$ \\
\hline
2 & $\frac{1}{3}\myvec{-1 \\ 4 \\ 1}$ \\
\hline
-2 & $\myvec{-3 \\ 0 \\ 3}$ \\
\hline
\end{tabular}
\end{table}

\item Find the position vector of the mid point of the vector joining the points $\vec{P}$(2, 3, 4)
and $\vec{Q}$(4, 1, –2).
\\
\solution
		\begin{align}
\vec{D}&=\frac{\vec{B}+\vec{C}}{2}
=\myvec{\frac{7}{2}\\[2pt] \frac{9}{2}}\\
\vec{E}&=\frac{\vec{A}+\vec{C}}{2}
=\myvec{\frac{5}{2}\\ 3}\\
\vec{F}&=\frac{\vec{A}+\vec{B}}{2}
=\myvec{5\\ \frac{7}{2}}
\\
\vec{P}
	&=\vec{Q}
=\vec{R}
=\frac{1}{3}\myvec{11\\11}
\\
\vec{G}&=\frac{\vec{A}+\vec{B}+\vec{C}}{3}
=\frac{1}{3}\myvec{11\\11}
\end{align} 
is the centroid.
See 
  \figref{fig:chapters/10/7/4/7/Figure}.
\begin{figure}[H]
\centering
\includegraphics[width=0.75\columnwidth]{chapters/10/7/4/7/figs/dj.pdf}
\caption{}
  \label{fig:chapters/10/7/4/7/Figure}
\end{figure}

\item Determine the ratio in which the line $2x+y  - 4=0$ divides the line segment joining the points  $\vec{A}(2, - 2)$  and  $\vec{B}(3, 7)$.
\\
\solution
	The given equation can be expressed as
\begin{align}
\label{eq:chapters/10/7/4/1vec}
    \myvec{2&1}\vec{x}&=4
\end{align}
Using section formula in
\eqref{eq:chapters/10/7/4/1vec},
\begin{align}
    \vec{n}^{\top}\myvec{\frac{k\vec{B+A}}{k+1}}&=c\\
    \implies k&=\frac{c-\vec{n}^{\top}\vec{A}}{\vec{n}^{\top}\vec{B}-c}
\end{align}
upon simplification.  Substituting numerical values, 
\begin{align}
    k=\frac{2}{9}
\end{align}
See 
\figref{fig:chapters/10/7/4/1vec}.
\begin{figure}[H]
\centering
\includegraphics[width=0.75\columnwidth]{chapters/10/7/4/1/figs/vec.pdf}
\caption{}
\label{fig:chapters/10/7/4/1vec}
\end{figure}


\item Let $\vec{A}(4, 2), \vec{B}(6, 5)$  and $ \vec{C}(1, 4)$ be the vertices of $\triangle ABC$.
\begin{enumerate}
\item The median from $\vec{A}$ meets $BC$ at $\vec{D}$. Find the coordinates of the point $\vec{D}$.
\item Find the coordinates of the point $\vec{P}$ on $AD$ such that $AP : PD = 2 : 1$.
\item Find the coordinates of points $\vec{Q}$ and $\vec{R}$ on medians $BE$ and $CF$ respectively such that $BQ : QE = 2 : 1$  and  $CR : RF = 2 : 1$.
\item What do you observe?
\item If $\vec{A}, \vec{B}$ and $\vec{C}$  are the vertices of $\triangle ABC$, find the coordinates of the centroid of the triangle.
\end{enumerate}
\solution
	\begin{align}
\vec{D}&=\frac{\vec{B}+\vec{C}}{2}
=\myvec{\frac{7}{2}\\[2pt] \frac{9}{2}}\\
\vec{E}&=\frac{\vec{A}+\vec{C}}{2}
=\myvec{\frac{5}{2}\\ 3}\\
\vec{F}&=\frac{\vec{A}+\vec{B}}{2}
=\myvec{5\\ \frac{7}{2}}
\\
\vec{P}
	&=\vec{Q}
=\vec{R}
=\frac{1}{3}\myvec{11\\11}
\\
\vec{G}&=\frac{\vec{A}+\vec{B}+\vec{C}}{3}
=\frac{1}{3}\myvec{11\\11}
\end{align} 
is the centroid.
See 
  \figref{fig:chapters/10/7/4/7/Figure}.
\begin{figure}[H]
\centering
\includegraphics[width=0.75\columnwidth]{chapters/10/7/4/7/figs/dj.pdf}
\caption{}
  \label{fig:chapters/10/7/4/7/Figure}
\end{figure}

\item Find the position vector of a point R which divides the line joining two points $P$ and $Q$ whose position vectors are $(2\vec{a}+\vec{b})$ and $(\vec{a}-3\vec{b})$
externally in the ratio 1 : 2. Also, show that $P$ is the mid point of the line segment $RQ$.\\
	\solution
		\begin{align}
\vec{R} = \frac{\vec{Q} -2\vec{P}}{-1} 
= \myvec{3\\5},
\\
\frac{ (\vec{R} + \vec{Q})}{2}
=\myvec{2\\1} =\vec{P}.
\end{align}
See 
\figref{fig:chapters/12/10/5/9/Figure1}.
\begin{figure}[H]
	\begin{center}
		\includegraphics[width=0.75\columnwidth]{chapters/12/10/5/9/figs/line.png}
	\end{center}
\caption{}
\label{fig:chapters/12/10/5/9/Figure1}
\end{figure}

\item The point which divides the line segment joining the points $\vec{P} (7, –6) $  and  $\vec{Q}(3, 4)$ in the 
ratio 1 : 2 internally lies in the
\begin{enumerate}
\item I quadrant
\item  II quadrant
\item  III quadrant
\item  IV quadrant
\end{enumerate}
\item If the point $\vec{P} (2, 1)$ lies on the line segment joining points $\vec{A} (4, 2)$  and $ \vec{B} (8, 4)$,
then
\begin{enumerate}
	\item $AP =\frac{1}{3}{AB}$ 
\item ${AP}={PE}$
\item ${PB}=\frac{1}{3}{AB}$
\item${AP}=\frac{1}{2}{AB}$
 \end{enumerate}
\item A line intersects the y-axis and x-axis at the points $\vec{P}$  and $\vec{Q}$, respectiveiy. lf $(2,5)$ is the mid-point of $\vec{PQ}$, then the coordinates of $\vec{P}$ and $ \vec{Q}$ are, respectively
\begin{enumerate}
	\item$(0,-5)$ and $(2,0)$
	\item$(0,-10)$ and $(-4,0)$
	\item$(0,4)$ and  $(-10,0)$
	\item$(0,-10)$ and $(4,0)$
\end{enumerate}
\item Point $\vec{P}(5,-3)$ is one of the two points of trisection of line segment joining the points $\vec{A}(7,-2)\text{ and }\vec{B}(1,-5)$
\item Points $\vec{A}(-6,10),\vec{B}(-4,6) \text{ and } \vec{C}(3,-8)$ are collinear such that $\vec{A}\vec{B}=  \frac{2}{9}\vec{A}\vec{C}$
\item In what ratio does the $x$-axis divide the line segment joining the points $(-4,-6)\text{ and }(-1,7)$? Find the coordinates of the point of division.
\item Find the ratio in which the point $\vec{P}\brak{\frac{3}{4},\frac{5}{12}}$ divides the line segment joining the points $\vec{A}\brak{\frac{1}{2},\frac{3}{2}}\text{ and } \vec{B}(2,-5)$.
\item If $\vec{P}(9a-2,-b)$ divides line segment joining $\vec{A}(3a+1,-3)\text{ and }\vec{B}(8a,5)$ in the ratio 3:1, find the values of $a$ and $b$.
\item The line segment joining the points $\vec{A}(3,2)\text{ and }\vec{B}(5,1)$ is divided at the point $\vec{P}$ in the ratio 1:2 which lies on $3x-18y+k=0$. Find the value of $k$.  
\item Find the coordinates of the point $\vec{R}$ on the line segment joining the points $\vec{P}(-1,3)\text{ and }\vec{Q}(2,5)$ such that $PR=\frac{3}{5}PQ$.
\item Find the ratio in which the line $2x+3y-5=0$ divides the line segment joining the points $(8,-9)$ and $(2,1)$. Also find the coordinates of the point of division.
\item If $\vec{a}$ and $\vec{b}$ are the postion vectors of $A$ and $B$, respectively, find the position vector of a point $C$ in $BA$ produced such that $BC=1.5BA$.
\item The position vector of the point which divides the join of points 2$\vec{a}$-3$\vec{b}$ $\text{and}$ $\vec{a}+\vec{b}$ in the ratio 3:1 is
	\begin{enumerate}
\item $\frac{3\vec{a}-2\vec{b}}{2}$
\item $\frac{7\vec{a}-8\vec{b}}{4}$
\item $\frac{\vec{3a}}{4}$
\item $\frac{\vec{5a}}{4}$
\end{enumerate}
\item Find the ratio in which the line segment joining $A(1,-5) \text{ and } B(-4,5)$ $\text{is divided by the x-axis}$. Also find the coordinates of the point of division.
\item Find the position vector of a point $\vec{R}$ which divides the line joining two points $\vec{P}$ and $\vec{Q}$ whose position vectors are $2\vec{a}+\vec{b}$ and $\vec{a}-3\vec{b}$ externally in the ratio $1:2$.
\end{enumerate}

\subsection{Rank}
\iffalse
\documentclass[journal,10pt,twocolumn]{article}
\usepackage{graphicx}
\usepackage{caption} 
\usepackage{hyperref}
\usepackage[margin=0.5in]{geometry}
\usepackage{booktabs}
\usepackage{array}
\usepackage{amsmath}   % for having text in math mode
\usepackage{mathtools}
\usepackage{enumitem}
\usepackage{atbegshi}% http://ctan.org/pkg/atbegshi
\AtBeginDocument{\AtBeginShipoutNext{\AtBeginShipoutDiscard}}
\newcommand{\myvec}[1]{\ensuremath{\begin{pmatrix}#1\end{pmatrix}}}
\let\vec\mathbf
\newcommand{\mydet}[1]{\ensuremath{\begin{vmatrix}#1\end{vmatrix}}}
\providecommand{\brak}[1]{\ensuremath{\left(#1\right)}}
\newcommand{\solution}{\noindent \textbf{Solution: }}
\let\vec\mathbf
\begin{document}
\begin{center}
\title{\textbf{Properties of Collinear}}
\date{\vspace{-5ex}} %Not to print date automatically
\maketitle
\end{center}
\setcounter{page}{1}
\section{10$^{th}$ Maths - Chapter 7}
\textbf{This is Problem-2 from Exercise 7.3.2}
\item In each of the following find the value of ‘k’, for which the points are collinear.

\item (7, –2), (5, 1), (3, k) \\
\item (8, 1), (k, – 4), (2, –5).\\
\fi
\begin{enumerate}
\item 
Let
\begin{align}  
	\vec{A}&=\myvec{7 \\-2},
\vec{B}=\myvec{5 \\ 1},
\vec{C}=\myvec{3 \\ k}
\end{align}
Then
\begin{align}  
\vec{D} &=\brak{\vec{A}-\vec{B}} = \brak{\myvec{7 \\-2 } - \myvec{5 \\1 } } = \myvec{2 \\ -3 }\\
\vec{E} &= \brak{\vec{A}-\vec{C}} = \brak{\myvec{7 \\ -2 } - \myvec{3 \\k} } = \myvec{4 \\-2-k}
\end{align}
Forming the collinearlity matrix,
\begin{align}
\vec{F} &={\myvec{\vec{D}\\ \vec{E}}}
=
\myvec{
2 & -3
 \\
4 & -2-k 
}
\end{align}
yielding
\begin{align}
\label{eq:chapters/10/7/3/2/chem_balance_mat_row1}
 \xleftrightarrow[]{R_2 = R_2-2R_1}
\myvec{
2 & -3
\\
0 & -k+4
}
\end{align}
For the matrix to be rank 1, 
\begin{align}
 -k+4 =0
\implies k =4 
\end{align}
This is verified in Fig. 
	  \ref{fig:chapters/10/7/3/2/line1.pdf}.
\begin{figure}[H]
	  \centering 
	  \includegraphics[width=0.75\columnwidth]{chapters/10/7/3/2/figs/line1.pdf}
	  \caption{}
	  \label{fig:chapters/10/7/3/2/line1.pdf}
	  \end{figure} 	 		  
%
 \item In this case,
\begin{align}  
\vec{A}=\myvec{8 \\ 1},
\vec{B}=\myvec{k \\ -4},
\vec{C}=\myvec{2 \\ -5}.
\end{align}
Since
\begin{align}  
 \vec{D} &=\brak{\vec{A}-\vec{B}} = \brak{\myvec{8 \\1 } - \myvec{k \\-4 } } = \myvec{8-k \\ 5 }\\
\vec{E} &= \brak{\vec{A}-\vec{C}} = \brak{\myvec{8 \\ 1 } - \myvec{2 \\-5 } } = \myvec{6 \\6}
\end{align}
the collinearity matrix is
\begin{align}
\vec{F} &={\myvec{\vec{D}\\ \vec{E}}}
=
\myvec{
8-k & 5
 \\
6 & 6
}
\end{align}
yielding
\begin{align}
\label{eq:chapters/10/7/3/2/chem_balance_mat_row}
 \xleftrightarrow[]{R_1=\frac{R_1}{8-k}}
\myvec{
1& \frac{5}{8-k}
\\
6 & 6
}
\\
\xleftrightarrow[]{R_2 = R_2-6R_1}
\myvec{
1 & \frac{5}{8-k}
\\
\\
0 & 6-\frac{30}{8-k}
}
\end{align}
For 
the matrix to be rank 1,
\begin{align}
6-\frac{30}{8-k}&=0
\\
\implies k &=3
\end{align}
This is verified in Fig. 
	  \ref{fig:chapters/10/7/3/2/line2.png}
\begin{figure}[H]
	  \centering 
	  \includegraphics[width=0.75\columnwidth]{chapters/10/7/3/2/figs/line2.pdf}
	  \caption{}
	  \label{fig:chapters/10/7/3/2/line2.png}
	  \end{figure}
\end{enumerate} 

\subsection{Length}
\input{chapters/vectors/length.tex}
%
\newpage
\section{Vector Multiplication}
\subsection{Formulae}
\input{chapters/formulae/vectors/product.tex}
\subsection{Scalar Product}
\begin{enumerate}[label=\thesubsection.\arabic*,ref=\thesubsection.\theenumi]
\item Find the angle between two vectors $\overrightarrow{a}$ and $\overrightarrow {b} $ with magnitudes $\sqrt{3}$ and 2 respectively having $\overrightarrow {a}\cdot\overrightarrow {b}=\sqrt{6}$.
		\label{prob:12/10/3/1/inner}
	\\
	\solution
		\solution 
See Fig. \ref{fig:10/7/4/8Fig3}. From 
  \eqref{eq:10/7/4/8det2f}, $PQRS$ is a parallelogram.
\begin{align}
  %\label{eq:10/7/4/8det2f}
  \vec{P}  = 
 \frac{3}{2},\, 
 \vec{Q}  = \myvec{
 2 \\
 4 \\
 } ,\,
 \vec{R}  = \myvec{
 5 \\
 \frac{3}{2}
 }   
  ,\,
 \vec{S}  = \myvec{
 2\\
 -1 \\
 }   
 \\
	\implies 
 \brak{\vec{Q}-\vec{P}}^\top\brak{\vec{R}-\vec{Q}}  \neq 0
 \\
 \brak{\vec{R}-\vec{P}}^\top\brak{\vec{S}-\vec{Q}}  = 0
\end{align}
Therefore $PQRS$ is a rhombus.
\begin{figure}[H]
	\begin{center}
		\includegraphics[width=0.75\columnwidth]{chapters/10/7/4/8/figs/problem1.pdf}
	\end{center}
\caption{}
\label{fig:10/7/4/8Fig3}
\end{figure}


\item Find the angle between the the vectors $\hat{i}-2\hat{j}+3\hat{k}$ and $3\hat{i}-2\hat{j}+\hat{k}$.
	\\
	\solution
		\solution 
See Fig. \ref{fig:10/7/4/8Fig3}. From 
  \eqref{eq:10/7/4/8det2f}, $PQRS$ is a parallelogram.
\begin{align}
  %\label{eq:10/7/4/8det2f}
  \vec{P}  = 
 \frac{3}{2},\, 
 \vec{Q}  = \myvec{
 2 \\
 4 \\
 } ,\,
 \vec{R}  = \myvec{
 5 \\
 \frac{3}{2}
 }   
  ,\,
 \vec{S}  = \myvec{
 2\\
 -1 \\
 }   
 \\
	\implies 
 \brak{\vec{Q}-\vec{P}}^\top\brak{\vec{R}-\vec{Q}}  \neq 0
 \\
 \brak{\vec{R}-\vec{P}}^\top\brak{\vec{S}-\vec{Q}}  = 0
\end{align}
Therefore $PQRS$ is a rhombus.
\begin{figure}[H]
	\begin{center}
		\includegraphics[width=0.75\columnwidth]{chapters/10/7/4/8/figs/problem1.pdf}
	\end{center}
\caption{}
\label{fig:10/7/4/8Fig3}
\end{figure}


\item Evaluate the product $(3\overrightarrow {a}-5\overrightarrow {b})\cdot (2\overrightarrow {a}+7\overrightarrow {b})$.
	\\
	\solution
		\solution 
See Fig. \ref{fig:10/7/4/8Fig3}. From 
  \eqref{eq:10/7/4/8det2f}, $PQRS$ is a parallelogram.
\begin{align}
  %\label{eq:10/7/4/8det2f}
  \vec{P}  = 
 \frac{3}{2},\, 
 \vec{Q}  = \myvec{
 2 \\
 4 \\
 } ,\,
 \vec{R}  = \myvec{
 5 \\
 \frac{3}{2}
 }   
  ,\,
 \vec{S}  = \myvec{
 2\\
 -1 \\
 }   
 \\
	\implies 
 \brak{\vec{Q}-\vec{P}}^\top\brak{\vec{R}-\vec{Q}}  \neq 0
 \\
 \brak{\vec{R}-\vec{P}}^\top\brak{\vec{S}-\vec{Q}}  = 0
\end{align}
Therefore $PQRS$ is a rhombus.
\begin{figure}[H]
	\begin{center}
		\includegraphics[width=0.75\columnwidth]{chapters/10/7/4/8/figs/problem1.pdf}
	\end{center}
\caption{}
\label{fig:10/7/4/8Fig3}
\end{figure}


\item If the vertices $A,B,C$ of a triangle $ABC$ are (1,2,3), (-1,0,0), (0,1,2), respectively, then find  $\angle{ABC}$.
	\\
	\solution
		\solution 
See Fig. \ref{fig:10/7/4/8Fig3}. From 
  \eqref{eq:10/7/4/8det2f}, $PQRS$ is a parallelogram.
\begin{align}
  %\label{eq:10/7/4/8det2f}
  \vec{P}  = 
 \frac{3}{2},\, 
 \vec{Q}  = \myvec{
 2 \\
 4 \\
 } ,\,
 \vec{R}  = \myvec{
 5 \\
 \frac{3}{2}
 }   
  ,\,
 \vec{S}  = \myvec{
 2\\
 -1 \\
 }   
 \\
	\implies 
 \brak{\vec{Q}-\vec{P}}^\top\brak{\vec{R}-\vec{Q}}  \neq 0
 \\
 \brak{\vec{R}-\vec{P}}^\top\brak{\vec{S}-\vec{Q}}  = 0
\end{align}
Therefore $PQRS$ is a rhombus.
\begin{figure}[H]
	\begin{center}
		\includegraphics[width=0.75\columnwidth]{chapters/10/7/4/8/figs/problem1.pdf}
	\end{center}
\caption{}
\label{fig:10/7/4/8Fig3}
\end{figure}


	\item The slope of a line is double of the slope of another line. If tangent of the angle between them is 1/3, find the slopes of the lines.
\label{chapters/11/10/1/11}
\\
\solution 
\iffalse
\documentclass[10pt, a4paper]{article}
\usepackage[a4paper,outer=1.5cm,inner=1.5cm,top=1.75cm,bottom=1.5cm]{geometry}

\twocolumn
\usepackage{graphicx}

\usepackage{hyperref}
\usepackage[utf8]{inputenc}
\usepackage{amsmath}
\usepackage{physics}
\usepackage{amssymb}
\begin{document}
\title{Assignment-4}
\author{Name:A.SUSI\and Email :  \url{susireddy9121@gmail.com}}
%\{ Wireless Communication (FWC)}
\date{30-sep-2022}
\maketitle



\section{Problem}
\fi
\solution 
\iffalse
\section{Solution}
\begin{center}
The input given 
\boldmath
\fi 
Let
\begin{align} 
\vec{A}=\myvec{ h\\ 0 },
\vec{B}=\myvec{ a\\ b },
\vec{C}=\myvec{ 0\\ k }
\end{align}
Forming the matrix in 
	\eqref{eq:normal_line-collinear}, we obtain, upon row reduction
	\iffalse
\begin{align}
\myvec{ h-a & -b\\ h & -k  } 
\end{align}
Using row reduction, 


In the problem they have given that three points lie on a line, thats means these three points are collinear.\\
If  points on a line  are  collinear, rank of matrix is "1"then the vectors are in linearlydependent.\\
For 2 × 2 matrix Rank =1 means Determinant is 0.\\
Through pivoting,we obtain\\
\fi
\begin{align}\label{eq:}
\myvec{ h-a & -b\\ h & -k  }  
	\xleftrightarrow[]{{\frac{R_1}{h-a}}}\myvec{
1 &\frac{-b}{h-a} \\ 
 h& -k
}
	\\
	\xleftrightarrow[]{R_2\rightarrow R_2-hR_1}
\myvec{
1 &\frac{-b}{h-a} \\ 
 0&-k+\frac{bh}{h-a} 
}
\end{align} 
For obtaining a rank 1 matrix, 
\iffalse

if the rank of the matrix is 1 means any one of the row must be zero.So, making the last element in the matrix to 0.\\
\fi
\begin{align}
	-k+\frac{bh}{h-a}&=0
	\\
	\implies \frac{a}{h}+\frac{b}{k}&=1 
\end{align} 
upon simplification.
\iffalse

Hence proved.\\
\section{Construction}
 \begin{figure}[H]
\centering
\includegraphics[width=0.75\columnwidth]{fig.png} 
\caption{}
\end{figure}
\section{Code}
*Verify the above proofs in the following code.\\
\framebox{
\url{https://github.com/Susi9121/FWC/tree/main/matrix/line}}	
\bibliographystyle{ieeetr}
\end{document}
\fi

\item    Find angle between the lines, $\sqrt{3}x+y=1$ and $x+\sqrt{3}y$=1.
\label{chapters/11/10/3/9}
\\
   \solution 
Given
\begin{align}
	c_1 = \frac{7}{3},\,
c_2 = -6.
\end{align}
	From \eqref{eq:parallel_lines},
we need to find $c$ such that,
\begin{align}
	\abs{c-c_1} = \abs{c-c_2} \implies c = \frac{c_1+c_2}{2}
	 = -\frac{11}{6}.
\end{align}
Hence, the desired equation is
\begin{align}
	\myvec{3 & 2}\vec{x} &= -\frac{11}{6}
\end{align}
	See \figref{fig:chapters/11/10/4/21/1}.
\begin{figure}[H]
	\centering
	\includegraphics[width=0.75\columnwidth]{chapters/11/10/4/21/figs/line_plot.jpg}
	\caption{}
	\label{fig:chapters/11/10/4/21/1}
\end{figure}

\item Find the angle between the vectors $2\hat{i}-\hat{j}+\hat{k}$ and $3\hat{i}+4\hat{j}-\hat{k}$.
\item The angles between two vectors $\vec{a}, \vec{b}$ with magnitude $\sqrt{3}, 4$ respectively, and $\vec{a} \cdot \vec{b}= 2\sqrt{3}$ is
	\begin{enumerate}
\item $\frac{\pi}{6}$
\item $\frac{\pi}{3}$
\item $\frac{\pi}{2}$ 
\item $\frac{5\pi}{2}$
\end{enumerate}
\item Find the angle between the lines 
\begin{align}
	\overrightarrow{r}&=3\hat{i}-2\hat{j}+6\hat{k}+\lambda(2\hat{i}+\hat{j}+2\hat{k})
	\text{ and}
	\\
	\overrightarrow{r}&=(2\hat{j}-5\hat{k})+\mu(6\hat{i}+3\hat{j}+2\hat{k})
\end{align}
%
\solution  The given lines can be expressed  in the form 
of 
	\eqref{eq:param-form}
	as
\begin{align}
	\vec{x} = \myvec{3 \\ -2 \\ 6} + \kappa_1 \myvec{2 \\ 1 \\ 2}
	\\
	\vec{x} = \myvec{0 \\ 2 \\ -5 } + \kappa_2 \myvec{6 \\ 3 \\ 2}
\end{align}
From the above, it is obvious that the direction vectors of the two lines are
\begin{align}
\vec{m}_1 =\myvec{2 \\ 1 \\ 2},\
	\vec{m}_2=\myvec{6 \\ 3 \\ 2}
\end{align}
	From \eqref{eq:angle-inner}, the angle between the two lines is  obtained as
\begin{align}
	\cos \theta = \frac{19}{21}
\end{align}
\item The vectors $\vec{a}=3\hat{i}-2\hat{j}+2\hat{k}$ $\text{ and }$ $\vec{b}=\hat{i}-2\hat{k}$ are the adjancent sides of a parallelogram. The acute angle between its diagonals is \rule{1cm}{0.15mm}.
\item The sine of the angle between the straight line 
\begin{align}
	\frac{x-2}{3}=\frac{y-3}{4}=\frac{z-4}{5} 
\end{align}
and the plane  
\begin{align}
2x-2y+z=5
\end{align}
is
\begin{enumerate}
	\item $\frac{10}{6\sqrt{5}}$ 
	\item $\frac{4}{5\sqrt{2}}$
	\item $\frac{2\sqrt{3}}{5}$
	\item $\frac{\sqrt{2}}{10}$
\end{enumerate}
\solution The given line can be expressed in the form 
	\eqref{eq:param-form}
	as
\begin{align}
	\vec{x} = \myvec{2 \\ 3 \\ 4} + \kappa_1 \myvec{3 \\ 4 \\ 5}
\end{align}
Hence the direction vector of this line is 
\begin{align}
\myvec{3 \\ 4 \\ 5}
\end{align}
	From \eqref{eq:normal-form}, the normal vector of the given plane is 
\begin{align}
\myvec{2 \\ -2 \\ 1}
\end{align}
Thus, the cosine of the angle between the two is 
obtained from \eqref{eq:angle-inner} as
\begin{align}
	\frac{\sqrt{2}}{10},
\end{align}
which is sine of the angle between the plane and the line.
\item The plane $2x-3y+6z-11=0$ makes an angle $\sin^{-1}(\alpha)$ with x-axis. The value of $\alpha$ is equal to 
\begin{enumerate}
	\item  $\frac{\sqrt{3}}{2}$
	\item  $\frac{\sqrt{2}}{3}$
	\item  $\frac{2}{7}$
	\item  $\frac{3}{7}$
\end{enumerate}
\item Find the angle between the vectors $2\hat{i}-\hat{j}+\hat{k}$ $\text{and}$ $3\hat{i}+4\hat{j}-\hat{k}$.
\item The angles between two vectors $\vec{a}$ $\text{and}$ $\vec{b}$ with magnitude $\sqrt{3}$ $\text{ and }$ 4, respectively, and $\vec{a}$, $\vec{b}$= $2\sqrt{3}$ is
	\begin{enumerate}
\item $\frac{\pi}{6}$
\item $\frac{\pi}{3}$
\item $\frac{\pi}{2}$ 
\item $\frac{5\pi}{2}$
\end{enumerate}

\item The angle between the line 
\begin{align}
	\overrightarrow{r}=(5\hat{i}-\hat{j}-4\hat{k})+\lambda(2\hat{i}-\hat{j}+\hat{k})
\end{align}
	and the plane 
\begin{align}
	\overrightarrow{r} \cdot (3\hat{i}-4\hat{j}-\hat{k})+5=0
\end{align}
	is $\sin^{-1}\brak{\frac{5}{2\sqrt{91}}}$.
\item The angle between the planes 
\begin{align}
	\overrightarrow{r} \cdot (2\hat{i}-3\hat{j}+\hat{k})&=1 
	\text{ and }
	\\
	\overrightarrow{r} \cdot (\hat{i}-\hat{j})&=4  
\end{align}
is
	$\cos^{-1} \brak{\frac{-5}{\sqrt{58}}}$.
\item Find the angle between the lines 
\begin{align}
	y&=(2-\sqrt{3})(x+5)\text{ and }
	\\
	y&=(2+\sqrt{3})(x-7).
\end{align}
\item The unit vector normal to the plane $x+2y+3z-6=0$ is $\frac{1}{\sqrt{14}}\hat{i} + \frac{2}{\sqrt{14}}\hat{j} + \frac{3}{\sqrt{14}}\hat{k}$.
\item The scalar product of the vector $\hat{i}+\hat{j}+\hat{k}$ with a unit vector along the sum of vectors $2\hat{i}+4\hat{j}-5\hat{k}$ and $\lambda\hat{i}+2\hat{j}+3\hat{k}$ is equal to one. Find the value of $\lambda$.
\end{enumerate}

\subsection{Orthogonality}
\input{chapters/vectors/ortho.tex}
\subsection{Vector Product}
\begin{enumerate}[label=\thesubsection.\arabic*,ref=\thesubsection.\theenumi]
		\item Find $\abs{\overrightarrow{a}\times\overrightarrow{b}},\text{ if }\overrightarrow{a}=\hat{i}-7\hat{j}+7\hat{k}\text{ and } \overrightarrow{b}=3\hat{i}-2\hat{j}+2\hat{k}$.
	\\
		\solution
		The area of the rhombus is
\begin{align}
                \norm{\myvec{\vec{A-D}}\times \myvec{\vec{B-A}}}=\mydet{5 & 1\\1 & 5} = 24
\end{align}
See 
\figref{fig:chapters/10/7/2/10/gFig1}.
\begin{figure}[H]
 \begin{center}
  \includegraphics[width=0.75\columnwidth]{chapters/10/7/2/10/figs/fig.pdf}
 \end{center}
\caption{}
\label{fig:chapters/10/7/2/10/gFig1}
\end{figure}

\item Find $\lambda$ and $\mu$ if $(2\hat{i}+6\hat{j}+27\hat{k})\times(\hat{i}+\lambda \hat{j} + \mu \hat{k})=\overrightarrow{0}$.
	\\
		\solution
		The area of the rhombus is
\begin{align}
                \norm{\myvec{\vec{A-D}}\times \myvec{\vec{B-A}}}=\mydet{5 & 1\\1 & 5} = 24
\end{align}
See 
\figref{fig:chapters/10/7/2/10/gFig1}.
\begin{figure}[H]
 \begin{center}
  \includegraphics[width=0.75\columnwidth]{chapters/10/7/2/10/figs/fig.pdf}
 \end{center}
\caption{}
\label{fig:chapters/10/7/2/10/gFig1}
\end{figure}

\item Find the area of the triangle with vertices $A(1, 1, 2), B(2, 3, 5)$ and $C(1, 5, 5)$.
	\\
		\solution
		The area of the rhombus is
\begin{align}
                \norm{\myvec{\vec{A-D}}\times \myvec{\vec{B-A}}}=\mydet{5 & 1\\1 & 5} = 24
\end{align}
See 
\figref{fig:chapters/10/7/2/10/gFig1}.
\begin{figure}[H]
 \begin{center}
  \includegraphics[width=0.75\columnwidth]{chapters/10/7/2/10/figs/fig.pdf}
 \end{center}
\caption{}
\label{fig:chapters/10/7/2/10/gFig1}
\end{figure}

\item Find the area of the parallelogram whose adjacent sides are determined by the vectors $\overrightarrow{a}=\hat{i}-\hat{j}+3\hat{k}$ and $\overrightarrow{b}=2\hat{i}-7\hat{j}+\hat{k}$.
	\\
		\solution
		The area of the rhombus is
\begin{align}
                \norm{\myvec{\vec{A-D}}\times \myvec{\vec{B-A}}}=\mydet{5 & 1\\1 & 5} = 24
\end{align}
See 
\figref{fig:chapters/10/7/2/10/gFig1}.
\begin{figure}[H]
 \begin{center}
  \includegraphics[width=0.75\columnwidth]{chapters/10/7/2/10/figs/fig.pdf}
 \end{center}
\caption{}
\label{fig:chapters/10/7/2/10/gFig1}
\end{figure}

\item Find the area of a rhombus if its vertices are $A(3,0), B(4,5), C(-1,4)$  and  $D(-2,-1)$ taken in order. 
	\\
		\solution
	The area of the rhombus is
\begin{align}
                \norm{\myvec{\vec{A-D}}\times \myvec{\vec{B-A}}}=\mydet{5 & 1\\1 & 5} = 24
\end{align}
See 
\figref{fig:chapters/10/7/2/10/gFig1}.
\begin{figure}[H]
 \begin{center}
  \includegraphics[width=0.75\columnwidth]{chapters/10/7/2/10/figs/fig.pdf}
 \end{center}
\caption{}
\label{fig:chapters/10/7/2/10/gFig1}
\end{figure}

\item Let the vectors $\overrightarrow{a}$ and $\overrightarrow{b}$ be such that $|\overrightarrow{a}| = 3$ and $|\overrightarrow{b}| = \dfrac{\sqrt{2}}{3}$, then $\overrightarrow{a} \times \overrightarrow{b}$ is a unit vector, if the angle between $\overrightarrow{a}$ and $\overrightarrow{b}$ is
\begin{enumerate}
\item $\frac{\pi}{6}$
\item $\frac{\pi}{4}$
\item $\frac{\pi}{3}$
\item $\frac{\pi}{2}$
\end{enumerate}
		\solution
		The area of the rhombus is
\begin{align}
                \norm{\myvec{\vec{A-D}}\times \myvec{\vec{B-A}}}=\mydet{5 & 1\\1 & 5} = 24
\end{align}
See 
\figref{fig:chapters/10/7/2/10/gFig1}.
\begin{figure}[H]
 \begin{center}
  \includegraphics[width=0.75\columnwidth]{chapters/10/7/2/10/figs/fig.pdf}
 \end{center}
\caption{}
\label{fig:chapters/10/7/2/10/gFig1}
\end{figure}

\item Area of a rectangle having vertices A, B, C and D with position vectors $ -\hat{i}+ \frac{1}{2} \hat{j}+4\hat{k}, \hat{i}+ \frac{1}{2} \hat{j}+4\hat{k}, \hat{i}-\frac{1}{2} \hat{j}+4\hat{k}$ and $-\hat{i}- \frac{1}{2} \hat{j}+4\hat{k}$, respectively is
\begin{enumerate}
\item $\frac{1}{2}$
\item 1
\item 2
\item 4
\end{enumerate}
		\solution
		\iffalse
\documentclass[journal,12pt,twocolumn]{IEEEtran}
%
\usepackage{setspace}
\usepackage{gensymb}
%\doublespacing
\singlespacing

%\usepackage{graphicx}
%\usepackage{amssymb}
%\usepackage{relsize}
\usepackage[cmex10]{amsmath}
%\usepackage{amsthm}
%\interdisplaylinepenalty=2500
%\savesymbol{iint}
%\usepackage{txfonts}
%\restoresymbol{TXF}{iint}
%\usepackage{wasysym}
\usepackage{amsthm}
%\usepackage{iithtlc}
\usepackage{mathrsfs}
\usepackage{txfonts}
\usepackage{stfloats}
\usepackage{bm}
\usepackage{cite}
\usepackage{cases}
\usepackage{subfig}
%\usepackage{xtab}
\usepackage{longtable}
\usepackage{multirow}
%\usepackage{algorithm}
%\usepackage{algpseudocode}
\usepackage{enumitem}
\usepackage{mathtools}
\usepackage{steinmetz}
\usepackage{tikz}
\usepackage{circuitikz}
\usepackage{verbatim}
\usepackage{tfrupee}
\usepackage[breaklinks=true]{hyperref}
%\usepackage{stmaryrd}
\usepackage{tkz-euclide} % loads  TikZ and tkz-base
%\usetkzobj{all}
\usetikzlibrary{calc,math}
\usepackage{listings}
    \usepackage{color}                                            %%
    \usepackage{array}                                            %%
    \usepackage{longtable}                                        %%
    \usepackage{calc}                                             %%
    \usepackage{multirow}                                         %%
    \usepackage{hhline}                                           %%
    \usepackage{ifthen}                                           %%
  %optionally (for landscape tables embedded in another document): %%
    \usepackage{lscape}     
\usepackage{multicol}
\usepackage{chngcntr}
%\usepackage{enumerate}

%\usepackage{wasysym}
%\newcounter{MYtempeqncnt}
\DeclareMathOperator*{\Res}{Res}
%\renewcommand{\baselinestretch}{2}
\renewcommand\thesection{\arabic{section}}
\renewcommand\thesubsection{\thesection.\arabic{subsection}}
\renewcommand\thesubsubsection{\thesubsection.\arabic{subsubsection}}

\renewcommand\thesectiondis{\arabic{section}}
\renewcommand\thesubsectiondis{\thesectiondis.\arabic{subsection}}
\renewcommand\thesubsubsectiondis{\thesubsectiondis.\arabic{subsubsection}}

% correct bad hyphenation here
\hyphenation{op-tical net-works semi-conduc-tor}
\def\inputGnumericTable{}                                 %%

\lstset{
%language=C,
frame=single, 
breaklines=true,
columns=fullflexible
}
%\lstset{
%language=tex,
%frame=single, 
%breaklines=true
%}

\begin{document}
%


\newtheorem{theorem}{Theorem}[section]
\newtheorem{problem}{Problem}
\newtheorem{proposition}{Proposition}[section]
\newtheorem{lemma}{Lemma}[section]
\newtheorem{corollary}[theorem]{Corollary}
\newtheorem{example}{Example}[section]
\newtheorem{definition}[problem]{Definition}
%\newtheorem{thm}{Theorem}[section] 
%\newtheorem{defn}[thm]{Definition}
%\newtheorem{algorithm}{Algorithm}[section]
%\newtheorem{cor}{Corollary}
\newcommand{\BEQA}{\begin{eqnarray}}
\newcommand{\EEQA}{\end{eqnarray}}
\newcommand{\define}{\stackrel{\triangle}{=}}

\bibliographystyle{IEEEtran}
%\bibliographystyle{ieeetr}


\providecommand{\mbf}{\mathbf}
\providecommand{\pr}[1]{\ensuremath{\Pr\left(#1\right)}}
\providecommand{\qfunc}[1]{\ensuremath{Q\left(#1\right)}}
\providecommand{\sbrak}[1]{\ensuremath{{}\left[#1\right]}}
\providecommand{\lsbrak}[1]{\ensuremath{{}\left[#1\right.}}
\providecommand{\rsbrak}[1]{\ensuremath{{}\left.#1\right]}}
\providecommand{\brak}[1]{\ensuremath{\left(#1\right)}}
\providecommand{\lbrak}[1]{\ensuremath{\left(#1\right.}}
\providecommand{\rbrak}[1]{\ensuremath{\left.#1\right)}}
\providecommand{\cbrak}[1]{\ensuremath{\left\{#1\right\}}}
\providecommand{\lcbrak}[1]{\ensuremath{\left\{#1\right.}}
\providecommand{\rcbrak}[1]{\ensuremath{\left.#1\right\}}}
\theoremstyle{remark}
\newtheorem{rem}{Remark}
\newcommand{\sgn}{\mathop{\mathrm{sgn}}}
\providecommand{\abs}[1]{\left\vert#1\right\vert}
\providecommand{\res}[1]{\Res\displaylimits_{#1}} 
\providecommand{\norm}[1]{\left\lVert#1\right\rVert}
%\providecommand{\norm}[1]{\lVert#1\rVert}
\providecommand{\mtx}[1]{\mathbf{#1}}
\providecommand{\mean}[1]{E\left[ #1 \right]}
\providecommand{\fourier}{\overset{\mathcal{F}}{ \rightleftharpoons}}
%\providecommand{\hilbert}{\overset{\mathcal{H}}{ \rightleftharpoons}}
\providecommand{\system}{\overset{\mathcal{H}}{ \longleftrightarrow}}
	%\newcommand{\solution}[2]{\textbf{Solution:}{#1}}
\newcommand{\solution}{\noindent \textbf{Solution: }}
\newcommand{\cosec}{\,\text{cosec}\,}
\providecommand{\dec}[2]{\ensuremath{\overset{#1}{\underset{#2}{\gtrless}}}}
\newcommand{\myvec}[1]{\ensuremath{\begin{pmatrix}#1\end{pmatrix}}}
\newcommand{\mydet}[1]{\ensuremath{\begin{vmatrix}#1\end{vmatrix}}}
%\numberwithin{equation}{section}
\numberwithin{equation}{subsection}
%\numberwithin{problem}{section}
%\numberwithin{definition}{section}
\makeatletter
\@addtoreset{figure}{problem}
\makeatother

\let\StandardTheFigure\thefigure
\let\vec\mathbf
%\renewcommand{\thefigure}{\theproblem.\arabic{figure}}
\renewcommand{\thefigure}{\theproblem}
%\setlist[enumerate,1]{before=\renewcommand\theequation{\theenumi.\arabic{equation}}
%\counterwithin{equation}{enumi}


%\renewcommand{\theequation}{\arabic{subsection}.\arabic{equation}}

\def\putbox#1#2#3{\makebox[0in][l]{\makebox[#1][l]{}\raisebox{\baselineskip}[0in][0in]{\raisebox{#2}[0in][0in]{#3}}}}
     \def\rightbox#1{\makebox[0in][r]{#1}}
     \def\centbox#1{\makebox[0in]{#1}}
     \def\topbox#1{\raisebox{-\baselineskip}[0in][0in]{#1}}
     \def\midbox#1{\raisebox{-0.5\baselineskip}[0in][0in]{#1}}

\vspace{3cm}


\title{Question: 12.10.4.12}
\author{Nikam Pratik Balasaheb (EE21BTECH11037)}





% make the title area
\maketitle

\newpage

%\tableofcontents

\bigskip

\renewcommand{\thefigure}{\theenumi}
\renewcommand{\thetable}{\theenumi}
%\renewcommand{\theequation}{\theenumi}

\section{Problem}
Find the area of rectangle having A,B,C,D with position vectors $\myvec{-1\\[1pt]\frac{1}{2} \\[1pt] 4}$ ,$\myvec{1\\[1pt]\frac{1}{2} \\[1pt] 4}$, $\myvec{1\\[1pt]\frac{-1}{2} \\[1pt] 4}$ and $\myvec{-1\\[1pt]\frac{-1}{2} \\[1pt] 4}$ respectively.  
\section{Solution}
\fi
Since
\begin{align}
\vec{A} - \vec{B} &= \myvec{-2\\0\\0}\\
\vec{C} -\vec{B} &= \myvec{0\\-1\\0}
\end{align}
area of the rectangle is
\begin{align}
 \norm{\brak{\vec{A} -\vec{B}} \times \brak{\vec{C}-\vec{D}}}
= 2
\end{align} 
See Fig. 
   \ref{fig:chapters/12/10/4/12Rect_ABCD}
\begin{figure}[H]
  \centering
   \includegraphics[width=0.75\columnwidth]{chapters/12/10/4/12/figs/Figure_1.png}
   \caption{}
   \label{fig:chapters/12/10/4/12Rect_ABCD}
\end{figure}





\item Find the area of the triangle whose vertices are 
\begin{enumerate}
\item $(2, 3), (–1, 0), (2, – 4)$
\item $(–5, –1), (3, –5), (5, 2)$ 
\end{enumerate}
		\label{10/7/3/1}
\solution
		    See \tabref{eq:10/7/3/1/area}.
\begin{table}[H]
    \centering
    \caption{}
    \label{eq:10/7/3/1/area}
    \begin{tabular}{|c|c|c|c|}
        \hline
	     & $\vec{A}-\vec{B}$  & $\vec{A}-\vec{C}$  & $\frac{1}{2}\|\brak{\vec{A}-\vec{B}} \times \brak{\vec{A}-\vec{C}}\|$ \\
        \hline
         a)& $\myvec{ 3 \\3 }$ & $\myvec{ 0 \\ 7 }$ & $\frac{21}{2}$ \\
        \hline
	    b)& $\myvec{
 -8 \\
 4 
 }$
         &$\myvec{
 -10 \\
 -3 
 }$
  &  $32$   \\
        \hline
    \end{tabular}
\end{table}


\item Find the area of the triangle formed by joining the mid-points of the sides of the triangle whose vertices are $A(0, –1), B(2, 1)$  and  $C(0, 3)$. Find the ratio of this area to the area of the given triangle.
	\\
\solution
		Using 
	  \eqref{eq:section_formula},
the mid point coordinates are given by
	\begin{align}
		\vec{P} = \frac{1}{2}\vec(\vec{A}+\vec{B})  = \myvec{1\\0}\\
		\vec{Q} = \frac{1}{2}\vec(\vec{B}+\vec{C}) = \myvec{1\\2}\\
		\vec{R} = \frac{1}{2}\vec(\vec{A}+\vec{C}) = \myvec{0\\1}
	\end{align}
	\begin{align}
\because		\vec{P}-\vec{Q} =  \myvec{
 0 \\
 -2 
 },\,
		\vec{Q}-\vec{R} =   \myvec{
 1 \\
 1 
 }
 \\
		ar(PQR)=\frac{1}{2}{\norm{\vec(\vec{P}-\vec{Q})\times\vec(\vec{Q}-\vec{R})}}
		=1
	\end{align}
	Similarly, 
	\begin{align}
		\vec{A}-\vec{B} = \myvec{
 -2 \\
 -2 
 }
 ,\,
		\vec{A}-\vec{C} =  \myvec{
 0 \\
 -4 
 }
 \\
 \implies
		ar(ABC)=\frac{1}{2}{\norm{\vec(\vec{A}-\vec{B})\times\vec(\vec{A}-\vec{C})}}
=4
\\
		\implies \frac{ar\brak{PQR}}{ar\brak{ABC}} = \frac{1}{4}
	\end{align}
	See 
\figref{fig:10/7/3/3Fig}
\begin{figure}[H]
	\begin{center} 
	    \includegraphics[width=0.75\columnwidth]{chapters/10/7/3/3/figs/trigraph.png}
	\end{center}
\caption{}
\label{fig:10/7/3/3Fig}
\end{figure}


\item Find the area of the quadrilateral whose vertices, taken in order, are $A(– 4, – 2), B(– 3, – 5), C(3, – 2)$  and $ D(2, 3)$.
	\\
\solution
		See 
\figref{fig:chapters/10/7/3/4/Fig1}
\begin{figure}[H]
 \begin{center}
  \includegraphics[width=0.75\columnwidth]{chapters/10/7/3/4/figs/fig.pdf}
 \end{center}
\caption{}
\label{fig:chapters/10/7/3/4/Fig1}
\end{figure}
\begin{align}
\because	\vec{A}- \vec{B} =\myvec{-1\\3},\,
	  \vec{A}- \vec{D} =\myvec{-6\\-5},
	  \\
	\vec{B}- \vec{C} =\myvec{-6\\-5},\,
	  \vec{B}- \vec{D} =\myvec{-3\\-8},
	  \\
	  ar(ABD)=\frac{1}{2} \norm{\brak{\vec{A}-\vec{B}}  \times 
   \brak{\vec{A}- \vec{D}}} 
	=	\frac{23}{2}
	\\
	  ar(BCD)=\frac{1}{2} \norm{\brak{\vec{B}-\vec{C}}  \times 
   \brak{\vec{B}- \vec{D}}} 
	=	\frac{33}{2}
	\\
\implies	ar(ABCD)=  ar(ABD) +  ar(BCD)
	= 28
\end{align}


\item Verify that a median of a triangle divides it into two triangles of equal areas for $\triangle ABC$ whose vertices are $\vec{A}(4, -6), \vec{B}(3, 2), \text{ and } \vec{C}(5, 2)$. 
		\label{10/7/3/5}
		\\
\solution
		\begin{align}
\vec{D}=\frac{\vec{B}+\vec{C}}{2}
=\myvec{4\\ 0},
\\
	\vec{A}- \vec{B} =\myvec{1\\ -4},\,
	  \vec{A}- \vec{D} =\myvec{0\\ -6}
	  \\
	  \implies
  ar(ABD)=\frac{1}{2} \norm{\brak{\vec{A}-\vec{B}}  \times 
   \brak{\vec{A}- \vec{D}}} 
	       =3	
	       \\
	\vec{A}- \vec{C} =\myvec{-1\\ -8},\,
	  \vec{A}- \vec{D} =\myvec{0\\ -6}
	  \\
	  \implies
  ar(ACD)=\frac{1}{2} \norm{\brak{\vec{A}-\vec{C}}  \times 
   \brak{\vec{A}- \vec{D}}} 
   \\
	= 3 =
ar(ABD)
\end{align}
See  
\figref{fig:10/7/3/5/}.
\begin{figure}[H]
\centering
\includegraphics[width=0.75\columnwidth]{chapters/10/7/3/5/figs/fig.pdf}
\caption{}
\label{fig:10/7/3/5/}
\end{figure} 

\item The two adjacent sides of a parallelogram are 
$\vec{a}= 2\hat{i}-4\hat{j}+5\hat{k}$  and  $\vec{b} =\hat{i}-2\hat{j}-3\hat{k}$.
Find the unit vector parallel to its diagonal. Also, find its area.\\
	\solution
		The area of the rhombus is
\begin{align}
                \norm{\myvec{\vec{A-D}}\times \myvec{\vec{B-A}}}=\mydet{5 & 1\\1 & 5} = 24
\end{align}
See 
\figref{fig:chapters/10/7/2/10/gFig1}.
\begin{figure}[H]
 \begin{center}
  \includegraphics[width=0.75\columnwidth]{chapters/10/7/2/10/figs/fig.pdf}
 \end{center}
\caption{}
\label{fig:chapters/10/7/2/10/gFig1}
\end{figure}

\item The vertices of a $\triangle ABC$ are $\vec{A}(4,6), \vec{B}(1,5)$ and  $\vec{C}(7,2)$. A line is drawn to intersect sides $AB$ and $AC$ at $\vec{D}$ and $\vec{E}$ respectively, such that $\frac{AD}{AB} = \frac{AE}{AC} = \frac{1}{4}$. Calculate the area of $\triangle ADE$ and compare it with the area of the $\triangle ABC$.
\\
\solution
	See  
\figref{fig:chapters/10/7/4/6Fig1}.
\begin{figure}[H]
 \begin{center}
 \includegraphics[width=0.75\columnwidth]{chapters/10/7/4/6/figs/fig.png}
 \end{center}
\caption{}
\label{fig:chapters/10/7/4/6Fig1}
\end{figure}
	Using section formula
	  \eqref{eq:section_formula},
\begin{align}
\vec{D} =\frac{3\vec{A}+\vec{B}}{4}
	=\frac{1}{4}\myvec{13\\ 23}
	\\
\vec{E} =\frac{3\vec{A}+\vec{C}}{4}
	=\frac{1}{4}\myvec{19\\ 20}
	\\
	\vec{A}- \vec{D} 
	=\frac{1}{4}\myvec{3\\ 1},\,
	  \vec{A}- \vec{E}  
	=\frac{1}{4}\myvec{-3\\ 1}
	\\
	\vec{A}- \vec{B} =\myvec{3\\1},
	  \vec{B}-\vec{C} =\myvec{-6\\3}
\end{align}
yielding
\begin{align}
ar(ABD) =\frac{1}{2} \norm{\brak{\vec{A}-\vec{D}}  \times 
   \brak{\vec{A}- \vec{E}}} 
	=	\frac{15}{32}
	\\
	  ar(ABC) =\frac{1}{2} \norm{\brak{\vec{A}-\vec{B}}  \times 
   \brak{\vec{B}- \vec{C}}} 
	=	\frac{15}{2}
	\\
	\implies \frac{ar\brak{ADE}}{ar\brak{ABC}}=\frac{1}{16}
\end{align}

    \item Draw a quadrilateral in the Cartesian plane, whose vertices are 
    \begin{align}
        \vec{A} = \myvec{-4\\5},\, \vec{B} = \myvec{0\\7},\, 
        \vec{C} = \myvec{5\\-5},\, \vec{D} = \myvec{-4\\-2}.
    \end{align}
    Also, find its area.
\label{chapters/11/10/1/1}
   \\ 
    \solution 
See \figref{fig:11/10/1/1quad}.
    From 
        \eqref{eq:11/10/1/1area-diag},
    \begin{align}
ar\brak{ABCD}
	       = \frac{121}{2}
        \label{eq:11/10/1/1ans}
    \end{align}
    \begin{figure}[H]
        \centering
        \includegraphics[width=0.75\columnwidth]{chapters/11/10/1/1/figs/quad.png}
        \caption{Plot of quadrilateral $ABCD$}
        \label{fig:11/10/1/1quad}
    \end{figure}

\item Find the area of region bounded by the triangle whose
	vertices are $(1, 0), (2, 2)$ and $(3, 1)$. 
\item Find the area of region bounded by the triangle whose vertices
	are $(– 1, 0), (1, 3)$  and  $(3, 2)$. 
\item Find the area of the $\triangle ABC$, coordinates of whose vertices are $\vec{A}(2, 0), \vec{B}(4, 5)$ and $\vec{C}(6, 3)$.
\item The area of a triangle with vertices $\vec{A}(3, 0), \vec{B}(7, 0)$ and  $\vec{C}(8, 4)$ is
\begin{enumerate}
\item 14
\item 28
\item 8
\item 6
\end{enumerate}
\item Find the area of the triangle whose vertices are $(-8,4),(-6,6)$ and $(-3,9)$.
\item If $\vec{D}\brak{\frac{-1}{2},\frac{5}{2}},\vec{E}(7,3)$ and $\vec{F}\brak{\frac{7}{2},\frac{7}{2}}$ are the midpoints of sides of $\triangle ABC$, find the area of the $\triangle ABC$.
\item Find the sine of the angle between the vectors $\vec{a}=3\hat{i}+\hat{j}+2\hat{k}$ $\text{ and }$ $\vec{b}=2\hat{i}-2\hat{j}+4\hat{k}$.
\item Using vectors, find the area of $\triangle{ABC}$ with vertices A(1,2,3), B(2,-1,4) and C(4,5,-1).
\item Find the area of the parallelogram whose diagonals are $2\hat{i}-\hat{j}+\hat{k}$ and $\hat{i}+3\hat{j}-\hat{k}$.

\item The vector from origin to the points A and B are $\vec{a}$ = $2\hat{i}-3\hat{j}+2\hat{k}$ and  $\vec{b}$ = $2\hat{i}+3\hat{j}+\hat{k}$, respectively, then the area of $\triangle {OAB}$ is
	\begin{enumerate}
\item 340 
\item $\sqrt{25}$
\item $\sqrt{229}$
\item $\frac{1}{2}\sqrt{229}$
\end{enumerate}
\item If $\vec{a} = \hat{i}+\hat{j}+\hat{k}$ and $\vec{b} = \hat{j}-\hat{k}$, find a vector $\vec{c}$ such that $\vec{a}\times\vec{c} = \vec{b}$ and $\vec{a}\cdot \vec{c}$ = 3.
%
\item The area of the quadrilateral ABCD, where A$(0,4,1)$, B$(2,3,-1)$, C$(4,5,0)$ and D$(2,6,2)$, is equal to 
\begin{enumerate}
	\item 9 sq. units
	\item 18 sq. units 
	\item 27 sq. units 
	\item 81 sq. units
\end{enumerate}
\item Find the area of region bounded by the triangle whose vertices are $(-1, 1), (0, 5)$ and $(3, 2)$.
\item The value of $\hat{i}\cdot(\hat{j}\times\hat{k})+\hat{j}\cdot(\hat{i}\times\hat{k})+\hat{k}\cdot(\hat{i}\times\hat{j})$ is
\begin{enumerate}
\item 0
\item -1
\item 1
\item 3
\end{enumerate}
\item The value of $\hat{i}\cdot (\hat{j}\times\hat{k})+\hat{j}\cdot (\hat{i}\times\hat{k})+\hat{k}\cdot (\hat{i}\times\hat{j})$ is
\begin{enumerate}
\item 0
\item -1
\item 1
\item 3
\end{enumerate}
\end{enumerate}


\subsection{Miscellaneous}
\begin{enumerate}[label=\thesubsection.\arabic*,ref=\thesubsection.\theenumi]


\item Find the values of $k$ for which the line 
\begin{align}
(k-3)x-(4-k^2)y+k^2-7k+6=0 \label{eq:chapters/11/10/4/1/1}
\end{align}
is
\begin{enumerate}
\item Parallel to the $x$-axis
\item Parallel to the $y$-axis
\item Passing through the origin
\end{enumerate}
    \solution 
		Given
\begin{align}
	c_1 = \frac{7}{3},\,
c_2 = -6.
\end{align}
	From \eqref{eq:parallel_lines},
we need to find $c$ such that,
\begin{align}
	\abs{c-c_1} = \abs{c-c_2} \implies c = \frac{c_1+c_2}{2}
	 = -\frac{11}{6}.
\end{align}
Hence, the desired equation is
\begin{align}
	\myvec{3 & 2}\vec{x} &= -\frac{11}{6}
\end{align}
	See \figref{fig:chapters/11/10/4/21/1}.
\begin{figure}[H]
	\centering
	\includegraphics[width=0.75\columnwidth]{chapters/11/10/4/21/figs/line_plot.jpg}
	\caption{}
	\label{fig:chapters/11/10/4/21/1}
\end{figure}

	\item Find the  equations of the lines, which cutoff intercepts on the axes  whose sum and product are 1 and -6 respectively.
\\
\solution
		Let the intercepts be $a$ and  $b$. Then
\begin{align}
a+b=1,
ab=-6 \label{eq:11/10/4/32a}
\\
\implies  a = 3, b = -2
\end{align}
Thus, the possible 
intercepts are
\begin{align}
\myvec{3\\0}, \myvec{0\\-2},
\myvec{-2\\0}, \myvec{0\\3}
\end{align}
From
		\eqref{prop:lin-eq-unit-mat},
\begin{align}
	\myvec{3 & 0 \\ 0 &-2}\vec{n} = \myvec{1 \\ 1}
	\\
	\implies \vec{n} = \myvec{\frac{1}{3} \\ -\frac{1}{2}}
	\\
	\text{or, } \myvec{2 & -3}\vec{x} = 6
\end{align}
using		\eqref{prop:lin-eq-unit}.
Similarly, the other line can be obtained
as
\begin{align}
	\myvec { 3 & -2 }  \vec{x}  = -6        
\end{align}
See  
\figref{fig:11/10/4/3line segmenta}.
\begin{figure}[H]
\centering
\includegraphics[width=0.75\columnwidth]{chapters/11/10/4/3/figs/inter.png}
\caption{}
\label{fig:11/10/4/3line segmenta}
\end{figure}

\item A ray of light passing through the point $\vec{P} = \brak{1, 2}$ reflects on the x-axis at point $\vec{A}$ and the reflected ray passes through the point $\vec{Q} =\brak{5, 3}$. Find the coordinates of $\vec{A}$.
\\
    \solution 
			From \eqref{eq:11/10/4/22},
the reflection of $\vec{Q}$ is 
\begin{align}
\vec{R}  
= \myvec{5\\-3}
\end{align}
Letting
\begin{align}
\vec{A} = \myvec{x\\0},
\end{align}
since 
$\vec{P},
\vec{A},  
\vec{R}  
$
are collinear, 
		from \eqref{prop:lin-dep-rank},
\begin{align}
	\myvec{
		1 & 1 & 2 
		\\ 
		1 & 5 & -3 
		\\
		1 & x & 0 }
	\xleftrightarrow[R_3=R_3 - R_1]{R_2 = R_2 - R_1}
	\myvec{
		1 & 1 & 2 
		\\ 
		0 & 4 & -5 
		\\
		0 & x-1 & -2 }
	\\
	\xleftrightarrow[]{R_3 = 4R_3 - \brak{x-1}R_2}
	\myvec{
		1 & 1 & 2 
		\\ 
		0 & 4 & -5 
		\\
		0 & 0 & 5x-13 }
	\implies x = \frac{13}{5}
\end{align}
See  
\figref{fig:chapters/11/10/4/22/1}.
\begin{figure}[H]
\centering
\includegraphics[width=0.75\columnwidth]{chapters/11/10/4/22/figs/fig.png}
\caption{}
\label{fig:chapters/11/10/4/22/1}
\end{figure}




\item Prove that in any $\triangle{ABC}$, cos A=$\frac{b^2+c^2-a^2}{2bc}$, where a,b,c are the magnitudes of the sides opposite to the vertices A,B,C respectively.
\item Distance of the point $(\alpha, \beta, \gamma)$ from y-axis is
\begin{enumerate}
	\item $\beta$ 
	\item $\abs{\beta}$
	\item $\abs{\beta+\gamma}$
	\item $\sqrt{\alpha^2+\gamma^2}$
\end{enumerate}
\item The reflection of the point $(\alpha, \beta, \gamma )$ in the xy-plane is 
\begin{enumerate}
	\item $\alpha,\beta,0)$
	\item $(0,0,\gamma)$
	\item $(-\alpha,-\beta,\gamma)$
	\item $(\alpha,\beta,-\gamma)$
\end{enumerate}
\item The plane $ax+by=0$ is rotated about its line of intersection with the plane $z=0$ through an angle $\alpha.$ Prove that the equation of the plane in its new position is 
\begin{align*}
	ax+by \pm (\sqrt{a^2+b^2} \tan\alpha)z=0.
\end{align*}
\item The locus represented by $xy+yz=0$ is 
\begin{enumerate}
	\item A pair of perpendicular lines
	\item A pair of parallel lines
	\item A pair of parallel planes 
	\item A pair of perpendicular planes
\end{enumerate}
\item For what values of $a$ and $b$ the intercepts cut off on the coordinate axes by the line $ax+by+8=0$ are equal in length but opposite in signs to those cut off by the line $2x-3y=0$ on the axes.
\item If the equation of the base of an equilateral triangle is $x+y=2$ and the vertex is (2,-1), then find the length of the side of the triangle. 
\item A variable line passes through a fixed point $\vec{P}$. The algebraic sum of the perpendiculars drawn from the points (2,0), (0,2) and (1,1) on the line is zero. Find the coordinates of the point $\vec{P}$.  
\item A straight line moves so that the sum of the reciprocals of its intercepts made on axes is constant. Show that the line passes through a fixed point. 
\item If the sum of the distances of a moving point in a plane from the axes is $l$, then finds the locus of the point.  
\item $\vec{P}_1,\vec{P}_2$ are points on either of the two lines $y-\sqrt{3}\abs{x}=2$ at a distance of 5 units from their point of intersection. Find the coordinates of the root of perpendiculars drawn from $P_1, P_2$ on the bisector of the angle between the given lines.
\item If $p$ is the length of perpendicular from the origin on the lien $\frac{x}{a}+\frac{y}{b}=1$ and $a^2,p^2,b^2$ are in A.P, then show that $a^4+b^4=0$.
\item The point (4,1) undergoes the following two successive transformations :
\begin{enumerate}
\item Reflection about the line $y=x$
\item Translation through a distance 2 units along the positive $x$-axis 
\end{enumerate}
Then the final coordinates of the point are
\begin{enumerate}
\item (4,3)
\item (3,4)
\item (1,4)
\item $\frac{7}{2}$,$\frac{7}{2}$
\end{enumerate}
\item One vertex of the equilateral with centroid at the origin and one side as $x+y-2=0$ is
\begin{enumerate}
\item (-1,-1)
\item (2,2)
\item (-2-2)
\item (2,-2)
\end{enumerate}
\item If $a,b,c$ are is A.P., then the straight lines $ax+by+c=0$ will always pass through \rule{1cm}{0.15mm}.
\item The points (3,4) and (2,-6) are situated on the \rule{1cm}{0.15mm} of the line $3x-4y-8=0$.
\item A point moves so that square of its distance from the point (3,-2) is numerically equal to its distance from the line $5x-12y=3$. The equation of its locus is %\rule{1cm}{0.15mm}.
\item Locus of the mid-points of the portion of the line $x\sin\theta+y\cos\theta=p$ intercepted between the axes is \rule{1cm}{0.15mm}.

State whether the following statements are true or false. Justify.
\item If the vertices of a triangle have integral coordinates, then the triangle can not be equilateral.
\item The line $\frac{x}{a}+\frac{y}{b}=1$ moves in such a way that $\frac{1}{a^2}+\frac{1}{b^2}=\frac{1}{c^2}$, where $c$ is a constant. The locus of the foot of the perpendicular from the origin on the given line is $x^2+y^2=c^2$.
\item 
Match the following
	\begin{table}[H]
\centering
	\resizebox{\columnwidth}{!}{
\begin{matchtabular}
  The coordinates of the points P and Q on the line x + 5y = 13 which are at a distance of 2 units from the line 12x – 5y + 26 = 0 are & (3,1),(-7,11)\\
  The coordinates of the point on the line x + y = 4, which are at a unit distance from the line 4x + 3y – 10 = 0 are & $-\frac{1}{11},\frac{11}{3}$ , $\frac{4}{3},\frac{7}{3}$\\
  The coordinates of the point on the line joining A (–2, 5) and B (3, 1) such that AP = PQ = QB are & 1,$\frac{12}{5}$ , $-3,\frac{16}{5}$\\
\end{matchtabular}
		}
		\caption{}
		\label{tab:lin-misc-1}
	\end{table}
\item The value of the $\lambda$, if the lines\\$(2x+3y+4)+\lambda(6x-y+12)=0$ are
	\begin{table}[H]
\centering
	\resizebox{\columnwidth}{!}{
\begin{matchtabular}
parallel to $y$-axis is & $\lambda =-\frac{3}{4}$\\
perpendicular to $7x+y-4=0$ is & $\lambda=-\frac{1}{3}$\\
passes through (1,2) is & $\lambda=-\frac{17}{41}$\\
parallel to $x$ axis is & $\lambda=3$\\
\end{matchtabular}
		}
		\caption{}
		\label{tab:lin-misc-2}
	\end{table}
\item The equation of the line through the intersection of the lines $2x-3y=0$ and $4x-5y=2$ and
	\begin{table}[H]
\centering
	\resizebox{\columnwidth}{!}{
\begin{matchtabular}
through the point (2,1) is & $2x-y=4$\\
perpendicular to the line & $x+y-5=0$\\
parallel to the line $3x-4y+5=0$ is & $x-y-1=0$\\
equally inclined to the axes is & $3x-4y-1=0$\\
\end{matchtabular}
		}
		\caption{}
		\label{tab:lin-misc-3}
	\end{table}
\item Point $\vec{R}\brak{h, k}$ divides a line segment between the axes in the ratio 1: 2. Find the equation of the line.
\label{chapters/11/10/2/19}
	\\
	\solution 
Given
\begin{align}
	c_1 = \frac{7}{3},\,
c_2 = -6.
\end{align}
	From \eqref{eq:parallel_lines},
we need to find $c$ such that,
\begin{align}
	\abs{c-c_1} = \abs{c-c_2} \implies c = \frac{c_1+c_2}{2}
	 = -\frac{11}{6}.
\end{align}
Hence, the desired equation is
\begin{align}
	\myvec{3 & 2}\vec{x} &= -\frac{11}{6}
\end{align}
	See \figref{fig:chapters/11/10/4/21/1}.
\begin{figure}[H]
	\centering
	\includegraphics[width=0.75\columnwidth]{chapters/11/10/4/21/figs/line_plot.jpg}
	\caption{}
	\label{fig:chapters/11/10/4/21/1}
\end{figure}

\item The tangent of angle between the lines whose intercepts on the axes are $a,-b$ and $b,-a$, respectively, is
\begin{enumerate}
\item $\frac{a^2-b^2}{ab}$
\item $\frac{b^2-a^2}{2}$
\item $\frac{b^2-a^2}{2ab}$
\item none of these 
\end{enumerate}
\item Prove that the line through the point $(x_1,y_1)$ and parallel to the line $Ax+By+C=0$ is $A(x-x_1)+B(y-y_1)=0$.
\label{chapters/11/10/3/11}
\\
\solution
Given
\begin{align}
	c_1 = \frac{7}{3},\,
c_2 = -6.
\end{align}
	From \eqref{eq:parallel_lines},
we need to find $c$ such that,
\begin{align}
	\abs{c-c_1} = \abs{c-c_2} \implies c = \frac{c_1+c_2}{2}
	 = -\frac{11}{6}.
\end{align}
Hence, the desired equation is
\begin{align}
	\myvec{3 & 2}\vec{x} &= -\frac{11}{6}
\end{align}
	See \figref{fig:chapters/11/10/4/21/1}.
\begin{figure}[H]
	\centering
	\includegraphics[width=0.75\columnwidth]{chapters/11/10/4/21/figs/line_plot.jpg}
	\caption{}
	\label{fig:chapters/11/10/4/21/1}
\end{figure}

\item  If ${p}$ and ${q}$ are the lengths of perpendiculars from the origin to the lines ${x}\cos\theta - {y}\sin\theta =  {k}\cos2\theta$ and ${x}\sec\theta + {y}\cosec\theta = {k}$, respectively, prove that ${p}^2 + 4{q}^2 = {k}^2$
\label{chapters/11/10/3/16}
\\
\solution
Given
\begin{align}
	c_1 = \frac{7}{3},\,
c_2 = -6.
\end{align}
	From \eqref{eq:parallel_lines},
we need to find $c$ such that,
\begin{align}
	\abs{c-c_1} = \abs{c-c_2} \implies c = \frac{c_1+c_2}{2}
	 = -\frac{11}{6}.
\end{align}
Hence, the desired equation is
\begin{align}
	\myvec{3 & 2}\vec{x} &= -\frac{11}{6}
\end{align}
	See \figref{fig:chapters/11/10/4/21/1}.
\begin{figure}[H]
	\centering
	\includegraphics[width=0.75\columnwidth]{chapters/11/10/4/21/figs/line_plot.jpg}
	\caption{}
	\label{fig:chapters/11/10/4/21/1}
\end{figure}

\item If $p$ is the length of perpendicular from origin to the line whose intercepts on the axes are $a$ and $b$, then show that 
\begin{align}
	\frac{1}{p^2} = \frac{1}{a^2}+ \frac{1}{b^2}
\label{eq:11/10/3/18}
\end{align}
\label{chapters/11/10/3/18}
\\
\solution
	From \eqref{eq:parallel_lines}, the desired values are available in
  \tabref{tab:11/10/3/6}.
\begin{table}[H]
  \centering
  \begin{tabular}{|c|c|c|c|c|}
    \hline
    & $\vec{n}$ & $c_1$ & $c_2$ & $d$ \\
    \hline
    a) & $\myvec{15 \\ 8}$ & 34 & -31 & $\frac{65}{17}$ \\
    \hline
    b) & $\myvec{1 \\ 1}$ & $\frac{-p}{l}$ & $\frac{r}{l}$ & $\frac{\lvert p-r \rvert}{l\sqrt{2}}$ \\
    \hline
  \end{tabular}
  \caption{}
  \label{tab:11/10/3/6}
\end{table}

\item Find perpendicular distance from the origin to the line joining the points $(\cos\theta,\sin\theta)$ and $(\cos\phi,\sin\phi)$.
\\
\solution
			From \eqref{eq:parallel_lines}, the desired values are available in
  \tabref{tab:11/10/3/6}.
\begin{table}[H]
  \centering
  \begin{tabular}{|c|c|c|c|c|}
    \hline
    & $\vec{n}$ & $c_1$ & $c_2$ & $d$ \\
    \hline
    a) & $\myvec{15 \\ 8}$ & 34 & -31 & $\frac{65}{17}$ \\
    \hline
    b) & $\myvec{1 \\ 1}$ & $\frac{-p}{l}$ & $\frac{r}{l}$ & $\frac{\lvert p-r \rvert}{l\sqrt{2}}$ \\
    \hline
  \end{tabular}
  \caption{}
  \label{tab:11/10/3/6}
\end{table}

	\item Prove that the products of the lengths of the perpendiculars drawn from the points $\myvec{\sqrt{a^2-b^2}& 0}^{\top}$ and $\myvec{-\sqrt{a^2-b^2} &0}^{\top}$ to the line $\frac{x}{a} \cos{\theta} + \frac{y}{b}\sin{\theta} =1 $ is $ b^2 $.
\\
    \solution 
			From \eqref{eq:parallel_lines}, the desired values are available in
  \tabref{tab:11/10/3/6}.
\begin{table}[H]
  \centering
  \begin{tabular}{|c|c|c|c|c|}
    \hline
    & $\vec{n}$ & $c_1$ & $c_2$ & $d$ \\
    \hline
    a) & $\myvec{15 \\ 8}$ & 34 & -31 & $\frac{65}{17}$ \\
    \hline
    b) & $\myvec{1 \\ 1}$ & $\frac{-p}{l}$ & $\frac{r}{l}$ & $\frac{\lvert p-r \rvert}{l\sqrt{2}}$ \\
    \hline
  \end{tabular}
  \caption{}
  \label{tab:11/10/3/6}
\end{table}

\item O is the origin and A is $(a,b,c)$. Find the direction cosines of the line OA and the equation of the plane through A at right angle at OA.
\item Two systems of rectangular axis have the same origin. If a plane cuts them at distances $a,b,c$ and $a^{\prime},b^{\prime},c^{\prime}$, respectively, from the origin, prove that $$\frac{1}{a^2}+\frac{1}{b^2}+\frac{1}{c^2}=\frac{1}{{a^{\prime}}^2}+\frac{1}{{b^{\prime}}^2}+\frac{1}{{c^{\prime}}^2}$$.
\item Equation of the line passing through the point $(a\cos^3\theta, a\sin^3\theta)$ and perpendicular to the line $x\sec\theta+y\csc\theta=a$ is $x\cos\theta-y\sin\theta=\alpha\sin2\theta$.
\item The distance between the lines $y=mx+c$,\text{ and }$y=mx+c^2$ is
\begin{enumerate}
\item $\frac{c_1-c_2}{\sqrt{m+1}}$
\item $\frac{\abs{c_1-c_2}}{\sqrt{1+m^2}}$
\item $\frac{c^2-c^1}{\sqrt{1+m^2}}$
\item 0
\end{enumerate}
	\item Find the area of triangle formed by the lines $y-x=0, x+y=0, \text{ and } x-k=0$.
		\\
\solution
		Given
\begin{align}
	c_1 = \frac{7}{3},\,
c_2 = -6.
\end{align}
	From \eqref{eq:parallel_lines},
we need to find $c$ such that,
\begin{align}
	\abs{c-c_1} = \abs{c-c_2} \implies c = \frac{c_1+c_2}{2}
	 = -\frac{11}{6}.
\end{align}
Hence, the desired equation is
\begin{align}
	\myvec{3 & 2}\vec{x} &= -\frac{11}{6}
\end{align}
	See \figref{fig:chapters/11/10/4/21/1}.
\begin{figure}[H]
	\centering
	\includegraphics[width=0.75\columnwidth]{chapters/11/10/4/21/figs/line_plot.jpg}
	\caption{}
	\label{fig:chapters/11/10/4/21/1}
\end{figure}

\item The lines $ax+2y+1=0$, $bx=3y+1=0\text{ and }cx+4y+1=0$ are concurrent if $a$, $b$, $c$ are in G.P.
\item 
$P(a,b)$ is the mid-point of the line segment between axes. Show that the equation of the line is $\frac{x}{a}+\frac{y}{b}=2$
\label{chapters/11/10/2/18}
\\
\solution
Given
\begin{align}
	c_1 = \frac{7}{3},\,
c_2 = -6.
\end{align}
	From \eqref{eq:parallel_lines},
we need to find $c$ such that,
\begin{align}
	\abs{c-c_1} = \abs{c-c_2} \implies c = \frac{c_1+c_2}{2}
	 = -\frac{11}{6}.
\end{align}
Hence, the desired equation is
\begin{align}
	\myvec{3 & 2}\vec{x} &= -\frac{11}{6}
\end{align}
	See \figref{fig:chapters/11/10/4/21/1}.
\begin{figure}[H]
	\centering
	\includegraphics[width=0.75\columnwidth]{chapters/11/10/4/21/figs/line_plot.jpg}
	\caption{}
	\label{fig:chapters/11/10/4/21/1}
\end{figure}

\end{enumerate}

\newpage
\section{Constructions}
\subsection{Formulae}
\input{chapters/formulae/const.tex}
\subsection{Triangle}
\begin{enumerate}[label=\thesubsection.\arabic*,ref=\thesubsection.\theenumi]
\item Construct a triangle $ABC$ in which $BC=7cm, \angle{B}=75\degree$ and $AB + AC = 13 cm$.
\label{chapters/9/11/2/1}
	\\
	\solution 
		From 
		\eqref{eq:9/11/2/1}
		and 
		\eqref{eq:9/11/2/1-final},
		we obtain
		\figref{fig:9/11/2/1}.
		\iffalse
		See
\begin{lstlisting}
	codes/triangle/const-aBsum.py
\end{lstlisting}
\fi
	\begin{figure}[H]
		\centering
 \includegraphics[width=0.75\columnwidth]{chapters/9/11/2/1/figs/vector.pdf}
		\caption{}
		\label{fig:9/11/2/1}
  	\end{figure}
	

%
\item Construct a triangle $ABC$ in which $BC=8cm, \angle{B}=45\degree$ and $AB - AC = 3.5 cm$.
\label{chapters/9/11/2/2}
\\
\solution
See \figref{fig:Fig1}.
\begin{figure}[H]
 \begin{center}
	 \includegraphics[width=0.75\columnwidth]{chapters/9/11/2/2/figs/vector.pdf}
 \end{center}
 \caption{}
 \label{fig:Fig1}
\end{figure}

%
\item Construct a triangle $ABC$ in which $BC=6cm, \angle{B}=60\degree$ and $AC - AB = 2cm$.
\label{chapters/9/11/2/3}
\\
\solution 
See 
		\figref{fig:9/11/2/3} obtained by substituting $K = -2$. 
	\begin{figure}[H]
		\centering
 \includegraphics[width=0.75\columnwidth]{chapters/9/11/2/3/figs/vector.pdf}
		\caption{}
		\label{fig:9/11/2/3}
  	\end{figure}

%
\item Construct a right triangle whose base is 12$cm$ and sum of its hypotenuse and other side is 18$cm$.
\label{chapters/9/11/2/5}
\\
\solution
For $a = 12, \angle B = 90\degree, b+c = 18$, we obtain 
		\figref{fig:9/11/2/5}.
	\begin{figure}[H]
		\centering
 \includegraphics[width=0.75\columnwidth]{chapters/9/11/2/5/figs/vector.pdf}
		\caption{}
		\label{fig:9/11/2/5}
  	\end{figure}

%
\item Construct a triangle $ABC$ in which $\angle{B}=30\degree, \angle{C}=90\degree$ and  $AB+BC+CA=11cm$.
\label{chapters/9/11/2/4}
\\
\solution 
From 
		\eqref{eq:9/11/2/4}
		and
		\eqref{eq:9/11/2/4-final},
		\figref{fig:9/11/2/4}
		is generated.
		\iffalse
		See
\begin{lstlisting}
	codes/triangle/const-BCsum.py
\end{lstlisting}
\fi
	\begin{figure}[H]
		\centering
 \includegraphics[width=0.75\columnwidth]{chapters/9/11/2/4/figs/vector.pdf}
		\caption{}
		\label{fig:9/11/2/4}
  	\end{figure}

%
\item Draw a right triangle ${ABC}$ in which $BC=12 cm$, $AB=5 cm$ and $\angle{B}=90\degree$.
\item Draw an isosceles triangle ${ABC}$ in which $AB=AC=6cm$ and $BC =6cm$.
\item Draw a triangle ${ABC}$ in which $AB=5cm,BC=6cm$ and $\angle {ABC}=60\degree$.
\item Draw a triangle ${ABC}$ in which $AB=4cm, BC=6cm$ and $AC=9cm$.
\item Draw a triangle ${ABC}$ in which $BC=6 cm, CA=5 cm$ and $AB=4 cm$. 
\item Is it possible to construct a triangle with lengths of its sides as 4$cm$, 3$cm$ and 7$cm$? Give reason for your answer.

\item Is it possible to construct a triangle with lengths of its sides as 9$cm$, 7$cm$ and 17$cm$? Give reason for your answer.

\item Is it possible to construct a triangle with lengths of its sides as 8$cm$, 7$cm$ and 4$cm$? Give reason for your answer.

\item Two sides of a triangle are of lengths 5$cm$ and 1.5$cm$. The length of the third side of the triangle cannot be
\begin{enumerate}
\item $3.6 cm$
\item $4.1 cm$
\item $3.8 cm$
\item $3.4 cm$
\end{enumerate}
\item The construction of a triangle $ABC$, given that $BC = 6 cm, \angle B = 45\degree$ is not possible when difference of $AB$ and $AC$ is equal to
		\begin{enumerate}
			\item 6.9$cm$
			\item 5.2$cm$
			\item 5.0$cm$
			\item 4.0$cm$
		\end{enumerate}
	\item The construction of a triangle $ABC$, given that $BC = 6 cm, \angle C = 60\degree$ is possible when difference of $AB$ and $AC$ is equal to
		\begin{enumerate}
			\item 3.2$cm$
			\item 3.1$cm$
			\item 3$cm$
			\item 2.8$cm$
		\end{enumerate}
\item Construct a triangle whose sides are $3.6 cm$, $3.0 cm$ and $4.8 cm$. Bisect the smallest angle and measure each part.
\item Construct a triangle $ABC$ in which $BC = 5 cm$, $\angle B = 60\degree$ and $AC+AB = 7.5cm$.
\end{enumerate}
Construct each of the following and give justification :
\begin{enumerate}[label=\thesubsection.\arabic*,ref=\thesubsection.\theenumi,resume*]
\item A triangle if its perimeter is 10.4$cm$ and two angles are 45\degree and 120\degree.
\item A triangle $PQR$ given that $QR$ = 3$cm$, $\angle PQR = 45\degree$ and $QP - PR = 2 cm$.
\item A right triangle when one side is 3.5$cm$ and sum of other sides and the hypotenuse
is 5.5$cm$.
\item An equilateral triangle if its altitude is 3.2$cm$.
\end{enumerate}                               
Write true or false in each of the following. Give reasons for your answer:
\begin{enumerate}[label=\thesubsection.\arabic*,ref=\thesubsection.\theenumi,resume*]
\item A triangle $ABC$ can be constructed in which $AB = 5cm, \angle A =45\degree$ and $BC + AC = 5cm$.
\item A triangle $ABC$ can be constructed in which $BC = 6cm, \angle B =30\degree$ and $AC - AB=4cm$.
\item A triangle $ABC$ can be constructed in which $\angle B =105\degree,\angle C =90\degree$ and $AB + BC + AC = 10cm$.        
\item A triangle $ABC$ can be constructed in which $\angle B =60\degree, \angle C =45 \degree$ and $AB + BC + AC = 12cm$.           
\item Draw a right triangle ${ABC}$ in which $BC=12$ cm, $AB=5$ cm and $\angle{B}=90\degree$.
\item Draw a triangle ${ABC}$ in which $AB$=4 cm, $BC=6 cm\text{ and }AC=9$. 
\item Draw a triangle ${ABC}$ in which $AB$=5 cm. $BC=6 cm\text{ and }\angle {ABC}=60\degree$. 
\item Draw a parallelogram ${ABCD}$ in which $BC=5$ cm, $AB=3$ cm and $\angle{ABC}=60\degree$, divide it into triangles ${ACB}\text{ and }{ABD}$ by the diagonal $BD$. 
Construct the triangle $BD'C'$ similar to $\triangle{BDC}$ with scale factor $\frac{4}{3}$. Draw the line segment $D'A'$ parallel to $DA$ where $\vec{A}$' lies on extended side $BA$. Is $A'BC'D'$ a parallelogram? 
\item Draw a triangle ${ABC}$ in which $BC=6$ cm, $CA=5$ cm and $AB=4$ cm. 
\end{enumerate}

\subsection{Quadrilateral}
\begin{enumerate}[label=\thesubsection.\arabic*,ref=\thesubsection.\theenumi]
	\item Draw a quadrilateral in the Cartesian plane, whose vertices are (-4, 5), (0, 7), (5, -5) and (-4, -2). 
\item Draw a parallelogram ${ABCD}$ in which $BC=5 cm, AB=3 cm$ and $\angle{ABC}=60\degree$, divide it into triangles ${ACB}\text{ and }{ABD}$ by the diagonal $BD$. 
\item Construct a square of side $3 cm$.
\item Construct  a rectangle whose adjacent sides are of lengths $5 cm$ and $3.5 cm$.
\item Construct a rhombus whose side is of length $3.4 cm$ and one of its angles is $45\degree$.
\item Construct a rhombus whose diagonals are 4 cm and 6 cm in lengths.
\end{enumerate}

\newpage
\section{Linear Forms}
\subsection{Formulae}
  See \tabref{tab:11/10/3/3}.
			\eqref{eq:PQ-final} was used for computing the distance from the origin.
			\begin{table}[H]
  \centering
  \begin{tabular}{|c|c|c|c|c|}
    \hline
    & $\vec{n}$ & Angle & $c$& Distance \\
    \hline
    a) & $\myvec{1 \\ -\sqrt{3}}$ & $\tan^{-1}(-\sqrt{3}) = \frac{2\pi}{3}$ &-8 & 4 \\
    \hline
    b) & $\myvec{0 \\ 1}$ & $\tan^{-1}\infty = \frac{\pi}{2}$ &2 & 2 \\
    \hline
    c) & $\myvec{1 \\ -1}$ & $\tan^{-1}(-1) = \frac{3\pi}{4}$ &4 & $2\sqrt{2}$ \\
    \hline
  \end{tabular}
  \caption{}
  \label{tab:11/10/3/3}
\end{table}


\subsection{Equation }
Find the equation of line 
\begin{enumerate}[label=\thesubsection.\arabic*,ref=\thesubsection.\theenumi]
	\item passing through the point $\vec{P} = (– 4, 3)$ with slope $\frac{1}{2}$.
\label{chapters/11/10/2/2}
\\
\solution
			From \eqref{eq:line-school-normal},
\begin{align}
\vec{n}\equiv \myvec{\frac{1}{2}\\ -1}
\implies \myvec{\frac{1}{2}&-1}{\vec{x}}&=-5
\end{align}
using \eqref{eq:geo-normal}.
See 
		\figref{fig:chapters/11/10/2/2/Figure}.
\begin{figure}[H]
\centering
\includegraphics[width=0.75\columnwidth]{chapters/11/10/2/2/figs/fig.pdf}
\caption{}
		\label{fig:chapters/11/10/2/2/Figure}
\end{figure}

	\item passing through $\myvec{0\\0}$ with slope $m$.\\
\label{chapters/11/10/2/3}
\solution
Given
\begin{align}
	c_1 = \frac{7}{3},\,
c_2 = -6.
\end{align}
	From \eqref{eq:parallel_lines},
we need to find $c$ such that,
\begin{align}
	\abs{c-c_1} = \abs{c-c_2} \implies c = \frac{c_1+c_2}{2}
	 = -\frac{11}{6}.
\end{align}
Hence, the desired equation is
\begin{align}
	\myvec{3 & 2}\vec{x} &= -\frac{11}{6}
\end{align}
	See \figref{fig:chapters/11/10/4/21/1}.
\begin{figure}[H]
	\centering
	\includegraphics[width=0.75\columnwidth]{chapters/11/10/4/21/figs/line_plot.jpg}
	\caption{}
	\label{fig:chapters/11/10/4/21/1}
\end{figure}

    \item passing through 
    $\vec{A} = \myvec{2\\2\sqrt{3}}$ and inclined with the x-axis at an angle 
    of 75\textdegree.
\label{chapters/11/10/2/4}
\\
    \solution 
    \begin{align}
	    \vec{n} &= \myvec{-1\\2+\sqrt{3}}
        \label{eq:11/10/2/4normal-vec}
	\\
        \implies \myvec{-1&2+\sqrt{3}}\vec{x} &=\myvec{-1&2+\sqrt{3}}\myvec{2\\2\sqrt{3}}  
	    \\
	    &= 4\brak{\sqrt{3}+1}
        \label{eq:11/10/2/4line}
    \end{align}
is the desired equation.  See \figref{fig:11/10/2/4line}.
    \begin{figure}[H]
        \centering
        \includegraphics[width=0.75\columnwidth]{chapters/11/10/2/4/figs/line.png}
        \caption{}
        \label{fig:11/10/2/4line}
    \end{figure}

\item intersecting the x-axis at a distance of 3 units to the left of origin with slope of -2.
\label{chapters/11/10/2/5}
\\
\solution 
		From the given information,
\begin{align}		
	\vec{A}=\myvec{-3\\0},\,
\vec{n} = \myvec{2 \\1}.
\end{align}
The desired equation of the line is
\begin{align}
\implies	\myvec { 2 & 1 } \brak{ \vec{x} - \myvec{ -3 \\ 0}} &= 0  \\
	\text{or, }	\myvec{ 2 & 1} \vec{x}  &= -6
        \label{eq:chapters/11/10/2/5/1}
\end{align}
See \figref{fig:chapters/11/10/2/5/Fig1}.
\begin{figure}[H]
	\begin{center}
		\includegraphics[width=0.75\columnwidth]{chapters/11/10/2/5/figs/line1.pdf}
	\end{center}
\caption{}
\label{fig:chapters/11/10/2/5/Fig1}
\end{figure}


\item intersecting the y-axis at a distance of 2 units above the origin and making an
angle of $30\degree$ with positive direction of the x-axis.
\\
\solution 
\begin{align}
    \vec{n} =  \myvec{-\frac{1}{\sqrt{3}} \\ 1},
    \vec{A} = \myvec{0 \\ 2}.
\end{align}
Hence, 
the equation of the line is given by
\begin{align}
\myvec{-\frac{1}{\sqrt{3}}&1}\brak{ \vec{x} - \myvec{0 \\ 2}} &= 0  \\
    \text{or, }	\myvec{-\frac{1}{\sqrt{3}}&1} \vec{x}  &= 2
\end{align}
%
See
    \figref{fig:chapters/11/10/2/6/line}.
\begin{figure}[H]
    \centering
    \includegraphics[width=0.75\columnwidth]{chapters/11/10/2/6/figs/line.png}
    \caption{}
    \label{fig:chapters/11/10/2/6/line}
\end{figure}


\item passing through (1,2) and making angle $30\degree$ with $y$-axis.
\item passing through the points $\vec{A}\myvec{-1\\1}$ and $\vec{B}\myvec{2\\-4}$.
\label{chapters/11/10/2/7}
\\
\solution 
		From \eqref{prop:lin-eq-unit-mat},
\begin{align}
	\myvec{ -1 & 1\\  2 & -4 }\vec{n} = \myvec{1 \\ 1 }
	\\
	\implies 
	\augvec{2}{1}{ 
	-1 & 1 & 1
	\\  
	2 & -4 & 1
	}
     \xleftrightarrow[]{R_2 \leftarrow R_2+2R_1}
	\augvec{2}{1}{ 
	-1 & 1 & 1
	\\ 
	0 & -2 & 3 
	}
	\\
     \xleftrightarrow[]{R_1 \leftarrow 2R_1+R_2}
	\augvec{2}{1}{ 
	-2 & 0 &5 
	\\ 
	0 & -2 & 3 
	}
	\implies \vec{n} = -\frac{1}{2}\myvec{ 5 \\ 3}
\end{align}
Thus, from
		\eqref{prop:lin-eq-unit},
the equation of the line is
\begin{align}
 \myvec{ 5 & 3}\vec{x}  &= -2
\end{align}
See 
   \figref{fig:chapters/11/10/2/7/Line_AB}.
\begin{figure}[H]
  \centering
   \includegraphics[width=0.75\columnwidth]{chapters/11/10/2/7/figs/Figure_1.png}
   \caption{}
   \label{fig:chapters/11/10/2/7/Line_AB}
\end{figure}





\item passing through the points $(3,4,-7)$ and $(1,-1,6)$. 
\item The vector equation of the line 
\begin{align*}
	\frac{x-5}{3}=\frac{y+4}{7}=\frac{z-6}{2} 
\end{align*}
is \noindent\rule{2cm}{0.4pt}. 
\item The vector equation of the line 
\begin{align*}
	\frac{x-5}{3}=\frac{y+4}{7}=\frac{z-6}{2}
\end{align*}
 is \noindent\rule{2cm}{0.4pt}.
\item 
The vertices of triangle $PQR$ are $\vec{P}(2,1), \vec{Q}(-2,3), \vec{R}(4,5)$. Find the equation of the median through $\vec{R}$.
\label{chapters/11/10/2/9}
\\
\solution
	\begin{figure}[H]
		\centering
 \includegraphics[width=0.75\columnwidth]{chapters/11/10/2/9/figs/line.png}
		\caption{}
		\label{fig:11/10/2/9}
  	\end{figure}
	See Fig. 
		\ref{fig:11/10/2/9}.
Using section formula, the mid point of $PQ$ is
\begin{align}
\vec{A} = \frac{\vec{P} +\vec{Q} }{2}
	= {\myvec{0\\2}}
\end{align} 
Following the approach in \probref{chapters/11/10/2/7},
\begin{align*}
	\augvec{2}{1}{ 
	4 & 5 & 1
	\\  
	0 & 2 & 1
	}
	\xleftrightarrow[R_2 \leftarrow 4R_2 ]{R_1 \leftarrow 2R_1 -5R_2}
	\augvec{2}{1}{ 
	8 & 0 & -3 
	\\ 
	0 & 8 & 4 
	}
	\implies \vec{n} = \frac{1}{8}\myvec{ -3 \\ 4}
\end{align*}
Thus,
the equation of the line is 
\begin{align}
	\myvec{-3 & 4}\vec{x} =8 
\end{align}

	\item Find the equations of the planes that pass through the points
\begin{enumerate}
\item $\vec{A}= \myvec{1\\1\\– 1}, \vec{B}=\myvec{6\\4\\– 5},\vec{C}= \myvec{– 4\\– 2\\3}$
\item $\vec{A}= \myvec{1\\1\\0}, \vec{B}= \myvec{1\\2\\1}, \vec{C}= \myvec{– 2\\2\\-1}$
\end{enumerate}
    \solution
		\begin{enumerate}
	\item From 
		\eqref{prop:lin-eq-unit-mat},
\begin{align}
\myvec{1&1&-1\\ 6&4&-5\\ -4&-2&3} \vec{n} = \myvec{1\\1\\1}
\end{align}
\begin{align*}
	\implies \myvec{1&1&-1&\vrule&1\\6&4&-5&\vrule&1\\-4&-2&3&\vrule&1}
	\\
\xleftrightarrow[R_3 \leftarrow R_3 + 4R_1]{R_2 \leftarrow R_2 - 6R_1}
\myvec{1&1&-1&\vrule&1\\0&-2&1&\vrule&-5\\0&2&-1&\vrule&5}\\ 
\xleftrightarrow[{R_1 \leftarrow 2R_1 + R_2}] {R_3 \leftarrow R_3 + R_2}
\myvec{2&0&-1&\vrule&-3\\0&2&-1&\vrule&5\\0&0&0&\vrule&0}
\end{align*}
Since we obtain a 0 row, 
the given points are collinear.
The direction vector of the line is
\begin{align}
\vec{m}=\vec{B}-\vec{C} \equiv \myvec{5\\3\\-4}
\end{align}
and the equation of a line is given by,
\begin{align}
	\vec{x}&=\vec{A}+  \kappa\vec{m}\\
&= \myvec{1\\1\\– 1} + \kappa \myvec{5\\3\\-4}
\end{align}
See 
     \figref{fig:chapters/12/11/3/6/1}.
\begin{figure}[H]
  \centering
   \includegraphics[width=0.75\columnwidth]{chapters/12/11/3/6/figs/collinear_points.png}
    \caption{}
     \label{fig:chapters/12/11/3/6/1}
     \end{figure}
     \item  In this case, 
\begin{align}
\myvec{1&1&0 \\ 1&2&1 \\ -2&2&-1} \vec{n}=\vec{1}
\end{align}
\begin{align*}
\implies
\myvec{1&1&0&\vrule&1\\1&2&1&\vrule&1\\-2&2&-1&\vrule&1}
\\
\xleftrightarrow[R_3 \leftarrow R_3 + 2R_1]{R_2 \leftarrow R_2 - R_1}
\myvec{1&1&0&\vrule&1\\0&1&1&\vrule&0\\0&4&-1&\vrule&3}
\\
	\xleftrightarrow[R_3 \leftarrow R_3 - 4R_2]{R_1 \leftarrow R_1- R_2}
\myvec{1&0&-1&\vrule&1\\0&1&1&\vrule&0\\0&0&-5&\vrule&3}\\
	\xleftrightarrow[R_2 \leftarrow 5R_2 + R_3]{R_1 \leftarrow 5R_1- R_3}
\myvec{5&0&0&\vrule&2\\0&5&0&\vrule&3\\0&0&5&\vrule&-3}
\end{align*}
Hence, the equation of the plane is
\begin{align}
\myvec{2 & 3 & -3} \vec{x} = 5
\end{align}
\end{enumerate}

\item Find the equation of the plane through the points $(2,1,0)$, $(3,-2,-2)$ and $(3,1,7)$.
\item A plane passes through the points $(2,0,0) (0,3,0)$ and $(0,0,4)$. The equation of the plane is \noindent\rule{2cm}{0.4pt}.
\item If the intercept of a line between the coordinate axes is divided by the point (-5,4) in the ratio 1:2 then find the equation of the line.
\item Find the equation of a line that cuts off equal intercepts on the coordinate axes and passes through the point $(2,3)$.  
	\\
\solution 
\label{chapters/11/10/2/12}
Let $(a,0)$  and  $(0,a)$ be the intercept points. 
\begin{align}
\vec{m} 
        &=   \myvec{
		a \\
		0 
		} - \myvec{
		   0 \\
		   a
		}  
        		  \equiv \myvec{
                           1 \\
			   -1 
		         } 
			 \\
			 \implies
\vec{n} &=  \myvec{
		     1 \\
		     1
	     } 
\end{align}
and 
the equation of the  line is
\begin{align}
	\myvec { 1 & 1 } \brak{ \vec{ x  - \myvec{ 2 \\
                                   3
			     }
		}}  &= 0  \\
\implies		\myvec{ 1 & 1} \vec{x}  &= 5 
        \label{eq:11/10/2/12/1}
\end{align}
See  \figref{fig:11/10/2/12/Fig1}.
\begin{figure}[H]
	\begin{center}
		\includegraphics[width=0.75\columnwidth]{chapters/11/10/2/12/figs/problem12.pdf}
	\end{center}
\caption{}
\label{fig:11/10/2/12/Fig1}
\end{figure}


\item 
Find the equation of a line passing through a point (2,2) and cutting off intercepts on the axes whose sum is 9.
\label{chapters/11/10/2/13}
	\\
	\solution 
Let  the intercept points be
\begin{align}
{\vec{P}}=\myvec{
  a\\
  0}
 , {\vec{Q}}=\myvec{
  0\\
  b}
  \text{ and }
   {\vec{R}}=\myvec{
  2\\
  2}
\end{align}
be the given point.  
Forming the collinearity matrix from 
		\eqref{prop:lin-dep-rank},
\begin{align}
	\myvec{ \vec{P}-\vec{Q} &\vec{P}-\vec{R}} 
	=
	 \myvec{
  a & a-2\\
  -b & -2
 }
\end{align}
which is singular if 
\begin{align}
 ab -2\brak{a+b} = 0
 \implies ab = 18
		\label{eq:11/10/2/13-a+b}
		\\
\because  a + b = 9.
\end{align}
$\therefore a,b$
are the roots of
\begin{align}
	x^2 -9x +18 = 0.
\end{align}
yielding
\begin{align}
	\myvec{a \\ b} = \myvec{6 \\ 3}, \myvec{3\\6}
\end{align}
Since 
\begin{align}
	\vec{m} = \myvec{a \\ -b},
	\vec{n} = \myvec{b \\ a} \equiv \myvec{1 \\ 2}, \myvec{2\\1}
\end{align}
Thus, the possible equations of the line are 
\begin{align}
\myvec{1 & 2}\vec{x} = 6
	\\
	\myvec{2&1}\vec{x} = 6
\end{align}
		See \figref{fig:11/10/2/13}.
	\begin{figure}[H]
		\centering
 \includegraphics[width=0.75\columnwidth]{chapters/11/10/2/13/figs/assign4.png}
		\caption{}
		\label{fig:11/10/2/13}
  	\end{figure}

\item Find the equation of the lines which passes the point (3,4) and cuts off intercepts from the coordinate axes such that their sum is 14.
\item Find the equation of the straight line which passes through the point (1, -2) and cuts off equal intercepts from axes.
\item Find the equation of the line which passes through the point (-4,3) and the portion of the line intercepted between the axes is divided internally in ratio 5:3 by this point.
\item Consider the following population and year graph. Find the slope of the line AB and using it, find what will be the population in the year 2010.
\\
\begin{figure}[H]
\centering
\includegraphics[width=0.75\columnwidth]{chapters/11/10/1/14/figs/fig.png}
\caption{}
\label{fig:chapters/11/10/1/14/1}
\end{figure}
\solution
Given
\begin{align}
	c_1 = \frac{7}{3},\,
c_2 = -6.
\end{align}
	From \eqref{eq:parallel_lines},
we need to find $c$ such that,
\begin{align}
	\abs{c-c_1} = \abs{c-c_2} \implies c = \frac{c_1+c_2}{2}
	 = -\frac{11}{6}.
\end{align}
Hence, the desired equation is
\begin{align}
	\myvec{3 & 2}\vec{x} &= -\frac{11}{6}
\end{align}
	See \figref{fig:chapters/11/10/4/21/1}.
\begin{figure}[H]
	\centering
	\includegraphics[width=0.75\columnwidth]{chapters/11/10/4/21/figs/line_plot.jpg}
	\caption{}
	\label{fig:chapters/11/10/4/21/1}
\end{figure}

\item Slope of a line which cuts off intercepts of equal length on the axes is 
\begin{enumerate}
\item -1
\item -0
\item 2
\item $\sqrt{3}$
\end{enumerate}
\item If the coordinates of middle point of the portion of a line intercepted between the coordinate axes is (3,2),then the equation of the line will be
\begin{enumerate}
\item $2x+3y=12$
\item $3x+2y=12$
\item $4x-3y=6$
\item $5x-2y=10$
\end{enumerate}
\item If the line $\frac{x}{a}+\frac{y}{b}=1$ passes the points (2,-3) and (4,-5), then $(a,b)$ is 
\begin{enumerate}
\item (1,1)
\item (-1,1)
\item (1,-1)
\item (-1,-1)
\end{enumerate}
\item The intercepts made by the plane $2x-3y+5z+4=0$ on the co-ordinate axis are $\brak{-2,\frac{4}{3},-\frac{4}{5}}$.
\item The line $\overrightarrow{r}=2\hat{i}-3\hat{j}-\hat{k}+\lambda(\hat{i}-\hat{j}+2\hat{k})$ lies in the plane $\overrightarrow{r} \cdot (3\hat{i}+\hat{j}-\hat{k})+2=0$.
\end{enumerate}

\subsection{Parallel}
\input{chapters/linear/parallel.tex}
\subsection{Perpendicular}
  See \tabref{tab:11/10/3/3}.
			\eqref{eq:PQ-final} was used for computing the distance from the origin.
			\begin{table}[H]
  \centering
  \begin{tabular}{|c|c|c|c|c|}
    \hline
    & $\vec{n}$ & Angle & $c$& Distance \\
    \hline
    a) & $\myvec{1 \\ -\sqrt{3}}$ & $\tan^{-1}(-\sqrt{3}) = \frac{2\pi}{3}$ &-8 & 4 \\
    \hline
    b) & $\myvec{0 \\ 1}$ & $\tan^{-1}\infty = \frac{\pi}{2}$ &2 & 2 \\
    \hline
    c) & $\myvec{1 \\ -1}$ & $\tan^{-1}(-1) = \frac{3\pi}{4}$ &4 & $2\sqrt{2}$ \\
    \hline
  \end{tabular}
  \caption{}
  \label{tab:11/10/3/3}
\end{table}


\subsection{Angle}
\input{chapters/linear/angle.tex}
\subsection{Intersection}
In this case, 
\begin{align}
	\vec{V} &= \myvec{ 0 & 0 \\ 0 & 1} \\
	\vec{u} &= \myvec{-2 \\ 0} \\
	f &= 0
\end{align}
For the given line $y=3$, the parameters are
\begin{align}
	\vec{h} = \myvec{0 \\ 3} , \vec{m} = \myvec{1 \\ 0 }
\end{align}
The intersection of 
the line with the conic is obtained from \eqref{eq:tangent_roots} 
as
\begin{align}
	\kappa  = \frac{9}{4} 
\end{align}
The point of contact is given as
\begin{align}
	\vec{a}_0 = \myvec{\frac{9}{4}  \\[1pt] \\ 3}
\end{align}
From \figref{fig:chapters/12/8/1/13/Fig1},
the desired area of the region is obtaioned as
\begin{align}
	\int_{0}^{3} \ \frac{y^2}{4} \,dy &= \frac{1}{12}\sbrak{y^3}_{0}^{3} \\
	&= \frac{1}{12}\brak{27-0} \\
	&= \frac{9}{4} \text{ sq.units}
\end{align}
\begin{figure}[H]
	\begin{center}
		\includegraphics[width=0.75\columnwidth]{chapters/12/8/1/13/figs/problem13.pdf}
	\end{center}
\caption{}
\label{fig:chapters/12/8/1/13/Fig1}
\end{figure}

\subsection{Miscellaneous }
\begin{enumerate}[label=\thesubsection.\arabic*,ref=\thesubsection.\theenumi]


\item Find the values of $k$ for which the line 
\begin{align}
(k-3)x-(4-k^2)y+k^2-7k+6=0 \label{eq:chapters/11/10/4/1/1}
\end{align}
is
\begin{enumerate}
\item Parallel to the $x$-axis
\item Parallel to the $y$-axis
\item Passing through the origin
\end{enumerate}
    \solution 
		Given
\begin{align}
	c_1 = \frac{7}{3},\,
c_2 = -6.
\end{align}
	From \eqref{eq:parallel_lines},
we need to find $c$ such that,
\begin{align}
	\abs{c-c_1} = \abs{c-c_2} \implies c = \frac{c_1+c_2}{2}
	 = -\frac{11}{6}.
\end{align}
Hence, the desired equation is
\begin{align}
	\myvec{3 & 2}\vec{x} &= -\frac{11}{6}
\end{align}
	See \figref{fig:chapters/11/10/4/21/1}.
\begin{figure}[H]
	\centering
	\includegraphics[width=0.75\columnwidth]{chapters/11/10/4/21/figs/line_plot.jpg}
	\caption{}
	\label{fig:chapters/11/10/4/21/1}
\end{figure}

	\item Find the  equations of the lines, which cutoff intercepts on the axes  whose sum and product are 1 and -6 respectively.
\\
\solution
		Let the intercepts be $a$ and  $b$. Then
\begin{align}
a+b=1,
ab=-6 \label{eq:11/10/4/32a}
\\
\implies  a = 3, b = -2
\end{align}
Thus, the possible 
intercepts are
\begin{align}
\myvec{3\\0}, \myvec{0\\-2},
\myvec{-2\\0}, \myvec{0\\3}
\end{align}
From
		\eqref{prop:lin-eq-unit-mat},
\begin{align}
	\myvec{3 & 0 \\ 0 &-2}\vec{n} = \myvec{1 \\ 1}
	\\
	\implies \vec{n} = \myvec{\frac{1}{3} \\ -\frac{1}{2}}
	\\
	\text{or, } \myvec{2 & -3}\vec{x} = 6
\end{align}
using		\eqref{prop:lin-eq-unit}.
Similarly, the other line can be obtained
as
\begin{align}
	\myvec { 3 & -2 }  \vec{x}  = -6        
\end{align}
See  
\figref{fig:11/10/4/3line segmenta}.
\begin{figure}[H]
\centering
\includegraphics[width=0.75\columnwidth]{chapters/11/10/4/3/figs/inter.png}
\caption{}
\label{fig:11/10/4/3line segmenta}
\end{figure}

\item A ray of light passing through the point $\vec{P} = \brak{1, 2}$ reflects on the x-axis at point $\vec{A}$ and the reflected ray passes through the point $\vec{Q} =\brak{5, 3}$. Find the coordinates of $\vec{A}$.
\\
    \solution 
			From \eqref{eq:11/10/4/22},
the reflection of $\vec{Q}$ is 
\begin{align}
\vec{R}  
= \myvec{5\\-3}
\end{align}
Letting
\begin{align}
\vec{A} = \myvec{x\\0},
\end{align}
since 
$\vec{P},
\vec{A},  
\vec{R}  
$
are collinear, 
		from \eqref{prop:lin-dep-rank},
\begin{align}
	\myvec{
		1 & 1 & 2 
		\\ 
		1 & 5 & -3 
		\\
		1 & x & 0 }
	\xleftrightarrow[R_3=R_3 - R_1]{R_2 = R_2 - R_1}
	\myvec{
		1 & 1 & 2 
		\\ 
		0 & 4 & -5 
		\\
		0 & x-1 & -2 }
	\\
	\xleftrightarrow[]{R_3 = 4R_3 - \brak{x-1}R_2}
	\myvec{
		1 & 1 & 2 
		\\ 
		0 & 4 & -5 
		\\
		0 & 0 & 5x-13 }
	\implies x = \frac{13}{5}
\end{align}
See  
\figref{fig:chapters/11/10/4/22/1}.
\begin{figure}[H]
\centering
\includegraphics[width=0.75\columnwidth]{chapters/11/10/4/22/figs/fig.png}
\caption{}
\label{fig:chapters/11/10/4/22/1}
\end{figure}




\item Prove that in any $\triangle{ABC}$, cos A=$\frac{b^2+c^2-a^2}{2bc}$, where a,b,c are the magnitudes of the sides opposite to the vertices A,B,C respectively.
\item Distance of the point $(\alpha, \beta, \gamma)$ from y-axis is
\begin{enumerate}
	\item $\beta$ 
	\item $\abs{\beta}$
	\item $\abs{\beta+\gamma}$
	\item $\sqrt{\alpha^2+\gamma^2}$
\end{enumerate}
\item The reflection of the point $(\alpha, \beta, \gamma )$ in the xy-plane is 
\begin{enumerate}
	\item $\alpha,\beta,0)$
	\item $(0,0,\gamma)$
	\item $(-\alpha,-\beta,\gamma)$
	\item $(\alpha,\beta,-\gamma)$
\end{enumerate}
\item The plane $ax+by=0$ is rotated about its line of intersection with the plane $z=0$ through an angle $\alpha.$ Prove that the equation of the plane in its new position is 
\begin{align*}
	ax+by \pm (\sqrt{a^2+b^2} \tan\alpha)z=0.
\end{align*}
\item The locus represented by $xy+yz=0$ is 
\begin{enumerate}
	\item A pair of perpendicular lines
	\item A pair of parallel lines
	\item A pair of parallel planes 
	\item A pair of perpendicular planes
\end{enumerate}
\item For what values of $a$ and $b$ the intercepts cut off on the coordinate axes by the line $ax+by+8=0$ are equal in length but opposite in signs to those cut off by the line $2x-3y=0$ on the axes.
\item If the equation of the base of an equilateral triangle is $x+y=2$ and the vertex is (2,-1), then find the length of the side of the triangle. 
\item A variable line passes through a fixed point $\vec{P}$. The algebraic sum of the perpendiculars drawn from the points (2,0), (0,2) and (1,1) on the line is zero. Find the coordinates of the point $\vec{P}$.  
\item A straight line moves so that the sum of the reciprocals of its intercepts made on axes is constant. Show that the line passes through a fixed point. 
\item If the sum of the distances of a moving point in a plane from the axes is $l$, then finds the locus of the point.  
\item $\vec{P}_1,\vec{P}_2$ are points on either of the two lines $y-\sqrt{3}\abs{x}=2$ at a distance of 5 units from their point of intersection. Find the coordinates of the root of perpendiculars drawn from $P_1, P_2$ on the bisector of the angle between the given lines.
\item If $p$ is the length of perpendicular from the origin on the lien $\frac{x}{a}+\frac{y}{b}=1$ and $a^2,p^2,b^2$ are in A.P, then show that $a^4+b^4=0$.
\item The point (4,1) undergoes the following two successive transformations :
\begin{enumerate}
\item Reflection about the line $y=x$
\item Translation through a distance 2 units along the positive $x$-axis 
\end{enumerate}
Then the final coordinates of the point are
\begin{enumerate}
\item (4,3)
\item (3,4)
\item (1,4)
\item $\frac{7}{2}$,$\frac{7}{2}$
\end{enumerate}
\item One vertex of the equilateral with centroid at the origin and one side as $x+y-2=0$ is
\begin{enumerate}
\item (-1,-1)
\item (2,2)
\item (-2-2)
\item (2,-2)
\end{enumerate}
\item If $a,b,c$ are is A.P., then the straight lines $ax+by+c=0$ will always pass through \rule{1cm}{0.15mm}.
\item The points (3,4) and (2,-6) are situated on the \rule{1cm}{0.15mm} of the line $3x-4y-8=0$.
\item A point moves so that square of its distance from the point (3,-2) is numerically equal to its distance from the line $5x-12y=3$. The equation of its locus is %\rule{1cm}{0.15mm}.
\item Locus of the mid-points of the portion of the line $x\sin\theta+y\cos\theta=p$ intercepted between the axes is \rule{1cm}{0.15mm}.

State whether the following statements are true or false. Justify.
\item If the vertices of a triangle have integral coordinates, then the triangle can not be equilateral.
\item The line $\frac{x}{a}+\frac{y}{b}=1$ moves in such a way that $\frac{1}{a^2}+\frac{1}{b^2}=\frac{1}{c^2}$, where $c$ is a constant. The locus of the foot of the perpendicular from the origin on the given line is $x^2+y^2=c^2$.
\item 
Match the following
	\begin{table}[H]
\centering
	\resizebox{\columnwidth}{!}{
\begin{matchtabular}
  The coordinates of the points P and Q on the line x + 5y = 13 which are at a distance of 2 units from the line 12x – 5y + 26 = 0 are & (3,1),(-7,11)\\
  The coordinates of the point on the line x + y = 4, which are at a unit distance from the line 4x + 3y – 10 = 0 are & $-\frac{1}{11},\frac{11}{3}$ , $\frac{4}{3},\frac{7}{3}$\\
  The coordinates of the point on the line joining A (–2, 5) and B (3, 1) such that AP = PQ = QB are & 1,$\frac{12}{5}$ , $-3,\frac{16}{5}$\\
\end{matchtabular}
		}
		\caption{}
		\label{tab:lin-misc-1}
	\end{table}
\item The value of the $\lambda$, if the lines\\$(2x+3y+4)+\lambda(6x-y+12)=0$ are
	\begin{table}[H]
\centering
	\resizebox{\columnwidth}{!}{
\begin{matchtabular}
parallel to $y$-axis is & $\lambda =-\frac{3}{4}$\\
perpendicular to $7x+y-4=0$ is & $\lambda=-\frac{1}{3}$\\
passes through (1,2) is & $\lambda=-\frac{17}{41}$\\
parallel to $x$ axis is & $\lambda=3$\\
\end{matchtabular}
		}
		\caption{}
		\label{tab:lin-misc-2}
	\end{table}
\item The equation of the line through the intersection of the lines $2x-3y=0$ and $4x-5y=2$ and
	\begin{table}[H]
\centering
	\resizebox{\columnwidth}{!}{
\begin{matchtabular}
through the point (2,1) is & $2x-y=4$\\
perpendicular to the line & $x+y-5=0$\\
parallel to the line $3x-4y+5=0$ is & $x-y-1=0$\\
equally inclined to the axes is & $3x-4y-1=0$\\
\end{matchtabular}
		}
		\caption{}
		\label{tab:lin-misc-3}
	\end{table}
\item Point $\vec{R}\brak{h, k}$ divides a line segment between the axes in the ratio 1: 2. Find the equation of the line.
\label{chapters/11/10/2/19}
	\\
	\solution 
Given
\begin{align}
	c_1 = \frac{7}{3},\,
c_2 = -6.
\end{align}
	From \eqref{eq:parallel_lines},
we need to find $c$ such that,
\begin{align}
	\abs{c-c_1} = \abs{c-c_2} \implies c = \frac{c_1+c_2}{2}
	 = -\frac{11}{6}.
\end{align}
Hence, the desired equation is
\begin{align}
	\myvec{3 & 2}\vec{x} &= -\frac{11}{6}
\end{align}
	See \figref{fig:chapters/11/10/4/21/1}.
\begin{figure}[H]
	\centering
	\includegraphics[width=0.75\columnwidth]{chapters/11/10/4/21/figs/line_plot.jpg}
	\caption{}
	\label{fig:chapters/11/10/4/21/1}
\end{figure}

\item The tangent of angle between the lines whose intercepts on the axes are $a,-b$ and $b,-a$, respectively, is
\begin{enumerate}
\item $\frac{a^2-b^2}{ab}$
\item $\frac{b^2-a^2}{2}$
\item $\frac{b^2-a^2}{2ab}$
\item none of these 
\end{enumerate}
\item Prove that the line through the point $(x_1,y_1)$ and parallel to the line $Ax+By+C=0$ is $A(x-x_1)+B(y-y_1)=0$.
\label{chapters/11/10/3/11}
\\
\solution
Given
\begin{align}
	c_1 = \frac{7}{3},\,
c_2 = -6.
\end{align}
	From \eqref{eq:parallel_lines},
we need to find $c$ such that,
\begin{align}
	\abs{c-c_1} = \abs{c-c_2} \implies c = \frac{c_1+c_2}{2}
	 = -\frac{11}{6}.
\end{align}
Hence, the desired equation is
\begin{align}
	\myvec{3 & 2}\vec{x} &= -\frac{11}{6}
\end{align}
	See \figref{fig:chapters/11/10/4/21/1}.
\begin{figure}[H]
	\centering
	\includegraphics[width=0.75\columnwidth]{chapters/11/10/4/21/figs/line_plot.jpg}
	\caption{}
	\label{fig:chapters/11/10/4/21/1}
\end{figure}

\item  If ${p}$ and ${q}$ are the lengths of perpendiculars from the origin to the lines ${x}\cos\theta - {y}\sin\theta =  {k}\cos2\theta$ and ${x}\sec\theta + {y}\cosec\theta = {k}$, respectively, prove that ${p}^2 + 4{q}^2 = {k}^2$
\label{chapters/11/10/3/16}
\\
\solution
Given
\begin{align}
	c_1 = \frac{7}{3},\,
c_2 = -6.
\end{align}
	From \eqref{eq:parallel_lines},
we need to find $c$ such that,
\begin{align}
	\abs{c-c_1} = \abs{c-c_2} \implies c = \frac{c_1+c_2}{2}
	 = -\frac{11}{6}.
\end{align}
Hence, the desired equation is
\begin{align}
	\myvec{3 & 2}\vec{x} &= -\frac{11}{6}
\end{align}
	See \figref{fig:chapters/11/10/4/21/1}.
\begin{figure}[H]
	\centering
	\includegraphics[width=0.75\columnwidth]{chapters/11/10/4/21/figs/line_plot.jpg}
	\caption{}
	\label{fig:chapters/11/10/4/21/1}
\end{figure}

\item If $p$ is the length of perpendicular from origin to the line whose intercepts on the axes are $a$ and $b$, then show that 
\begin{align}
	\frac{1}{p^2} = \frac{1}{a^2}+ \frac{1}{b^2}
\label{eq:11/10/3/18}
\end{align}
\label{chapters/11/10/3/18}
\\
\solution
	From \eqref{eq:parallel_lines}, the desired values are available in
  \tabref{tab:11/10/3/6}.
\begin{table}[H]
  \centering
  \begin{tabular}{|c|c|c|c|c|}
    \hline
    & $\vec{n}$ & $c_1$ & $c_2$ & $d$ \\
    \hline
    a) & $\myvec{15 \\ 8}$ & 34 & -31 & $\frac{65}{17}$ \\
    \hline
    b) & $\myvec{1 \\ 1}$ & $\frac{-p}{l}$ & $\frac{r}{l}$ & $\frac{\lvert p-r \rvert}{l\sqrt{2}}$ \\
    \hline
  \end{tabular}
  \caption{}
  \label{tab:11/10/3/6}
\end{table}

\item Find perpendicular distance from the origin to the line joining the points $(\cos\theta,\sin\theta)$ and $(\cos\phi,\sin\phi)$.
\\
\solution
			From \eqref{eq:parallel_lines}, the desired values are available in
  \tabref{tab:11/10/3/6}.
\begin{table}[H]
  \centering
  \begin{tabular}{|c|c|c|c|c|}
    \hline
    & $\vec{n}$ & $c_1$ & $c_2$ & $d$ \\
    \hline
    a) & $\myvec{15 \\ 8}$ & 34 & -31 & $\frac{65}{17}$ \\
    \hline
    b) & $\myvec{1 \\ 1}$ & $\frac{-p}{l}$ & $\frac{r}{l}$ & $\frac{\lvert p-r \rvert}{l\sqrt{2}}$ \\
    \hline
  \end{tabular}
  \caption{}
  \label{tab:11/10/3/6}
\end{table}

	\item Prove that the products of the lengths of the perpendiculars drawn from the points $\myvec{\sqrt{a^2-b^2}& 0}^{\top}$ and $\myvec{-\sqrt{a^2-b^2} &0}^{\top}$ to the line $\frac{x}{a} \cos{\theta} + \frac{y}{b}\sin{\theta} =1 $ is $ b^2 $.
\\
    \solution 
			From \eqref{eq:parallel_lines}, the desired values are available in
  \tabref{tab:11/10/3/6}.
\begin{table}[H]
  \centering
  \begin{tabular}{|c|c|c|c|c|}
    \hline
    & $\vec{n}$ & $c_1$ & $c_2$ & $d$ \\
    \hline
    a) & $\myvec{15 \\ 8}$ & 34 & -31 & $\frac{65}{17}$ \\
    \hline
    b) & $\myvec{1 \\ 1}$ & $\frac{-p}{l}$ & $\frac{r}{l}$ & $\frac{\lvert p-r \rvert}{l\sqrt{2}}$ \\
    \hline
  \end{tabular}
  \caption{}
  \label{tab:11/10/3/6}
\end{table}

\item O is the origin and A is $(a,b,c)$. Find the direction cosines of the line OA and the equation of the plane through A at right angle at OA.
\item Two systems of rectangular axis have the same origin. If a plane cuts them at distances $a,b,c$ and $a^{\prime},b^{\prime},c^{\prime}$, respectively, from the origin, prove that $$\frac{1}{a^2}+\frac{1}{b^2}+\frac{1}{c^2}=\frac{1}{{a^{\prime}}^2}+\frac{1}{{b^{\prime}}^2}+\frac{1}{{c^{\prime}}^2}$$.
\item Equation of the line passing through the point $(a\cos^3\theta, a\sin^3\theta)$ and perpendicular to the line $x\sec\theta+y\csc\theta=a$ is $x\cos\theta-y\sin\theta=\alpha\sin2\theta$.
\item The distance between the lines $y=mx+c$,\text{ and }$y=mx+c^2$ is
\begin{enumerate}
\item $\frac{c_1-c_2}{\sqrt{m+1}}$
\item $\frac{\abs{c_1-c_2}}{\sqrt{1+m^2}}$
\item $\frac{c^2-c^1}{\sqrt{1+m^2}}$
\item 0
\end{enumerate}
	\item Find the area of triangle formed by the lines $y-x=0, x+y=0, \text{ and } x-k=0$.
		\\
\solution
		Given
\begin{align}
	c_1 = \frac{7}{3},\,
c_2 = -6.
\end{align}
	From \eqref{eq:parallel_lines},
we need to find $c$ such that,
\begin{align}
	\abs{c-c_1} = \abs{c-c_2} \implies c = \frac{c_1+c_2}{2}
	 = -\frac{11}{6}.
\end{align}
Hence, the desired equation is
\begin{align}
	\myvec{3 & 2}\vec{x} &= -\frac{11}{6}
\end{align}
	See \figref{fig:chapters/11/10/4/21/1}.
\begin{figure}[H]
	\centering
	\includegraphics[width=0.75\columnwidth]{chapters/11/10/4/21/figs/line_plot.jpg}
	\caption{}
	\label{fig:chapters/11/10/4/21/1}
\end{figure}

\item The lines $ax+2y+1=0$, $bx=3y+1=0\text{ and }cx+4y+1=0$ are concurrent if $a$, $b$, $c$ are in G.P.
\item 
$P(a,b)$ is the mid-point of the line segment between axes. Show that the equation of the line is $\frac{x}{a}+\frac{y}{b}=2$
\label{chapters/11/10/2/18}
\\
\solution
Given
\begin{align}
	c_1 = \frac{7}{3},\,
c_2 = -6.
\end{align}
	From \eqref{eq:parallel_lines},
we need to find $c$ such that,
\begin{align}
	\abs{c-c_1} = \abs{c-c_2} \implies c = \frac{c_1+c_2}{2}
	 = -\frac{11}{6}.
\end{align}
Hence, the desired equation is
\begin{align}
	\myvec{3 & 2}\vec{x} &= -\frac{11}{6}
\end{align}
	See \figref{fig:chapters/11/10/4/21/1}.
\begin{figure}[H]
	\centering
	\includegraphics[width=0.75\columnwidth]{chapters/11/10/4/21/figs/line_plot.jpg}
	\caption{}
	\label{fig:chapters/11/10/4/21/1}
\end{figure}

\end{enumerate}

\newpage
\section{Skew Lines}
\subsection{Formulae}
 The given lines  can be written as
\begin{align}
\label{eq:chapters/12/11/2/15/lines}
\begin{split}
\vec{x} &= \myvec{-1\\-1\\-1} + \kappa_1\myvec{7\\-6\\1}\\
\vec{x} &= \myvec{3\\5\\7} + \kappa_2\myvec{1\\-2\\1} 
\end{split}
\end{align}
with
\begin{align}
\vec{A} = \myvec{-1\\-1\\-1},\, \vec{B} &= \myvec{3\\5\\7},\,
\vec{M} = \myvec{7&1\\-6&-2\\1&1}
\end{align}
%
Substituting the above in 
	    \eqref{eq:chapters/12/11/2/16/lsq/rank},
\begin{align}
\myvec{7&1&\vrule&4\\-6&-2&\vrule&6\\1&1&\vrule&8}
\xleftrightarrow[R_3 \leftarrow R_3 - \frac{1}{7}R_1]{R_2 \leftarrow R_2 + \frac{6}{7}R_1}\\
	\myvec{7&1&\vrule&4\\[1ex]0&-\frac{8}{7}&\vrule&\frac{66}{7}\\[1ex]0&\frac{6}{7}&\vrule&-\frac{52}{7}}
\xleftrightarrow{R_3 \leftarrow R_3 + \frac{3}{4}R_2}\\
\myvec{2&3&\vrule&1\\[1ex]0&-\frac{7}{2}&\vrule&\frac{1}{2}\\[1ex]0&0&\vrule&-\frac{5}{14}}
\end{align}
The rank of the matrix is 3. So the given lines are skew.
        From \eqref{eq:chapters/12/11/2/16/lsq/vec-eqn}
\begin{align}
\myvec{7&-6&1\\1&-2&1} \myvec{7&1\\-6&-2\\1&1}\bm{\kappa} &= \myvec{7&-6&1\\1&-2&1} \myvec{4\\6\\8}\\
\implies \myvec{86&20\\20&6}\bm{\kappa} &= \myvec{0\\0}
\label{eq:chapters/12/11/2/15/3}
\\
\implies \myvec{\kappa_1\\-\kappa_2} &= \myvec{0\\0}
\end{align}
From \eqref{eq:chapters/12/11/2/15/lines}, 
the closest points are $\vec{A}$  and $\vec{B}$ and  
the minimum distance between the lines is given by
\begin{align}
\norm{\vec{B}-\vec{A}} &= \norm{\myvec{4\\6\\8}}
= 2\sqrt{29}
\end{align}
%
See \figref{fig:chapters/12/11/2/15/}.
\begin{figure}[H]
\centering
\includegraphics[width=0.75\columnwidth]{chapters/12/11/2/15/figs/Figure_1.png}
\caption{}
\label{fig:chapters/12/11/2/15/}
\end{figure}


\subsection{Least Squares}
 The given lines  can be written as
\begin{align}
\label{eq:chapters/12/11/2/15/lines}
\begin{split}
\vec{x} &= \myvec{-1\\-1\\-1} + \kappa_1\myvec{7\\-6\\1}\\
\vec{x} &= \myvec{3\\5\\7} + \kappa_2\myvec{1\\-2\\1} 
\end{split}
\end{align}
with
\begin{align}
\vec{A} = \myvec{-1\\-1\\-1},\, \vec{B} &= \myvec{3\\5\\7},\,
\vec{M} = \myvec{7&1\\-6&-2\\1&1}
\end{align}
%
Substituting the above in 
	    \eqref{eq:chapters/12/11/2/16/lsq/rank},
\begin{align}
\myvec{7&1&\vrule&4\\-6&-2&\vrule&6\\1&1&\vrule&8}
\xleftrightarrow[R_3 \leftarrow R_3 - \frac{1}{7}R_1]{R_2 \leftarrow R_2 + \frac{6}{7}R_1}\\
	\myvec{7&1&\vrule&4\\[1ex]0&-\frac{8}{7}&\vrule&\frac{66}{7}\\[1ex]0&\frac{6}{7}&\vrule&-\frac{52}{7}}
\xleftrightarrow{R_3 \leftarrow R_3 + \frac{3}{4}R_2}\\
\myvec{2&3&\vrule&1\\[1ex]0&-\frac{7}{2}&\vrule&\frac{1}{2}\\[1ex]0&0&\vrule&-\frac{5}{14}}
\end{align}
The rank of the matrix is 3. So the given lines are skew.
        From \eqref{eq:chapters/12/11/2/16/lsq/vec-eqn}
\begin{align}
\myvec{7&-6&1\\1&-2&1} \myvec{7&1\\-6&-2\\1&1}\bm{\kappa} &= \myvec{7&-6&1\\1&-2&1} \myvec{4\\6\\8}\\
\implies \myvec{86&20\\20&6}\bm{\kappa} &= \myvec{0\\0}
\label{eq:chapters/12/11/2/15/3}
\\
\implies \myvec{\kappa_1\\-\kappa_2} &= \myvec{0\\0}
\end{align}
From \eqref{eq:chapters/12/11/2/15/lines}, 
the closest points are $\vec{A}$  and $\vec{B}$ and  
the minimum distance between the lines is given by
\begin{align}
\norm{\vec{B}-\vec{A}} &= \norm{\myvec{4\\6\\8}}
= 2\sqrt{29}
\end{align}
%
See \figref{fig:chapters/12/11/2/15/}.
\begin{figure}[H]
\centering
\includegraphics[width=0.75\columnwidth]{chapters/12/11/2/15/figs/Figure_1.png}
\caption{}
\label{fig:chapters/12/11/2/15/}
\end{figure}


\subsection{Singular Value Decomposition}
\input{chapters/skew/svd.tex}
\newpage
\section{Circle}
\subsection{Formulae}
In the each of the following exercises, find the coordinates of the focus, vertex, eccentricity, axis of the conic section, the equation of the directrix and the length of the latus rectum.
\begin{enumerate}[label=\thesubsection.\arabic*,ref=\thesubsection.\theenumi]
\item $y^2=12x$ 
\label{chapters/11/11/2/1}
\\
\solution
See 
\tabref{tab:std-conic-params-sol}
and 
\figref{fig:11/11/2/1Fig1}.
\begin{figure}[H]
	\begin{center}
		\includegraphics[width=0.75\columnwidth]{chapters/11/11/2/1/figs/problem1.pdf}
	\end{center}
\caption{}
\label{fig:11/11/2/1Fig1}
\end{figure}

\item 
$y^2 = –8x$
\\
\solution
See \tabref{tab:std-conic-params-sol} and 
\figref{fig:chapters/11/11/2/3/1}.
\begin{figure}[H]
\centering
\includegraphics[width=0.75\columnwidth]{chapters/11/11/2/3/figs/fig.png}
\caption{Graph}
\label{fig:chapters/11/11/2/3/1}
\end{figure}

  \item $\frac{x^2}{36}+\frac{y^2}{16}=1$
\\
\solution
See 
\tabref{tab:std-conic-params-sol}
and 
\figref{fig:chapters/11/11/3/1/Fig1}.
\begin{figure}[H]
	\begin{center}
		\includegraphics[width=0.75\columnwidth]{chapters/11/11/3/1/figs/problem1.pdf}
	\end{center}
\caption{}
\label{fig:chapters/11/11/3/1/Fig1}
\end{figure}

  \item $\frac{x^2}{16}+\frac{y^2}{9}=1$
\\
\solution
See \tabref{tab:std-conic-params-sol}
and
\figref{fig:chapters/11/11/3/3/Fig1}.
\begin{figure}[H]
	\begin{center}
		\includegraphics[width=0.75\columnwidth]{chapters/11/11/3/3/figs/conic.png}
	\end{center}
\caption{}
\label{fig:chapters/11/11/3/3/Fig1}
\end{figure}

	\item $\frac{x^2}{16}-\frac{y^2}{9} = 1$. \\ 
		\solution
		See 
\tabref{tab:std-conic-params-sol}
and
\figref{fig:11/11/4/1Fig1}.
\begin{figure}[H]
	\begin{center}
		\includegraphics[width=0.75\columnwidth]{chapters/11/11/4/1/figs/problem1.pdf}
	\end{center}
\caption{}
\label{fig:11/11/4/1Fig1}
\end{figure}

\begin{table}[H]
\centering
\caption{}
\label{tab:std-conic-params-sol}
\resizebox{\columnwidth}{!}{%
		\input{chapters/conics/tables/std.tex}
		}
\end{table}
  \item $\frac{x^2}{4}+\frac{y^2}{25}=1$
\\
\solution
From \tabref{tab:rot-conic-params-sol}, it can be seen that this is not a standard ellipse, since $\lambda_1 > \lambda_2$.  Hence $\vec{P}$ plays a role and we need to use the affine transformation
\begin{align}
\vec{x} = \vec{P}\vec{y}
\end{align}
So the value of $\lambda_1$ and $\lambda_2$ need to be interchanged for all calculations and 
in
					\eqref{eq:dx-ell-hyp},
					$\vec{e}_2$ becomes the normal vector.
See \figref{fig:chapters/11/11/3/2/Fig1}.
\begin{figure}[H]
	\begin{center} 
	    \includegraphics[width=0.75\columnwidth]{chapters/11/11/3/2/figs/ellipse}
	\end{center}
\caption{}
\label{fig:chapters/11/11/3/2/Fig1}
\end{figure}

	\item $5{y^2}-9{x^2}=36$.
		\\
		\solution
		\\
		See \tabref{tab:rot-conic-params-sol}
and 
\figref{fig:chapters/11/11/4/5/1}.
In
\tabref{tab:rot-conic-params-sol}, $\vec{P}$ shifts the negative eigenvalue 
to get the hyperbola in standard form.
\begin{figure}[H]
	\begin{center} 
	    \includegraphics[width=0.75\columnwidth]{chapters/11/11/4/5/figs/hyperbola.png}
	\end{center}
\caption{}
\label{fig:chapters/11/11/4/5/1}
\end{figure}


	\item $\frac{y^2}{9}-\frac{x^2}{27}=1$.
		\\
		\solution
		\\
		See \tabref{tab:rot-conic-params-sol}
and 
\figref{fig:chapters/11/11/4/2/Fig1}.
\begin{figure}[H]
	\begin{center} 
	    \includegraphics[width=0.75\columnwidth]{chapters/11/11/4/2/figs/hyperbola}
	\end{center}
\caption{}
\label{fig:chapters/11/11/4/2/Fig1}
\end{figure}




\item $x^2=-16y$
\\
\solution
See \tabref{tab:rot-conic-params-sol}
and 
\figref{fig:chapters/11/11/2/4/Fig1}.
\begin{figure}[H]
	\begin{center} 
	    \includegraphics[width=0.75\columnwidth]{chapters/11/11/2/4/figs/parabola}
	\end{center}
\caption{}
\label{fig:chapters/11/11/2/4/Fig1}
\end{figure}

\item $x^2$=6y 
\\
\solution
See \tabref{tab:rot-conic-params-sol}
and
\figref{fig:chapters/11/11/2/2/Fig1}.
\begin{figure}[H]
	\begin{center} 
	    \includegraphics[width=0.75\columnwidth]{chapters/11/11/2/2/figs/parabola}
	\end{center}
\caption{}
\label{fig:chapters/11/11/2/2/Fig1}
\end{figure}





\begin{table}[H]
\centering
\caption{}
\label{tab:rot-conic-params-sol}
\resizebox{\columnwidth}{!}{%
		\input{chapters/conics/tables/rot.tex}
		}
\end{table}
\item $x^2$=-9y  
  \item $\frac{x^2}{25}+\frac{y^2}{100}=1$
  \item $\frac{x^2}{49}+\frac{y^2}{36}=1$
  \item $\frac{x^2}{100}+\frac{y^2}{400}=1$
  \item $36x^2+4y^2=144$
  \item $16x^2+y^2=16$
  \item $4x^2+9y^2=36$
\item $y^2=10x$  
\end{enumerate}

In each of the following exercises, find the equation of the conic, that satisfies the given conditions.

\begin{enumerate}[label=\thesubsection.\arabic*,ref=\thesubsection.\theenumi,resume*]
\item  foci \brak{\pm 4, 0}, latus rectum of length 12.
\\
\solution
		The given information is available in 
\tabref{tab:chapters/11/11/4/13/1}.
Since two foci are given, the conic cannot be a parabola.
\begin{enumerate}
\item The direction vector of $F_1F_2$ is the normal vector of the directrix.  Hence, 
\begin{align}
\vec{n} = \vec{F_1} - \vec{F_2}
	\equiv \vec{e}_1
\end{align}
Substituting in 
  \eqref{eq:conic_quad_form_v},
\eqref{eq:conic_quad_form_u}
and
\eqref{eq:conic_quad_form_f},
\begin{align}
	\vec{V} &= \myvec{1-e^2&0\\0&1} \label{eq:chapters/11/11/4/13/6} 
	\\
	\vec{u} &= ce^2\vec{e}_1-\vec{F}
\label{eq:chapters/11/11/4/13/6/u} 
	\\
	f&=16-c^2e^2
\label{eq:chapters/11/11/4/13/6/f} 
\end{align}
\item From
\eqref{eq:chapters/11/11/4/13/6},
\begin{align}
\lambda_1 &= 1-e^2,\
\lambda_2 = 1
\label{eq:chapters/11/11/4/13/12}
\end{align}
which upon substituting
in
			\eqref{eq:latus-ellipse}, along with the value of the latus rectum 
from \tabref{tab:chapters/11/11/4/13/1}
		\begin{align}
	6\brak{1-e^2} = \sqrt{\abs{f}}
\label{eq:chapters/11/11/4/13/12/f}
\end{align}
\item  The centre of the conic is given by
\begin{align}
\vec{c} = \frac{\vec{F_1} + \vec{F_2}}{2}
= \vec{0}
\label{eq:chapters/11/11/4/13/5}
\end{align}
From \eqref{eq:chapters/11/11/4/13/6}, it is obvious that  
$\vec{V}$ is invertible.  Hence,  
from \eqref{eq:chapters/11/11/4/13/5}
and 
\eqref{eq:conic_parmas_c_def},
\begin{align}
\vec{u} = \vec{0}
	\label{eq:chapters/11/11/4/13/7/u}
\end{align}
Substituting the above in \eqref{eq:chapters/11/11/4/13/6/u}, 
\begin{align}
\vec{F} = ce^2\vec{e}_1 
\implies 
	\norm{\vec{F}} = 4 = ce^2
	\label{eq:chapters/11/11/4/13/7}
\end{align}
\item 
	From 
      \eqref{eq:f0}, 
	\eqref{eq:chapters/11/11/4/13/7/u}
and
\eqref{eq:chapters/11/11/4/13/6/f},
		\begin{align}
	36\brak{1-e^2}^2 = 16-c^2e^2
\label{eq:chapters/11/11/4/13/12/ec}
\end{align}
From
	\eqref{eq:chapters/11/11/4/13/7}
	and
\eqref{eq:chapters/11/11/4/13/12/ec}
\begin{align}
\frac{4}{e\sqrt{e^2-1}} &= 6
\\
\implies 9e^2\brak{e^2-1} &= 4\\
\implies 9e^4-9e^2-4 &= 0
\\
	\text{or, }\brak{3e^2-4}
	\brak{12e^2+1} &=0
\label{eq:chapters/11/11/4/13/14}
\end{align}
yielding
\begin{align}
e = \frac{4}{3}
\end{align}
as the only viable solution.
\end{enumerate}
The equation of the conic is then obtained as
\begin{align}
\vec{x}^\top\myvec{-\frac{1}{3}&0\\0&1}\vec{x} +4 = 0
\end{align}
See \figref{fig:chapters/11/11/4/13/1}.
\begin{figure}[H]
\centering
\includegraphics[width=0.75\columnwidth]{chapters/11/11/4/13/figs/fig1.png}
\caption{}
\label{fig:chapters/11/11/4/13/1}
\end{figure}
\begin{table}[H]
\centering
\input{chapters/11/11/4/13/tables/table1.tex}
\caption{}
\label{tab:chapters/11/11/4/13/1}
\end{table}

    \item eccentricity $e = \frac{4}{3}$,
    vertices 
    \begin{align}
        \vec{P_1} = \myvec{7\\0},\ \vec{P_2} = \myvec{-7\\0}
        \label{eq:chapters/11/11/4/14/vert}
    \end{align}
\\
\solution
		    The major axis of a conic is the chord which passes through the vertices of the conic.
    The direction vector of the major axis in this case is
    \begin{align}
        \vec{P}_2-\vec{P}_1 \equiv \vec{e}_1 = \vec{n}
\label{eq:chapters/11/11/4/13/6/n} 
    \end{align}
    which is the normal vector for the directrix.
    Since $e > 1$, the conic is a hyperbola.
Substituting  
\eqref{eq:chapters/11/11/4/13/6/n} 
in
  \eqref{eq:conic_quad_form_v},
\eqref{eq:conic_quad_form_u}
and
\eqref{eq:conic_quad_form_f},
\begin{align}
	\vec{V} &= \myvec{1-e^2&0\\0&1} = \myvec{-\frac{7}{9}&0\\0&1} \label{eq:chapters/11/11/4/14/6} 
	\\
	\vec{u} &= ce^2\vec{e}_1-\vec{F}
\label{eq:chapters/11/11/4/14/6/u} 
	\\
	f&=16-c^2e^2
\label{eq:chapters/11/11/4/14/6/f} 
\end{align}
    Thus,
    \begin{align}
        \vec{V} = \myvec{1-e^2&0\\0&1} \label{eq:chapters/11/11/4/14/V-val} \\
	    \vec{u} = ce^2\vec{e}_1 - \vec{F} \label{eq:chapters/11/11/4/14/u-val} \\
        f = \norm{\vec{F}}^2 - c^2e^2 \label{eq:chapters/11/11/4/14/f-val}
    \end{align}
    The centre of the hyperbola is 
\begin{align}
	\vec{c} = \frac{\vec{P}_1+\vec{P}_2}{2} = \vec{0} = \vec{u}
\end{align}
from \eqref{eq:conic_parmas_c_def}.      Substituting $\vec{P}_1$ and $\vec{P}_2$ in 
    \eqref{eq:conic_quad_form},
    \begin{align}
        \vec{P}_1^\top\vec{VP}_1 + 2\vec{u}^\top\vec{P}_1 + f &= 0 \label{eq:chapters/11/11/4/14/ep1} \\
        \vec{P}_2^\top\vec{VP}_2 + 2\vec{u}^\top\vec{P}_2 + f &= 0 \label{eq:chapters/11/11/4/14/ep2}
	\\
	    \implies f = \vec{P}_1^\top\vec{VP}_1  = 49\brak{e^2-1}&=\frac{343}{9}
    \end{align}
    upon adding 
    \eqref{eq:chapters/11/11/4/14/ep2} and \eqref{eq:chapters/11/11/4/14/ep1}
    and simplifying.
    Therefore, the equation of the conic is
    \begin{align}
        \vec{x}^\top\myvec{-\frac{7}{9}&0\\0&1}\vec{x} + \frac{343}{9} = 0
    \end{align}
See \figref{fig:chapters/11/11/4/14/hyperbola}.
    \begin{figure}[H]
        \centering
        \includegraphics[width=0.75\columnwidth]{chapters/11/11/4/14/figs/hyperbola.png}
        \caption{}
        \label{fig:chapters/11/11/4/14/hyperbola}
    \end{figure}

\item centre at $\vec{c}(0,0)$, major axis on the y-axis and passes through the points $\vec{P}(3,2)$ and $\vec{Q}(1,6)$.
\\
\solution
Since the major axis is along the $y$-axis,
\begin{align}
\vec{n} = \vec{e}_2
\end{align}
Thus,
\begin{align}
\vec{V} = \myvec{1&0\\0&1-e^2} \label{eq:chapters/11/11/3/19/5} 
\end{align}
Since
\begin{align}
\vec{c} = \vec{0}, \vec{u}=\vec{0}.
\label{eq:chapters/11/11/3/19/8}
\end{align}
    From \eqref{eq:conic_quad_form},
    \begin{align}
        \vec{P}^\top\vec{VP} + 2\vec{u}^\top\vec{P} + f &= 0 \label{eq:chapters/11/11/3/19/ep1} \\
        \vec{Q}^\top\vec{VQ} + 2\vec{u}^\top\vec{Q} + f &= 0 \label{eq:chapters/11/11/3/19/ep2}
    \end{align}
    yielding
\begin{align}
4e^2 - f = 13 \label{eq:chapters/11/11/3/19/10}
\\
36e^2 - f = 37 \label{eq:chapters/11/11/3/19/11}
\end{align}
which can be formulated as the matrix equation
\begin{align}
\myvec{4&-1\\36&-1}\myvec{e^2\\f} = \myvec{13\\37}
\label{eq:chapters/11/11/3/19/12}
\end{align}
The augmented matrix is given by,
\begin{align*}
\myvec{4&-1&\vline&13\\36&-1&\vline&37}
\xleftrightarrow[]{R_1\leftarrow-\frac{R_1}{8}} \myvec{4&0&\vline&3\\36&-1&\vline&37} 
\\
\xleftrightarrow[]{R_2\leftarrow R_2-9R_1}
\myvec{4&0&\vline&3\\0&-1&\vline&10} 
\xleftrightarrow[R_2\leftarrow -R_2]{R_1\leftarrow \frac{R_1}{4}}
\myvec{1&0&\vline&\frac{3}{4}\\0&1&\vline&-10}
\end{align*}
Thus,
\begin{align}
e^2 = \frac{3}{4},\ f = -10
\end{align}
and the equation of the conic is given by
\begin{align}
\vec{x}^\top\myvec{1&0\\0&\frac{1}{4}}\vec{x} - 10 = 0
\end{align}
See  
\figref{fig:chapters/11/11/3/19/1}.
\begin{figure}[H]
\centering
\includegraphics[width=0.75\columnwidth]{chapters/11/11/3/19/figs/fig1.png}
\caption{Graph}
\label{fig:chapters/11/11/3/19/1}
\end{figure}

\item major axis on the x-axis and passes through the points (4,3) and (6,2).
\\
\solution
In this case, 
    \begin{align}
        \vec{n} = \myvec{1\\0}
    \end{align}
    Thus,
    \begin{align}
        \vec{V} = \myvec{1-e^2&0\\0&1} \label{eq:chapters/11/11/3/20/V-val} \\
    \end{align}
Since
\begin{align}
\vec{c} = \vec{0}, \vec{u}=\vec{0}.
\label{eq:chapters/11/11/3/20/8}
    \end{align}
    From \eqref{eq:conic_quad_form},
    \begin{align}
        \vec{P}^\top\vec{VP} + 2\vec{u}^\top\vec{P} + f &= 0 \label{eq:chapters/11/11/3/20/ep1} \\
        \vec{Q}^\top\vec{VQ} + 2\vec{u}^\top\vec{Q} + f &= 0 \label{eq:chapters/11/11/3/20/ep2}
    \end{align}
    yielding
    \begin{align}
        16e^2 - f = 25 \label{eq:chapters/11/11/3/20/e1}
	\\
        36e^2 - f = 40 \label{eq:chapters/11/11/3/20/e2}
    \end{align}
which can be formulated as the matrix equation
    \begin{align}
        \myvec{16&-1\\36&-1}\myvec{e^2\\f} = \myvec{25\\40}
        \label{eq:chapters/11/11/3/20/mtx-eqn}
    \end{align}
    and can be solved using the augmented matrix.
    \begin{align*}
        \myvec{16&-1&25\\36&-1&40} \xleftrightarrow[]{R_1\leftarrow R_1-R_2} \myvec{-20&0&-15\\36&-1&40} \\
                 \xleftrightarrow[]{\substack{R_1\leftarrow\frac{R_1}{-5}\\R_2\leftarrow -R_2}} \myvec{4&0&3\\-36&1&-40} 
                 \xleftrightarrow[]{R_2\leftarrow R_2+9R_1}\myvec{4&0&3\\0&1&-13} \\
                 \xleftrightarrow[]{R_1\leftarrow\frac{R_1}{4}}\myvec{1&0&\frac{3}{4}\\0&1&-13}
    \end{align*}
    Thus,
    \begin{align}
        e^2 = \frac{3}{4},\ f = -13
    \end{align}
    and the equation of the conic is given by
    \begin{align}
        \vec{x}^\top\myvec{\frac{1}{4}&0\\0&1}\vec{x} - 13 = 0
    \end{align}
    See \figref{fig:chapters/11/11/3/20/ellipse}.
    \begin{figure}[H]
        \centering
        \includegraphics[width=0.75\columnwidth]{chapters/11/11/3/20/figs/ellipse.png}
        \caption{Locus of the required ellipse.}
        \label{fig:chapters/11/11/3/20/ellipse}
    \end{figure}

\item vertices $\myvec{0\\\pm 3}$ and foci $\myvec{0\\\pm5}$.
	\\
\solution
		Following the approach in the earlier problems, it is obvious that
	\begin{align}
		\vec{n} 
			= \vec{e}_2,
	\vec{c} =\vec{u}=\vec{0}.
\end{align}
Consequently,
%
\begin{align}
	\vec{V} &= \myvec{1 &0\\ 0 & 1-e^2}
	\\
	\vec{F} &= ce^2\vec{e}_2 \implies \norm{\vec{F}} = ce^2=5
\label{eq:chapters/11/11/4/9/F}
	\\
	f 
	  &= 25 - c^2 e^2
\label{eq:chapters/11/11/4/9/f}
\end{align}
%
Since the vertices are  on the conic,
\begin{align}
	\vec{v_1}^{\top}\vec{V}\vec{v_1} +2\vec{u}^{\top}\vec{v_1}+f &= 0\\
\implies 9\brak{1-e^2} + f &= 0\\
 \label{eq:chapters/11/11/4/9/1}
\end{align}
Solving \eqref{eq:chapters/11/11/4/9/1},
\eqref{eq:chapters/11/11/4/9/F}
and
\eqref{eq:chapters/11/11/4/9/f},
\begin{align}
	c = \frac{9}{5},\ 
	e = \frac{5}{3},
\end{align}
%
yielding
\begin{align}
	\vec{V} = \myvec{1&0\\0& -\frac{16}{9}} ,\
	\vec{u} = \myvec{0\\0},\
	f = 16.
\end{align}
%
Thus, the desired equation of the hyperbola is
\begin{align}
	\vec{x}^{\top} \myvec{1&0\\ 0 & -\frac{16}{9}} \vec{x} +16 =0
\end{align}
%
See
%
    \figref{fig:chapters/11/11/4/9/}.
\begin{figure}[H]
  \centering
    \includegraphics[width=0.75\columnwidth]{chapters/11/11/4/9/figs/Figure_1.png}
    \caption{Figure 1}
    \label{fig:chapters/11/11/4/9/}
\end{figure}
%




\item vertices $\brak{0,\pm 5}, \text{foci} \brak{0,\pm 8}$.  
\item focus (6,0); directrix x=-6 
\item focus (0,-3); directrix y=3
\item vertex (0,0); focus (3,0)
\item vertex (0,0); focus (-2,0) 
\item vertex (0,0) passing through (2,3) and axis is along x-axis
\item vertex (0,0) passing through (5,2) symmetric with respect to y-axis
\item vertices $(\pm5,0),\text{foci} (\pm4,0)$.
\item vertices $(\pm0,13),\text{foci} (0,\pm5)$.
\item vertices $(\pm6,0),\text{foci} (\pm4,0)$.
\item ends of major axis $(\pm3,0)$, ends of minor axis $(0,\pm2)$.
\item ends of major axis $(0,\pm \sqrt{5})$, ends of minor axis $(\pm1,0)$.
\item length of major axis 26, foci $(\pm5,0)$.
\item length of minor axis 16, foci $(0,\pm6)$.
\item foci $(\pm3,0), a=4$.
\item vertex (0,4),  focus (0,2). 
\item vertex (-3,0),  directrix $x+5=0$.
\item focus (0,-3) and directrix $y=3$.
\item  directrix x=0, focus at (6,0).
\item  vertex  at (0,4), focus at (0,2).
\item  focus at (-1,2), directrix $x-2y+3=0$.
	 \item  vertices $(\pm5,0)$, foci $(\pm 7,0)$.
	 \item vertices $(0\pm7)$ ,e =$\frac{4}{3}$. 
	 \item  foci (0,$\pm\sqrt{10})$, passing through (2,3).
\item vertices at $(0,\pm6)$,  eccentricity $\frac{5}{3}$.
\item focus (-1,-2),  directrix $x-2y+3=0$.
\item eccentricity $\frac{3}{2}$, foci $(\pm2,0)$.
 \item eccentricity $\frac{2}{3}$, latus rectum 5, centre  (0,0).
\item If the parabola $y^2=4ax$ passes through the point (3,2), then the length of its latus rectum is
\item Find the eccentricity of the hyperbola $9y^2-4x^2=36$.
	\item Equation of the hyperbola with eccentricty $\frac{3}{2}$ and foci at ($\pm2,0)$ is
\begin{enumerate} 
	\item $\frac{x^2}{4}-\frac{y^2}{5}=\frac{4}{9}$

	\item  $\frac{x^2}{9}-\frac{y^2}{9}=\frac{4}{9}$
	\item  $\frac{x^2}{4}-\frac{y^2}{9}=1$
\item  none of these.
\end{enumerate}
 \item Given the ellipse with equation $9x^2+25y^2=225,$ find the eccentricity and foci.
 \item Find the equation of the set of all points whose distance from (0,4) is $\frac{2}{3}$ of their distance from the line $y=9$.
\item The equation of the ellipse whose focus is (1,-1), directrix $x-y-3
	=0$ and eccentricity $\frac{1}{2}$ is
\begin{enumerate}
\item $7x^2+2xy+7y^2-10x+10y+7=0$
\item $7x^2+2xy+7y^2+7=0$
\item $7x^2+2xy+7y^2+10x-10y-7=0$ 
\item none
\end{enumerate}
\item The length of the latus rectum of the ellipse $3x^2+y^2=12$ is
\begin{enumerate}
\item 4
\item 3
\item 8
\item $4\sqrt{3}$
\end{enumerate}
\end{enumerate}

\subsection{Equation}
In the each of the following exercises, find the coordinates of the focus, vertex, eccentricity, axis of the conic section, the equation of the directrix and the length of the latus rectum.
\begin{enumerate}[label=\thesubsection.\arabic*,ref=\thesubsection.\theenumi]
\item $y^2=12x$ 
\label{chapters/11/11/2/1}
\\
\solution
See 
\tabref{tab:std-conic-params-sol}
and 
\figref{fig:11/11/2/1Fig1}.
\begin{figure}[H]
	\begin{center}
		\includegraphics[width=0.75\columnwidth]{chapters/11/11/2/1/figs/problem1.pdf}
	\end{center}
\caption{}
\label{fig:11/11/2/1Fig1}
\end{figure}

\item 
$y^2 = –8x$
\\
\solution
See \tabref{tab:std-conic-params-sol} and 
\figref{fig:chapters/11/11/2/3/1}.
\begin{figure}[H]
\centering
\includegraphics[width=0.75\columnwidth]{chapters/11/11/2/3/figs/fig.png}
\caption{Graph}
\label{fig:chapters/11/11/2/3/1}
\end{figure}

  \item $\frac{x^2}{36}+\frac{y^2}{16}=1$
\\
\solution
See 
\tabref{tab:std-conic-params-sol}
and 
\figref{fig:chapters/11/11/3/1/Fig1}.
\begin{figure}[H]
	\begin{center}
		\includegraphics[width=0.75\columnwidth]{chapters/11/11/3/1/figs/problem1.pdf}
	\end{center}
\caption{}
\label{fig:chapters/11/11/3/1/Fig1}
\end{figure}

  \item $\frac{x^2}{16}+\frac{y^2}{9}=1$
\\
\solution
See \tabref{tab:std-conic-params-sol}
and
\figref{fig:chapters/11/11/3/3/Fig1}.
\begin{figure}[H]
	\begin{center}
		\includegraphics[width=0.75\columnwidth]{chapters/11/11/3/3/figs/conic.png}
	\end{center}
\caption{}
\label{fig:chapters/11/11/3/3/Fig1}
\end{figure}

	\item $\frac{x^2}{16}-\frac{y^2}{9} = 1$. \\ 
		\solution
		See 
\tabref{tab:std-conic-params-sol}
and
\figref{fig:11/11/4/1Fig1}.
\begin{figure}[H]
	\begin{center}
		\includegraphics[width=0.75\columnwidth]{chapters/11/11/4/1/figs/problem1.pdf}
	\end{center}
\caption{}
\label{fig:11/11/4/1Fig1}
\end{figure}

\begin{table}[H]
\centering
\caption{}
\label{tab:std-conic-params-sol}
\resizebox{\columnwidth}{!}{%
		\input{chapters/conics/tables/std.tex}
		}
\end{table}
  \item $\frac{x^2}{4}+\frac{y^2}{25}=1$
\\
\solution
From \tabref{tab:rot-conic-params-sol}, it can be seen that this is not a standard ellipse, since $\lambda_1 > \lambda_2$.  Hence $\vec{P}$ plays a role and we need to use the affine transformation
\begin{align}
\vec{x} = \vec{P}\vec{y}
\end{align}
So the value of $\lambda_1$ and $\lambda_2$ need to be interchanged for all calculations and 
in
					\eqref{eq:dx-ell-hyp},
					$\vec{e}_2$ becomes the normal vector.
See \figref{fig:chapters/11/11/3/2/Fig1}.
\begin{figure}[H]
	\begin{center} 
	    \includegraphics[width=0.75\columnwidth]{chapters/11/11/3/2/figs/ellipse}
	\end{center}
\caption{}
\label{fig:chapters/11/11/3/2/Fig1}
\end{figure}

	\item $5{y^2}-9{x^2}=36$.
		\\
		\solution
		\\
		See \tabref{tab:rot-conic-params-sol}
and 
\figref{fig:chapters/11/11/4/5/1}.
In
\tabref{tab:rot-conic-params-sol}, $\vec{P}$ shifts the negative eigenvalue 
to get the hyperbola in standard form.
\begin{figure}[H]
	\begin{center} 
	    \includegraphics[width=0.75\columnwidth]{chapters/11/11/4/5/figs/hyperbola.png}
	\end{center}
\caption{}
\label{fig:chapters/11/11/4/5/1}
\end{figure}


	\item $\frac{y^2}{9}-\frac{x^2}{27}=1$.
		\\
		\solution
		\\
		See \tabref{tab:rot-conic-params-sol}
and 
\figref{fig:chapters/11/11/4/2/Fig1}.
\begin{figure}[H]
	\begin{center} 
	    \includegraphics[width=0.75\columnwidth]{chapters/11/11/4/2/figs/hyperbola}
	\end{center}
\caption{}
\label{fig:chapters/11/11/4/2/Fig1}
\end{figure}




\item $x^2=-16y$
\\
\solution
See \tabref{tab:rot-conic-params-sol}
and 
\figref{fig:chapters/11/11/2/4/Fig1}.
\begin{figure}[H]
	\begin{center} 
	    \includegraphics[width=0.75\columnwidth]{chapters/11/11/2/4/figs/parabola}
	\end{center}
\caption{}
\label{fig:chapters/11/11/2/4/Fig1}
\end{figure}

\item $x^2$=6y 
\\
\solution
See \tabref{tab:rot-conic-params-sol}
and
\figref{fig:chapters/11/11/2/2/Fig1}.
\begin{figure}[H]
	\begin{center} 
	    \includegraphics[width=0.75\columnwidth]{chapters/11/11/2/2/figs/parabola}
	\end{center}
\caption{}
\label{fig:chapters/11/11/2/2/Fig1}
\end{figure}





\begin{table}[H]
\centering
\caption{}
\label{tab:rot-conic-params-sol}
\resizebox{\columnwidth}{!}{%
		\input{chapters/conics/tables/rot.tex}
		}
\end{table}
\item $x^2$=-9y  
  \item $\frac{x^2}{25}+\frac{y^2}{100}=1$
  \item $\frac{x^2}{49}+\frac{y^2}{36}=1$
  \item $\frac{x^2}{100}+\frac{y^2}{400}=1$
  \item $36x^2+4y^2=144$
  \item $16x^2+y^2=16$
  \item $4x^2+9y^2=36$
\item $y^2=10x$  
\end{enumerate}

In each of the following exercises, find the equation of the conic, that satisfies the given conditions.

\begin{enumerate}[label=\thesubsection.\arabic*,ref=\thesubsection.\theenumi,resume*]
\item  foci \brak{\pm 4, 0}, latus rectum of length 12.
\\
\solution
		The given information is available in 
\tabref{tab:chapters/11/11/4/13/1}.
Since two foci are given, the conic cannot be a parabola.
\begin{enumerate}
\item The direction vector of $F_1F_2$ is the normal vector of the directrix.  Hence, 
\begin{align}
\vec{n} = \vec{F_1} - \vec{F_2}
	\equiv \vec{e}_1
\end{align}
Substituting in 
  \eqref{eq:conic_quad_form_v},
\eqref{eq:conic_quad_form_u}
and
\eqref{eq:conic_quad_form_f},
\begin{align}
	\vec{V} &= \myvec{1-e^2&0\\0&1} \label{eq:chapters/11/11/4/13/6} 
	\\
	\vec{u} &= ce^2\vec{e}_1-\vec{F}
\label{eq:chapters/11/11/4/13/6/u} 
	\\
	f&=16-c^2e^2
\label{eq:chapters/11/11/4/13/6/f} 
\end{align}
\item From
\eqref{eq:chapters/11/11/4/13/6},
\begin{align}
\lambda_1 &= 1-e^2,\
\lambda_2 = 1
\label{eq:chapters/11/11/4/13/12}
\end{align}
which upon substituting
in
			\eqref{eq:latus-ellipse}, along with the value of the latus rectum 
from \tabref{tab:chapters/11/11/4/13/1}
		\begin{align}
	6\brak{1-e^2} = \sqrt{\abs{f}}
\label{eq:chapters/11/11/4/13/12/f}
\end{align}
\item  The centre of the conic is given by
\begin{align}
\vec{c} = \frac{\vec{F_1} + \vec{F_2}}{2}
= \vec{0}
\label{eq:chapters/11/11/4/13/5}
\end{align}
From \eqref{eq:chapters/11/11/4/13/6}, it is obvious that  
$\vec{V}$ is invertible.  Hence,  
from \eqref{eq:chapters/11/11/4/13/5}
and 
\eqref{eq:conic_parmas_c_def},
\begin{align}
\vec{u} = \vec{0}
	\label{eq:chapters/11/11/4/13/7/u}
\end{align}
Substituting the above in \eqref{eq:chapters/11/11/4/13/6/u}, 
\begin{align}
\vec{F} = ce^2\vec{e}_1 
\implies 
	\norm{\vec{F}} = 4 = ce^2
	\label{eq:chapters/11/11/4/13/7}
\end{align}
\item 
	From 
      \eqref{eq:f0}, 
	\eqref{eq:chapters/11/11/4/13/7/u}
and
\eqref{eq:chapters/11/11/4/13/6/f},
		\begin{align}
	36\brak{1-e^2}^2 = 16-c^2e^2
\label{eq:chapters/11/11/4/13/12/ec}
\end{align}
From
	\eqref{eq:chapters/11/11/4/13/7}
	and
\eqref{eq:chapters/11/11/4/13/12/ec}
\begin{align}
\frac{4}{e\sqrt{e^2-1}} &= 6
\\
\implies 9e^2\brak{e^2-1} &= 4\\
\implies 9e^4-9e^2-4 &= 0
\\
	\text{or, }\brak{3e^2-4}
	\brak{12e^2+1} &=0
\label{eq:chapters/11/11/4/13/14}
\end{align}
yielding
\begin{align}
e = \frac{4}{3}
\end{align}
as the only viable solution.
\end{enumerate}
The equation of the conic is then obtained as
\begin{align}
\vec{x}^\top\myvec{-\frac{1}{3}&0\\0&1}\vec{x} +4 = 0
\end{align}
See \figref{fig:chapters/11/11/4/13/1}.
\begin{figure}[H]
\centering
\includegraphics[width=0.75\columnwidth]{chapters/11/11/4/13/figs/fig1.png}
\caption{}
\label{fig:chapters/11/11/4/13/1}
\end{figure}
\begin{table}[H]
\centering
\input{chapters/11/11/4/13/tables/table1.tex}
\caption{}
\label{tab:chapters/11/11/4/13/1}
\end{table}

    \item eccentricity $e = \frac{4}{3}$,
    vertices 
    \begin{align}
        \vec{P_1} = \myvec{7\\0},\ \vec{P_2} = \myvec{-7\\0}
        \label{eq:chapters/11/11/4/14/vert}
    \end{align}
\\
\solution
		    The major axis of a conic is the chord which passes through the vertices of the conic.
    The direction vector of the major axis in this case is
    \begin{align}
        \vec{P}_2-\vec{P}_1 \equiv \vec{e}_1 = \vec{n}
\label{eq:chapters/11/11/4/13/6/n} 
    \end{align}
    which is the normal vector for the directrix.
    Since $e > 1$, the conic is a hyperbola.
Substituting  
\eqref{eq:chapters/11/11/4/13/6/n} 
in
  \eqref{eq:conic_quad_form_v},
\eqref{eq:conic_quad_form_u}
and
\eqref{eq:conic_quad_form_f},
\begin{align}
	\vec{V} &= \myvec{1-e^2&0\\0&1} = \myvec{-\frac{7}{9}&0\\0&1} \label{eq:chapters/11/11/4/14/6} 
	\\
	\vec{u} &= ce^2\vec{e}_1-\vec{F}
\label{eq:chapters/11/11/4/14/6/u} 
	\\
	f&=16-c^2e^2
\label{eq:chapters/11/11/4/14/6/f} 
\end{align}
    Thus,
    \begin{align}
        \vec{V} = \myvec{1-e^2&0\\0&1} \label{eq:chapters/11/11/4/14/V-val} \\
	    \vec{u} = ce^2\vec{e}_1 - \vec{F} \label{eq:chapters/11/11/4/14/u-val} \\
        f = \norm{\vec{F}}^2 - c^2e^2 \label{eq:chapters/11/11/4/14/f-val}
    \end{align}
    The centre of the hyperbola is 
\begin{align}
	\vec{c} = \frac{\vec{P}_1+\vec{P}_2}{2} = \vec{0} = \vec{u}
\end{align}
from \eqref{eq:conic_parmas_c_def}.      Substituting $\vec{P}_1$ and $\vec{P}_2$ in 
    \eqref{eq:conic_quad_form},
    \begin{align}
        \vec{P}_1^\top\vec{VP}_1 + 2\vec{u}^\top\vec{P}_1 + f &= 0 \label{eq:chapters/11/11/4/14/ep1} \\
        \vec{P}_2^\top\vec{VP}_2 + 2\vec{u}^\top\vec{P}_2 + f &= 0 \label{eq:chapters/11/11/4/14/ep2}
	\\
	    \implies f = \vec{P}_1^\top\vec{VP}_1  = 49\brak{e^2-1}&=\frac{343}{9}
    \end{align}
    upon adding 
    \eqref{eq:chapters/11/11/4/14/ep2} and \eqref{eq:chapters/11/11/4/14/ep1}
    and simplifying.
    Therefore, the equation of the conic is
    \begin{align}
        \vec{x}^\top\myvec{-\frac{7}{9}&0\\0&1}\vec{x} + \frac{343}{9} = 0
    \end{align}
See \figref{fig:chapters/11/11/4/14/hyperbola}.
    \begin{figure}[H]
        \centering
        \includegraphics[width=0.75\columnwidth]{chapters/11/11/4/14/figs/hyperbola.png}
        \caption{}
        \label{fig:chapters/11/11/4/14/hyperbola}
    \end{figure}

\item centre at $\vec{c}(0,0)$, major axis on the y-axis and passes through the points $\vec{P}(3,2)$ and $\vec{Q}(1,6)$.
\\
\solution
Since the major axis is along the $y$-axis,
\begin{align}
\vec{n} = \vec{e}_2
\end{align}
Thus,
\begin{align}
\vec{V} = \myvec{1&0\\0&1-e^2} \label{eq:chapters/11/11/3/19/5} 
\end{align}
Since
\begin{align}
\vec{c} = \vec{0}, \vec{u}=\vec{0}.
\label{eq:chapters/11/11/3/19/8}
\end{align}
    From \eqref{eq:conic_quad_form},
    \begin{align}
        \vec{P}^\top\vec{VP} + 2\vec{u}^\top\vec{P} + f &= 0 \label{eq:chapters/11/11/3/19/ep1} \\
        \vec{Q}^\top\vec{VQ} + 2\vec{u}^\top\vec{Q} + f &= 0 \label{eq:chapters/11/11/3/19/ep2}
    \end{align}
    yielding
\begin{align}
4e^2 - f = 13 \label{eq:chapters/11/11/3/19/10}
\\
36e^2 - f = 37 \label{eq:chapters/11/11/3/19/11}
\end{align}
which can be formulated as the matrix equation
\begin{align}
\myvec{4&-1\\36&-1}\myvec{e^2\\f} = \myvec{13\\37}
\label{eq:chapters/11/11/3/19/12}
\end{align}
The augmented matrix is given by,
\begin{align*}
\myvec{4&-1&\vline&13\\36&-1&\vline&37}
\xleftrightarrow[]{R_1\leftarrow-\frac{R_1}{8}} \myvec{4&0&\vline&3\\36&-1&\vline&37} 
\\
\xleftrightarrow[]{R_2\leftarrow R_2-9R_1}
\myvec{4&0&\vline&3\\0&-1&\vline&10} 
\xleftrightarrow[R_2\leftarrow -R_2]{R_1\leftarrow \frac{R_1}{4}}
\myvec{1&0&\vline&\frac{3}{4}\\0&1&\vline&-10}
\end{align*}
Thus,
\begin{align}
e^2 = \frac{3}{4},\ f = -10
\end{align}
and the equation of the conic is given by
\begin{align}
\vec{x}^\top\myvec{1&0\\0&\frac{1}{4}}\vec{x} - 10 = 0
\end{align}
See  
\figref{fig:chapters/11/11/3/19/1}.
\begin{figure}[H]
\centering
\includegraphics[width=0.75\columnwidth]{chapters/11/11/3/19/figs/fig1.png}
\caption{Graph}
\label{fig:chapters/11/11/3/19/1}
\end{figure}

\item major axis on the x-axis and passes through the points (4,3) and (6,2).
\\
\solution
In this case, 
    \begin{align}
        \vec{n} = \myvec{1\\0}
    \end{align}
    Thus,
    \begin{align}
        \vec{V} = \myvec{1-e^2&0\\0&1} \label{eq:chapters/11/11/3/20/V-val} \\
    \end{align}
Since
\begin{align}
\vec{c} = \vec{0}, \vec{u}=\vec{0}.
\label{eq:chapters/11/11/3/20/8}
    \end{align}
    From \eqref{eq:conic_quad_form},
    \begin{align}
        \vec{P}^\top\vec{VP} + 2\vec{u}^\top\vec{P} + f &= 0 \label{eq:chapters/11/11/3/20/ep1} \\
        \vec{Q}^\top\vec{VQ} + 2\vec{u}^\top\vec{Q} + f &= 0 \label{eq:chapters/11/11/3/20/ep2}
    \end{align}
    yielding
    \begin{align}
        16e^2 - f = 25 \label{eq:chapters/11/11/3/20/e1}
	\\
        36e^2 - f = 40 \label{eq:chapters/11/11/3/20/e2}
    \end{align}
which can be formulated as the matrix equation
    \begin{align}
        \myvec{16&-1\\36&-1}\myvec{e^2\\f} = \myvec{25\\40}
        \label{eq:chapters/11/11/3/20/mtx-eqn}
    \end{align}
    and can be solved using the augmented matrix.
    \begin{align*}
        \myvec{16&-1&25\\36&-1&40} \xleftrightarrow[]{R_1\leftarrow R_1-R_2} \myvec{-20&0&-15\\36&-1&40} \\
                 \xleftrightarrow[]{\substack{R_1\leftarrow\frac{R_1}{-5}\\R_2\leftarrow -R_2}} \myvec{4&0&3\\-36&1&-40} 
                 \xleftrightarrow[]{R_2\leftarrow R_2+9R_1}\myvec{4&0&3\\0&1&-13} \\
                 \xleftrightarrow[]{R_1\leftarrow\frac{R_1}{4}}\myvec{1&0&\frac{3}{4}\\0&1&-13}
    \end{align*}
    Thus,
    \begin{align}
        e^2 = \frac{3}{4},\ f = -13
    \end{align}
    and the equation of the conic is given by
    \begin{align}
        \vec{x}^\top\myvec{\frac{1}{4}&0\\0&1}\vec{x} - 13 = 0
    \end{align}
    See \figref{fig:chapters/11/11/3/20/ellipse}.
    \begin{figure}[H]
        \centering
        \includegraphics[width=0.75\columnwidth]{chapters/11/11/3/20/figs/ellipse.png}
        \caption{Locus of the required ellipse.}
        \label{fig:chapters/11/11/3/20/ellipse}
    \end{figure}

\item vertices $\myvec{0\\\pm 3}$ and foci $\myvec{0\\\pm5}$.
	\\
\solution
		Following the approach in the earlier problems, it is obvious that
	\begin{align}
		\vec{n} 
			= \vec{e}_2,
	\vec{c} =\vec{u}=\vec{0}.
\end{align}
Consequently,
%
\begin{align}
	\vec{V} &= \myvec{1 &0\\ 0 & 1-e^2}
	\\
	\vec{F} &= ce^2\vec{e}_2 \implies \norm{\vec{F}} = ce^2=5
\label{eq:chapters/11/11/4/9/F}
	\\
	f 
	  &= 25 - c^2 e^2
\label{eq:chapters/11/11/4/9/f}
\end{align}
%
Since the vertices are  on the conic,
\begin{align}
	\vec{v_1}^{\top}\vec{V}\vec{v_1} +2\vec{u}^{\top}\vec{v_1}+f &= 0\\
\implies 9\brak{1-e^2} + f &= 0\\
 \label{eq:chapters/11/11/4/9/1}
\end{align}
Solving \eqref{eq:chapters/11/11/4/9/1},
\eqref{eq:chapters/11/11/4/9/F}
and
\eqref{eq:chapters/11/11/4/9/f},
\begin{align}
	c = \frac{9}{5},\ 
	e = \frac{5}{3},
\end{align}
%
yielding
\begin{align}
	\vec{V} = \myvec{1&0\\0& -\frac{16}{9}} ,\
	\vec{u} = \myvec{0\\0},\
	f = 16.
\end{align}
%
Thus, the desired equation of the hyperbola is
\begin{align}
	\vec{x}^{\top} \myvec{1&0\\ 0 & -\frac{16}{9}} \vec{x} +16 =0
\end{align}
%
See
%
    \figref{fig:chapters/11/11/4/9/}.
\begin{figure}[H]
  \centering
    \includegraphics[width=0.75\columnwidth]{chapters/11/11/4/9/figs/Figure_1.png}
    \caption{Figure 1}
    \label{fig:chapters/11/11/4/9/}
\end{figure}
%




\item vertices $\brak{0,\pm 5}, \text{foci} \brak{0,\pm 8}$.  
\item focus (6,0); directrix x=-6 
\item focus (0,-3); directrix y=3
\item vertex (0,0); focus (3,0)
\item vertex (0,0); focus (-2,0) 
\item vertex (0,0) passing through (2,3) and axis is along x-axis
\item vertex (0,0) passing through (5,2) symmetric with respect to y-axis
\item vertices $(\pm5,0),\text{foci} (\pm4,0)$.
\item vertices $(\pm0,13),\text{foci} (0,\pm5)$.
\item vertices $(\pm6,0),\text{foci} (\pm4,0)$.
\item ends of major axis $(\pm3,0)$, ends of minor axis $(0,\pm2)$.
\item ends of major axis $(0,\pm \sqrt{5})$, ends of minor axis $(\pm1,0)$.
\item length of major axis 26, foci $(\pm5,0)$.
\item length of minor axis 16, foci $(0,\pm6)$.
\item foci $(\pm3,0), a=4$.
\item vertex (0,4),  focus (0,2). 
\item vertex (-3,0),  directrix $x+5=0$.
\item focus (0,-3) and directrix $y=3$.
\item  directrix x=0, focus at (6,0).
\item  vertex  at (0,4), focus at (0,2).
\item  focus at (-1,2), directrix $x-2y+3=0$.
	 \item  vertices $(\pm5,0)$, foci $(\pm 7,0)$.
	 \item vertices $(0\pm7)$ ,e =$\frac{4}{3}$. 
	 \item  foci (0,$\pm\sqrt{10})$, passing through (2,3).
\item vertices at $(0,\pm6)$,  eccentricity $\frac{5}{3}$.
\item focus (-1,-2),  directrix $x-2y+3=0$.
\item eccentricity $\frac{3}{2}$, foci $(\pm2,0)$.
 \item eccentricity $\frac{2}{3}$, latus rectum 5, centre  (0,0).
\item If the parabola $y^2=4ax$ passes through the point (3,2), then the length of its latus rectum is
\item Find the eccentricity of the hyperbola $9y^2-4x^2=36$.
	\item Equation of the hyperbola with eccentricty $\frac{3}{2}$ and foci at ($\pm2,0)$ is
\begin{enumerate} 
	\item $\frac{x^2}{4}-\frac{y^2}{5}=\frac{4}{9}$

	\item  $\frac{x^2}{9}-\frac{y^2}{9}=\frac{4}{9}$
	\item  $\frac{x^2}{4}-\frac{y^2}{9}=1$
\item  none of these.
\end{enumerate}
 \item Given the ellipse with equation $9x^2+25y^2=225,$ find the eccentricity and foci.
 \item Find the equation of the set of all points whose distance from (0,4) is $\frac{2}{3}$ of their distance from the line $y=9$.
\item The equation of the ellipse whose focus is (1,-1), directrix $x-y-3
	=0$ and eccentricity $\frac{1}{2}$ is
\begin{enumerate}
\item $7x^2+2xy+7y^2-10x+10y+7=0$
\item $7x^2+2xy+7y^2+7=0$
\item $7x^2+2xy+7y^2+10x-10y-7=0$ 
\item none
\end{enumerate}
\item The length of the latus rectum of the ellipse $3x^2+y^2=12$ is
\begin{enumerate}
\item 4
\item 3
\item 8
\item $4\sqrt{3}$
\end{enumerate}
\end{enumerate}

\subsection{Miscellaneous}
\begin{enumerate}[label=\thesubsection.\arabic*,ref=\thesubsection.\theenumi]


\item Find the values of $k$ for which the line 
\begin{align}
(k-3)x-(4-k^2)y+k^2-7k+6=0 \label{eq:chapters/11/10/4/1/1}
\end{align}
is
\begin{enumerate}
\item Parallel to the $x$-axis
\item Parallel to the $y$-axis
\item Passing through the origin
\end{enumerate}
    \solution 
		Given
\begin{align}
	c_1 = \frac{7}{3},\,
c_2 = -6.
\end{align}
	From \eqref{eq:parallel_lines},
we need to find $c$ such that,
\begin{align}
	\abs{c-c_1} = \abs{c-c_2} \implies c = \frac{c_1+c_2}{2}
	 = -\frac{11}{6}.
\end{align}
Hence, the desired equation is
\begin{align}
	\myvec{3 & 2}\vec{x} &= -\frac{11}{6}
\end{align}
	See \figref{fig:chapters/11/10/4/21/1}.
\begin{figure}[H]
	\centering
	\includegraphics[width=0.75\columnwidth]{chapters/11/10/4/21/figs/line_plot.jpg}
	\caption{}
	\label{fig:chapters/11/10/4/21/1}
\end{figure}

	\item Find the  equations of the lines, which cutoff intercepts on the axes  whose sum and product are 1 and -6 respectively.
\\
\solution
		Let the intercepts be $a$ and  $b$. Then
\begin{align}
a+b=1,
ab=-6 \label{eq:11/10/4/32a}
\\
\implies  a = 3, b = -2
\end{align}
Thus, the possible 
intercepts are
\begin{align}
\myvec{3\\0}, \myvec{0\\-2},
\myvec{-2\\0}, \myvec{0\\3}
\end{align}
From
		\eqref{prop:lin-eq-unit-mat},
\begin{align}
	\myvec{3 & 0 \\ 0 &-2}\vec{n} = \myvec{1 \\ 1}
	\\
	\implies \vec{n} = \myvec{\frac{1}{3} \\ -\frac{1}{2}}
	\\
	\text{or, } \myvec{2 & -3}\vec{x} = 6
\end{align}
using		\eqref{prop:lin-eq-unit}.
Similarly, the other line can be obtained
as
\begin{align}
	\myvec { 3 & -2 }  \vec{x}  = -6        
\end{align}
See  
\figref{fig:11/10/4/3line segmenta}.
\begin{figure}[H]
\centering
\includegraphics[width=0.75\columnwidth]{chapters/11/10/4/3/figs/inter.png}
\caption{}
\label{fig:11/10/4/3line segmenta}
\end{figure}

\item A ray of light passing through the point $\vec{P} = \brak{1, 2}$ reflects on the x-axis at point $\vec{A}$ and the reflected ray passes through the point $\vec{Q} =\brak{5, 3}$. Find the coordinates of $\vec{A}$.
\\
    \solution 
			From \eqref{eq:11/10/4/22},
the reflection of $\vec{Q}$ is 
\begin{align}
\vec{R}  
= \myvec{5\\-3}
\end{align}
Letting
\begin{align}
\vec{A} = \myvec{x\\0},
\end{align}
since 
$\vec{P},
\vec{A},  
\vec{R}  
$
are collinear, 
		from \eqref{prop:lin-dep-rank},
\begin{align}
	\myvec{
		1 & 1 & 2 
		\\ 
		1 & 5 & -3 
		\\
		1 & x & 0 }
	\xleftrightarrow[R_3=R_3 - R_1]{R_2 = R_2 - R_1}
	\myvec{
		1 & 1 & 2 
		\\ 
		0 & 4 & -5 
		\\
		0 & x-1 & -2 }
	\\
	\xleftrightarrow[]{R_3 = 4R_3 - \brak{x-1}R_2}
	\myvec{
		1 & 1 & 2 
		\\ 
		0 & 4 & -5 
		\\
		0 & 0 & 5x-13 }
	\implies x = \frac{13}{5}
\end{align}
See  
\figref{fig:chapters/11/10/4/22/1}.
\begin{figure}[H]
\centering
\includegraphics[width=0.75\columnwidth]{chapters/11/10/4/22/figs/fig.png}
\caption{}
\label{fig:chapters/11/10/4/22/1}
\end{figure}




\item Prove that in any $\triangle{ABC}$, cos A=$\frac{b^2+c^2-a^2}{2bc}$, where a,b,c are the magnitudes of the sides opposite to the vertices A,B,C respectively.
\item Distance of the point $(\alpha, \beta, \gamma)$ from y-axis is
\begin{enumerate}
	\item $\beta$ 
	\item $\abs{\beta}$
	\item $\abs{\beta+\gamma}$
	\item $\sqrt{\alpha^2+\gamma^2}$
\end{enumerate}
\item The reflection of the point $(\alpha, \beta, \gamma )$ in the xy-plane is 
\begin{enumerate}
	\item $\alpha,\beta,0)$
	\item $(0,0,\gamma)$
	\item $(-\alpha,-\beta,\gamma)$
	\item $(\alpha,\beta,-\gamma)$
\end{enumerate}
\item The plane $ax+by=0$ is rotated about its line of intersection with the plane $z=0$ through an angle $\alpha.$ Prove that the equation of the plane in its new position is 
\begin{align*}
	ax+by \pm (\sqrt{a^2+b^2} \tan\alpha)z=0.
\end{align*}
\item The locus represented by $xy+yz=0$ is 
\begin{enumerate}
	\item A pair of perpendicular lines
	\item A pair of parallel lines
	\item A pair of parallel planes 
	\item A pair of perpendicular planes
\end{enumerate}
\item For what values of $a$ and $b$ the intercepts cut off on the coordinate axes by the line $ax+by+8=0$ are equal in length but opposite in signs to those cut off by the line $2x-3y=0$ on the axes.
\item If the equation of the base of an equilateral triangle is $x+y=2$ and the vertex is (2,-1), then find the length of the side of the triangle. 
\item A variable line passes through a fixed point $\vec{P}$. The algebraic sum of the perpendiculars drawn from the points (2,0), (0,2) and (1,1) on the line is zero. Find the coordinates of the point $\vec{P}$.  
\item A straight line moves so that the sum of the reciprocals of its intercepts made on axes is constant. Show that the line passes through a fixed point. 
\item If the sum of the distances of a moving point in a plane from the axes is $l$, then finds the locus of the point.  
\item $\vec{P}_1,\vec{P}_2$ are points on either of the two lines $y-\sqrt{3}\abs{x}=2$ at a distance of 5 units from their point of intersection. Find the coordinates of the root of perpendiculars drawn from $P_1, P_2$ on the bisector of the angle between the given lines.
\item If $p$ is the length of perpendicular from the origin on the lien $\frac{x}{a}+\frac{y}{b}=1$ and $a^2,p^2,b^2$ are in A.P, then show that $a^4+b^4=0$.
\item The point (4,1) undergoes the following two successive transformations :
\begin{enumerate}
\item Reflection about the line $y=x$
\item Translation through a distance 2 units along the positive $x$-axis 
\end{enumerate}
Then the final coordinates of the point are
\begin{enumerate}
\item (4,3)
\item (3,4)
\item (1,4)
\item $\frac{7}{2}$,$\frac{7}{2}$
\end{enumerate}
\item One vertex of the equilateral with centroid at the origin and one side as $x+y-2=0$ is
\begin{enumerate}
\item (-1,-1)
\item (2,2)
\item (-2-2)
\item (2,-2)
\end{enumerate}
\item If $a,b,c$ are is A.P., then the straight lines $ax+by+c=0$ will always pass through \rule{1cm}{0.15mm}.
\item The points (3,4) and (2,-6) are situated on the \rule{1cm}{0.15mm} of the line $3x-4y-8=0$.
\item A point moves so that square of its distance from the point (3,-2) is numerically equal to its distance from the line $5x-12y=3$. The equation of its locus is %\rule{1cm}{0.15mm}.
\item Locus of the mid-points of the portion of the line $x\sin\theta+y\cos\theta=p$ intercepted between the axes is \rule{1cm}{0.15mm}.

State whether the following statements are true or false. Justify.
\item If the vertices of a triangle have integral coordinates, then the triangle can not be equilateral.
\item The line $\frac{x}{a}+\frac{y}{b}=1$ moves in such a way that $\frac{1}{a^2}+\frac{1}{b^2}=\frac{1}{c^2}$, where $c$ is a constant. The locus of the foot of the perpendicular from the origin on the given line is $x^2+y^2=c^2$.
\item 
Match the following
	\begin{table}[H]
\centering
	\resizebox{\columnwidth}{!}{
\begin{matchtabular}
  The coordinates of the points P and Q on the line x + 5y = 13 which are at a distance of 2 units from the line 12x – 5y + 26 = 0 are & (3,1),(-7,11)\\
  The coordinates of the point on the line x + y = 4, which are at a unit distance from the line 4x + 3y – 10 = 0 are & $-\frac{1}{11},\frac{11}{3}$ , $\frac{4}{3},\frac{7}{3}$\\
  The coordinates of the point on the line joining A (–2, 5) and B (3, 1) such that AP = PQ = QB are & 1,$\frac{12}{5}$ , $-3,\frac{16}{5}$\\
\end{matchtabular}
		}
		\caption{}
		\label{tab:lin-misc-1}
	\end{table}
\item The value of the $\lambda$, if the lines\\$(2x+3y+4)+\lambda(6x-y+12)=0$ are
	\begin{table}[H]
\centering
	\resizebox{\columnwidth}{!}{
\begin{matchtabular}
parallel to $y$-axis is & $\lambda =-\frac{3}{4}$\\
perpendicular to $7x+y-4=0$ is & $\lambda=-\frac{1}{3}$\\
passes through (1,2) is & $\lambda=-\frac{17}{41}$\\
parallel to $x$ axis is & $\lambda=3$\\
\end{matchtabular}
		}
		\caption{}
		\label{tab:lin-misc-2}
	\end{table}
\item The equation of the line through the intersection of the lines $2x-3y=0$ and $4x-5y=2$ and
	\begin{table}[H]
\centering
	\resizebox{\columnwidth}{!}{
\begin{matchtabular}
through the point (2,1) is & $2x-y=4$\\
perpendicular to the line & $x+y-5=0$\\
parallel to the line $3x-4y+5=0$ is & $x-y-1=0$\\
equally inclined to the axes is & $3x-4y-1=0$\\
\end{matchtabular}
		}
		\caption{}
		\label{tab:lin-misc-3}
	\end{table}
\item Point $\vec{R}\brak{h, k}$ divides a line segment between the axes in the ratio 1: 2. Find the equation of the line.
\label{chapters/11/10/2/19}
	\\
	\solution 
Given
\begin{align}
	c_1 = \frac{7}{3},\,
c_2 = -6.
\end{align}
	From \eqref{eq:parallel_lines},
we need to find $c$ such that,
\begin{align}
	\abs{c-c_1} = \abs{c-c_2} \implies c = \frac{c_1+c_2}{2}
	 = -\frac{11}{6}.
\end{align}
Hence, the desired equation is
\begin{align}
	\myvec{3 & 2}\vec{x} &= -\frac{11}{6}
\end{align}
	See \figref{fig:chapters/11/10/4/21/1}.
\begin{figure}[H]
	\centering
	\includegraphics[width=0.75\columnwidth]{chapters/11/10/4/21/figs/line_plot.jpg}
	\caption{}
	\label{fig:chapters/11/10/4/21/1}
\end{figure}

\item The tangent of angle between the lines whose intercepts on the axes are $a,-b$ and $b,-a$, respectively, is
\begin{enumerate}
\item $\frac{a^2-b^2}{ab}$
\item $\frac{b^2-a^2}{2}$
\item $\frac{b^2-a^2}{2ab}$
\item none of these 
\end{enumerate}
\item Prove that the line through the point $(x_1,y_1)$ and parallel to the line $Ax+By+C=0$ is $A(x-x_1)+B(y-y_1)=0$.
\label{chapters/11/10/3/11}
\\
\solution
Given
\begin{align}
	c_1 = \frac{7}{3},\,
c_2 = -6.
\end{align}
	From \eqref{eq:parallel_lines},
we need to find $c$ such that,
\begin{align}
	\abs{c-c_1} = \abs{c-c_2} \implies c = \frac{c_1+c_2}{2}
	 = -\frac{11}{6}.
\end{align}
Hence, the desired equation is
\begin{align}
	\myvec{3 & 2}\vec{x} &= -\frac{11}{6}
\end{align}
	See \figref{fig:chapters/11/10/4/21/1}.
\begin{figure}[H]
	\centering
	\includegraphics[width=0.75\columnwidth]{chapters/11/10/4/21/figs/line_plot.jpg}
	\caption{}
	\label{fig:chapters/11/10/4/21/1}
\end{figure}

\item  If ${p}$ and ${q}$ are the lengths of perpendiculars from the origin to the lines ${x}\cos\theta - {y}\sin\theta =  {k}\cos2\theta$ and ${x}\sec\theta + {y}\cosec\theta = {k}$, respectively, prove that ${p}^2 + 4{q}^2 = {k}^2$
\label{chapters/11/10/3/16}
\\
\solution
Given
\begin{align}
	c_1 = \frac{7}{3},\,
c_2 = -6.
\end{align}
	From \eqref{eq:parallel_lines},
we need to find $c$ such that,
\begin{align}
	\abs{c-c_1} = \abs{c-c_2} \implies c = \frac{c_1+c_2}{2}
	 = -\frac{11}{6}.
\end{align}
Hence, the desired equation is
\begin{align}
	\myvec{3 & 2}\vec{x} &= -\frac{11}{6}
\end{align}
	See \figref{fig:chapters/11/10/4/21/1}.
\begin{figure}[H]
	\centering
	\includegraphics[width=0.75\columnwidth]{chapters/11/10/4/21/figs/line_plot.jpg}
	\caption{}
	\label{fig:chapters/11/10/4/21/1}
\end{figure}

\item If $p$ is the length of perpendicular from origin to the line whose intercepts on the axes are $a$ and $b$, then show that 
\begin{align}
	\frac{1}{p^2} = \frac{1}{a^2}+ \frac{1}{b^2}
\label{eq:11/10/3/18}
\end{align}
\label{chapters/11/10/3/18}
\\
\solution
	From \eqref{eq:parallel_lines}, the desired values are available in
  \tabref{tab:11/10/3/6}.
\begin{table}[H]
  \centering
  \begin{tabular}{|c|c|c|c|c|}
    \hline
    & $\vec{n}$ & $c_1$ & $c_2$ & $d$ \\
    \hline
    a) & $\myvec{15 \\ 8}$ & 34 & -31 & $\frac{65}{17}$ \\
    \hline
    b) & $\myvec{1 \\ 1}$ & $\frac{-p}{l}$ & $\frac{r}{l}$ & $\frac{\lvert p-r \rvert}{l\sqrt{2}}$ \\
    \hline
  \end{tabular}
  \caption{}
  \label{tab:11/10/3/6}
\end{table}

\item Find perpendicular distance from the origin to the line joining the points $(\cos\theta,\sin\theta)$ and $(\cos\phi,\sin\phi)$.
\\
\solution
			From \eqref{eq:parallel_lines}, the desired values are available in
  \tabref{tab:11/10/3/6}.
\begin{table}[H]
  \centering
  \begin{tabular}{|c|c|c|c|c|}
    \hline
    & $\vec{n}$ & $c_1$ & $c_2$ & $d$ \\
    \hline
    a) & $\myvec{15 \\ 8}$ & 34 & -31 & $\frac{65}{17}$ \\
    \hline
    b) & $\myvec{1 \\ 1}$ & $\frac{-p}{l}$ & $\frac{r}{l}$ & $\frac{\lvert p-r \rvert}{l\sqrt{2}}$ \\
    \hline
  \end{tabular}
  \caption{}
  \label{tab:11/10/3/6}
\end{table}

	\item Prove that the products of the lengths of the perpendiculars drawn from the points $\myvec{\sqrt{a^2-b^2}& 0}^{\top}$ and $\myvec{-\sqrt{a^2-b^2} &0}^{\top}$ to the line $\frac{x}{a} \cos{\theta} + \frac{y}{b}\sin{\theta} =1 $ is $ b^2 $.
\\
    \solution 
			From \eqref{eq:parallel_lines}, the desired values are available in
  \tabref{tab:11/10/3/6}.
\begin{table}[H]
  \centering
  \begin{tabular}{|c|c|c|c|c|}
    \hline
    & $\vec{n}$ & $c_1$ & $c_2$ & $d$ \\
    \hline
    a) & $\myvec{15 \\ 8}$ & 34 & -31 & $\frac{65}{17}$ \\
    \hline
    b) & $\myvec{1 \\ 1}$ & $\frac{-p}{l}$ & $\frac{r}{l}$ & $\frac{\lvert p-r \rvert}{l\sqrt{2}}$ \\
    \hline
  \end{tabular}
  \caption{}
  \label{tab:11/10/3/6}
\end{table}

\item O is the origin and A is $(a,b,c)$. Find the direction cosines of the line OA and the equation of the plane through A at right angle at OA.
\item Two systems of rectangular axis have the same origin. If a plane cuts them at distances $a,b,c$ and $a^{\prime},b^{\prime},c^{\prime}$, respectively, from the origin, prove that $$\frac{1}{a^2}+\frac{1}{b^2}+\frac{1}{c^2}=\frac{1}{{a^{\prime}}^2}+\frac{1}{{b^{\prime}}^2}+\frac{1}{{c^{\prime}}^2}$$.
\item Equation of the line passing through the point $(a\cos^3\theta, a\sin^3\theta)$ and perpendicular to the line $x\sec\theta+y\csc\theta=a$ is $x\cos\theta-y\sin\theta=\alpha\sin2\theta$.
\item The distance between the lines $y=mx+c$,\text{ and }$y=mx+c^2$ is
\begin{enumerate}
\item $\frac{c_1-c_2}{\sqrt{m+1}}$
\item $\frac{\abs{c_1-c_2}}{\sqrt{1+m^2}}$
\item $\frac{c^2-c^1}{\sqrt{1+m^2}}$
\item 0
\end{enumerate}
	\item Find the area of triangle formed by the lines $y-x=0, x+y=0, \text{ and } x-k=0$.
		\\
\solution
		Given
\begin{align}
	c_1 = \frac{7}{3},\,
c_2 = -6.
\end{align}
	From \eqref{eq:parallel_lines},
we need to find $c$ such that,
\begin{align}
	\abs{c-c_1} = \abs{c-c_2} \implies c = \frac{c_1+c_2}{2}
	 = -\frac{11}{6}.
\end{align}
Hence, the desired equation is
\begin{align}
	\myvec{3 & 2}\vec{x} &= -\frac{11}{6}
\end{align}
	See \figref{fig:chapters/11/10/4/21/1}.
\begin{figure}[H]
	\centering
	\includegraphics[width=0.75\columnwidth]{chapters/11/10/4/21/figs/line_plot.jpg}
	\caption{}
	\label{fig:chapters/11/10/4/21/1}
\end{figure}

\item The lines $ax+2y+1=0$, $bx=3y+1=0\text{ and }cx+4y+1=0$ are concurrent if $a$, $b$, $c$ are in G.P.
\item 
$P(a,b)$ is the mid-point of the line segment between axes. Show that the equation of the line is $\frac{x}{a}+\frac{y}{b}=2$
\label{chapters/11/10/2/18}
\\
\solution
Given
\begin{align}
	c_1 = \frac{7}{3},\,
c_2 = -6.
\end{align}
	From \eqref{eq:parallel_lines},
we need to find $c$ such that,
\begin{align}
	\abs{c-c_1} = \abs{c-c_2} \implies c = \frac{c_1+c_2}{2}
	 = -\frac{11}{6}.
\end{align}
Hence, the desired equation is
\begin{align}
	\myvec{3 & 2}\vec{x} &= -\frac{11}{6}
\end{align}
	See \figref{fig:chapters/11/10/4/21/1}.
\begin{figure}[H]
	\centering
	\includegraphics[width=0.75\columnwidth]{chapters/11/10/4/21/figs/line_plot.jpg}
	\caption{}
	\label{fig:chapters/11/10/4/21/1}
\end{figure}

\end{enumerate}

\newpage
\section{Conics}
\subsection{Formulae}
In the each of the following exercises, find the coordinates of the focus, vertex, eccentricity, axis of the conic section, the equation of the directrix and the length of the latus rectum.
\begin{enumerate}[label=\thesubsection.\arabic*,ref=\thesubsection.\theenumi]
\item $y^2=12x$ 
\label{chapters/11/11/2/1}
\\
\solution
See 
\tabref{tab:std-conic-params-sol}
and 
\figref{fig:11/11/2/1Fig1}.
\begin{figure}[H]
	\begin{center}
		\includegraphics[width=0.75\columnwidth]{chapters/11/11/2/1/figs/problem1.pdf}
	\end{center}
\caption{}
\label{fig:11/11/2/1Fig1}
\end{figure}

\item 
$y^2 = –8x$
\\
\solution
See \tabref{tab:std-conic-params-sol} and 
\figref{fig:chapters/11/11/2/3/1}.
\begin{figure}[H]
\centering
\includegraphics[width=0.75\columnwidth]{chapters/11/11/2/3/figs/fig.png}
\caption{Graph}
\label{fig:chapters/11/11/2/3/1}
\end{figure}

  \item $\frac{x^2}{36}+\frac{y^2}{16}=1$
\\
\solution
See 
\tabref{tab:std-conic-params-sol}
and 
\figref{fig:chapters/11/11/3/1/Fig1}.
\begin{figure}[H]
	\begin{center}
		\includegraphics[width=0.75\columnwidth]{chapters/11/11/3/1/figs/problem1.pdf}
	\end{center}
\caption{}
\label{fig:chapters/11/11/3/1/Fig1}
\end{figure}

  \item $\frac{x^2}{16}+\frac{y^2}{9}=1$
\\
\solution
See \tabref{tab:std-conic-params-sol}
and
\figref{fig:chapters/11/11/3/3/Fig1}.
\begin{figure}[H]
	\begin{center}
		\includegraphics[width=0.75\columnwidth]{chapters/11/11/3/3/figs/conic.png}
	\end{center}
\caption{}
\label{fig:chapters/11/11/3/3/Fig1}
\end{figure}

	\item $\frac{x^2}{16}-\frac{y^2}{9} = 1$. \\ 
		\solution
		See 
\tabref{tab:std-conic-params-sol}
and
\figref{fig:11/11/4/1Fig1}.
\begin{figure}[H]
	\begin{center}
		\includegraphics[width=0.75\columnwidth]{chapters/11/11/4/1/figs/problem1.pdf}
	\end{center}
\caption{}
\label{fig:11/11/4/1Fig1}
\end{figure}

\begin{table}[H]
\centering
\caption{}
\label{tab:std-conic-params-sol}
\resizebox{\columnwidth}{!}{%
		\input{chapters/conics/tables/std.tex}
		}
\end{table}
  \item $\frac{x^2}{4}+\frac{y^2}{25}=1$
\\
\solution
From \tabref{tab:rot-conic-params-sol}, it can be seen that this is not a standard ellipse, since $\lambda_1 > \lambda_2$.  Hence $\vec{P}$ plays a role and we need to use the affine transformation
\begin{align}
\vec{x} = \vec{P}\vec{y}
\end{align}
So the value of $\lambda_1$ and $\lambda_2$ need to be interchanged for all calculations and 
in
					\eqref{eq:dx-ell-hyp},
					$\vec{e}_2$ becomes the normal vector.
See \figref{fig:chapters/11/11/3/2/Fig1}.
\begin{figure}[H]
	\begin{center} 
	    \includegraphics[width=0.75\columnwidth]{chapters/11/11/3/2/figs/ellipse}
	\end{center}
\caption{}
\label{fig:chapters/11/11/3/2/Fig1}
\end{figure}

	\item $5{y^2}-9{x^2}=36$.
		\\
		\solution
		\\
		See \tabref{tab:rot-conic-params-sol}
and 
\figref{fig:chapters/11/11/4/5/1}.
In
\tabref{tab:rot-conic-params-sol}, $\vec{P}$ shifts the negative eigenvalue 
to get the hyperbola in standard form.
\begin{figure}[H]
	\begin{center} 
	    \includegraphics[width=0.75\columnwidth]{chapters/11/11/4/5/figs/hyperbola.png}
	\end{center}
\caption{}
\label{fig:chapters/11/11/4/5/1}
\end{figure}


	\item $\frac{y^2}{9}-\frac{x^2}{27}=1$.
		\\
		\solution
		\\
		See \tabref{tab:rot-conic-params-sol}
and 
\figref{fig:chapters/11/11/4/2/Fig1}.
\begin{figure}[H]
	\begin{center} 
	    \includegraphics[width=0.75\columnwidth]{chapters/11/11/4/2/figs/hyperbola}
	\end{center}
\caption{}
\label{fig:chapters/11/11/4/2/Fig1}
\end{figure}




\item $x^2=-16y$
\\
\solution
See \tabref{tab:rot-conic-params-sol}
and 
\figref{fig:chapters/11/11/2/4/Fig1}.
\begin{figure}[H]
	\begin{center} 
	    \includegraphics[width=0.75\columnwidth]{chapters/11/11/2/4/figs/parabola}
	\end{center}
\caption{}
\label{fig:chapters/11/11/2/4/Fig1}
\end{figure}

\item $x^2$=6y 
\\
\solution
See \tabref{tab:rot-conic-params-sol}
and
\figref{fig:chapters/11/11/2/2/Fig1}.
\begin{figure}[H]
	\begin{center} 
	    \includegraphics[width=0.75\columnwidth]{chapters/11/11/2/2/figs/parabola}
	\end{center}
\caption{}
\label{fig:chapters/11/11/2/2/Fig1}
\end{figure}





\begin{table}[H]
\centering
\caption{}
\label{tab:rot-conic-params-sol}
\resizebox{\columnwidth}{!}{%
		\input{chapters/conics/tables/rot.tex}
		}
\end{table}
\item $x^2$=-9y  
  \item $\frac{x^2}{25}+\frac{y^2}{100}=1$
  \item $\frac{x^2}{49}+\frac{y^2}{36}=1$
  \item $\frac{x^2}{100}+\frac{y^2}{400}=1$
  \item $36x^2+4y^2=144$
  \item $16x^2+y^2=16$
  \item $4x^2+9y^2=36$
\item $y^2=10x$  
\end{enumerate}

In each of the following exercises, find the equation of the conic, that satisfies the given conditions.

\begin{enumerate}[label=\thesubsection.\arabic*,ref=\thesubsection.\theenumi,resume*]
\item  foci \brak{\pm 4, 0}, latus rectum of length 12.
\\
\solution
		The given information is available in 
\tabref{tab:chapters/11/11/4/13/1}.
Since two foci are given, the conic cannot be a parabola.
\begin{enumerate}
\item The direction vector of $F_1F_2$ is the normal vector of the directrix.  Hence, 
\begin{align}
\vec{n} = \vec{F_1} - \vec{F_2}
	\equiv \vec{e}_1
\end{align}
Substituting in 
  \eqref{eq:conic_quad_form_v},
\eqref{eq:conic_quad_form_u}
and
\eqref{eq:conic_quad_form_f},
\begin{align}
	\vec{V} &= \myvec{1-e^2&0\\0&1} \label{eq:chapters/11/11/4/13/6} 
	\\
	\vec{u} &= ce^2\vec{e}_1-\vec{F}
\label{eq:chapters/11/11/4/13/6/u} 
	\\
	f&=16-c^2e^2
\label{eq:chapters/11/11/4/13/6/f} 
\end{align}
\item From
\eqref{eq:chapters/11/11/4/13/6},
\begin{align}
\lambda_1 &= 1-e^2,\
\lambda_2 = 1
\label{eq:chapters/11/11/4/13/12}
\end{align}
which upon substituting
in
			\eqref{eq:latus-ellipse}, along with the value of the latus rectum 
from \tabref{tab:chapters/11/11/4/13/1}
		\begin{align}
	6\brak{1-e^2} = \sqrt{\abs{f}}
\label{eq:chapters/11/11/4/13/12/f}
\end{align}
\item  The centre of the conic is given by
\begin{align}
\vec{c} = \frac{\vec{F_1} + \vec{F_2}}{2}
= \vec{0}
\label{eq:chapters/11/11/4/13/5}
\end{align}
From \eqref{eq:chapters/11/11/4/13/6}, it is obvious that  
$\vec{V}$ is invertible.  Hence,  
from \eqref{eq:chapters/11/11/4/13/5}
and 
\eqref{eq:conic_parmas_c_def},
\begin{align}
\vec{u} = \vec{0}
	\label{eq:chapters/11/11/4/13/7/u}
\end{align}
Substituting the above in \eqref{eq:chapters/11/11/4/13/6/u}, 
\begin{align}
\vec{F} = ce^2\vec{e}_1 
\implies 
	\norm{\vec{F}} = 4 = ce^2
	\label{eq:chapters/11/11/4/13/7}
\end{align}
\item 
	From 
      \eqref{eq:f0}, 
	\eqref{eq:chapters/11/11/4/13/7/u}
and
\eqref{eq:chapters/11/11/4/13/6/f},
		\begin{align}
	36\brak{1-e^2}^2 = 16-c^2e^2
\label{eq:chapters/11/11/4/13/12/ec}
\end{align}
From
	\eqref{eq:chapters/11/11/4/13/7}
	and
\eqref{eq:chapters/11/11/4/13/12/ec}
\begin{align}
\frac{4}{e\sqrt{e^2-1}} &= 6
\\
\implies 9e^2\brak{e^2-1} &= 4\\
\implies 9e^4-9e^2-4 &= 0
\\
	\text{or, }\brak{3e^2-4}
	\brak{12e^2+1} &=0
\label{eq:chapters/11/11/4/13/14}
\end{align}
yielding
\begin{align}
e = \frac{4}{3}
\end{align}
as the only viable solution.
\end{enumerate}
The equation of the conic is then obtained as
\begin{align}
\vec{x}^\top\myvec{-\frac{1}{3}&0\\0&1}\vec{x} +4 = 0
\end{align}
See \figref{fig:chapters/11/11/4/13/1}.
\begin{figure}[H]
\centering
\includegraphics[width=0.75\columnwidth]{chapters/11/11/4/13/figs/fig1.png}
\caption{}
\label{fig:chapters/11/11/4/13/1}
\end{figure}
\begin{table}[H]
\centering
\input{chapters/11/11/4/13/tables/table1.tex}
\caption{}
\label{tab:chapters/11/11/4/13/1}
\end{table}

    \item eccentricity $e = \frac{4}{3}$,
    vertices 
    \begin{align}
        \vec{P_1} = \myvec{7\\0},\ \vec{P_2} = \myvec{-7\\0}
        \label{eq:chapters/11/11/4/14/vert}
    \end{align}
\\
\solution
		    The major axis of a conic is the chord which passes through the vertices of the conic.
    The direction vector of the major axis in this case is
    \begin{align}
        \vec{P}_2-\vec{P}_1 \equiv \vec{e}_1 = \vec{n}
\label{eq:chapters/11/11/4/13/6/n} 
    \end{align}
    which is the normal vector for the directrix.
    Since $e > 1$, the conic is a hyperbola.
Substituting  
\eqref{eq:chapters/11/11/4/13/6/n} 
in
  \eqref{eq:conic_quad_form_v},
\eqref{eq:conic_quad_form_u}
and
\eqref{eq:conic_quad_form_f},
\begin{align}
	\vec{V} &= \myvec{1-e^2&0\\0&1} = \myvec{-\frac{7}{9}&0\\0&1} \label{eq:chapters/11/11/4/14/6} 
	\\
	\vec{u} &= ce^2\vec{e}_1-\vec{F}
\label{eq:chapters/11/11/4/14/6/u} 
	\\
	f&=16-c^2e^2
\label{eq:chapters/11/11/4/14/6/f} 
\end{align}
    Thus,
    \begin{align}
        \vec{V} = \myvec{1-e^2&0\\0&1} \label{eq:chapters/11/11/4/14/V-val} \\
	    \vec{u} = ce^2\vec{e}_1 - \vec{F} \label{eq:chapters/11/11/4/14/u-val} \\
        f = \norm{\vec{F}}^2 - c^2e^2 \label{eq:chapters/11/11/4/14/f-val}
    \end{align}
    The centre of the hyperbola is 
\begin{align}
	\vec{c} = \frac{\vec{P}_1+\vec{P}_2}{2} = \vec{0} = \vec{u}
\end{align}
from \eqref{eq:conic_parmas_c_def}.      Substituting $\vec{P}_1$ and $\vec{P}_2$ in 
    \eqref{eq:conic_quad_form},
    \begin{align}
        \vec{P}_1^\top\vec{VP}_1 + 2\vec{u}^\top\vec{P}_1 + f &= 0 \label{eq:chapters/11/11/4/14/ep1} \\
        \vec{P}_2^\top\vec{VP}_2 + 2\vec{u}^\top\vec{P}_2 + f &= 0 \label{eq:chapters/11/11/4/14/ep2}
	\\
	    \implies f = \vec{P}_1^\top\vec{VP}_1  = 49\brak{e^2-1}&=\frac{343}{9}
    \end{align}
    upon adding 
    \eqref{eq:chapters/11/11/4/14/ep2} and \eqref{eq:chapters/11/11/4/14/ep1}
    and simplifying.
    Therefore, the equation of the conic is
    \begin{align}
        \vec{x}^\top\myvec{-\frac{7}{9}&0\\0&1}\vec{x} + \frac{343}{9} = 0
    \end{align}
See \figref{fig:chapters/11/11/4/14/hyperbola}.
    \begin{figure}[H]
        \centering
        \includegraphics[width=0.75\columnwidth]{chapters/11/11/4/14/figs/hyperbola.png}
        \caption{}
        \label{fig:chapters/11/11/4/14/hyperbola}
    \end{figure}

\item centre at $\vec{c}(0,0)$, major axis on the y-axis and passes through the points $\vec{P}(3,2)$ and $\vec{Q}(1,6)$.
\\
\solution
Since the major axis is along the $y$-axis,
\begin{align}
\vec{n} = \vec{e}_2
\end{align}
Thus,
\begin{align}
\vec{V} = \myvec{1&0\\0&1-e^2} \label{eq:chapters/11/11/3/19/5} 
\end{align}
Since
\begin{align}
\vec{c} = \vec{0}, \vec{u}=\vec{0}.
\label{eq:chapters/11/11/3/19/8}
\end{align}
    From \eqref{eq:conic_quad_form},
    \begin{align}
        \vec{P}^\top\vec{VP} + 2\vec{u}^\top\vec{P} + f &= 0 \label{eq:chapters/11/11/3/19/ep1} \\
        \vec{Q}^\top\vec{VQ} + 2\vec{u}^\top\vec{Q} + f &= 0 \label{eq:chapters/11/11/3/19/ep2}
    \end{align}
    yielding
\begin{align}
4e^2 - f = 13 \label{eq:chapters/11/11/3/19/10}
\\
36e^2 - f = 37 \label{eq:chapters/11/11/3/19/11}
\end{align}
which can be formulated as the matrix equation
\begin{align}
\myvec{4&-1\\36&-1}\myvec{e^2\\f} = \myvec{13\\37}
\label{eq:chapters/11/11/3/19/12}
\end{align}
The augmented matrix is given by,
\begin{align*}
\myvec{4&-1&\vline&13\\36&-1&\vline&37}
\xleftrightarrow[]{R_1\leftarrow-\frac{R_1}{8}} \myvec{4&0&\vline&3\\36&-1&\vline&37} 
\\
\xleftrightarrow[]{R_2\leftarrow R_2-9R_1}
\myvec{4&0&\vline&3\\0&-1&\vline&10} 
\xleftrightarrow[R_2\leftarrow -R_2]{R_1\leftarrow \frac{R_1}{4}}
\myvec{1&0&\vline&\frac{3}{4}\\0&1&\vline&-10}
\end{align*}
Thus,
\begin{align}
e^2 = \frac{3}{4},\ f = -10
\end{align}
and the equation of the conic is given by
\begin{align}
\vec{x}^\top\myvec{1&0\\0&\frac{1}{4}}\vec{x} - 10 = 0
\end{align}
See  
\figref{fig:chapters/11/11/3/19/1}.
\begin{figure}[H]
\centering
\includegraphics[width=0.75\columnwidth]{chapters/11/11/3/19/figs/fig1.png}
\caption{Graph}
\label{fig:chapters/11/11/3/19/1}
\end{figure}

\item major axis on the x-axis and passes through the points (4,3) and (6,2).
\\
\solution
In this case, 
    \begin{align}
        \vec{n} = \myvec{1\\0}
    \end{align}
    Thus,
    \begin{align}
        \vec{V} = \myvec{1-e^2&0\\0&1} \label{eq:chapters/11/11/3/20/V-val} \\
    \end{align}
Since
\begin{align}
\vec{c} = \vec{0}, \vec{u}=\vec{0}.
\label{eq:chapters/11/11/3/20/8}
    \end{align}
    From \eqref{eq:conic_quad_form},
    \begin{align}
        \vec{P}^\top\vec{VP} + 2\vec{u}^\top\vec{P} + f &= 0 \label{eq:chapters/11/11/3/20/ep1} \\
        \vec{Q}^\top\vec{VQ} + 2\vec{u}^\top\vec{Q} + f &= 0 \label{eq:chapters/11/11/3/20/ep2}
    \end{align}
    yielding
    \begin{align}
        16e^2 - f = 25 \label{eq:chapters/11/11/3/20/e1}
	\\
        36e^2 - f = 40 \label{eq:chapters/11/11/3/20/e2}
    \end{align}
which can be formulated as the matrix equation
    \begin{align}
        \myvec{16&-1\\36&-1}\myvec{e^2\\f} = \myvec{25\\40}
        \label{eq:chapters/11/11/3/20/mtx-eqn}
    \end{align}
    and can be solved using the augmented matrix.
    \begin{align*}
        \myvec{16&-1&25\\36&-1&40} \xleftrightarrow[]{R_1\leftarrow R_1-R_2} \myvec{-20&0&-15\\36&-1&40} \\
                 \xleftrightarrow[]{\substack{R_1\leftarrow\frac{R_1}{-5}\\R_2\leftarrow -R_2}} \myvec{4&0&3\\-36&1&-40} 
                 \xleftrightarrow[]{R_2\leftarrow R_2+9R_1}\myvec{4&0&3\\0&1&-13} \\
                 \xleftrightarrow[]{R_1\leftarrow\frac{R_1}{4}}\myvec{1&0&\frac{3}{4}\\0&1&-13}
    \end{align*}
    Thus,
    \begin{align}
        e^2 = \frac{3}{4},\ f = -13
    \end{align}
    and the equation of the conic is given by
    \begin{align}
        \vec{x}^\top\myvec{\frac{1}{4}&0\\0&1}\vec{x} - 13 = 0
    \end{align}
    See \figref{fig:chapters/11/11/3/20/ellipse}.
    \begin{figure}[H]
        \centering
        \includegraphics[width=0.75\columnwidth]{chapters/11/11/3/20/figs/ellipse.png}
        \caption{Locus of the required ellipse.}
        \label{fig:chapters/11/11/3/20/ellipse}
    \end{figure}

\item vertices $\myvec{0\\\pm 3}$ and foci $\myvec{0\\\pm5}$.
	\\
\solution
		Following the approach in the earlier problems, it is obvious that
	\begin{align}
		\vec{n} 
			= \vec{e}_2,
	\vec{c} =\vec{u}=\vec{0}.
\end{align}
Consequently,
%
\begin{align}
	\vec{V} &= \myvec{1 &0\\ 0 & 1-e^2}
	\\
	\vec{F} &= ce^2\vec{e}_2 \implies \norm{\vec{F}} = ce^2=5
\label{eq:chapters/11/11/4/9/F}
	\\
	f 
	  &= 25 - c^2 e^2
\label{eq:chapters/11/11/4/9/f}
\end{align}
%
Since the vertices are  on the conic,
\begin{align}
	\vec{v_1}^{\top}\vec{V}\vec{v_1} +2\vec{u}^{\top}\vec{v_1}+f &= 0\\
\implies 9\brak{1-e^2} + f &= 0\\
 \label{eq:chapters/11/11/4/9/1}
\end{align}
Solving \eqref{eq:chapters/11/11/4/9/1},
\eqref{eq:chapters/11/11/4/9/F}
and
\eqref{eq:chapters/11/11/4/9/f},
\begin{align}
	c = \frac{9}{5},\ 
	e = \frac{5}{3},
\end{align}
%
yielding
\begin{align}
	\vec{V} = \myvec{1&0\\0& -\frac{16}{9}} ,\
	\vec{u} = \myvec{0\\0},\
	f = 16.
\end{align}
%
Thus, the desired equation of the hyperbola is
\begin{align}
	\vec{x}^{\top} \myvec{1&0\\ 0 & -\frac{16}{9}} \vec{x} +16 =0
\end{align}
%
See
%
    \figref{fig:chapters/11/11/4/9/}.
\begin{figure}[H]
  \centering
    \includegraphics[width=0.75\columnwidth]{chapters/11/11/4/9/figs/Figure_1.png}
    \caption{Figure 1}
    \label{fig:chapters/11/11/4/9/}
\end{figure}
%




\item vertices $\brak{0,\pm 5}, \text{foci} \brak{0,\pm 8}$.  
\item focus (6,0); directrix x=-6 
\item focus (0,-3); directrix y=3
\item vertex (0,0); focus (3,0)
\item vertex (0,0); focus (-2,0) 
\item vertex (0,0) passing through (2,3) and axis is along x-axis
\item vertex (0,0) passing through (5,2) symmetric with respect to y-axis
\item vertices $(\pm5,0),\text{foci} (\pm4,0)$.
\item vertices $(\pm0,13),\text{foci} (0,\pm5)$.
\item vertices $(\pm6,0),\text{foci} (\pm4,0)$.
\item ends of major axis $(\pm3,0)$, ends of minor axis $(0,\pm2)$.
\item ends of major axis $(0,\pm \sqrt{5})$, ends of minor axis $(\pm1,0)$.
\item length of major axis 26, foci $(\pm5,0)$.
\item length of minor axis 16, foci $(0,\pm6)$.
\item foci $(\pm3,0), a=4$.
\item vertex (0,4),  focus (0,2). 
\item vertex (-3,0),  directrix $x+5=0$.
\item focus (0,-3) and directrix $y=3$.
\item  directrix x=0, focus at (6,0).
\item  vertex  at (0,4), focus at (0,2).
\item  focus at (-1,2), directrix $x-2y+3=0$.
	 \item  vertices $(\pm5,0)$, foci $(\pm 7,0)$.
	 \item vertices $(0\pm7)$ ,e =$\frac{4}{3}$. 
	 \item  foci (0,$\pm\sqrt{10})$, passing through (2,3).
\item vertices at $(0,\pm6)$,  eccentricity $\frac{5}{3}$.
\item focus (-1,-2),  directrix $x-2y+3=0$.
\item eccentricity $\frac{3}{2}$, foci $(\pm2,0)$.
 \item eccentricity $\frac{2}{3}$, latus rectum 5, centre  (0,0).
\item If the parabola $y^2=4ax$ passes through the point (3,2), then the length of its latus rectum is
\item Find the eccentricity of the hyperbola $9y^2-4x^2=36$.
	\item Equation of the hyperbola with eccentricty $\frac{3}{2}$ and foci at ($\pm2,0)$ is
\begin{enumerate} 
	\item $\frac{x^2}{4}-\frac{y^2}{5}=\frac{4}{9}$

	\item  $\frac{x^2}{9}-\frac{y^2}{9}=\frac{4}{9}$
	\item  $\frac{x^2}{4}-\frac{y^2}{9}=1$
\item  none of these.
\end{enumerate}
 \item Given the ellipse with equation $9x^2+25y^2=225,$ find the eccentricity and foci.
 \item Find the equation of the set of all points whose distance from (0,4) is $\frac{2}{3}$ of their distance from the line $y=9$.
\item The equation of the ellipse whose focus is (1,-1), directrix $x-y-3
	=0$ and eccentricity $\frac{1}{2}$ is
\begin{enumerate}
\item $7x^2+2xy+7y^2-10x+10y+7=0$
\item $7x^2+2xy+7y^2+7=0$
\item $7x^2+2xy+7y^2+10x-10y-7=0$ 
\item none
\end{enumerate}
\item The length of the latus rectum of the ellipse $3x^2+y^2=12$ is
\begin{enumerate}
\item 4
\item 3
\item 8
\item $4\sqrt{3}$
\end{enumerate}
\end{enumerate}

\subsection{Equation}
In the each of the following exercises, find the coordinates of the focus, vertex, eccentricity, axis of the conic section, the equation of the directrix and the length of the latus rectum.
\begin{enumerate}[label=\thesubsection.\arabic*,ref=\thesubsection.\theenumi]
\item $y^2=12x$ 
\label{chapters/11/11/2/1}
\\
\solution
See 
\tabref{tab:std-conic-params-sol}
and 
\figref{fig:11/11/2/1Fig1}.
\begin{figure}[H]
	\begin{center}
		\includegraphics[width=0.75\columnwidth]{chapters/11/11/2/1/figs/problem1.pdf}
	\end{center}
\caption{}
\label{fig:11/11/2/1Fig1}
\end{figure}

\item 
$y^2 = –8x$
\\
\solution
See \tabref{tab:std-conic-params-sol} and 
\figref{fig:chapters/11/11/2/3/1}.
\begin{figure}[H]
\centering
\includegraphics[width=0.75\columnwidth]{chapters/11/11/2/3/figs/fig.png}
\caption{Graph}
\label{fig:chapters/11/11/2/3/1}
\end{figure}

  \item $\frac{x^2}{36}+\frac{y^2}{16}=1$
\\
\solution
See 
\tabref{tab:std-conic-params-sol}
and 
\figref{fig:chapters/11/11/3/1/Fig1}.
\begin{figure}[H]
	\begin{center}
		\includegraphics[width=0.75\columnwidth]{chapters/11/11/3/1/figs/problem1.pdf}
	\end{center}
\caption{}
\label{fig:chapters/11/11/3/1/Fig1}
\end{figure}

  \item $\frac{x^2}{16}+\frac{y^2}{9}=1$
\\
\solution
See \tabref{tab:std-conic-params-sol}
and
\figref{fig:chapters/11/11/3/3/Fig1}.
\begin{figure}[H]
	\begin{center}
		\includegraphics[width=0.75\columnwidth]{chapters/11/11/3/3/figs/conic.png}
	\end{center}
\caption{}
\label{fig:chapters/11/11/3/3/Fig1}
\end{figure}

	\item $\frac{x^2}{16}-\frac{y^2}{9} = 1$. \\ 
		\solution
		See 
\tabref{tab:std-conic-params-sol}
and
\figref{fig:11/11/4/1Fig1}.
\begin{figure}[H]
	\begin{center}
		\includegraphics[width=0.75\columnwidth]{chapters/11/11/4/1/figs/problem1.pdf}
	\end{center}
\caption{}
\label{fig:11/11/4/1Fig1}
\end{figure}

\begin{table}[H]
\centering
\caption{}
\label{tab:std-conic-params-sol}
\resizebox{\columnwidth}{!}{%
		\input{chapters/conics/tables/std.tex}
		}
\end{table}
  \item $\frac{x^2}{4}+\frac{y^2}{25}=1$
\\
\solution
From \tabref{tab:rot-conic-params-sol}, it can be seen that this is not a standard ellipse, since $\lambda_1 > \lambda_2$.  Hence $\vec{P}$ plays a role and we need to use the affine transformation
\begin{align}
\vec{x} = \vec{P}\vec{y}
\end{align}
So the value of $\lambda_1$ and $\lambda_2$ need to be interchanged for all calculations and 
in
					\eqref{eq:dx-ell-hyp},
					$\vec{e}_2$ becomes the normal vector.
See \figref{fig:chapters/11/11/3/2/Fig1}.
\begin{figure}[H]
	\begin{center} 
	    \includegraphics[width=0.75\columnwidth]{chapters/11/11/3/2/figs/ellipse}
	\end{center}
\caption{}
\label{fig:chapters/11/11/3/2/Fig1}
\end{figure}

	\item $5{y^2}-9{x^2}=36$.
		\\
		\solution
		\\
		See \tabref{tab:rot-conic-params-sol}
and 
\figref{fig:chapters/11/11/4/5/1}.
In
\tabref{tab:rot-conic-params-sol}, $\vec{P}$ shifts the negative eigenvalue 
to get the hyperbola in standard form.
\begin{figure}[H]
	\begin{center} 
	    \includegraphics[width=0.75\columnwidth]{chapters/11/11/4/5/figs/hyperbola.png}
	\end{center}
\caption{}
\label{fig:chapters/11/11/4/5/1}
\end{figure}


	\item $\frac{y^2}{9}-\frac{x^2}{27}=1$.
		\\
		\solution
		\\
		See \tabref{tab:rot-conic-params-sol}
and 
\figref{fig:chapters/11/11/4/2/Fig1}.
\begin{figure}[H]
	\begin{center} 
	    \includegraphics[width=0.75\columnwidth]{chapters/11/11/4/2/figs/hyperbola}
	\end{center}
\caption{}
\label{fig:chapters/11/11/4/2/Fig1}
\end{figure}




\item $x^2=-16y$
\\
\solution
See \tabref{tab:rot-conic-params-sol}
and 
\figref{fig:chapters/11/11/2/4/Fig1}.
\begin{figure}[H]
	\begin{center} 
	    \includegraphics[width=0.75\columnwidth]{chapters/11/11/2/4/figs/parabola}
	\end{center}
\caption{}
\label{fig:chapters/11/11/2/4/Fig1}
\end{figure}

\item $x^2$=6y 
\\
\solution
See \tabref{tab:rot-conic-params-sol}
and
\figref{fig:chapters/11/11/2/2/Fig1}.
\begin{figure}[H]
	\begin{center} 
	    \includegraphics[width=0.75\columnwidth]{chapters/11/11/2/2/figs/parabola}
	\end{center}
\caption{}
\label{fig:chapters/11/11/2/2/Fig1}
\end{figure}





\begin{table}[H]
\centering
\caption{}
\label{tab:rot-conic-params-sol}
\resizebox{\columnwidth}{!}{%
		\input{chapters/conics/tables/rot.tex}
		}
\end{table}
\item $x^2$=-9y  
  \item $\frac{x^2}{25}+\frac{y^2}{100}=1$
  \item $\frac{x^2}{49}+\frac{y^2}{36}=1$
  \item $\frac{x^2}{100}+\frac{y^2}{400}=1$
  \item $36x^2+4y^2=144$
  \item $16x^2+y^2=16$
  \item $4x^2+9y^2=36$
\item $y^2=10x$  
\end{enumerate}

In each of the following exercises, find the equation of the conic, that satisfies the given conditions.

\begin{enumerate}[label=\thesubsection.\arabic*,ref=\thesubsection.\theenumi,resume*]
\item  foci \brak{\pm 4, 0}, latus rectum of length 12.
\\
\solution
		The given information is available in 
\tabref{tab:chapters/11/11/4/13/1}.
Since two foci are given, the conic cannot be a parabola.
\begin{enumerate}
\item The direction vector of $F_1F_2$ is the normal vector of the directrix.  Hence, 
\begin{align}
\vec{n} = \vec{F_1} - \vec{F_2}
	\equiv \vec{e}_1
\end{align}
Substituting in 
  \eqref{eq:conic_quad_form_v},
\eqref{eq:conic_quad_form_u}
and
\eqref{eq:conic_quad_form_f},
\begin{align}
	\vec{V} &= \myvec{1-e^2&0\\0&1} \label{eq:chapters/11/11/4/13/6} 
	\\
	\vec{u} &= ce^2\vec{e}_1-\vec{F}
\label{eq:chapters/11/11/4/13/6/u} 
	\\
	f&=16-c^2e^2
\label{eq:chapters/11/11/4/13/6/f} 
\end{align}
\item From
\eqref{eq:chapters/11/11/4/13/6},
\begin{align}
\lambda_1 &= 1-e^2,\
\lambda_2 = 1
\label{eq:chapters/11/11/4/13/12}
\end{align}
which upon substituting
in
			\eqref{eq:latus-ellipse}, along with the value of the latus rectum 
from \tabref{tab:chapters/11/11/4/13/1}
		\begin{align}
	6\brak{1-e^2} = \sqrt{\abs{f}}
\label{eq:chapters/11/11/4/13/12/f}
\end{align}
\item  The centre of the conic is given by
\begin{align}
\vec{c} = \frac{\vec{F_1} + \vec{F_2}}{2}
= \vec{0}
\label{eq:chapters/11/11/4/13/5}
\end{align}
From \eqref{eq:chapters/11/11/4/13/6}, it is obvious that  
$\vec{V}$ is invertible.  Hence,  
from \eqref{eq:chapters/11/11/4/13/5}
and 
\eqref{eq:conic_parmas_c_def},
\begin{align}
\vec{u} = \vec{0}
	\label{eq:chapters/11/11/4/13/7/u}
\end{align}
Substituting the above in \eqref{eq:chapters/11/11/4/13/6/u}, 
\begin{align}
\vec{F} = ce^2\vec{e}_1 
\implies 
	\norm{\vec{F}} = 4 = ce^2
	\label{eq:chapters/11/11/4/13/7}
\end{align}
\item 
	From 
      \eqref{eq:f0}, 
	\eqref{eq:chapters/11/11/4/13/7/u}
and
\eqref{eq:chapters/11/11/4/13/6/f},
		\begin{align}
	36\brak{1-e^2}^2 = 16-c^2e^2
\label{eq:chapters/11/11/4/13/12/ec}
\end{align}
From
	\eqref{eq:chapters/11/11/4/13/7}
	and
\eqref{eq:chapters/11/11/4/13/12/ec}
\begin{align}
\frac{4}{e\sqrt{e^2-1}} &= 6
\\
\implies 9e^2\brak{e^2-1} &= 4\\
\implies 9e^4-9e^2-4 &= 0
\\
	\text{or, }\brak{3e^2-4}
	\brak{12e^2+1} &=0
\label{eq:chapters/11/11/4/13/14}
\end{align}
yielding
\begin{align}
e = \frac{4}{3}
\end{align}
as the only viable solution.
\end{enumerate}
The equation of the conic is then obtained as
\begin{align}
\vec{x}^\top\myvec{-\frac{1}{3}&0\\0&1}\vec{x} +4 = 0
\end{align}
See \figref{fig:chapters/11/11/4/13/1}.
\begin{figure}[H]
\centering
\includegraphics[width=0.75\columnwidth]{chapters/11/11/4/13/figs/fig1.png}
\caption{}
\label{fig:chapters/11/11/4/13/1}
\end{figure}
\begin{table}[H]
\centering
\input{chapters/11/11/4/13/tables/table1.tex}
\caption{}
\label{tab:chapters/11/11/4/13/1}
\end{table}

    \item eccentricity $e = \frac{4}{3}$,
    vertices 
    \begin{align}
        \vec{P_1} = \myvec{7\\0},\ \vec{P_2} = \myvec{-7\\0}
        \label{eq:chapters/11/11/4/14/vert}
    \end{align}
\\
\solution
		    The major axis of a conic is the chord which passes through the vertices of the conic.
    The direction vector of the major axis in this case is
    \begin{align}
        \vec{P}_2-\vec{P}_1 \equiv \vec{e}_1 = \vec{n}
\label{eq:chapters/11/11/4/13/6/n} 
    \end{align}
    which is the normal vector for the directrix.
    Since $e > 1$, the conic is a hyperbola.
Substituting  
\eqref{eq:chapters/11/11/4/13/6/n} 
in
  \eqref{eq:conic_quad_form_v},
\eqref{eq:conic_quad_form_u}
and
\eqref{eq:conic_quad_form_f},
\begin{align}
	\vec{V} &= \myvec{1-e^2&0\\0&1} = \myvec{-\frac{7}{9}&0\\0&1} \label{eq:chapters/11/11/4/14/6} 
	\\
	\vec{u} &= ce^2\vec{e}_1-\vec{F}
\label{eq:chapters/11/11/4/14/6/u} 
	\\
	f&=16-c^2e^2
\label{eq:chapters/11/11/4/14/6/f} 
\end{align}
    Thus,
    \begin{align}
        \vec{V} = \myvec{1-e^2&0\\0&1} \label{eq:chapters/11/11/4/14/V-val} \\
	    \vec{u} = ce^2\vec{e}_1 - \vec{F} \label{eq:chapters/11/11/4/14/u-val} \\
        f = \norm{\vec{F}}^2 - c^2e^2 \label{eq:chapters/11/11/4/14/f-val}
    \end{align}
    The centre of the hyperbola is 
\begin{align}
	\vec{c} = \frac{\vec{P}_1+\vec{P}_2}{2} = \vec{0} = \vec{u}
\end{align}
from \eqref{eq:conic_parmas_c_def}.      Substituting $\vec{P}_1$ and $\vec{P}_2$ in 
    \eqref{eq:conic_quad_form},
    \begin{align}
        \vec{P}_1^\top\vec{VP}_1 + 2\vec{u}^\top\vec{P}_1 + f &= 0 \label{eq:chapters/11/11/4/14/ep1} \\
        \vec{P}_2^\top\vec{VP}_2 + 2\vec{u}^\top\vec{P}_2 + f &= 0 \label{eq:chapters/11/11/4/14/ep2}
	\\
	    \implies f = \vec{P}_1^\top\vec{VP}_1  = 49\brak{e^2-1}&=\frac{343}{9}
    \end{align}
    upon adding 
    \eqref{eq:chapters/11/11/4/14/ep2} and \eqref{eq:chapters/11/11/4/14/ep1}
    and simplifying.
    Therefore, the equation of the conic is
    \begin{align}
        \vec{x}^\top\myvec{-\frac{7}{9}&0\\0&1}\vec{x} + \frac{343}{9} = 0
    \end{align}
See \figref{fig:chapters/11/11/4/14/hyperbola}.
    \begin{figure}[H]
        \centering
        \includegraphics[width=0.75\columnwidth]{chapters/11/11/4/14/figs/hyperbola.png}
        \caption{}
        \label{fig:chapters/11/11/4/14/hyperbola}
    \end{figure}

\item centre at $\vec{c}(0,0)$, major axis on the y-axis and passes through the points $\vec{P}(3,2)$ and $\vec{Q}(1,6)$.
\\
\solution
Since the major axis is along the $y$-axis,
\begin{align}
\vec{n} = \vec{e}_2
\end{align}
Thus,
\begin{align}
\vec{V} = \myvec{1&0\\0&1-e^2} \label{eq:chapters/11/11/3/19/5} 
\end{align}
Since
\begin{align}
\vec{c} = \vec{0}, \vec{u}=\vec{0}.
\label{eq:chapters/11/11/3/19/8}
\end{align}
    From \eqref{eq:conic_quad_form},
    \begin{align}
        \vec{P}^\top\vec{VP} + 2\vec{u}^\top\vec{P} + f &= 0 \label{eq:chapters/11/11/3/19/ep1} \\
        \vec{Q}^\top\vec{VQ} + 2\vec{u}^\top\vec{Q} + f &= 0 \label{eq:chapters/11/11/3/19/ep2}
    \end{align}
    yielding
\begin{align}
4e^2 - f = 13 \label{eq:chapters/11/11/3/19/10}
\\
36e^2 - f = 37 \label{eq:chapters/11/11/3/19/11}
\end{align}
which can be formulated as the matrix equation
\begin{align}
\myvec{4&-1\\36&-1}\myvec{e^2\\f} = \myvec{13\\37}
\label{eq:chapters/11/11/3/19/12}
\end{align}
The augmented matrix is given by,
\begin{align*}
\myvec{4&-1&\vline&13\\36&-1&\vline&37}
\xleftrightarrow[]{R_1\leftarrow-\frac{R_1}{8}} \myvec{4&0&\vline&3\\36&-1&\vline&37} 
\\
\xleftrightarrow[]{R_2\leftarrow R_2-9R_1}
\myvec{4&0&\vline&3\\0&-1&\vline&10} 
\xleftrightarrow[R_2\leftarrow -R_2]{R_1\leftarrow \frac{R_1}{4}}
\myvec{1&0&\vline&\frac{3}{4}\\0&1&\vline&-10}
\end{align*}
Thus,
\begin{align}
e^2 = \frac{3}{4},\ f = -10
\end{align}
and the equation of the conic is given by
\begin{align}
\vec{x}^\top\myvec{1&0\\0&\frac{1}{4}}\vec{x} - 10 = 0
\end{align}
See  
\figref{fig:chapters/11/11/3/19/1}.
\begin{figure}[H]
\centering
\includegraphics[width=0.75\columnwidth]{chapters/11/11/3/19/figs/fig1.png}
\caption{Graph}
\label{fig:chapters/11/11/3/19/1}
\end{figure}

\item major axis on the x-axis and passes through the points (4,3) and (6,2).
\\
\solution
In this case, 
    \begin{align}
        \vec{n} = \myvec{1\\0}
    \end{align}
    Thus,
    \begin{align}
        \vec{V} = \myvec{1-e^2&0\\0&1} \label{eq:chapters/11/11/3/20/V-val} \\
    \end{align}
Since
\begin{align}
\vec{c} = \vec{0}, \vec{u}=\vec{0}.
\label{eq:chapters/11/11/3/20/8}
    \end{align}
    From \eqref{eq:conic_quad_form},
    \begin{align}
        \vec{P}^\top\vec{VP} + 2\vec{u}^\top\vec{P} + f &= 0 \label{eq:chapters/11/11/3/20/ep1} \\
        \vec{Q}^\top\vec{VQ} + 2\vec{u}^\top\vec{Q} + f &= 0 \label{eq:chapters/11/11/3/20/ep2}
    \end{align}
    yielding
    \begin{align}
        16e^2 - f = 25 \label{eq:chapters/11/11/3/20/e1}
	\\
        36e^2 - f = 40 \label{eq:chapters/11/11/3/20/e2}
    \end{align}
which can be formulated as the matrix equation
    \begin{align}
        \myvec{16&-1\\36&-1}\myvec{e^2\\f} = \myvec{25\\40}
        \label{eq:chapters/11/11/3/20/mtx-eqn}
    \end{align}
    and can be solved using the augmented matrix.
    \begin{align*}
        \myvec{16&-1&25\\36&-1&40} \xleftrightarrow[]{R_1\leftarrow R_1-R_2} \myvec{-20&0&-15\\36&-1&40} \\
                 \xleftrightarrow[]{\substack{R_1\leftarrow\frac{R_1}{-5}\\R_2\leftarrow -R_2}} \myvec{4&0&3\\-36&1&-40} 
                 \xleftrightarrow[]{R_2\leftarrow R_2+9R_1}\myvec{4&0&3\\0&1&-13} \\
                 \xleftrightarrow[]{R_1\leftarrow\frac{R_1}{4}}\myvec{1&0&\frac{3}{4}\\0&1&-13}
    \end{align*}
    Thus,
    \begin{align}
        e^2 = \frac{3}{4},\ f = -13
    \end{align}
    and the equation of the conic is given by
    \begin{align}
        \vec{x}^\top\myvec{\frac{1}{4}&0\\0&1}\vec{x} - 13 = 0
    \end{align}
    See \figref{fig:chapters/11/11/3/20/ellipse}.
    \begin{figure}[H]
        \centering
        \includegraphics[width=0.75\columnwidth]{chapters/11/11/3/20/figs/ellipse.png}
        \caption{Locus of the required ellipse.}
        \label{fig:chapters/11/11/3/20/ellipse}
    \end{figure}

\item vertices $\myvec{0\\\pm 3}$ and foci $\myvec{0\\\pm5}$.
	\\
\solution
		Following the approach in the earlier problems, it is obvious that
	\begin{align}
		\vec{n} 
			= \vec{e}_2,
	\vec{c} =\vec{u}=\vec{0}.
\end{align}
Consequently,
%
\begin{align}
	\vec{V} &= \myvec{1 &0\\ 0 & 1-e^2}
	\\
	\vec{F} &= ce^2\vec{e}_2 \implies \norm{\vec{F}} = ce^2=5
\label{eq:chapters/11/11/4/9/F}
	\\
	f 
	  &= 25 - c^2 e^2
\label{eq:chapters/11/11/4/9/f}
\end{align}
%
Since the vertices are  on the conic,
\begin{align}
	\vec{v_1}^{\top}\vec{V}\vec{v_1} +2\vec{u}^{\top}\vec{v_1}+f &= 0\\
\implies 9\brak{1-e^2} + f &= 0\\
 \label{eq:chapters/11/11/4/9/1}
\end{align}
Solving \eqref{eq:chapters/11/11/4/9/1},
\eqref{eq:chapters/11/11/4/9/F}
and
\eqref{eq:chapters/11/11/4/9/f},
\begin{align}
	c = \frac{9}{5},\ 
	e = \frac{5}{3},
\end{align}
%
yielding
\begin{align}
	\vec{V} = \myvec{1&0\\0& -\frac{16}{9}} ,\
	\vec{u} = \myvec{0\\0},\
	f = 16.
\end{align}
%
Thus, the desired equation of the hyperbola is
\begin{align}
	\vec{x}^{\top} \myvec{1&0\\ 0 & -\frac{16}{9}} \vec{x} +16 =0
\end{align}
%
See
%
    \figref{fig:chapters/11/11/4/9/}.
\begin{figure}[H]
  \centering
    \includegraphics[width=0.75\columnwidth]{chapters/11/11/4/9/figs/Figure_1.png}
    \caption{Figure 1}
    \label{fig:chapters/11/11/4/9/}
\end{figure}
%




\item vertices $\brak{0,\pm 5}, \text{foci} \brak{0,\pm 8}$.  
\item focus (6,0); directrix x=-6 
\item focus (0,-3); directrix y=3
\item vertex (0,0); focus (3,0)
\item vertex (0,0); focus (-2,0) 
\item vertex (0,0) passing through (2,3) and axis is along x-axis
\item vertex (0,0) passing through (5,2) symmetric with respect to y-axis
\item vertices $(\pm5,0),\text{foci} (\pm4,0)$.
\item vertices $(\pm0,13),\text{foci} (0,\pm5)$.
\item vertices $(\pm6,0),\text{foci} (\pm4,0)$.
\item ends of major axis $(\pm3,0)$, ends of minor axis $(0,\pm2)$.
\item ends of major axis $(0,\pm \sqrt{5})$, ends of minor axis $(\pm1,0)$.
\item length of major axis 26, foci $(\pm5,0)$.
\item length of minor axis 16, foci $(0,\pm6)$.
\item foci $(\pm3,0), a=4$.
\item vertex (0,4),  focus (0,2). 
\item vertex (-3,0),  directrix $x+5=0$.
\item focus (0,-3) and directrix $y=3$.
\item  directrix x=0, focus at (6,0).
\item  vertex  at (0,4), focus at (0,2).
\item  focus at (-1,2), directrix $x-2y+3=0$.
	 \item  vertices $(\pm5,0)$, foci $(\pm 7,0)$.
	 \item vertices $(0\pm7)$ ,e =$\frac{4}{3}$. 
	 \item  foci (0,$\pm\sqrt{10})$, passing through (2,3).
\item vertices at $(0,\pm6)$,  eccentricity $\frac{5}{3}$.
\item focus (-1,-2),  directrix $x-2y+3=0$.
\item eccentricity $\frac{3}{2}$, foci $(\pm2,0)$.
 \item eccentricity $\frac{2}{3}$, latus rectum 5, centre  (0,0).
\item If the parabola $y^2=4ax$ passes through the point (3,2), then the length of its latus rectum is
\item Find the eccentricity of the hyperbola $9y^2-4x^2=36$.
	\item Equation of the hyperbola with eccentricty $\frac{3}{2}$ and foci at ($\pm2,0)$ is
\begin{enumerate} 
	\item $\frac{x^2}{4}-\frac{y^2}{5}=\frac{4}{9}$

	\item  $\frac{x^2}{9}-\frac{y^2}{9}=\frac{4}{9}$
	\item  $\frac{x^2}{4}-\frac{y^2}{9}=1$
\item  none of these.
\end{enumerate}
 \item Given the ellipse with equation $9x^2+25y^2=225,$ find the eccentricity and foci.
 \item Find the equation of the set of all points whose distance from (0,4) is $\frac{2}{3}$ of their distance from the line $y=9$.
\item The equation of the ellipse whose focus is (1,-1), directrix $x-y-3
	=0$ and eccentricity $\frac{1}{2}$ is
\begin{enumerate}
\item $7x^2+2xy+7y^2-10x+10y+7=0$
\item $7x^2+2xy+7y^2+7=0$
\item $7x^2+2xy+7y^2+10x-10y-7=0$ 
\item none
\end{enumerate}
\item The length of the latus rectum of the ellipse $3x^2+y^2=12$ is
\begin{enumerate}
\item 4
\item 3
\item 8
\item $4\sqrt{3}$
\end{enumerate}
\end{enumerate}

\subsection{Miscellaneous}
\begin{enumerate}[label=\thesubsection.\arabic*,ref=\thesubsection.\theenumi]


\item Find the values of $k$ for which the line 
\begin{align}
(k-3)x-(4-k^2)y+k^2-7k+6=0 \label{eq:chapters/11/10/4/1/1}
\end{align}
is
\begin{enumerate}
\item Parallel to the $x$-axis
\item Parallel to the $y$-axis
\item Passing through the origin
\end{enumerate}
    \solution 
		Given
\begin{align}
	c_1 = \frac{7}{3},\,
c_2 = -6.
\end{align}
	From \eqref{eq:parallel_lines},
we need to find $c$ such that,
\begin{align}
	\abs{c-c_1} = \abs{c-c_2} \implies c = \frac{c_1+c_2}{2}
	 = -\frac{11}{6}.
\end{align}
Hence, the desired equation is
\begin{align}
	\myvec{3 & 2}\vec{x} &= -\frac{11}{6}
\end{align}
	See \figref{fig:chapters/11/10/4/21/1}.
\begin{figure}[H]
	\centering
	\includegraphics[width=0.75\columnwidth]{chapters/11/10/4/21/figs/line_plot.jpg}
	\caption{}
	\label{fig:chapters/11/10/4/21/1}
\end{figure}

	\item Find the  equations of the lines, which cutoff intercepts on the axes  whose sum and product are 1 and -6 respectively.
\\
\solution
		Let the intercepts be $a$ and  $b$. Then
\begin{align}
a+b=1,
ab=-6 \label{eq:11/10/4/32a}
\\
\implies  a = 3, b = -2
\end{align}
Thus, the possible 
intercepts are
\begin{align}
\myvec{3\\0}, \myvec{0\\-2},
\myvec{-2\\0}, \myvec{0\\3}
\end{align}
From
		\eqref{prop:lin-eq-unit-mat},
\begin{align}
	\myvec{3 & 0 \\ 0 &-2}\vec{n} = \myvec{1 \\ 1}
	\\
	\implies \vec{n} = \myvec{\frac{1}{3} \\ -\frac{1}{2}}
	\\
	\text{or, } \myvec{2 & -3}\vec{x} = 6
\end{align}
using		\eqref{prop:lin-eq-unit}.
Similarly, the other line can be obtained
as
\begin{align}
	\myvec { 3 & -2 }  \vec{x}  = -6        
\end{align}
See  
\figref{fig:11/10/4/3line segmenta}.
\begin{figure}[H]
\centering
\includegraphics[width=0.75\columnwidth]{chapters/11/10/4/3/figs/inter.png}
\caption{}
\label{fig:11/10/4/3line segmenta}
\end{figure}

\item A ray of light passing through the point $\vec{P} = \brak{1, 2}$ reflects on the x-axis at point $\vec{A}$ and the reflected ray passes through the point $\vec{Q} =\brak{5, 3}$. Find the coordinates of $\vec{A}$.
\\
    \solution 
			From \eqref{eq:11/10/4/22},
the reflection of $\vec{Q}$ is 
\begin{align}
\vec{R}  
= \myvec{5\\-3}
\end{align}
Letting
\begin{align}
\vec{A} = \myvec{x\\0},
\end{align}
since 
$\vec{P},
\vec{A},  
\vec{R}  
$
are collinear, 
		from \eqref{prop:lin-dep-rank},
\begin{align}
	\myvec{
		1 & 1 & 2 
		\\ 
		1 & 5 & -3 
		\\
		1 & x & 0 }
	\xleftrightarrow[R_3=R_3 - R_1]{R_2 = R_2 - R_1}
	\myvec{
		1 & 1 & 2 
		\\ 
		0 & 4 & -5 
		\\
		0 & x-1 & -2 }
	\\
	\xleftrightarrow[]{R_3 = 4R_3 - \brak{x-1}R_2}
	\myvec{
		1 & 1 & 2 
		\\ 
		0 & 4 & -5 
		\\
		0 & 0 & 5x-13 }
	\implies x = \frac{13}{5}
\end{align}
See  
\figref{fig:chapters/11/10/4/22/1}.
\begin{figure}[H]
\centering
\includegraphics[width=0.75\columnwidth]{chapters/11/10/4/22/figs/fig.png}
\caption{}
\label{fig:chapters/11/10/4/22/1}
\end{figure}




\item Prove that in any $\triangle{ABC}$, cos A=$\frac{b^2+c^2-a^2}{2bc}$, where a,b,c are the magnitudes of the sides opposite to the vertices A,B,C respectively.
\item Distance of the point $(\alpha, \beta, \gamma)$ from y-axis is
\begin{enumerate}
	\item $\beta$ 
	\item $\abs{\beta}$
	\item $\abs{\beta+\gamma}$
	\item $\sqrt{\alpha^2+\gamma^2}$
\end{enumerate}
\item The reflection of the point $(\alpha, \beta, \gamma )$ in the xy-plane is 
\begin{enumerate}
	\item $\alpha,\beta,0)$
	\item $(0,0,\gamma)$
	\item $(-\alpha,-\beta,\gamma)$
	\item $(\alpha,\beta,-\gamma)$
\end{enumerate}
\item The plane $ax+by=0$ is rotated about its line of intersection with the plane $z=0$ through an angle $\alpha.$ Prove that the equation of the plane in its new position is 
\begin{align*}
	ax+by \pm (\sqrt{a^2+b^2} \tan\alpha)z=0.
\end{align*}
\item The locus represented by $xy+yz=0$ is 
\begin{enumerate}
	\item A pair of perpendicular lines
	\item A pair of parallel lines
	\item A pair of parallel planes 
	\item A pair of perpendicular planes
\end{enumerate}
\item For what values of $a$ and $b$ the intercepts cut off on the coordinate axes by the line $ax+by+8=0$ are equal in length but opposite in signs to those cut off by the line $2x-3y=0$ on the axes.
\item If the equation of the base of an equilateral triangle is $x+y=2$ and the vertex is (2,-1), then find the length of the side of the triangle. 
\item A variable line passes through a fixed point $\vec{P}$. The algebraic sum of the perpendiculars drawn from the points (2,0), (0,2) and (1,1) on the line is zero. Find the coordinates of the point $\vec{P}$.  
\item A straight line moves so that the sum of the reciprocals of its intercepts made on axes is constant. Show that the line passes through a fixed point. 
\item If the sum of the distances of a moving point in a plane from the axes is $l$, then finds the locus of the point.  
\item $\vec{P}_1,\vec{P}_2$ are points on either of the two lines $y-\sqrt{3}\abs{x}=2$ at a distance of 5 units from their point of intersection. Find the coordinates of the root of perpendiculars drawn from $P_1, P_2$ on the bisector of the angle between the given lines.
\item If $p$ is the length of perpendicular from the origin on the lien $\frac{x}{a}+\frac{y}{b}=1$ and $a^2,p^2,b^2$ are in A.P, then show that $a^4+b^4=0$.
\item The point (4,1) undergoes the following two successive transformations :
\begin{enumerate}
\item Reflection about the line $y=x$
\item Translation through a distance 2 units along the positive $x$-axis 
\end{enumerate}
Then the final coordinates of the point are
\begin{enumerate}
\item (4,3)
\item (3,4)
\item (1,4)
\item $\frac{7}{2}$,$\frac{7}{2}$
\end{enumerate}
\item One vertex of the equilateral with centroid at the origin and one side as $x+y-2=0$ is
\begin{enumerate}
\item (-1,-1)
\item (2,2)
\item (-2-2)
\item (2,-2)
\end{enumerate}
\item If $a,b,c$ are is A.P., then the straight lines $ax+by+c=0$ will always pass through \rule{1cm}{0.15mm}.
\item The points (3,4) and (2,-6) are situated on the \rule{1cm}{0.15mm} of the line $3x-4y-8=0$.
\item A point moves so that square of its distance from the point (3,-2) is numerically equal to its distance from the line $5x-12y=3$. The equation of its locus is %\rule{1cm}{0.15mm}.
\item Locus of the mid-points of the portion of the line $x\sin\theta+y\cos\theta=p$ intercepted between the axes is \rule{1cm}{0.15mm}.

State whether the following statements are true or false. Justify.
\item If the vertices of a triangle have integral coordinates, then the triangle can not be equilateral.
\item The line $\frac{x}{a}+\frac{y}{b}=1$ moves in such a way that $\frac{1}{a^2}+\frac{1}{b^2}=\frac{1}{c^2}$, where $c$ is a constant. The locus of the foot of the perpendicular from the origin on the given line is $x^2+y^2=c^2$.
\item 
Match the following
	\begin{table}[H]
\centering
	\resizebox{\columnwidth}{!}{
\begin{matchtabular}
  The coordinates of the points P and Q on the line x + 5y = 13 which are at a distance of 2 units from the line 12x – 5y + 26 = 0 are & (3,1),(-7,11)\\
  The coordinates of the point on the line x + y = 4, which are at a unit distance from the line 4x + 3y – 10 = 0 are & $-\frac{1}{11},\frac{11}{3}$ , $\frac{4}{3},\frac{7}{3}$\\
  The coordinates of the point on the line joining A (–2, 5) and B (3, 1) such that AP = PQ = QB are & 1,$\frac{12}{5}$ , $-3,\frac{16}{5}$\\
\end{matchtabular}
		}
		\caption{}
		\label{tab:lin-misc-1}
	\end{table}
\item The value of the $\lambda$, if the lines\\$(2x+3y+4)+\lambda(6x-y+12)=0$ are
	\begin{table}[H]
\centering
	\resizebox{\columnwidth}{!}{
\begin{matchtabular}
parallel to $y$-axis is & $\lambda =-\frac{3}{4}$\\
perpendicular to $7x+y-4=0$ is & $\lambda=-\frac{1}{3}$\\
passes through (1,2) is & $\lambda=-\frac{17}{41}$\\
parallel to $x$ axis is & $\lambda=3$\\
\end{matchtabular}
		}
		\caption{}
		\label{tab:lin-misc-2}
	\end{table}
\item The equation of the line through the intersection of the lines $2x-3y=0$ and $4x-5y=2$ and
	\begin{table}[H]
\centering
	\resizebox{\columnwidth}{!}{
\begin{matchtabular}
through the point (2,1) is & $2x-y=4$\\
perpendicular to the line & $x+y-5=0$\\
parallel to the line $3x-4y+5=0$ is & $x-y-1=0$\\
equally inclined to the axes is & $3x-4y-1=0$\\
\end{matchtabular}
		}
		\caption{}
		\label{tab:lin-misc-3}
	\end{table}
\item Point $\vec{R}\brak{h, k}$ divides a line segment between the axes in the ratio 1: 2. Find the equation of the line.
\label{chapters/11/10/2/19}
	\\
	\solution 
Given
\begin{align}
	c_1 = \frac{7}{3},\,
c_2 = -6.
\end{align}
	From \eqref{eq:parallel_lines},
we need to find $c$ such that,
\begin{align}
	\abs{c-c_1} = \abs{c-c_2} \implies c = \frac{c_1+c_2}{2}
	 = -\frac{11}{6}.
\end{align}
Hence, the desired equation is
\begin{align}
	\myvec{3 & 2}\vec{x} &= -\frac{11}{6}
\end{align}
	See \figref{fig:chapters/11/10/4/21/1}.
\begin{figure}[H]
	\centering
	\includegraphics[width=0.75\columnwidth]{chapters/11/10/4/21/figs/line_plot.jpg}
	\caption{}
	\label{fig:chapters/11/10/4/21/1}
\end{figure}

\item The tangent of angle between the lines whose intercepts on the axes are $a,-b$ and $b,-a$, respectively, is
\begin{enumerate}
\item $\frac{a^2-b^2}{ab}$
\item $\frac{b^2-a^2}{2}$
\item $\frac{b^2-a^2}{2ab}$
\item none of these 
\end{enumerate}
\item Prove that the line through the point $(x_1,y_1)$ and parallel to the line $Ax+By+C=0$ is $A(x-x_1)+B(y-y_1)=0$.
\label{chapters/11/10/3/11}
\\
\solution
Given
\begin{align}
	c_1 = \frac{7}{3},\,
c_2 = -6.
\end{align}
	From \eqref{eq:parallel_lines},
we need to find $c$ such that,
\begin{align}
	\abs{c-c_1} = \abs{c-c_2} \implies c = \frac{c_1+c_2}{2}
	 = -\frac{11}{6}.
\end{align}
Hence, the desired equation is
\begin{align}
	\myvec{3 & 2}\vec{x} &= -\frac{11}{6}
\end{align}
	See \figref{fig:chapters/11/10/4/21/1}.
\begin{figure}[H]
	\centering
	\includegraphics[width=0.75\columnwidth]{chapters/11/10/4/21/figs/line_plot.jpg}
	\caption{}
	\label{fig:chapters/11/10/4/21/1}
\end{figure}

\item  If ${p}$ and ${q}$ are the lengths of perpendiculars from the origin to the lines ${x}\cos\theta - {y}\sin\theta =  {k}\cos2\theta$ and ${x}\sec\theta + {y}\cosec\theta = {k}$, respectively, prove that ${p}^2 + 4{q}^2 = {k}^2$
\label{chapters/11/10/3/16}
\\
\solution
Given
\begin{align}
	c_1 = \frac{7}{3},\,
c_2 = -6.
\end{align}
	From \eqref{eq:parallel_lines},
we need to find $c$ such that,
\begin{align}
	\abs{c-c_1} = \abs{c-c_2} \implies c = \frac{c_1+c_2}{2}
	 = -\frac{11}{6}.
\end{align}
Hence, the desired equation is
\begin{align}
	\myvec{3 & 2}\vec{x} &= -\frac{11}{6}
\end{align}
	See \figref{fig:chapters/11/10/4/21/1}.
\begin{figure}[H]
	\centering
	\includegraphics[width=0.75\columnwidth]{chapters/11/10/4/21/figs/line_plot.jpg}
	\caption{}
	\label{fig:chapters/11/10/4/21/1}
\end{figure}

\item If $p$ is the length of perpendicular from origin to the line whose intercepts on the axes are $a$ and $b$, then show that 
\begin{align}
	\frac{1}{p^2} = \frac{1}{a^2}+ \frac{1}{b^2}
\label{eq:11/10/3/18}
\end{align}
\label{chapters/11/10/3/18}
\\
\solution
	From \eqref{eq:parallel_lines}, the desired values are available in
  \tabref{tab:11/10/3/6}.
\begin{table}[H]
  \centering
  \begin{tabular}{|c|c|c|c|c|}
    \hline
    & $\vec{n}$ & $c_1$ & $c_2$ & $d$ \\
    \hline
    a) & $\myvec{15 \\ 8}$ & 34 & -31 & $\frac{65}{17}$ \\
    \hline
    b) & $\myvec{1 \\ 1}$ & $\frac{-p}{l}$ & $\frac{r}{l}$ & $\frac{\lvert p-r \rvert}{l\sqrt{2}}$ \\
    \hline
  \end{tabular}
  \caption{}
  \label{tab:11/10/3/6}
\end{table}

\item Find perpendicular distance from the origin to the line joining the points $(\cos\theta,\sin\theta)$ and $(\cos\phi,\sin\phi)$.
\\
\solution
			From \eqref{eq:parallel_lines}, the desired values are available in
  \tabref{tab:11/10/3/6}.
\begin{table}[H]
  \centering
  \begin{tabular}{|c|c|c|c|c|}
    \hline
    & $\vec{n}$ & $c_1$ & $c_2$ & $d$ \\
    \hline
    a) & $\myvec{15 \\ 8}$ & 34 & -31 & $\frac{65}{17}$ \\
    \hline
    b) & $\myvec{1 \\ 1}$ & $\frac{-p}{l}$ & $\frac{r}{l}$ & $\frac{\lvert p-r \rvert}{l\sqrt{2}}$ \\
    \hline
  \end{tabular}
  \caption{}
  \label{tab:11/10/3/6}
\end{table}

	\item Prove that the products of the lengths of the perpendiculars drawn from the points $\myvec{\sqrt{a^2-b^2}& 0}^{\top}$ and $\myvec{-\sqrt{a^2-b^2} &0}^{\top}$ to the line $\frac{x}{a} \cos{\theta} + \frac{y}{b}\sin{\theta} =1 $ is $ b^2 $.
\\
    \solution 
			From \eqref{eq:parallel_lines}, the desired values are available in
  \tabref{tab:11/10/3/6}.
\begin{table}[H]
  \centering
  \begin{tabular}{|c|c|c|c|c|}
    \hline
    & $\vec{n}$ & $c_1$ & $c_2$ & $d$ \\
    \hline
    a) & $\myvec{15 \\ 8}$ & 34 & -31 & $\frac{65}{17}$ \\
    \hline
    b) & $\myvec{1 \\ 1}$ & $\frac{-p}{l}$ & $\frac{r}{l}$ & $\frac{\lvert p-r \rvert}{l\sqrt{2}}$ \\
    \hline
  \end{tabular}
  \caption{}
  \label{tab:11/10/3/6}
\end{table}

\item O is the origin and A is $(a,b,c)$. Find the direction cosines of the line OA and the equation of the plane through A at right angle at OA.
\item Two systems of rectangular axis have the same origin. If a plane cuts them at distances $a,b,c$ and $a^{\prime},b^{\prime},c^{\prime}$, respectively, from the origin, prove that $$\frac{1}{a^2}+\frac{1}{b^2}+\frac{1}{c^2}=\frac{1}{{a^{\prime}}^2}+\frac{1}{{b^{\prime}}^2}+\frac{1}{{c^{\prime}}^2}$$.
\item Equation of the line passing through the point $(a\cos^3\theta, a\sin^3\theta)$ and perpendicular to the line $x\sec\theta+y\csc\theta=a$ is $x\cos\theta-y\sin\theta=\alpha\sin2\theta$.
\item The distance between the lines $y=mx+c$,\text{ and }$y=mx+c^2$ is
\begin{enumerate}
\item $\frac{c_1-c_2}{\sqrt{m+1}}$
\item $\frac{\abs{c_1-c_2}}{\sqrt{1+m^2}}$
\item $\frac{c^2-c^1}{\sqrt{1+m^2}}$
\item 0
\end{enumerate}
	\item Find the area of triangle formed by the lines $y-x=0, x+y=0, \text{ and } x-k=0$.
		\\
\solution
		Given
\begin{align}
	c_1 = \frac{7}{3},\,
c_2 = -6.
\end{align}
	From \eqref{eq:parallel_lines},
we need to find $c$ such that,
\begin{align}
	\abs{c-c_1} = \abs{c-c_2} \implies c = \frac{c_1+c_2}{2}
	 = -\frac{11}{6}.
\end{align}
Hence, the desired equation is
\begin{align}
	\myvec{3 & 2}\vec{x} &= -\frac{11}{6}
\end{align}
	See \figref{fig:chapters/11/10/4/21/1}.
\begin{figure}[H]
	\centering
	\includegraphics[width=0.75\columnwidth]{chapters/11/10/4/21/figs/line_plot.jpg}
	\caption{}
	\label{fig:chapters/11/10/4/21/1}
\end{figure}

\item The lines $ax+2y+1=0$, $bx=3y+1=0\text{ and }cx+4y+1=0$ are concurrent if $a$, $b$, $c$ are in G.P.
\item 
$P(a,b)$ is the mid-point of the line segment between axes. Show that the equation of the line is $\frac{x}{a}+\frac{y}{b}=2$
\label{chapters/11/10/2/18}
\\
\solution
Given
\begin{align}
	c_1 = \frac{7}{3},\,
c_2 = -6.
\end{align}
	From \eqref{eq:parallel_lines},
we need to find $c$ such that,
\begin{align}
	\abs{c-c_1} = \abs{c-c_2} \implies c = \frac{c_1+c_2}{2}
	 = -\frac{11}{6}.
\end{align}
Hence, the desired equation is
\begin{align}
	\myvec{3 & 2}\vec{x} &= -\frac{11}{6}
\end{align}
	See \figref{fig:chapters/11/10/4/21/1}.
\begin{figure}[H]
	\centering
	\includegraphics[width=0.75\columnwidth]{chapters/11/10/4/21/figs/line_plot.jpg}
	\caption{}
	\label{fig:chapters/11/10/4/21/1}
\end{figure}

\end{enumerate}

\newpage
\section{Intersection of Conics}
\subsection{Formulae}
\begin{enumerate}[label=\thesubsection.\arabic*,ref=\thesubsection.\theenumi]
		\item
  The points of intersection of the line 
\begin{align}
L: \quad \vec{x} = \vec{h} + \kappa \vec{m} \quad \kappa \in \mathbb{R}
\label{eq:conic_tangent}
\end{align}
with the conic section in \eqref{eq:conic_quad_form} are given by
\begin{align}
\vec{x}_i = \vec{h} + \kappa_i \vec{m}
	\label{eq:chord-pts}
\end{align}
%
where
\begin{multline}
\kappa_i = \frac{1}
{
\vec{m}^{\top}\vec{V}\vec{m}
}
\lbrak{-\vec{m}^{\top}\brak{\vec{V}\vec{h}+\vec{u}}}
%\\
\pm
%{\small
\rbrak{\sqrt{
\sbrak{
\vec{m}^{\top}\brak{\vec{V}\vec{h}+\vec{u}}
}^2
	-\text{g}
\brak
{\vec{h}
%\vec{h}^{\top}\vec{V}\vec{h} + 2\vec{u}^{\top}\vec{h} +f
}
\brak{\vec{m}^{\top}\vec{V}\vec{m}}
}
}
%}
\label{eq:tangent_roots}
\end{multline}
See 
	 \ref{prop:chord}
	 for proof.

  \item
\eqref{eq:conic_quad_form} represents a pair of straight lines if 
the matrix 
  \begin{align} 
	  \myvec{\vec{V} & \vec{u}\\ \vec{u}^{\top} & f}  
	  \label{eq:pair-mat-sing}
  \end{align} 
  is singular.
\item The intersection of two conics 
with parameters $\vec{V}_i, \vec{u}_i, f_i,\ i = 1,2$
	is defined
as
\begin{align}
	\vec{x}^{\top}\brak{\vec{V}_1 + \mu\vec{V}_2}\vec{x}+2 \brak{\vec{u}_1+\mu \vec{u}_2}^{\top} \vec{x} 
	+ \brak{f_1+\mu f_2}= 0
	  \label{eq:pair-mat-sing-conic}
    \end{align}
	  
	  
\item From \eqref{eq:pair-mat-sing}, \eqref{eq:pair-mat-sing-conic} represents a pair of straight lines if
\begin{align}
	  \label{eq:pair-mat-sing-conic-det}
\mydet{\vec{V}_1 + \mu\vec{V}_2 & \vec{u}_1+\mu \vec{u}_2\\ \brak{\vec{u}_1+\mu \vec{u}_2}^{\top} & f_1 + \mu f_2} &= 0
\end{align}
\end{enumerate}

\subsection{Chords}
\input{chapters/inter/examples/chord.tex}
\subsection{Curves}
\begin{enumerate}[label=\thesubsection.\arabic*,ref=\thesubsection.\theenumi]
		\item
  The points of intersection of the line 
\begin{align}
L: \quad \vec{x} = \vec{h} + \kappa \vec{m} \quad \kappa \in \mathbb{R}
\label{eq:conic_tangent}
\end{align}
with the conic section in \eqref{eq:conic_quad_form} are given by
\begin{align}
\vec{x}_i = \vec{h} + \kappa_i \vec{m}
	\label{eq:chord-pts}
\end{align}
%
where
\begin{multline}
\kappa_i = \frac{1}
{
\vec{m}^{\top}\vec{V}\vec{m}
}
\lbrak{-\vec{m}^{\top}\brak{\vec{V}\vec{h}+\vec{u}}}
%\\
\pm
%{\small
\rbrak{\sqrt{
\sbrak{
\vec{m}^{\top}\brak{\vec{V}\vec{h}+\vec{u}}
}^2
	-\text{g}
\brak
{\vec{h}
%\vec{h}^{\top}\vec{V}\vec{h} + 2\vec{u}^{\top}\vec{h} +f
}
\brak{\vec{m}^{\top}\vec{V}\vec{m}}
}
}
%}
\label{eq:tangent_roots}
\end{multline}
See 
	 \ref{prop:chord}
	 for proof.

  \item
\eqref{eq:conic_quad_form} represents a pair of straight lines if 
the matrix 
  \begin{align} 
	  \myvec{\vec{V} & \vec{u}\\ \vec{u}^{\top} & f}  
	  \label{eq:pair-mat-sing}
  \end{align} 
  is singular.
\item The intersection of two conics 
with parameters $\vec{V}_i, \vec{u}_i, f_i,\ i = 1,2$
	is defined
as
\begin{align}
	\vec{x}^{\top}\brak{\vec{V}_1 + \mu\vec{V}_2}\vec{x}+2 \brak{\vec{u}_1+\mu \vec{u}_2}^{\top} \vec{x} 
	+ \brak{f_1+\mu f_2}= 0
	  \label{eq:pair-mat-sing-conic}
    \end{align}
	  
	  
\item From \eqref{eq:pair-mat-sing}, \eqref{eq:pair-mat-sing-conic} represents a pair of straight lines if
\begin{align}
	  \label{eq:pair-mat-sing-conic-det}
\mydet{\vec{V}_1 + \mu\vec{V}_2 & \vec{u}_1+\mu \vec{u}_2\\ \brak{\vec{u}_1+\mu \vec{u}_2}^{\top} & f_1 + \mu f_2} &= 0
\end{align}
\end{enumerate}


\newpage
\section{Tangent And Normal}
\subsection{Formulae}
The parameters for the given conic are
\begin{align}
	\label{12/6/3/25eq:V_matrix}
	\vec{V} &= \myvec{0 & 0\\0 & 1},
	\\
	\label{12/6/3/25eq:u_vector}
	\vec{u} &= \myvec{-3/2\\0},
	\\
	\label{12/6/3/25eq:f_value}
	f &= 2
	%\\
\end{align}
which represent a parabola. 
Following the approach in 
\probref{chapters/12/6/3/15},
   \begin{align}
     \vec{p_1} = \myvec{1\\0},\
     \vec{n} = \myvec{-2\\1},
    \end{align}
yielding the matrix equation
\begin{align}
	\label{12/6/3/25eq:vertex_system}
	\myvec{-3&0\\0& 0\\0& 1}\vec{q} = \myvec{-41/16\\0 \\3/4}\\
\end{align}
The augmented matrix for \eqref{12/6/3/25eq:vertex_system} can be expressed as
\begin{align*}
	%\label{12/6/3/25eq:vertex_solv1}
	%\myvec{-3&0&\vrule&2\\0&0&\vrule&0\\0&1&\vrule&0}\\ 	
	%\label{12/6/3/25eq:vertex_solv2}
	\xleftrightarrow[]{R_2 \leftrightarrow R_3}\myvec{-3&0&\vrule&-41/16\\0&1&\vrule&0\\0&0&\vrule&3/4}
	\xleftrightarrow[]{-\frac{R_1}{-3} \leftarrow R_2}\myvec{1&0&\vrule&41/48\\0&1&\vrule&0\\0&0&\vrule&3/4}\\
	\implies\vec{q} = \myvec{\frac{41}{48}\\\frac{3}{4}}
\end{align*}
The equation of tangent is then obtained as
\begin{align}
	\myvec{-2 & 1}\vec{x} +\frac{23}{24} = 0 
\end{align}
See  
		\figref{fig:12/6/3/25}.
	\begin{figure}[H]
		\centering
 \includegraphics[width=0.75\columnwidth]{chapters/12/6/3/25/figs/conic.pdf}
		\caption{}
		\label{fig:12/6/3/25}
  	\end{figure}

\subsection{Circle}
Substituting numerical values
	in \eqref{eq:circ-cr},
\begin{align}
	f
	=\frac{11}{36}
\end{align}
	Thus, the equation of the circle is
\begin{align}
	\norm{\vec{x}}^2 + \myvec{-1 & -\frac{1}{2}}\vec{x}+\frac{11}{36}=0
\end{align}
See \figref{fig:chapters/11/11/1/3/Fig1}.
\begin{figure}[H]
\begin{center}
\includegraphics[width=0.75\columnwidth]{chapters/11/11/1/3/figs/fig.pdf}
\end{center}
\caption{}
\label{fig:chapters/11/11/1/3/Fig1}
\end{figure}

\subsection{Conic}
	\begin{figure}[H]
		\centering
 \includegraphics[width=0.75\columnwidth]{chapters/12/8/1/8/figs/conics1.png}
		\caption{}
		\label{fig:12/8/1/8}
  	\end{figure}
The given conic parameters are
\begin{align}
 \vec{V} = \myvec{0 & 0\\0 & 1},
	\vec{u} = -\frac{1}{2}\vec{e}_1
 f = 0
\end{align}
The parameters of the lines are
\begin{align}
\vec{q}_2=\myvec{a\\0},
\vec{m}_2=\vec{e}_2
\end{align}
Substituting the above values in 
\eqref{eq:tangent_roots},
\begin{align}
\mu_i=a,-a
\end{align}
yielding  the points of  intersection as
\begin{align}
\vec{a_0}=\myvec{a\\a},
\vec{a_1}=\myvec{a\\-a}
\end{align}
Similarly, for the line $x-4=0$, 
\begin{align}
\vec{q_1}=\myvec{4\\0},
\vec{m_1}=\vec{e}_2
\end{align}
yielding
\begin{align}
\mu_i=2,-2
\end{align}
and
\begin{align}
\vec{a}_3=\myvec{4\\2},
\vec{a}_2=\myvec{4\\-2}.
\end{align}
Area between parabola and the line $x=4$ is divided equally by the line $x=a$.  Thus, 
		from \figref{fig:12/8/1/8},
\begin{align}
	A_1&=\int_{0}^{a} \ \sqrt{x} \,dx
	\\
	A_2&=\int_{a}^{4} \ \sqrt{x} \,dx
	\\
	\text{ and }
	A_1&=A_2 \\
\implies 
	a&=4^\frac{2}{3}
\end{align}

\iffalse
\section*{\large Construction}

{
\setlength\extrarowheight{5pt}
\begin{tabular}{|l|c|}
    \hline 
    \textbf{Points} & \textbf{intersection points} \\ \hline
   a0 & $\myvec{
   a\\
   a
   } $ \\\hline
   a1 & $\myvec{
   a\\
   -a
   } $ \\\hline
    
   a3 & $\myvec{
   4\\
   2
   } $ \\\hline
   a2 & $\myvec{
   4\\
   -2
   } $ \\\hline
      
      \end{tabular}
}

\end{document}
\fi

\subsection{Construction}
\input{chapters/tangent/examples/const.tex}
%
\appendices
\section{Triangle}
%\numberwithin{equation}{section}
Consider a triangle with vertices
		\begin{align}
			\label{eq:app-tri-pts}
			\vec{A} = \myvec{1 \\ -1},\,
			\vec{B} = \myvec{-4 \\ 6},\,
			\vec{C} = \myvec{-3 \\ -5}
		\end{align}
\subsection{Sides}
%\renewcommand{\theequation}{\theenumi}
\begin{enumerate}[label=\thesubsection.\arabic*.,ref=\thesubsection.\theenumi]
%\numberwithin{equation}{enumi}
\item The direction vector of $AB$ is defined as
		\begin{align}
			\vec{B}-
			\vec{A}
		\end{align}
Find the direction vectors of $AB, BC$ and $CA$.
\\
\solution 
\begin{enumerate} 
\item  The Direction vector of $AB$ is 
	\begin{align}  \vec{B} - \vec{A} 
		=\myvec{ -4\\ 6 } - \myvec{ 1\\ -1 }
 = \myvec{ -4 - 1\\ 6 - (-1) } = \myvec{ -5\\ 7 }
		\label{eq:app-geo-dir-vec-ab}
 \end{align}
\item The Direction vector of $BC$ is
	\begin{align} \vec{C} - \vec{B}=\myvec{ -3\\ -5} - \myvec{ -4\\ 6 }
 = \myvec{ -3 - (-4)\\ -5 - 6 } = \myvec{1\\ -11 }
		\label{eq:app-geo-dir-vec-bc}
  \end{align}
  \item  The Direction vector of $CA$  is
	  \begin{align}  \vec{A} - \vec{C} =\myvec{ 1\\ -1 }-\myvec{ -3\\ -5}
 = \myvec{ 1 - (-3)\\ -1 - (-5) } = \myvec{ 4\\ 4 }
		\label{eq:app-geo-dir-vec-ca}
  \end{align}
 \end{enumerate}
%	\input{solutions/1/1/1/prob_1.tex}
	\item The length of side $BC$ is 
		\label{prob:side-length}
		\begin{align}
			c = \norm{\vec{B}-\vec{A}} \triangleq \sqrt{\brak{\vec{B}-\vec{A}}^{\top}\brak{\vec{B}-\vec{A}}}
		\end{align}
		where
		\begin{align}
			\vec{A}^{\top}\triangleq\myvec{1 & -1}
		\end{align}
		Similarly, 
		\begin{align}
b = \norm{\vec{C}-\vec{B}},\,
a = \norm{\vec{A}-\vec{C}}
		\end{align}
		Find $a, b, c$.
\begin{enumerate}
	\item 
	From 	
		\eqref{eq:app-geo-dir-vec-ab},
\begin{align}
\vec{A}-\vec{B} &= \myvec{5\\-7}, \\
\implies 	c &= 	\norm{\vec{B}-\vec{A}} = \norm{\vec{A}-\vec{B}} 
	\\
	&= \sqrt{\myvec{5 & -7}\myvec{5\\-7}}
= \sqrt{\brak{5}^2 +\brak{7}^2}\\
	&=\sqrt{74}
		\label{eq:app-geo-norm-ab}
\end{align}
	\item Similarly, from 
		\eqref{eq:app-geo-dir-vec-bc},
\begin{align}
	a &= \norm{\vec{B}-\vec{C}} 
	= \sqrt{\myvec{-1 & 11}\myvec{-1\\11}}
\\
&= \sqrt{\brak{1}^2+\brak{11}^2}
	= \sqrt{122}
		\label{eq:app-geo-norm-bc}
\end{align}
and
		from 		\eqref{eq:app-geo-dir-vec-ca},
	\item 
		\begin{align}
			b &= \norm{\vec{A}-\vec{C}} = \sqrt{\myvec{4 & 4}\myvec{4\\4}}
\\
&= \sqrt{\brak{4}^2+\brak{4}^2}
	=\sqrt{32}
		\label{eq:app-geo-norm-ca}
\end{align}
\end{enumerate}
%  \\            
  %\documentclass[journal,12pt,twocolumn]{IEEEtran}
\usepackage{setspace}
\usepackage{gensymb}
\usepackage{xcolor}
\usepackage{caption}
\singlespacing
\usepackage{siunitx}
\usepackage[cmex10]{amsmath}
\usepackage{mathtools}
\usepackage{hyperref}
\usepackage{amsthm}
\usepackage{mathrsfs}
\usepackage{txfonts}
\usepackage{stfloats}
\usepackage{cite}
\usepackage{cases}
\usepackage{subfig}
\usepackage{longtable}
\usepackage{multirow}
\usepackage{enumitem}
\usepackage{bm}
\usepackage{mathtools}
\usepackage{listings}
\usepackage{tikz}
\usetikzlibrary{shapes,arrows,positioning}
\usepackage{circuitikz}
\renewcommand{\vec}[1]{\boldsymbol{\mathbf{#1}}}
\DeclareMathOperator*{\Res}{Res}
\renewcommand\thesection{\arabic{section}}
\renewcommand\thesubsection{\thesection.\arabic{subsection}}
\renewcommand\thesubsubsection{\thesubsection.\arabic{subsubsection}}

\renewcommand\thesectiondis{\arabic{section}}
\renewcommand\thesubsectiondis{\thesectiondis.\arabic{subsection}}
\renewcommand\thesubsubsectiondis{\thesubsectiondis.\arabic{subsubsection}}
\hyphenation{op-tical net-works semi-conduc-tor}

\lstset{
language=Python,
frame=single, 
breaklines=true,
columns=fullflexible
}
\begin{document}
\theoremstyle{definition}
\newtheorem{theorem}{Theorem}[section]
\newtheorem{problem}{Problem}
\newtheorem{proposition}{Proposition}[section]
\newtheorem{lemma}{Lemma}[section]
\newtheorem{corollary}[theorem]{Corollary}
\newtheorem{example}{Example}[section]
\newtheorem{definition}{Definition}[section]
\newcommand{\BEQA}{\begin{eqnarray}}
        \newcommand{\EEQA}{\end{eqnarray}}
\newcommand{\define}{\stackrel{\triangle}{=}}
\newcommand{\myvec}[1]{\ensuremath{\begin{pmatrix}#1\end{pmatrix}}}
\newcommand{\mydet}[1]{\ensuremath{\begin{vmatrix}#1\end{vmatrix}}}
\bibliographystyle{IEEEtran}
\providecommand{\nCr}[2]{\,^{#1}C_{#2}} % nCr
\providecommand{\nPr}[2]{\,^{#1}P_{#2}} % nPr
\providecommand{\mbf}{\mathbf}
\providecommand{\pr}[1]{\ensuremath{\Pr\left(#1\right)}}
\providecommand{\qfunc}[1]{\ensuremath{Q\left(#1\right)}}
\providecommand{\sbrak}[1]{\ensuremath{{}\left[#1\right]}}
\providecommand{\lsbrak}[1]{\ensuremath{{}\left[#1\right.}}
\providecommand{\rsbrak}[1]{\ensuremath{{}\left.#1\right]}}
\providecommand{\brak}[1]{\ensuremath{\left(#1\right)}}
\providecommand{\lbrak}[1]{\ensuremath{\left(#1\right.}}
\providecommand{\rbrak}[1]{\ensuremath{\left.#1\right)}}
\providecommand{\cbrak}[1]{\ensuremath{\left\{#1\right\}}}
\providecommand{\lcbrak}[1]{\ensuremath{\left\{#1\right.}}
\providecommand{\rcbrak}[1]{\ensuremath{\left.#1\right\}}}
\theoremstyle{remark}
\newtheorem{rem}{Remark}
\newcommand{\sgn}{\mathop{\mathrm{sgn}}}
\newcommand{\rect}{\mathop{\mathrm{rect}}}
\newcommand{\sinc}{\mathop{\mathrm{sinc}}}
\providecommand{\abs}[1]{\left\vert#1\right\vert}
\providecommand{\res}[1]{\Res\displaylimits_{#1}}
\providecommand{\norm}[1]{\lVert#1\rVert}
\providecommand{\mtx}[1]{\mathbf{#1}}
\providecommand{\mean}[1]{E\left[ #1 \right]}
\providecommand{\fourier}{\overset{\mathcal{F}}{ \rightleftharpoons}}
\providecommand{\ztrans}{\overset{\mathcal{Z}}{ \rightleftharpoons}}
\providecommand{\system}[1]{\overset{\mathcal{#1}}{ \longleftrightarrow}}
\newcommand{\solution}{\noindent \textbf{Solution: }}
\providecommand{\dec}[2]{\ensuremath{\overset{#1}{\underset{#2}{\gtrless}}}}
\let\StandardTheFigure\thefigure
\def\putbox#1#2#3{\makebox[0in][l]{\makebox[#1][l]{}\raisebox{\baselineskip}[0in][0in]{\raisebox{#2}[0in][0in]{#3}}}}
\def\rightbox#1{\makebox[0in][r]{#1}}
\def\centbox#1{\makebox[0in]{#1}}
\def\topbox#1{\raisebox{-\baselineskip}[0in][0in]{#1}}
\def\midbox#1{\raisebox{-0.5\baselineskip}[0in][0in]{#1}}

\vspace{3cm}
\title{11.11.2.5}
\author{Lokesh Surana}
\maketitle
\section*{Class 11, Chapter 11, Exercise 2.5}

Q. Find the coordinates of the focus, axis of the parabola, the equation of the directrix and the length of the latus rectum $y^2 = 10x$

\solution
The given equation of the parabola can be rearranged as
\begin{align}
    \label{eq:1} y^2-10x = 0
\end{align}
The above equation can be equated to the generic equation of conic sections
\begin{align}
    \label{eq:2} g\brak{\vec{x}} = \vec{x}^T\vec{V}\vec{x} + 2\vec{u}^T\vec{x} + f = 0 
\end{align}

Comparing coefficients of \eqref{eq:1} and \eqref{eq:2},

\begin{align}
    \label{eq:3}
	\vec{V} &= \myvec{ 0 & 0 \\ 0 & 1} \\
	\label{eq:4}
	\vec{u} &= -\myvec{5 \\ 0} \\
	\label{eq:5}
	f &= 0 
\end{align}

\begin{enumerate}
\item From \eqref{eq:3}, since $\vec{V}$ is already diagonalized, the Eigen values $\lambda_1$ and $\lambda_2$ are given as 
\begin{align}
	\lambda_1 &= 0 \\
	\lambda_2 &= 1 
\end{align}
and the eigenvector matrix
\begin{align}
	\vec{P} = \vec{I}.
\end{align}

\begin{align}
	\therefore 
	\vec{n} &= \sqrt{\lambda_2}\vec{p_1} \\
	&= \myvec{1 \\ 0} 
\end{align}

Since
\begin{align}
	\label{eq:c}
	c = \frac{\norm{\vec{u}}^2-\lambda_2f}{2\vec{u}^\top\vec{n}},
\end{align}

Substituting values of $\vec{u}, \vec{n}, \lambda_2 \text{ and } f$ in \eqref{eq:c}, we get
\begin{align}
	c &= \frac{5^2-1\brak{0}}{-2 \myvec{5 & 0}\myvec{1 \\ 0}} = -\frac{5}{2} \\
\end{align}

The focus $\vec{F}$ of parabola is expressed as
\begin{align}
	\vec{F} &= \frac{ce^2\vec{n}-\vec{u}}{\lambda_2} \\
	&= \frac{-\frac{5}{2}\brak{1}^2\myvec{1 \\0} + \myvec{5 \\ 0}}{1} \\
	&= \myvec{\frac{5}{2} \\ 0}
\end{align}

\item  The directrix is given by
\begin{align}
	\vec{n}^\top\vec{x} &= c \\
\implies	\myvec{1 & 0}\vec{x} &= -\frac{5}{2} \\
\end{align}

\item The equation for the axis of parabola passing through $\vec{F}$ and orthogonal to the directrix is given as  
\begin{align}
	\vec{m}^\top\brak{\vec{x}-\vec{F}} &= 0
\end{align}
where $\vec{m}$ is the normal vector to the axis and also the slope of the directrix.
\begin{align}
	\because \vec{n} = \myvec{1 \\ 0 }, \vec{m} &= \myvec{0 \\ 1} \\
	\implies \myvec{0 & 1}\myvec{\vec{x} - \myvec{\frac{5}{2} \\ 0}} &= 0\\
	\text{or, }	\myvec{0 & 1}\vec{x} &= 0 
\end{align}

\item The latus rectum of a parabola is given by 
\begin{align}
	l &= \frac{\eta}{\lambda_2}  
	 = -\frac{2\vec{u}^\top\vec{p_1}}{\lambda_2} \\
	 &= -\frac{2\myvec{-5 & 0}\myvec{1 \\ 0}}{1} \\
	 &= 10 \text{ units }
\end{align}
\end{enumerate}

\begin{figure}[H]
    \centering
    \includegraphics[width=0.75\columnwidth]{figs/parabola.png}
    \caption{Parabola $y^2 = 10x$}
    \label{fig:parabola}
\end{figure}

\end{document}
\item   Points $\vec{A}, \vec{B}, \vec{C}$ are defined to be collinear if 
		\begin{align}
			\label{eq:app-app-line-rank}
			\rank{\myvec{1 & 1 & 1 \\ \vec{A}& \vec{B}&\vec{C}}} = 2
		\end{align}
Are the given points in
			\eqref{eq:app-tri-pts}
collinear?
\\
\solution 
From 
			\eqref{eq:app-tri-pts},
\begin{align}
    \label{eq:app-1.1.3,2}
\myvec{
    1 & 1 & 1\\
    \vec{A} & \vec{B} & \vec{C} \\
    } 
    =
    %\label{eq:app-matthrowoperations}
    \myvec{
    1 & 1 & 1
    \\
    1 & -4 & -3
    \\
    -1 & 6 & -5
    }
     \xleftrightarrow[]{R_3 \leftarrow R_3+R_2}
    \myvec{
    1 & 1 & 1
    \\
    1 & -4 & -3
    \\
    0 & 2 & -8 
    }
    \\
     \xleftrightarrow[]{R_2\leftarrow R_1-R_2}
    \myvec{
    1 & 1 & 1
    \\
    0 & 5 & 4
    \\
    0 & 2 & -8 
    }
     \xleftrightarrow[]{R_3\leftarrow R_3-\frac{2}{5}R_2}
    \myvec{
    1 & 1 & 1
    \\
    0 & 5 & 4
    \\
    0 & 0 & \frac{-48}{5}
    }
\end{align}
There are no zero rows. So,
\begin{align}
    \text{rank}\myvec{
    1 & 1 & 1\\
    \vec{A} & \vec{B} & \vec{C} \\
    } &= 3 
\end{align}  
Hence,  the points $\vec{A},\vec{B},\vec{C}$ are not collinear. 
This is visible in 
\figref{fig1:Triangle}.
\begin{figure}[H]
\centering
\includegraphics[width=0.75\columnwidth]{figs/triangle/vector.pdf}
\caption{$\triangle ABC$}
\label{fig1:Triangle}
\end{figure}
% \\		\solution 
From 
			\eqref{eq:tri-pts},
\begin{align}
    \label{eq:1.1.3,2}
\myvec{
    1 & 1 & 1\\
    \vec{A} & \vec{B} & \vec{C} \\
    } 
    =
    %\label{eq:matthrowoperations}
    \myvec{
    1 & 1 & 1
    \\
    1 & -4 & -3
    \\
    -1 & 6 & -5
    }
     \xleftrightarrow[]{R_3 \leftarrow R_3+R_2}
    \myvec{
    1 & 1 & 1
    \\
    1 & -4 & -3
    \\
    0 & 2 & -8 
    }
    \\
     \xleftrightarrow[]{R_2\leftarrow R_1-R_2}
    \myvec{
    1 & 1 & 1
    \\
    0 & 5 & 4
    \\
    0 & 2 & -8 
    }
     \xleftrightarrow[]{R_3\leftarrow R_3-\frac{2}{5}R_2}
    \myvec{
    1 & 1 & 1
    \\
    0 & 5 & 4
    \\
    0 & 0 & \frac{-48}{5}
    }
\end{align}
There are no zero rows. So,
\begin{align}
    \text{rank}\myvec{
    1 & 1 & 1\\
    \vec{A} & \vec{B} & \vec{C} \\
    } &= 3 
\end{align}  
Hence,  the points $\vec{A},\vec{B},\vec{C}$ are not collinear. 
This is visible in 
\figref{fig1:Triangle}.
\begin{figure}[H]
\centering
\includegraphics[width=0.75\columnwidth]{figs/triangle/vector.pdf}
\caption{$\triangle ABC$}
\label{fig1:Triangle}
\end{figure}

\item The parameteric form of the equation  of $AB$ is 
		\begin{align}
			\label{eq:app-geo-param}
			\vec{x}=\vec{A}+k\vec{m} \quad k \ne 0,
		\end{align}
		where
		\begin{align}
\vec{m}=\vec{B}-\vec{A}
		\end{align}
is the direction vector of $AB$.
Find the parameteric equations of $AB, BC$ and $CA$.
\\
\solution
From 
			\eqref{eq:app-geo-param} and
		\eqref{eq:app-geo-dir-vec-ab},
the parametric equation for $AB$ is given by
\begin{align}
AB: \vec{x} = &\myvec{1\\-1} + k \myvec{-5\\7}
\end{align}
Similarly, from 
		\eqref{eq:app-geo-dir-vec-bc} and
		\eqref{eq:app-geo-dir-vec-ca},
\begin{align}
BC: \vec{x} = &\myvec{-4\\6} + k \myvec{1\\-11}\\
CA: \vec{x} = &\myvec{-3\\-5} + k \myvec{4\\4}
\end{align}

%		\documentclass[journal,12pt,twocolumn]{IEEEtran}
\usepackage{setspace}
\usepackage{gensymb}
\usepackage{xcolor}
\usepackage{caption}
\singlespacing
\usepackage{siunitx}
\usepackage[cmex10]{amsmath}
\usepackage{mathtools}
\usepackage{hyperref}
\usepackage{amsthm}
\usepackage{mathrsfs}
\usepackage{txfonts}
\usepackage{stfloats}
\usepackage{cite}
\usepackage{cases}
\usepackage{subfig}
\usepackage{longtable}
\usepackage{multirow}
\usepackage{enumitem}
\usepackage{bm}
\usepackage{mathtools}
\usepackage{listings}
\usepackage{tikz}
\usetikzlibrary{shapes,arrows,positioning}
\usepackage{circuitikz}
\renewcommand{\vec}[1]{\boldsymbol{\mathbf{#1}}}
\DeclareMathOperator*{\Res}{Res}
\renewcommand\thesection{\arabic{section}}
\renewcommand\thesubsection{\thesection.\arabic{subsection}}
\renewcommand\thesubsubsection{\thesubsection.\arabic{subsubsection}}

\renewcommand\thesectiondis{\arabic{section}}
\renewcommand\thesubsectiondis{\thesectiondis.\arabic{subsection}}
\renewcommand\thesubsubsectiondis{\thesubsectiondis.\arabic{subsubsection}}
\hyphenation{op-tical net-works semi-conduc-tor}

\lstset{
language=Python,
frame=single, 
breaklines=true,
columns=fullflexible
}
\begin{document}
\theoremstyle{definition}
\newtheorem{theorem}{Theorem}[section]
\newtheorem{problem}{Problem}
\newtheorem{proposition}{Proposition}[section]
\newtheorem{lemma}{Lemma}[section]
\newtheorem{corollary}[theorem]{Corollary}
\newtheorem{example}{Example}[section]
\newtheorem{definition}{Definition}[section]
\newcommand{\BEQA}{\begin{eqnarray}}
        \newcommand{\EEQA}{\end{eqnarray}}
\newcommand{\define}{\stackrel{\triangle}{=}}
\newcommand{\myvec}[1]{\ensuremath{\begin{pmatrix}#1\end{pmatrix}}}
\newcommand{\mydet}[1]{\ensuremath{\begin{vmatrix}#1\end{vmatrix}}}
\bibliographystyle{IEEEtran}
\providecommand{\nCr}[2]{\,^{#1}C_{#2}} % nCr
\providecommand{\nPr}[2]{\,^{#1}P_{#2}} % nPr
\providecommand{\mbf}{\mathbf}
\providecommand{\pr}[1]{\ensuremath{\Pr\left(#1\right)}}
\providecommand{\qfunc}[1]{\ensuremath{Q\left(#1\right)}}
\providecommand{\sbrak}[1]{\ensuremath{{}\left[#1\right]}}
\providecommand{\lsbrak}[1]{\ensuremath{{}\left[#1\right.}}
\providecommand{\rsbrak}[1]{\ensuremath{{}\left.#1\right]}}
\providecommand{\brak}[1]{\ensuremath{\left(#1\right)}}
\providecommand{\lbrak}[1]{\ensuremath{\left(#1\right.}}
\providecommand{\rbrak}[1]{\ensuremath{\left.#1\right)}}
\providecommand{\cbrak}[1]{\ensuremath{\left\{#1\right\}}}
\providecommand{\lcbrak}[1]{\ensuremath{\left\{#1\right.}}
\providecommand{\rcbrak}[1]{\ensuremath{\left.#1\right\}}}
\theoremstyle{remark}
\newtheorem{rem}{Remark}
\newcommand{\sgn}{\mathop{\mathrm{sgn}}}
\newcommand{\rect}{\mathop{\mathrm{rect}}}
\newcommand{\sinc}{\mathop{\mathrm{sinc}}}
\providecommand{\abs}[1]{\left\vert#1\right\vert}
\providecommand{\res}[1]{\Res\displaylimits_{#1}}
\providecommand{\norm}[1]{\lVert#1\rVert}
\providecommand{\mtx}[1]{\mathbf{#1}}
\providecommand{\mean}[1]{E\left[ #1 \right]}
\providecommand{\fourier}{\overset{\mathcal{F}}{ \rightleftharpoons}}
\providecommand{\ztrans}{\overset{\mathcal{Z}}{ \rightleftharpoons}}
\providecommand{\system}[1]{\overset{\mathcal{#1}}{ \longleftrightarrow}}
\newcommand{\solution}{\noindent \textbf{Solution: }}
\providecommand{\dec}[2]{\ensuremath{\overset{#1}{\underset{#2}{\gtrless}}}}
\let\StandardTheFigure\thefigure
\def\putbox#1#2#3{\makebox[0in][l]{\makebox[#1][l]{}\raisebox{\baselineskip}[0in][0in]{\raisebox{#2}[0in][0in]{#3}}}}
\def\rightbox#1{\makebox[0in][r]{#1}}
\def\centbox#1{\makebox[0in]{#1}}
\def\topbox#1{\raisebox{-\baselineskip}[0in][0in]{#1}}
\def\midbox#1{\raisebox{-0.5\baselineskip}[0in][0in]{#1}}

\vspace{3cm}
\title{11.11.2.5}
\author{Lokesh Surana}
\maketitle
\section*{Class 11, Chapter 11, Exercise 2.5}

Q. Find the coordinates of the focus, axis of the parabola, the equation of the directrix and the length of the latus rectum $y^2 = 10x$

\solution
The given equation of the parabola can be rearranged as
\begin{align}
    \label{eq:1} y^2-10x = 0
\end{align}
The above equation can be equated to the generic equation of conic sections
\begin{align}
    \label{eq:2} g\brak{\vec{x}} = \vec{x}^T\vec{V}\vec{x} + 2\vec{u}^T\vec{x} + f = 0 
\end{align}

Comparing coefficients of \eqref{eq:1} and \eqref{eq:2},

\begin{align}
    \label{eq:3}
	\vec{V} &= \myvec{ 0 & 0 \\ 0 & 1} \\
	\label{eq:4}
	\vec{u} &= -\myvec{5 \\ 0} \\
	\label{eq:5}
	f &= 0 
\end{align}

\begin{enumerate}
\item From \eqref{eq:3}, since $\vec{V}$ is already diagonalized, the Eigen values $\lambda_1$ and $\lambda_2$ are given as 
\begin{align}
	\lambda_1 &= 0 \\
	\lambda_2 &= 1 
\end{align}
and the eigenvector matrix
\begin{align}
	\vec{P} = \vec{I}.
\end{align}

\begin{align}
	\therefore 
	\vec{n} &= \sqrt{\lambda_2}\vec{p_1} \\
	&= \myvec{1 \\ 0} 
\end{align}

Since
\begin{align}
	\label{eq:c}
	c = \frac{\norm{\vec{u}}^2-\lambda_2f}{2\vec{u}^\top\vec{n}},
\end{align}

Substituting values of $\vec{u}, \vec{n}, \lambda_2 \text{ and } f$ in \eqref{eq:c}, we get
\begin{align}
	c &= \frac{5^2-1\brak{0}}{-2 \myvec{5 & 0}\myvec{1 \\ 0}} = -\frac{5}{2} \\
\end{align}

The focus $\vec{F}$ of parabola is expressed as
\begin{align}
	\vec{F} &= \frac{ce^2\vec{n}-\vec{u}}{\lambda_2} \\
	&= \frac{-\frac{5}{2}\brak{1}^2\myvec{1 \\0} + \myvec{5 \\ 0}}{1} \\
	&= \myvec{\frac{5}{2} \\ 0}
\end{align}

\item  The directrix is given by
\begin{align}
	\vec{n}^\top\vec{x} &= c \\
\implies	\myvec{1 & 0}\vec{x} &= -\frac{5}{2} \\
\end{align}

\item The equation for the axis of parabola passing through $\vec{F}$ and orthogonal to the directrix is given as  
\begin{align}
	\vec{m}^\top\brak{\vec{x}-\vec{F}} &= 0
\end{align}
where $\vec{m}$ is the normal vector to the axis and also the slope of the directrix.
\begin{align}
	\because \vec{n} = \myvec{1 \\ 0 }, \vec{m} &= \myvec{0 \\ 1} \\
	\implies \myvec{0 & 1}\myvec{\vec{x} - \myvec{\frac{5}{2} \\ 0}} &= 0\\
	\text{or, }	\myvec{0 & 1}\vec{x} &= 0 
\end{align}

\item The latus rectum of a parabola is given by 
\begin{align}
	l &= \frac{\eta}{\lambda_2}  
	 = -\frac{2\vec{u}^\top\vec{p_1}}{\lambda_2} \\
	 &= -\frac{2\myvec{-5 & 0}\myvec{1 \\ 0}}{1} \\
	 &= 10 \text{ units }
\end{align}
\end{enumerate}

\begin{figure}[H]
    \centering
    \includegraphics[width=0.75\columnwidth]{figs/parabola.png}
    \caption{Parabola $y^2 = 10x$}
    \label{fig:parabola}
\end{figure}

\end{document}
\item The normal form of the equation of $AB$  is 
		\begin{align}
			\label{eq:app-geo-normal}
			\vec{n}^{\top}\brak{	\vec{x}-\vec{A}} = 0
		\end{align}
		where 
		\begin{align}
			\vec{n}^{\top}\vec{m}&=\vec{n}^{\top}\brak{\vec{B}-\vec{A}} = 0
			\\
			\text{or, } \vec{n}&=\myvec{0 & 1 \\ -1 & 0} \vec{m}
			\label{eq:app-geo-norm-vec}
		\end{align}
Find the normal form of the equations of $AB, BC$ and $CA$.
\\
\solution
\begin{enumerate}
	\item
From
		\eqref{eq:app-geo-dir-vec-bc}, 
the direction vector of side $\vec{BC}$ is
\begin{align}
\vec{m}
	&=\myvec{1\\-11}
	\\
\implies \vec{n} &= \myvec{0 & 1\\
  -1 & 0}\myvec{1\\-11}
 = \myvec{-11\\-1}
		\label{eq:app-geo-norm-vec-bc}
\end{align}
from 
			\eqref{eq:app-geo-norm-vec}.
Hence, from 
			\eqref{eq:app-geo-normal},
the normal equation of side $BC$ is 
\begin{align}
	\vec{n}^{\top}\brak{	\vec{x}-\vec{B}} &= 0
			\\
\implies    \myvec{-11 & -1}\vec{x}&=\myvec{-11 & -1}\myvec{-4\\6}\\
    \implies
BC: \quad    \myvec{11 & 1}\vec{x}&=-38
\end{align}
\item Similarly, for $AB$,
from 
		\eqref{eq:app-geo-dir-vec-ab}, 
\begin{align}
	\vec{m} &= \myvec{-5\\7}
	\\
\implies        \vec{n} 
                &= \myvec{0&1\\-1&0}\myvec{-5\\7}
                = \myvec{7\\5}
		\label{eq:app-geo-norm-vec-ab}
\end{align}
and 
\begin{align}
	\vec{n}^{\top}\brak{	\vec{x}-\vec{A}} &= 0
	\\
	\implies
                AB: \quad  \vec{n}^{\top}\vec{x} &= \myvec{7&5}\myvec{1\\-1}\\    
       \implies\myvec{7&5}\vec{x} &= 2
\end{align}
\item For 
$CA$, 
from 
		\eqref{eq:app-geo-dir-vec-ca}, 
\begin{align}
\vec{m} &= \myvec{1 \\ 1}
\\
		\label{eq:app-geo-norm-vec-ca}
\implies \vec{n} 
&= \myvec{0&1 \\ -1&0}\myvec{1 \\ 1}
= \myvec{1 \\ -1}\\
\\
\implies	\vec{n}^{\top}\brak{	\vec{x}-\vec{C}} &= 0
\\
\implies \myvec{1&-1}{\vec{x}} &= \myvec{1&-1}\myvec{-3 \\ -5} 
= 2 
\end{align}
\end{enumerate}

%\input{solutions/1/1/5/assign1.tex}
\item The area of $\triangle ABC$ is defined as
		\begin{align}
			\label{eq:app-tri-area-cross}
			\frac{1}{2}\norm{{\brak{\vec{A}-\vec{B}}\times \brak{\vec{A}-\vec{C}}}}
		\end{align}
		where
		\begin{align}
			\vec{A}\times\vec{B} \triangleq \mydet{1 & -4 \\-1 & 6}
		\end{align}
		Find the area of $\triangle ABC$.\\
\solution
From
		\eqref{eq:app-geo-dir-vec-ab}
		and
		\eqref{eq:app-geo-dir-vec-ca},
\begin{align}
	\vec{A}-\vec{B}=\myvec{5\\-7},
	\vec{A}-\vec{C}&=\myvec{4\\4}\\
\implies (\vec{A}-\vec{B})\times(\vec{A}-\vec{C}) &=\mydet{5 & 4\\-7 & 4}\\
&=5\times 4-4\times (-7)\\&=48\\
\implies\frac{1}{2}\norm{(\vec{A}-\vec{B})\times(\vec{A}-\vec{C})}&=\frac{48}{2}=24
\end{align}
which is the desired area.

%  		\documentclass[journal,12pt,twocolumn]{IEEEtran}
\usepackage{setspace}
\usepackage{gensymb}
\usepackage{xcolor}
\usepackage{caption}
\singlespacing
\usepackage{siunitx}
\usepackage[cmex10]{amsmath}
\usepackage{mathtools}
\usepackage{hyperref}
\usepackage{amsthm}
\usepackage{mathrsfs}
\usepackage{txfonts}
\usepackage{stfloats}
\usepackage{cite}
\usepackage{cases}
\usepackage{subfig}
\usepackage{longtable}
\usepackage{multirow}
\usepackage{enumitem}
\usepackage{bm}
\usepackage{mathtools}
\usepackage{listings}
\usepackage{tikz}
\usetikzlibrary{shapes,arrows,positioning}
\usepackage{circuitikz}
\renewcommand{\vec}[1]{\boldsymbol{\mathbf{#1}}}
\DeclareMathOperator*{\Res}{Res}
\renewcommand\thesection{\arabic{section}}
\renewcommand\thesubsection{\thesection.\arabic{subsection}}
\renewcommand\thesubsubsection{\thesubsection.\arabic{subsubsection}}

\renewcommand\thesectiondis{\arabic{section}}
\renewcommand\thesubsectiondis{\thesectiondis.\arabic{subsection}}
\renewcommand\thesubsubsectiondis{\thesubsectiondis.\arabic{subsubsection}}
\hyphenation{op-tical net-works semi-conduc-tor}

\lstset{
language=Python,
frame=single, 
breaklines=true,
columns=fullflexible
}
\begin{document}
\theoremstyle{definition}
\newtheorem{theorem}{Theorem}[section]
\newtheorem{problem}{Problem}
\newtheorem{proposition}{Proposition}[section]
\newtheorem{lemma}{Lemma}[section]
\newtheorem{corollary}[theorem]{Corollary}
\newtheorem{example}{Example}[section]
\newtheorem{definition}{Definition}[section]
\newcommand{\BEQA}{\begin{eqnarray}}
        \newcommand{\EEQA}{\end{eqnarray}}
\newcommand{\define}{\stackrel{\triangle}{=}}
\newcommand{\myvec}[1]{\ensuremath{\begin{pmatrix}#1\end{pmatrix}}}
\newcommand{\mydet}[1]{\ensuremath{\begin{vmatrix}#1\end{vmatrix}}}
\bibliographystyle{IEEEtran}
\providecommand{\nCr}[2]{\,^{#1}C_{#2}} % nCr
\providecommand{\nPr}[2]{\,^{#1}P_{#2}} % nPr
\providecommand{\mbf}{\mathbf}
\providecommand{\pr}[1]{\ensuremath{\Pr\left(#1\right)}}
\providecommand{\qfunc}[1]{\ensuremath{Q\left(#1\right)}}
\providecommand{\sbrak}[1]{\ensuremath{{}\left[#1\right]}}
\providecommand{\lsbrak}[1]{\ensuremath{{}\left[#1\right.}}
\providecommand{\rsbrak}[1]{\ensuremath{{}\left.#1\right]}}
\providecommand{\brak}[1]{\ensuremath{\left(#1\right)}}
\providecommand{\lbrak}[1]{\ensuremath{\left(#1\right.}}
\providecommand{\rbrak}[1]{\ensuremath{\left.#1\right)}}
\providecommand{\cbrak}[1]{\ensuremath{\left\{#1\right\}}}
\providecommand{\lcbrak}[1]{\ensuremath{\left\{#1\right.}}
\providecommand{\rcbrak}[1]{\ensuremath{\left.#1\right\}}}
\theoremstyle{remark}
\newtheorem{rem}{Remark}
\newcommand{\sgn}{\mathop{\mathrm{sgn}}}
\newcommand{\rect}{\mathop{\mathrm{rect}}}
\newcommand{\sinc}{\mathop{\mathrm{sinc}}}
\providecommand{\abs}[1]{\left\vert#1\right\vert}
\providecommand{\res}[1]{\Res\displaylimits_{#1}}
\providecommand{\norm}[1]{\lVert#1\rVert}
\providecommand{\mtx}[1]{\mathbf{#1}}
\providecommand{\mean}[1]{E\left[ #1 \right]}
\providecommand{\fourier}{\overset{\mathcal{F}}{ \rightleftharpoons}}
\providecommand{\ztrans}{\overset{\mathcal{Z}}{ \rightleftharpoons}}
\providecommand{\system}[1]{\overset{\mathcal{#1}}{ \longleftrightarrow}}
\newcommand{\solution}{\noindent \textbf{Solution: }}
\providecommand{\dec}[2]{\ensuremath{\overset{#1}{\underset{#2}{\gtrless}}}}
\let\StandardTheFigure\thefigure
\def\putbox#1#2#3{\makebox[0in][l]{\makebox[#1][l]{}\raisebox{\baselineskip}[0in][0in]{\raisebox{#2}[0in][0in]{#3}}}}
\def\rightbox#1{\makebox[0in][r]{#1}}
\def\centbox#1{\makebox[0in]{#1}}
\def\topbox#1{\raisebox{-\baselineskip}[0in][0in]{#1}}
\def\midbox#1{\raisebox{-0.5\baselineskip}[0in][0in]{#1}}

\vspace{3cm}
\title{11.11.2.5}
\author{Lokesh Surana}
\maketitle
\section*{Class 11, Chapter 11, Exercise 2.5}

Q. Find the coordinates of the focus, axis of the parabola, the equation of the directrix and the length of the latus rectum $y^2 = 10x$

\solution
The given equation of the parabola can be rearranged as
\begin{align}
    \label{eq:1} y^2-10x = 0
\end{align}
The above equation can be equated to the generic equation of conic sections
\begin{align}
    \label{eq:2} g\brak{\vec{x}} = \vec{x}^T\vec{V}\vec{x} + 2\vec{u}^T\vec{x} + f = 0 
\end{align}

Comparing coefficients of \eqref{eq:1} and \eqref{eq:2},

\begin{align}
    \label{eq:3}
	\vec{V} &= \myvec{ 0 & 0 \\ 0 & 1} \\
	\label{eq:4}
	\vec{u} &= -\myvec{5 \\ 0} \\
	\label{eq:5}
	f &= 0 
\end{align}

\begin{enumerate}
\item From \eqref{eq:3}, since $\vec{V}$ is already diagonalized, the Eigen values $\lambda_1$ and $\lambda_2$ are given as 
\begin{align}
	\lambda_1 &= 0 \\
	\lambda_2 &= 1 
\end{align}
and the eigenvector matrix
\begin{align}
	\vec{P} = \vec{I}.
\end{align}

\begin{align}
	\therefore 
	\vec{n} &= \sqrt{\lambda_2}\vec{p_1} \\
	&= \myvec{1 \\ 0} 
\end{align}

Since
\begin{align}
	\label{eq:c}
	c = \frac{\norm{\vec{u}}^2-\lambda_2f}{2\vec{u}^\top\vec{n}},
\end{align}

Substituting values of $\vec{u}, \vec{n}, \lambda_2 \text{ and } f$ in \eqref{eq:c}, we get
\begin{align}
	c &= \frac{5^2-1\brak{0}}{-2 \myvec{5 & 0}\myvec{1 \\ 0}} = -\frac{5}{2} \\
\end{align}

The focus $\vec{F}$ of parabola is expressed as
\begin{align}
	\vec{F} &= \frac{ce^2\vec{n}-\vec{u}}{\lambda_2} \\
	&= \frac{-\frac{5}{2}\brak{1}^2\myvec{1 \\0} + \myvec{5 \\ 0}}{1} \\
	&= \myvec{\frac{5}{2} \\ 0}
\end{align}

\item  The directrix is given by
\begin{align}
	\vec{n}^\top\vec{x} &= c \\
\implies	\myvec{1 & 0}\vec{x} &= -\frac{5}{2} \\
\end{align}

\item The equation for the axis of parabola passing through $\vec{F}$ and orthogonal to the directrix is given as  
\begin{align}
	\vec{m}^\top\brak{\vec{x}-\vec{F}} &= 0
\end{align}
where $\vec{m}$ is the normal vector to the axis and also the slope of the directrix.
\begin{align}
	\because \vec{n} = \myvec{1 \\ 0 }, \vec{m} &= \myvec{0 \\ 1} \\
	\implies \myvec{0 & 1}\myvec{\vec{x} - \myvec{\frac{5}{2} \\ 0}} &= 0\\
	\text{or, }	\myvec{0 & 1}\vec{x} &= 0 
\end{align}

\item The latus rectum of a parabola is given by 
\begin{align}
	l &= \frac{\eta}{\lambda_2}  
	 = -\frac{2\vec{u}^\top\vec{p_1}}{\lambda_2} \\
	 &= -\frac{2\myvec{-5 & 0}\myvec{1 \\ 0}}{1} \\
	 &= 10 \text{ units }
\end{align}
\end{enumerate}

\begin{figure}[H]
    \centering
    \includegraphics[width=0.75\columnwidth]{figs/parabola.png}
    \caption{Parabola $y^2 = 10x$}
    \label{fig:parabola}
\end{figure}

\end{document}
	\item Find the angles $A, B, C$ if 
%    \label{prop:angle2d}
  \begin{align}
    \label{eq:app-angle2d}
			\cos A \triangleq 
\frac{\brak{\vec{B}-\vec{A}}^{\top}{\vec{C}-\vec{A}}}{\norm{\vec{B}-\vec{A}}\norm{\vec{C}-\vec{A}}}
  \end{align}\\
  \solution
\begin{enumerate}
	\item From 
		\eqref{eq:app-geo-dir-vec-ab},
		\eqref{eq:app-geo-dir-vec-ca},
		\eqref{eq:app-geo-norm-ab}
		and
		\eqref{eq:app-geo-norm-ca}
\begin{align}
	(\vec{B}-\vec{A})^{\top}(\vec{C}-\vec{A})&=\myvec{-5&7}\myvec{-4\\-4}\\
	&=-8
	\\
	\implies
	\cos{A}&= \frac{-8}{\sqrt{74} \sqrt{32}}
	= \frac{-1}{\sqrt{37}}\\
	\implies A&=\cos^{-1}{\frac{-1}{\sqrt{37}}}
\end{align}
	\item From 
		\eqref{eq:app-geo-dir-vec-ab},
		\eqref{eq:app-geo-dir-vec-bc},
		\eqref{eq:app-geo-norm-ab}
		and
		\eqref{eq:app-geo-norm-bc}
\begin{align}
	(\vec{C}-\vec{B})^{\top}(\vec{A}-\vec{B})&=\myvec{1&-11}\myvec{5\\-7}\\
	&= 82
	\\
	\implies
	\cos{B}&= \frac{82}{\sqrt{74} \sqrt{122}}
	= \frac{41}{\sqrt{2257}}\\
	\implies B&=\cos^{-1}{\frac{41}{\sqrt{2257}}}
\end{align}
	\item From 
		\eqref{eq:app-geo-dir-vec-bc},
		\eqref{eq:app-geo-dir-vec-ca},
		\eqref{eq:app-geo-norm-bc}
		and
		\eqref{eq:app-geo-norm-ca}
\begin{align}
	(\vec{A}-\vec{C})^{\top}(\vec{B}-\vec{C})&=\myvec{4&4}\myvec{-1\\11}\\
	&=40
	\\
\implies	\cos{C}&= \frac{40}{\sqrt{32} \sqrt{122}}
	= \frac{5}{\sqrt{61}}\\
	\implies C&=\cos^{-1}{\frac{5}{\sqrt{61}}}
\end{align}

\end{enumerate}
%  	\documentclass[journal,12pt,twocolumn]{IEEEtran}
\usepackage{setspace}
\usepackage{gensymb}
\usepackage{xcolor}
\usepackage{caption}
\singlespacing
\usepackage{siunitx}
\usepackage[cmex10]{amsmath}
\usepackage{mathtools}
\usepackage{hyperref}
\usepackage{amsthm}
\usepackage{mathrsfs}
\usepackage{txfonts}
\usepackage{stfloats}
\usepackage{cite}
\usepackage{cases}
\usepackage{subfig}
\usepackage{longtable}
\usepackage{multirow}
\usepackage{enumitem}
\usepackage{bm}
\usepackage{mathtools}
\usepackage{listings}
\usepackage{tikz}
\usetikzlibrary{shapes,arrows,positioning}
\usepackage{circuitikz}
\renewcommand{\vec}[1]{\boldsymbol{\mathbf{#1}}}
\DeclareMathOperator*{\Res}{Res}
\renewcommand\thesection{\arabic{section}}
\renewcommand\thesubsection{\thesection.\arabic{subsection}}
\renewcommand\thesubsubsection{\thesubsection.\arabic{subsubsection}}

\renewcommand\thesectiondis{\arabic{section}}
\renewcommand\thesubsectiondis{\thesectiondis.\arabic{subsection}}
\renewcommand\thesubsubsectiondis{\thesubsectiondis.\arabic{subsubsection}}
\hyphenation{op-tical net-works semi-conduc-tor}

\lstset{
language=Python,
frame=single, 
breaklines=true,
columns=fullflexible
}
\begin{document}
\theoremstyle{definition}
\newtheorem{theorem}{Theorem}[section]
\newtheorem{problem}{Problem}
\newtheorem{proposition}{Proposition}[section]
\newtheorem{lemma}{Lemma}[section]
\newtheorem{corollary}[theorem]{Corollary}
\newtheorem{example}{Example}[section]
\newtheorem{definition}{Definition}[section]
\newcommand{\BEQA}{\begin{eqnarray}}
        \newcommand{\EEQA}{\end{eqnarray}}
\newcommand{\define}{\stackrel{\triangle}{=}}
\newcommand{\myvec}[1]{\ensuremath{\begin{pmatrix}#1\end{pmatrix}}}
\newcommand{\mydet}[1]{\ensuremath{\begin{vmatrix}#1\end{vmatrix}}}
\bibliographystyle{IEEEtran}
\providecommand{\nCr}[2]{\,^{#1}C_{#2}} % nCr
\providecommand{\nPr}[2]{\,^{#1}P_{#2}} % nPr
\providecommand{\mbf}{\mathbf}
\providecommand{\pr}[1]{\ensuremath{\Pr\left(#1\right)}}
\providecommand{\qfunc}[1]{\ensuremath{Q\left(#1\right)}}
\providecommand{\sbrak}[1]{\ensuremath{{}\left[#1\right]}}
\providecommand{\lsbrak}[1]{\ensuremath{{}\left[#1\right.}}
\providecommand{\rsbrak}[1]{\ensuremath{{}\left.#1\right]}}
\providecommand{\brak}[1]{\ensuremath{\left(#1\right)}}
\providecommand{\lbrak}[1]{\ensuremath{\left(#1\right.}}
\providecommand{\rbrak}[1]{\ensuremath{\left.#1\right)}}
\providecommand{\cbrak}[1]{\ensuremath{\left\{#1\right\}}}
\providecommand{\lcbrak}[1]{\ensuremath{\left\{#1\right.}}
\providecommand{\rcbrak}[1]{\ensuremath{\left.#1\right\}}}
\theoremstyle{remark}
\newtheorem{rem}{Remark}
\newcommand{\sgn}{\mathop{\mathrm{sgn}}}
\newcommand{\rect}{\mathop{\mathrm{rect}}}
\newcommand{\sinc}{\mathop{\mathrm{sinc}}}
\providecommand{\abs}[1]{\left\vert#1\right\vert}
\providecommand{\res}[1]{\Res\displaylimits_{#1}}
\providecommand{\norm}[1]{\lVert#1\rVert}
\providecommand{\mtx}[1]{\mathbf{#1}}
\providecommand{\mean}[1]{E\left[ #1 \right]}
\providecommand{\fourier}{\overset{\mathcal{F}}{ \rightleftharpoons}}
\providecommand{\ztrans}{\overset{\mathcal{Z}}{ \rightleftharpoons}}
\providecommand{\system}[1]{\overset{\mathcal{#1}}{ \longleftrightarrow}}
\newcommand{\solution}{\noindent \textbf{Solution: }}
\providecommand{\dec}[2]{\ensuremath{\overset{#1}{\underset{#2}{\gtrless}}}}
\let\StandardTheFigure\thefigure
\def\putbox#1#2#3{\makebox[0in][l]{\makebox[#1][l]{}\raisebox{\baselineskip}[0in][0in]{\raisebox{#2}[0in][0in]{#3}}}}
\def\rightbox#1{\makebox[0in][r]{#1}}
\def\centbox#1{\makebox[0in]{#1}}
\def\topbox#1{\raisebox{-\baselineskip}[0in][0in]{#1}}
\def\midbox#1{\raisebox{-0.5\baselineskip}[0in][0in]{#1}}

\vspace{3cm}
\title{11.11.2.5}
\author{Lokesh Surana}
\maketitle
\section*{Class 11, Chapter 11, Exercise 2.5}

Q. Find the coordinates of the focus, axis of the parabola, the equation of the directrix and the length of the latus rectum $y^2 = 10x$

\solution
The given equation of the parabola can be rearranged as
\begin{align}
    \label{eq:1} y^2-10x = 0
\end{align}
The above equation can be equated to the generic equation of conic sections
\begin{align}
    \label{eq:2} g\brak{\vec{x}} = \vec{x}^T\vec{V}\vec{x} + 2\vec{u}^T\vec{x} + f = 0 
\end{align}

Comparing coefficients of \eqref{eq:1} and \eqref{eq:2},

\begin{align}
    \label{eq:3}
	\vec{V} &= \myvec{ 0 & 0 \\ 0 & 1} \\
	\label{eq:4}
	\vec{u} &= -\myvec{5 \\ 0} \\
	\label{eq:5}
	f &= 0 
\end{align}

\begin{enumerate}
\item From \eqref{eq:3}, since $\vec{V}$ is already diagonalized, the Eigen values $\lambda_1$ and $\lambda_2$ are given as 
\begin{align}
	\lambda_1 &= 0 \\
	\lambda_2 &= 1 
\end{align}
and the eigenvector matrix
\begin{align}
	\vec{P} = \vec{I}.
\end{align}

\begin{align}
	\therefore 
	\vec{n} &= \sqrt{\lambda_2}\vec{p_1} \\
	&= \myvec{1 \\ 0} 
\end{align}

Since
\begin{align}
	\label{eq:c}
	c = \frac{\norm{\vec{u}}^2-\lambda_2f}{2\vec{u}^\top\vec{n}},
\end{align}

Substituting values of $\vec{u}, \vec{n}, \lambda_2 \text{ and } f$ in \eqref{eq:c}, we get
\begin{align}
	c &= \frac{5^2-1\brak{0}}{-2 \myvec{5 & 0}\myvec{1 \\ 0}} = -\frac{5}{2} \\
\end{align}

The focus $\vec{F}$ of parabola is expressed as
\begin{align}
	\vec{F} &= \frac{ce^2\vec{n}-\vec{u}}{\lambda_2} \\
	&= \frac{-\frac{5}{2}\brak{1}^2\myvec{1 \\0} + \myvec{5 \\ 0}}{1} \\
	&= \myvec{\frac{5}{2} \\ 0}
\end{align}

\item  The directrix is given by
\begin{align}
	\vec{n}^\top\vec{x} &= c \\
\implies	\myvec{1 & 0}\vec{x} &= -\frac{5}{2} \\
\end{align}

\item The equation for the axis of parabola passing through $\vec{F}$ and orthogonal to the directrix is given as  
\begin{align}
	\vec{m}^\top\brak{\vec{x}-\vec{F}} &= 0
\end{align}
where $\vec{m}$ is the normal vector to the axis and also the slope of the directrix.
\begin{align}
	\because \vec{n} = \myvec{1 \\ 0 }, \vec{m} &= \myvec{0 \\ 1} \\
	\implies \myvec{0 & 1}\myvec{\vec{x} - \myvec{\frac{5}{2} \\ 0}} &= 0\\
	\text{or, }	\myvec{0 & 1}\vec{x} &= 0 
\end{align}

\item The latus rectum of a parabola is given by 
\begin{align}
	l &= \frac{\eta}{\lambda_2}  
	 = -\frac{2\vec{u}^\top\vec{p_1}}{\lambda_2} \\
	 &= -\frac{2\myvec{-5 & 0}\myvec{1 \\ 0}}{1} \\
	 &= 10 \text{ units }
\end{align}
\end{enumerate}

\begin{figure}[H]
    \centering
    \includegraphics[width=0.75\columnwidth]{figs/parabola.png}
    \caption{Parabola $y^2 = 10x$}
    \label{fig:parabola}
\end{figure}

\end{document}
All codes for this section are available at
\begin{lstlisting}
	codes/triangle/sides.py
\end{lstlisting}
\end{enumerate}

\subsection{Formulae}
Given
\begin{align}
	c_1 = \frac{7}{3},\,
c_2 = -6.
\end{align}
	From \eqref{eq:parallel_lines},
we need to find $c$ such that,
\begin{align}
	\abs{c-c_1} = \abs{c-c_2} \implies c = \frac{c_1+c_2}{2}
	 = -\frac{11}{6}.
\end{align}
Hence, the desired equation is
\begin{align}
	\myvec{3 & 2}\vec{x} &= -\frac{11}{6}
\end{align}
	See \figref{fig:chapters/11/10/4/21/1}.
\begin{figure}[H]
	\centering
	\includegraphics[width=0.75\columnwidth]{chapters/11/10/4/21/figs/line_plot.jpg}
	\caption{}
	\label{fig:chapters/11/10/4/21/1}
\end{figure}

\subsection{Median}
\input{chapters/triangle/median}
\subsection{Altitude}
\input{chapters/triangle/altitude}
\subsection{Perpendicular Bisector}
\input{chapters/triangle/perp-bisect}
\subsection{Angle Bisector}
\input{chapters/triangle/angle-bisect}
\subsection{Eigenvalues and Eigenvectors}
\input{chapters/triangle/eigen}
\subsection{Formulae}
The parameters for the given conic are
\begin{align}
	\label{12/6/3/25eq:V_matrix}
	\vec{V} &= \myvec{0 & 0\\0 & 1},
	\\
	\label{12/6/3/25eq:u_vector}
	\vec{u} &= \myvec{-3/2\\0},
	\\
	\label{12/6/3/25eq:f_value}
	f &= 2
	%\\
\end{align}
which represent a parabola. 
Following the approach in 
\probref{chapters/12/6/3/15},
   \begin{align}
     \vec{p_1} = \myvec{1\\0},\
     \vec{n} = \myvec{-2\\1},
    \end{align}
yielding the matrix equation
\begin{align}
	\label{12/6/3/25eq:vertex_system}
	\myvec{-3&0\\0& 0\\0& 1}\vec{q} = \myvec{-41/16\\0 \\3/4}\\
\end{align}
The augmented matrix for \eqref{12/6/3/25eq:vertex_system} can be expressed as
\begin{align*}
	%\label{12/6/3/25eq:vertex_solv1}
	%\myvec{-3&0&\vrule&2\\0&0&\vrule&0\\0&1&\vrule&0}\\ 	
	%\label{12/6/3/25eq:vertex_solv2}
	\xleftrightarrow[]{R_2 \leftrightarrow R_3}\myvec{-3&0&\vrule&-41/16\\0&1&\vrule&0\\0&0&\vrule&3/4}
	\xleftrightarrow[]{-\frac{R_1}{-3} \leftarrow R_2}\myvec{1&0&\vrule&41/48\\0&1&\vrule&0\\0&0&\vrule&3/4}\\
	\implies\vec{q} = \myvec{\frac{41}{48}\\\frac{3}{4}}
\end{align*}
The equation of tangent is then obtained as
\begin{align}
	\myvec{-2 & 1}\vec{x} +\frac{23}{24} = 0 
\end{align}
See  
		\figref{fig:12/6/3/25}.
	\begin{figure}[H]
		\centering
 \includegraphics[width=0.75\columnwidth]{chapters/12/6/3/25/figs/conic.pdf}
		\caption{}
		\label{fig:12/6/3/25}
  	\end{figure}

\subsection{Matrices}
\iffalse
\documentclass[10pt, a4paper]{article}
\usepackage[a4paper,outer=1.5cm,inner=1.5cm,top=1.75cm,bottom=1.5cm]{geometry}

\twocolumn
\usepackage{graphicx}

\usepackage{hyperref}
\usepackage[utf8]{inputenc}
\usepackage{amsmath}
\usepackage{physics}
\usepackage{amssymb}
\begin{document}
\title{Assignment-4}
\author{Name:A.SUSI\and Email :  \url{susireddy9121@gmail.com}}
%\{ Wireless Communication (FWC)}
\date{30-sep-2022}
\maketitle



\section{Problem}
\fi
\solution 
\iffalse
\section{Solution}
\begin{center}
The input given 
\boldmath
\fi 
Let
\begin{align} 
\vec{A}=\myvec{ h\\ 0 },
\vec{B}=\myvec{ a\\ b },
\vec{C}=\myvec{ 0\\ k }
\end{align}
Forming the matrix in 
	\eqref{eq:normal_line-collinear}, we obtain, upon row reduction
	\iffalse
\begin{align}
\myvec{ h-a & -b\\ h & -k  } 
\end{align}
Using row reduction, 


In the problem they have given that three points lie on a line, thats means these three points are collinear.\\
If  points on a line  are  collinear, rank of matrix is "1"then the vectors are in linearlydependent.\\
For 2 × 2 matrix Rank =1 means Determinant is 0.\\
Through pivoting,we obtain\\
\fi
\begin{align}\label{eq:}
\myvec{ h-a & -b\\ h & -k  }  
	\xleftrightarrow[]{{\frac{R_1}{h-a}}}\myvec{
1 &\frac{-b}{h-a} \\ 
 h& -k
}
	\\
	\xleftrightarrow[]{R_2\rightarrow R_2-hR_1}
\myvec{
1 &\frac{-b}{h-a} \\ 
 0&-k+\frac{bh}{h-a} 
}
\end{align} 
For obtaining a rank 1 matrix, 
\iffalse

if the rank of the matrix is 1 means any one of the row must be zero.So, making the last element in the matrix to 0.\\
\fi
\begin{align}
	-k+\frac{bh}{h-a}&=0
	\\
	\implies \frac{a}{h}+\frac{b}{k}&=1 
\end{align} 
upon simplification.
\iffalse

Hence proved.\\
\section{Construction}
 \begin{figure}[H]
\centering
\includegraphics[width=0.75\columnwidth]{fig.png} 
\caption{}
\end{figure}
\section{Code}
*Verify the above proofs in the following code.\\
\framebox{
\url{https://github.com/Susi9121/FWC/tree/main/matrix/line}}	
\bibliographystyle{ieeetr}
\end{document}
\fi

\newpage
\section{Conic Section}
\subsection{Equation}
\begin{enumerate}[label=\thesubsection.\arabic*.,ref=\thesubsection.\theenumi]
\item
  Let $\vec{q}$ be a point such that the ratio of its distance from a fixed point $\vec{F}$ and the distance ($d$) from a fixed line 
	\begin{align}
L: \vec{n}^{\top}\vec{x}=c 
	\end{align}
		is constant, given by 
\label{conics/30/def}
\begin{align}
\frac{\norm{\vec{q}-\vec{F}}}{d} = e    
\end{align}
The locus of $\vec{q}$ is known as a conic section. The line $L$ is known as the directrix and the point $\vec{F}$ is the focus. $e$ is defined to be 
the eccentricity of the conic.  
\begin{enumerate}
    \item For $e = 1$, the conic is a parabola
    \item For $e < 1$, the conic is an ellipse
    \item For $e > 1$, the conic is a hyperbola
\end{enumerate}

\item
The equation of  a conic with directrix $\vec{n}^{\top}\vec{x} = c$, eccentricity $e$ and focus $\vec{F}$ is given by 
\begin{align}
    \label{eq:app-conic_quad_form}
	\text{g}\brak{\vec{x}} = \vec{x}^{\top}\vec{V}\vec{x}+2\vec{u}^{\top}\vec{x}+f=0
    \end{align}
where     
\begin{align}
  \label{eq:app-conic_quad_form_v}
\vec{V} &=\norm{\vec{n}}^2\vec{I}-e^2\vec{n}\vec{n}^{\top}, 
\\
\label{eq:app-conic_quad_form_u}
\vec{u} &= ce^2\vec{n}-\norm{\vec{n}}^2\vec{F}, 
\\
\label{eq:app-conic_quad_form_f}
f &= \norm{\vec{n}}^2\norm{\vec{F}}^2-c^2e^2
    \end{align}
    \solution
  Using Definition \ref{conics/30/def} and 
			\eqref{eq:PQ-final},
for any point $\vec{x}$ on the conic,
\begin{align}
	\norm{\vec{x}-\vec{F}}^2&=e^2 \frac{\brak{{\vec{n}^{\top}\vec{x} - c}}^2}{\norm{\vec{n}}^2}\label{conics/30/eq:1} \\
	\implies \norm{\vec{n}}^2\brak{\vec{x}-\vec{F}}^{\top}\brak{\vec{x}-\vec{F}}&=e^2\brak{\vec{n}^{\top}\vec{x} - c}^2
\\
\implies \norm{\vec{n}}^2\brak{\vec{x}^{\top}\vec{x}-2\vec{F}^{\top}\vec{x}+\norm{\vec{F}}^2}
	&=e^2\brak{c^2+\brak{\vec{n}^{\top}\vec{x} }^2-2c\vec{n}^{\top}\vec{x}} \\
	&=e^2\brak{c^2+\brak{\vec{x}^{\top}\vec{n}\vec{n}^{\top}\vec{x} }-2c\vec{n}^{\top}\vec{x}}
\end{align}
%
which can be expressed as \eqref{eq:app-conic_quad_form} after simplification.
\item
  The eccentricity, directrices and foci of \eqref{eq:app-conic_quad_form} are given by 
\begin{align}
  \label{eq:app-conic_quad_form_e} 
  e&= \sqrt{1-\frac{\lambda_1}{\lambda_2}}
\\
\label{eq:app-conic_quad_form_nc} 
	\begin{split}
  \vec{n}&= \sqrt{\lambda_2}\vec{p}_1,  
  \\
	c &= 
  \begin{cases}
    \frac{e\vec{u}^{\top}\vec{n} \pm \sqrt{e^2\brak{\vec{u}^{\top}\vec{n}}^2-\lambda_2\brak{e^2-1}\brak{\norm{\vec{u}}^2 - \lambda_2 f}}}{\lambda_2e\brak{e^2-1}} & e \ne 1
    \\
    \frac{\norm{\vec{u}}^2 - \lambda_2 f   }{2\vec{u}^{\top}\vec{n}} & e = 1
  \end{cases}
	\end{split}
  \\
  \label{eq:app-conic_quad_form_F} 
  \vec{F}  &= \frac{ce^2\vec{n}-\vec{u}}{\lambda_2}
\end{align}  
	\label{app:conic-parameters}
	\solution
	From \eqref{eq:app-conic_quad_form_v}, using the fact that $\vec{V}$ is symmetric with $\vec{V} = \vec{V}^{\top}$,
  \begin{align}
	  \vec{V}^{\top} \vec{V}&=\brak{\norm{\vec{n}}^2\vec{I}-e^2\vec{n}\vec{n}^{\top}}^{\top}
	  \brak{\norm{\vec{n}}^2\vec{I}-e^2\vec{n}\vec{n}^{\top}}
    \\
	  \implies \vec{V}^{2} &= \norm{\vec{n}}^4\vec{I}+e^4\vec{n}\vec{n}^{\top}\vec{n}\vec{n}^{\top}
	  -2e^2\norm{\vec{n}}^2\vec{n}\vec{n}^{\top}
    \\
	  &= \norm{\vec{n}}^4\vec{I} + e^4\norm{\vec{n}}^2\vec{n}\vec{n}^{\top}
	%  \\
	  - 2e^2\norm{\vec{n}}^2\vec{n}\vec{n}^{\top}
    \\
	  &= \norm{\vec{n}}^4\vec{I} + e^2\brak{e^2 - 2}\norm{\vec{n}}^2\vec{n}\vec{n}^{\top}
    \\
	  &= \norm{\vec{n}}^4\vec{I} + \brak{e^2 - 2}\norm{\vec{n}}^2\brak{\norm{\vec{n}}^2\vec{I}- \vec{V}}
    \end{align}
%    
which can be expressed as
\begin{align}
  \vec{V}^{2} + \brak{e^2 - 2}\norm{\vec{n}}^2\vec{V} - \brak{e^2 - 1}\norm{\vec{n}}^4\vec{I}=0
  \label{eq:app-conic_quad_form_e_cayley}
\end{align}
	Using the Cayley-Hamilton theorem,
	\eqref{eq:app-conic_quad_form_e_cayley} results in the characteristic equation, 
\begin{align}
  \lambda^{2} - \brak{2-e^2}\norm{\vec{n}}^2\lambda + \brak{1-e^2 }\norm{\vec{n}}^4=0
\end{align}
which can be expressed as
\begin{align}
\brak{\frac{\lambda}{\norm{\vec{n}}^2}}^2 - \brak{2-e^2 }\brak{\frac{\lambda}{\norm{\vec{n}}^2}} 
	+ \brak{1-e^2 } = 0
	\\
	\implies \frac{\lambda}{\norm{\vec{n}}^2} = 1-e^2, 1
  \\
	\text{or, }\lambda_2 = \norm{\vec{n}}^2, \lambda_1 = \brak{1-e^2}\lambda_2 
  \label{eq:app-conic_quad_form_lam_cayley}
\end{align}
From   \eqref{eq:app-conic_quad_form_lam_cayley}, the eccentricity of \eqref{eq:app-conic_quad_form} is given by 
\eqref{eq:app-conic_quad_form_e}.   
%
Multiplying both sides of    \eqref{eq:app-conic_quad_form_v} by $\vec{n}$,
\begin{align}
\vec{V} \vec{n}&=\norm{\vec{n}}^2\vec{n}-e^2\vec{n}\vec{n}^{\top}\vec{n} 
\\
&=\norm{\vec{n}}^2\brak{1-e^2}\vec{n} 
 \\
  &=\lambda_1 \vec{n} 
	\\
  \label{eq:eigevecn}
\end{align}  
from \eqref{eq:app-conic_quad_form_lam_cayley}.
Thus,  $\lambda_1$ is the corresponding eigenvalue for $\vec{n}$.  From       \eqref{eq:eigevecP} and \eqref{eq:eigevecn}, this implies that 
\begin{align}  
	\vec{p}_1 &= \frac{\vec{n}}{\norm{\vec{n}}} 
	\\
	\text{or, }
   \vec{n}&= \norm{\vec{n}}\vec{p}_1  = \sqrt{\lambda_2}\vec{p}_1 
\end{align}  
from   \eqref{eq:app-conic_quad_form_lam_cayley} .
From \eqref{eq:app-conic_quad_form_u} and \eqref{eq:app-conic_quad_form_lam_cayley},
\begin{align}
\vec{F}  &= \frac{ce^2\vec{n}-\vec{u}}{\lambda_2}
 \\
 \implies \norm{\vec{F}}^2  &= \frac{\brak{ce^2\vec{n}-\vec{u}}^{\top}\brak{ce^2\vec{n}-\vec{u}}}{\lambda_2^2}
 \\
 \implies \lambda_2^2\norm{\vec{F}}^2  &= c^2e^4\lambda_2-2ce^2\vec{u}^{\top}\vec{n}+\norm{\vec{u}}^2
 \label{eq:app-conic_quad_form_u_temp}
    \end{align}
    Also, \eqref{eq:app-conic_quad_form_f} can be expressed as
    \begin{align}
    \lambda_2\norm{\vec{F}}^2 &= f+c^2e^2
    \label{eq:app-conic_quad_form_f_temp}
\end{align}
From  \eqref{eq:app-conic_quad_form_u_temp} and     \eqref{eq:app-conic_quad_form_f_temp},
\begin{align}
c^2e^4\lambda_2-2ce^2\vec{u}^{\top}\vec{n}+\norm{\vec{u}}^2 = \lambda_2\brak{f+c^2e^2}
\end{align}
\begin{align}
\implies \lambda_2e^2\brak{e^2-1}c^2-2ce^2\vec{u}^{\top}\vec{n}
	+\norm{\vec{u}}^2 - \lambda_2 f = 0
\end{align}
yielding
  \eqref{eq:app-conic_quad_form_F}. 
\item
\eqref{eq:app-conic_quad_form} represents 
	\begin{enumerate}
		\item a parabola for $\mydet{\vec{V}} = 0 $,
		\item ellipse for $\mydet{\vec{V}} > 0 $ and 
		\item hyperbola for $\mydet{\vec{V}} < 0 $.
	\end{enumerate}
\solution
  From \eqref{eq:app-conic_quad_form_e},
\begin{align}
  \frac{\lambda_1}{\lambda_2} = 1 - e^2
\end{align}
Also, 
\begin{align}
	\mydet{\vec{V}} =   \lambda_1\lambda_2 
\end{align}
	yielding \tabref{table:det}.
\begin{table}[H]
\centering
\resizebox{\columnwidth}{!}{%
\input{tables/det.tex}
	}
	\caption{}
\label{table:det}
\end{table}
			\item Using the affine transformation in
					\label{app:std-prm-P}
	\eqref{eq:conic_affine},
	the conic in     \eqref{eq:app-conic_quad_form} can be expressed in standard form 
	%(centre/vertex at the origin, major axis - $x$ axis)
	as
  \begin{align}
    %\begin{aligned}
    \label{eq:app-conic_simp_temp_nonparab}
	    \vec{y}^{\top}\brak{\frac{\vec{D}}{f_0}}\vec{y} &= 1   &  \abs{\vec{V}} &\ne 0
    \\
	    \vec{y}^{\top}\vec{D}\vec{y} &=  -\eta\vec{e}_1^{\top}\vec{y}   & \abs{\vec{V}} &= 0
    \label{eq:app-conic_simp_temp_parab}
    %\end{aligned}
    \end{align}
    where
  \begin{align}
      %\begin{split}
      \label{eq:app-f0}
	  f_0 &=\vec{u}^{\top}\vec{V}^{-1}\vec{u} -f \ne 0
	  \\
      \label{eq:app-eta}
       \eta &=2\vec{u}^{\top}\vec{p}_1
       \\
       \vec{e}_1 &=\myvec{1 \\ 0}
      \end{align}
      \solution
  \label{app:parab}
	Using 
\eqref{eq:conic_affine},
\eqref{eq:app-conic_quad_form} can be expressed as
\begin{align}
\brak{\vec{P}\vec{y}+\vec{c}}^{\top}\vec{V}\brak{\vec{P}\vec{y}+\vec{c}}+2\vec{u}^{\top}\brak{\vec{P}\vec{y}+\vec{c}}+ f
	= 0, 
\end{align}
yielding 
\begin{align}
\vec{y}^{\top}\vec{P}^{\top}\vec{V}\vec{P}\vec{y}+2\brak{\vec{V}\vec{c}+\vec{u}}^{\top}\vec{P}\vec{y}
+  \vec{c}^{\top}\vec{V}\vec{c} + 2\vec{u}^{\top}\vec{c} + f= 0
\label{eq:conic_simp_one}
\end{align}
%
From \eqref{eq:conic_simp_one} and \eqref{eq:conic_parmas_eig_def},
\begin{align}
\vec{y}^{\top}\vec{D}\vec{y}+2\brak{\vec{V}\vec{c}+\vec{u}}^{\top}\vec{P}\vec{y}
+  \vec{c}^{\top}\brak{\vec{V}\vec{c} + \vec{u}}+ \vec{u}^{\top}\vec{c} + f= 0
\label{eq:conic_simp}
\end{align}
When $\vec{V}^{-1}$ exists, choosing
\begin{align}
%\begin{split}
\vec{V}\vec{c}+\vec{u} &= \vec{0}, \quad \text{or}, \vec{c} = -\vec{V}^{-1}\vec{u},
\label{eq:conic_parmas_c_def}
\end{align}
%
%%From \eqref{eq:conic_parmas_k_def} and 
%%
and substituting \eqref{eq:conic_parmas_c_def}
in \eqref{eq:conic_simp}
yields \eqref{eq:app-conic_simp_temp_nonparab}. 
  %See Appendix \ref{app:parab}.
When $\abs{\vec{V}} = 0, \lambda_1 = 0$ and 
\begin{align}
\vec{V}\vec{p}_1 = 0, 
\vec{V}\vec{p}_2 = \lambda_2\vec{p}_2.
\label{eq:conic_parab_eig_prop} 
\end{align}
Substituting \eqref{eq:eig_matrix}
in \eqref{eq:conic_simp},
\begin{align*}
	\vec{y}^{\top}\vec{D}\vec{y}+2\brak{\vec{c}^{\top}\vec{V}+\vec{u}^{\top}}\myvec{\vec{p}_1 & \vec{p}_2}\vec{y}
%	\\
	+  \vec{c}^{\top}\brak{\vec{V}\vec{c} + \vec{u}}+ \vec{u}^{\top}\vec{c} + f= 0
\\
\implies \vec{y}^{\top}\vec{D}\vec{y}
+2\myvec{\brak{\vec{c}^{\top}\vec{V}+\vec{u}^{\top}}\vec{p}_1  \brak{\vec{c}^{\top}\vec{V}+\vec{u}^{\top}}\vec{p}_2}\vec{y}
%\\
	+  \vec{c}^{\top}\brak{\vec{V}\vec{c} + \vec{u}}+ \vec{u}^{\top}\vec{c} + f= 0
\\
\implies \vec{y}^{\top}\vec{D}\vec{y}
+2\myvec{\vec{u}^{\top}\vec{p}_1 & \brak{\lambda_2\vec{c}^{\top}+\vec{u}^{\top}}\vec{p}_2}\vec{y}
%\\
	+  \vec{c}^{\top}\brak{\vec{V}\vec{c} + \vec{u}}+ \vec{u}^{\top}\vec{c} + f= 0
\end{align*}
upon substituting from 
 \eqref{eq:conic_parab_eig_prop}, yielding
\begin{multline}
\lambda_2y_2^2+2\brak{\vec{u}^{\top}\vec{p}_1}y_1+  2y_2\brak{\lambda_2\vec{c}+\vec{u}}^{\top}\vec{p}_2
%\\
	+  \vec{c}^{\top}\brak{\vec{V}\vec{c} + \vec{u}}+ \vec{u}^{\top}\vec{c} + f= 0
\label{eq:conic_parab_foc_len_temp} 
\end{multline}
%which is the equation of a parabola. 
Thus, \eqref{eq:conic_parab_foc_len_temp} 
can be expressed as \eqref{eq:app-conic_simp_temp_parab} by choosing
\begin{align}
%\label{eq:app-eta}
\eta = 2\vec{u}^{\top}\vec{p}_1
\end{align}
%Choosing 
%\begin{align}
%\vec{u} + \lambda_2\vec{c} = 0,
%\vec{c}^{\top}\brak{\vec{V}\vec{c} + \vec{u}}+ \vec{u}^{\top}\vec{c} + f = 0,
%\end{align}
% the above equation becomes
%\begin{align}
%y_2^2= -\frac{2\vec{u}^{\top}\vec{p}_1}{ \lambda_2} \brak{y_1
%+  \frac{\vec{u}^{\top}\vec{V}\vec{u} - 2\lambda_2\vec{u}^{\top}\vec{u} + f\lambda_2^2}{2\vec{u}^{\top}\vec{p}_1\lambda_2^2}}
%\\
%or \eta = 2\vec{u}^{\top}\vec{p}_1
%%\label{eq:conic_simp_parab_new}
%\end{align}
and $\vec{c}$ in \eqref{eq:conic_simp} such that
\begin{align}
\label{eq:conic_parab_one}
2\vec{P}^{\top}\brak{\vec{V}\vec{c}+\vec{u}} &= \eta\myvec{1\\0}
\\
\vec{c}^{\top}\brak{\vec{V}\vec{c} + \vec{u}}+ \vec{u}^{\top}\vec{c} + f&= 0
\label{eq:conic_parab_two}
\end{align}
%we obtain  \eqref{eq:app-conic_simp_temp_parab}.
	\item
		The center/vertex of a conic section are given by
  \begin{align}
    \label{eq:app-conic_nonparab_c}
	    \vec{c} &= - \vec{V}^{-1}\vec{u}  & \mydet{\vec{V}} \ne 0
    \\
	    \myvec{ \vec{u}^{\top}+\frac{\eta}{2}\vec{p}_1^{\top} \\ \vec{v}}\vec{c} &= \myvec{-f \\ \frac{\eta}{2}\vec{p}_1-\vec{u}}  
& \mydet{\vec{V}} = 0
    \label{eq:conic_parab_c}
    \end{align}	
\solution	
$\because
\vec{P}^{\top}\vec{P} = \vec{I}$,
multiplying \eqref{eq:conic_parab_one} by $\vec{P}$ yields
\begin{align}
\label{eq:conic_parab_one_eig}
	\brak{\vec{V}\vec{c}+\vec{u}} &= \frac{\eta}{2}\vec{p}_1,
\end{align}
which, upon substituting in \eqref{eq:conic_parab_two}
results in 
\begin{align}
\frac{\eta}{2}\vec{c}^{\top}\vec{p}_1 + \vec{u}^{\top}\vec{c} + f&= 0
\label{eq:conic_parab_two_eig}
\end{align}
\eqref{eq:conic_parab_one_eig} and \eqref{eq:conic_parab_two_eig} can be clubbed together to obtain \eqref{eq:conic_parab_c}.
\item
			In 
			\eqref{eq:conic_affine}, substituting $\vec{y} = \vec{0}$, the center/vertex for the quadratic form is obtained as
    \begin{align}
	    \vec{x} = \vec{c}, 
    \end{align}
			where $\vec{c}$ is derived as 
    \eqref{eq:app-conic_nonparab_c}
    and 
    \eqref{eq:conic_parab_c}
in Appendix  \ref{app:parab}.
		\end{enumerate}
\subsection{Standard Conic}
\begin{enumerate}[label=\thesubsection.\arabic*.,ref=\thesubsection.\theenumi]
	  \item
		For the standard conic, 
				\begin{align}
					\label{eq:app-std-prm-P}
					\vec{P} &= \vec{I}
					\\
					\vec{u} &= 
				\begin{cases}
				0 & e \ne 1
       \\
				\frac{\eta}{2} \vec{e}_1 & e = 1
				\end{cases}
				\label{eq:std-prm-u}
				\\
				\lambda_1 &  
					\begin{cases}
						=0 & e = 1
						\\
						\ne 0 & e \ne 1
					\end{cases}
				\label{eq:std-prm-lam1}
				\end{align}
				where 
				\begin{align}
					\vec{I} = \myvec{\vec{e}_1 & \vec{e}_2}
				\end{align}
				is the identity matrix.
	\item
			\label{corr:center}
			The center of the standard ellipse/hyperbola, defined to be the mid point of the line joining the foci, is the origin.
	
	\item
		\label{corr:axis}
			The principal (major) axis of the standard ellipse/hyperbola, defined to be the line joining the two foci   is the $x$-axis.  
	
	\begin{proof}
		From 	\eqref{eq:app-F-ell-hyp-parab}, it is obvious that the line joining the foci passes through the origin.  Also, the direction vector of this line is $\vec{e}_1$.  Thus, the principal axis is the $x$-axis. 
	\end{proof}
	\item
		\label{corr:minor-axis}
			The minor axis of the standard ellipse/hyperbola, defined to be the line orthogonal to the $x$-axis is the $y$-axis. 
	


	\item
			The axis of symmetry of the standard parabola, defined to be the line perpendicular to the directrix and passing through the focus,  is the $x$- axis.
	
	\begin{proof}
	From \eqref{eq:n-parab} and 	
					\eqref{eq:app-F-ell-hyp-parab}, 
					the axis of the parabola  can be expressed 
     as 
		\begin{align}
			\vec{e}_2^{\top}\brak{\vec{y}  
			+\frac{\eta}{4\lambda_2}\vec{e}_1} &= 0
			\\
			\implies \vec{e}_2^{\top}\vec{y} &= 0
					\label{eq:axis-std-parab}, 
		\end{align}
		which is the equation of the $x$-axis.
	\end{proof}


	\item
			\label{corr:center-parab}
 The point where the parabola intersects its axis of symmetry is called the vertex. For the standard parabola, the vertex is the origin.
	
	\begin{proof}
					\eqref{eq:axis-std-parab} can be expressed as 
    \begin{align}
			\vec{y}= \alpha \vec{e}_1. 
					\label{eq:axis-std-parab-dir} 
    \end{align}
					Substituting \eqref{eq:axis-std-parab-dir} in 
    \eqref{eq:app-conic_simp_temp_parab}, 
    \begin{align}
	     \alpha^2 \vec{e}_1^{\top}\vec{D} \vec{e}_1 &=  -\eta\alpha \vec{e}_1^{\top} \vec{e}_1   
	     \\
	     \implies \alpha &=0, \text{ or, } \vec{y} = \vec{0}.
    %\end{aligned}
    \end{align}
	\end{proof}
    \item\leavevmode
		\begin{enumerate}
			\item The directrices for the  standard conic are given by 
				\begin{align}
					\label{eq:app-dx-ell-hyp}
					\vec{e}_1^{\top}\vec{y} &=  
					%\pm\sqrt{\abs{\frac{f_0\lambda_2}{\lambda_1\brak{\lambda_2-\lambda_1}}}} & e \ne 1
					\pm \frac{1}{e}\sqrt{\frac{\abs{f_0}}{\lambda_2\brak{1-e^2}}} & e \ne 1
					\\
					\vec{e}_1^{\top}\vec{y} &= \frac{\eta}{2\lambda_2} & e = 1
					\label{eq:app-dx-parab}
				\end{align}
    \item The foci of the standard ellipse and hyperbola are given by 
				\begin{align}
					\label{eq:app-F-ell-hyp-parab}
					\vec{F} 
=
					\begin{cases}
						\pm e\sqrt{\frac{\abs{f_0}}{\lambda_2\brak{1-e^2}}}\vec{e}_1 & e \ne 1
					%	\pm \sqrt{\abs{\frac{f_0}{\lambda_1}\brak{1 - \frac{\lambda_1}{\lambda_2}}}}\vec{e}_1 & e \ne 1
						\\
						 -\frac{\eta}{4\lambda_2}\vec{e}_1 & e = 1
					\end{cases}
				\end{align}
	
		\end{enumerate}
	%	where, without loss of generality, $f_0 < 0$ for the hyperbola.
    
	\begin{proof}%\leavevmode
  \label{app:foc-dir}
%  \input{appendix.tex}
		\begin{enumerate}
			\item For the standard hyperbola/ellipse in \eqref{eq:app-conic_simp_temp_nonparab}, from 
					\eqref{eq:app-std-prm-P},
\eqref{eq:app-conic_quad_form_nc}
and 
					\eqref{eq:std-prm-u},
				\begin{align}
\label{eq:n-ell-hyp}
					\vec{n} &= \sqrt{\frac{\lambda_2}{f_0}} \vec{e}_1 
					\\
					c &= 
					%\pm \frac{\sqrt{-\lambda_2\brak{e^2-1}\brak{\lambda_2 f_0}}}{\lambda_2e\brak{e^2-1}}
					\pm \frac{\sqrt{-\frac{\lambda_2}{f_0}\brak{e^2-1}\brak{\frac{\lambda_2}{ f_0}}}}{\frac{\lambda_2}{f_0}e\brak{e^2-1}}
					\\
					&=\pm \frac{1}{e\sqrt{1-e^2}}
%					\\
%					&=\pm\sqrt{\abs{\frac{f_0}{\brak{1 - \frac{\lambda_1}{\lambda_2}}\frac{\lambda_1}{\lambda_2}}}}
\label{eq:c-ell-hyp}
				\end{align}
				yielding 
					\eqref{eq:app-dx-ell-hyp} upon substituting from 
\eqref{eq:app-conic_quad_form_e} and simplifying.
For the standard parabola in \eqref{eq:app-conic_simp_temp_parab},  from 
					\eqref{eq:app-std-prm-P},
\eqref{eq:app-conic_quad_form_nc}
and 
					\eqref{eq:std-prm-u}, noting that $f = 0$,

				\begin{align}
\label{eq:n-parab}
					\vec{n} &= \sqrt{\lambda_2} \vec{e}_1 
					\\
					c &=
	\frac{\norm{\frac{\eta}{2} \vec{e}_1}^2   }{2\vec{\brak{\frac{\eta}{2}} \brak{\vec{e}_1}^{\top}\vec{n}}} 
\\
					\\
					&= \frac{\eta}{4\sqrt{\lambda_2}}
\label{eq:c-parab}
				\end{align}
				yielding 
					\eqref{eq:app-dx-parab}.

				\item 	For the standard ellipse/hyperbola, substituting from
\eqref{eq:c-ell-hyp},
\eqref{eq:n-ell-hyp},
\eqref{eq:std-prm-u}
and \eqref{eq:app-conic_quad_form_e}
in \eqref{eq:app-conic_quad_form_F},
				\begin{align}
					\vec{F} &= \pm \frac{\brak{\frac{1}{e\sqrt{1-e^2}}}\brak{e^2}\sqrt{\frac{\lambda_2}{f_0}}\vec{e}_1}{\frac{\lambda_2}{f_0}}
					%\pm\sqrt{\abs{\frac{f_0}{\brak{1 - \frac{\lambda_1}{\lambda_2}}\frac{\lambda_1}{\lambda_2}}}}
					%\brak{1 - \frac{\lambda_1}{\lambda_2}}\frac{\sqrt{\lambda_2}}{\lambda_2}\vec{e}_1
 			\end{align}
			yielding
					\eqref{eq:app-F-ell-hyp-parab}
					after simplification.
					For the standard parabola, substituting from 
\eqref{eq:c-parab},
\eqref{eq:n-parab},
\eqref{eq:std-prm-u}
and \eqref{eq:app-conic_quad_form_e}
in \eqref{eq:app-conic_quad_form_F},			
				\begin{align}
	\vec{F}  &= \frac{\brak{\frac{\eta}{4\sqrt{\lambda_2}}}\sqrt{\lambda_2}\vec{e}_1-\vec{\frac{\eta}{2} \vec{e}_1}}{\lambda_2}
\\
				\end{align}
				yielding 
					\eqref{eq:app-F-ell-hyp-parab} after simplification.
		\end{enumerate}
	\end{proof}
	\item
			\label{corr:foclen}
	 The {\em focal length} of the standard parabola, , defined to be the distance between the vertex and the focus, measured along the axis of symmetry, is $\abs{\frac{\eta}{4 \lambda_2}}$


		\end{enumerate}
\subsection{Conic Lines}
\begin{enumerate}[label=\thesubsection.\arabic*.,ref=\thesubsection.\theenumi]
 \item
	 \label{prop:chord}
  The points of intersection of the line 
\begin{align}
L: \quad \vec{x} = \vec{h} + \kappa \vec{m} \quad \kappa \in \mathbb{R}
\label{eq:app-conic_tangent}
\end{align}
with the conic section in \eqref{eq:app-conic_quad_form} are given by
\begin{align}
\vec{x}_i = \vec{h} + \kappa_i \vec{m}
	\label{eq:app-chord-pts}
\end{align}
%
where
\begin{multline}
\kappa_i = \frac{1}
{
\vec{m}^{\top}\vec{V}\vec{m}
}
\lbrak{-\vec{m}^{\top}\brak{\vec{V}\vec{h}+\vec{u}}}
%\\
\pm
%{\small
\rbrak{\sqrt{
\sbrak{
\vec{m}^{\top}\brak{\vec{V}\vec{h}+\vec{u}}
}^2
	-\text{g}
\brak
{\vec{h}
%\vec{h}^{\top}\vec{V}\vec{h} + 2\vec{u}^{\top}\vec{h} +f
}
\brak{\vec{m}^{\top}\vec{V}\vec{m}}
}
}
%}
\label{eq:app-tangent_roots}
\end{multline}
\solution
  Substituting \eqref{eq:app-conic_tangent}
in \eqref{eq:app-conic_quad_form}, 
\begin{align}
\brak{\vec{h} + \kappa \vec{m}}^{\top}\vec{V}\brak{\vec{h} + \kappa \vec{m}}  + 2 \vec{u}^{\top}\brak{\vec{h} + \kappa \vec{m}}+f &= 0
\\
\implies \kappa^2\vec{m}^{\top}\vec{V}\vec{m} + 2 \kappa\vec{m}^{\top}\brak{\vec{V}\vec{h}+\vec{u}} 
+ \vec{h}^{\top}\vec{V}\vec{h} + 2\vec{u}^{\top}\vec{h} +f &= 0
	\\
	\text{or, }
\kappa^2\vec{m}^{\top}\vec{V}\vec{m} + 2 \kappa\vec{m}^{\top}\brak{\vec{V}\vec{h}+\vec{u}} 
	+ \text{g}\brak{\vec{h}} &=0
	%^{\top}\vec{V}\vec{h} + 2\vec{u}^{\top}\vec{h} +f &= 0
\label{eq:conic_intercept}
\end{align}
for g defined in \eqref{eq:app-conic_quad_form}.
Solving the above quadratic in \eqref{eq:conic_intercept}
yields \eqref{eq:app-tangent_roots}.
	\item
		The length of the chord in 
\eqref{eq:app-conic_tangent}
is given by 
\begin{align}
 \frac{2\sqrt{
\sbrak{
\vec{m}^{\top}\brak{\vec{V}\vec{h}+\vec{u}}
}^2
-
\brak
{
\vec{h}^{\top}\vec{V}\vec{h} + 2\vec{u}^{\top}\vec{h} +f
}
\brak{\vec{m}^{\top}\vec{V}\vec{m}}
}
}
{
\vec{m}^{\top}\vec{V}\vec{m}
}\norm{\vec{m}}
\label{eq:chord-len}
  \end{align}
	
\begin{proof}
The distance between the points in 
	\eqref{eq:app-chord-pts}
is given by 
\begin{align}
	\norm{\vec{x}_1-\vec{x}_2} =  \abs{\kappa_1-\kappa_2} \norm{\vec{m}}
\label{eq:app-conic_tangent_pts_dist}
\end{align}
Substituing $\kappa_i$ from 
\eqref{eq:app-tangent_roots} in
\eqref{eq:app-conic_tangent_pts_dist}
yields
	\eqref{eq:chord-len}.
\end{proof}
	\item
 The affine transform for the conic section, preserves the norm.  This implies that the length of any chord of a conic
	is invariant to translation and/or rotation.
	
	\begin{proof}
	Let 
%From \eqref{eq:conic_affine}, 
\begin{align}
\vec{x}_i = \vec{P}\vec{y}_i+\vec{c} 
\label{eq:conic_affine_pts}
\end{align}
be any two points on the conic.  Then the distance between the points is given by 
\begin{align}
	\norm{\vec{x}_1-\vec{x}_2 } &= \norm{\vec{P}\brak{	\vec{y}_1 -\vec{y}_2 }}
\end{align}
which can be expressed as 
\begin{align}
	\norm{\vec{x}_1-\vec{x}_2 }^2 &= 		\brak{\vec{y}_1 -\vec{y}_2 }^{\top}\vec{P}^{\top}\vec{P}\brak{\vec{y}_1 -\vec{y}_2 }
	\\
	&= 		\norm{\vec{y}_1 -\vec{y}_2 }^2
\label{eq:conic_affine_norm_preserve}
\end{align}
since 
\begin{align}
	\vec{P}^{\top}\vec{P} = \vec{I}
\end{align}
	\end{proof}
    \item For the standard hyperbola/ellipse, the length of the major axis is 
  \begin{align}
\label{eq:app-chord-len-major}
 2\sqrt{\abs{\frac{
f_0}
{\lambda_1}
	  }}
  \end{align}
  and the minor axis is 
  \begin{align}
\label{eq:app-chord-len-minor}
 2\sqrt{\abs{\frac{
f_0}
{\lambda_2}
	  }}
  \end{align}
		\label{app:major}
%	See Appendix \ref{app:major}
		\solution
		Since the major axis passes through the origin, 
  \begin{align}
	  \vec{q} =			\vec{0} 
\end{align}  
Further, from Corollary  
		\eqref{corr:axis},
  \begin{align}
  \vec{m}&= \vec{e}_2,  
\end{align} and
from 
    \eqref{eq:app-conic_simp_temp_nonparab},
  \begin{align}
	  \vec{V} =     \frac{\vec{D} }{f_0}, 
	   \vec{u} = 0, 
	   f = -1
	    \label{eq:latus_rectum_ellipse_param}
\end{align}  
Substituting the above in
\eqref{eq:chord-len}, 
\begin{align}
 \frac{2\sqrt{
\vec{e}_1^{\top}\frac{\vec{D}}{f_0}\vec{e}_1
}
}
{
\vec{e}_1^{\top}\frac{\vec{D}}{f_0}\vec{e}_1
}\norm{\vec{e}_1}
  \end{align}
  yielding 
\eqref{eq:app-chord-len-major}.
Similarly, for the minor axis, the only different parameter is 
  \begin{align}
  \vec{m}&= \vec{e}_2,  
\end{align} 
Substituting the above in
\eqref{eq:chord-len}, 
\begin{align}
 \frac{2\sqrt{
\vec{e}_2^{\top}\frac{\vec{D}}{f_0}\vec{e}_2
}
}
{
\vec{e}_2^{\top}\frac{\vec{D}}{f_0}\vec{e}_2
}\norm{\vec{e}_2}
  \end{align}
  yielding 
\eqref{eq:app-chord-len-minor}.
    \item The equation of the minor and major  axes for the ellipse/hyperbola are respectively given by 
  \begin{align}
\vec{p}_i^{\top}\brak{\vec{x}-\vec{c}} = 0, i = 1,2
	  \label{eq:app-major-minor-axis-quad}
  \end{align}
  The axis of symmetry for the parabola is also given by 
	  \eqref{eq:app-major-minor-axis-quad}.

		\begin{proof}
From		\eqref{corr:axis}, the major/symmetry axis for the hyperbola/ellipse/parabola can be expressed using 
	\eqref{eq:conic_affine}
 as
  \begin{align}
	  \vec{e}_2^{\top}
		  \vec{P}^{\top}\brak{\vec{x}-\vec{c}} &= 0
		  \\
	  \implies 		  \brak{\vec{P}\vec{e}_2}^{\top}\brak{\vec{x}-\vec{c}} &= 0
  \end{align}
yielding	  \eqref{eq:app-major-minor-axis-quad}, and the proof for the minor axis is similar.
		\end{proof}
\item
    The latus rectum of a conic section is the chord that passes through the focus and is perpendicular to the major axis.
	The length of the latus rectum for a conic is given by
		\begin{align}
			l =
			\begin{cases}
				2\frac{\sqrt{\abs{f_0\lambda_1}}}{\lambda_2} & e \ne 1
			\\
			\frac{\eta}{\lambda_2} & e = 1
			\end{cases}
			\label{eq:app-latus-ellipse}
		\end{align}
%			See Appendix \ref{app:latus}.
		%\section{}
		\label{app:latus}
		\solution
			The latus rectum is perpendicular to the major axis for the standard conic.  Hence, from Corollary  
		\eqref{corr:axis},
  \begin{align}
  \vec{m}&= \vec{e}_2,  
\end{align}  
Since it passes through the focus, from 
					\eqref{eq:app-F-ell-hyp-parab}
  \begin{align}
	  \vec{q} =			\vec{F} 
=
					 \pm e\sqrt{\frac{f_0}{\lambda_2\brak{1-e^2}}} \vec{e }_1
%					 \frac{e}{\sqrt{f_0\lambda_2\brak{1-e^2}}}\vec{e }_1
\end{align}  
for the standard hyperbola/ellipse.  Also, 
from 
    \eqref{eq:app-conic_simp_temp_nonparab},
  \begin{align}
	  \vec{V} =     \frac{\vec{D} }{f_0}, 
	   \vec{u} = 0, 
	   f = -1
	    \label{eq:latus_rectum_ellipse_param-new}
\end{align}  
Substituting the above in
\eqref{eq:chord-len}, 
we obtain
%\eqref{eq:chord-len-sub-ell}.
%\begin{figure*}[!t]
\begin{align}
 \frac{2\sqrt{
\sbrak{
\vec{e}_2^{\top}\brak{\frac{\vec{D}}{f_0} e\sqrt{\frac{f_0}{\lambda_2\brak{1-e^2}}} \vec{e }_1}
}^2
-
\brak
{
 e\sqrt{\frac{f_0}{\lambda_2\brak{1-e^2}}} \vec{e }_1^{\top}\frac{\vec{D}}{f_0} e\sqrt{\frac{f_0}{\lambda_2\brak{1-e^2}}} \vec{e }_1 -1 
}
\brak{\vec{e}_2^{\top}\frac{\vec{D}}{f_0}\vec{e}_2}
}
}
{
\vec{e}_2^{\top}\frac{\vec{D}}{f_0}\vec{e}_2
}\norm{\vec{e}_2}
\label{eq:chord-len-sub-ell}
  \end{align}
%\end{figure*}
  Since 
  \begin{align}
\vec{e}_2^{\top}\vec{D}\vec{e}_1 = 0, 
%\vec{e}_2^{\top}\vec{e}_2 = 0,
\vec{e}_1^{\top}\vec{D}\vec{e}_1 = \lambda_1,
\vec{e}_1^{\top}\vec{e}_1 = 1,
	  \norm{\vec{e}_2} = 1,
\vec{e}_2^{\top}\vec{D}\vec{e}_2 = \lambda_2,
  \end{align}
\eqref{eq:chord-len-sub-ell} can be expressed as 
  \begin{align}
	&		\frac{2\sqrt{\brak{1-\frac{\lambda_1e^2}{{\lambda_2\brak{1-e^2}}}}\brak{\frac{\lambda_2}{f_0}}}}
{
	\frac{\lambda_2}{f_0}
	} 	
	\\
	&=		2\frac{\sqrt{
		f_0\lambda_1}}{\lambda_2}
 & \brak{ \because e^2 = 1-\frac{\lambda_1}{\lambda_2}}
		   \end{align}
For the standard parabola, the parameters in 
\eqref{eq:chord-len} are
\begin{align}  
	\vec{q} =\vec{F} =  -\frac{\eta}{4\lambda_2}\vec{e}_1, \vec{m} = \vec{e}_1, \vec{V} = \vec{D},
	\vec{u} = \frac{\eta}{2}\vec{e}_1^{\top}, f = 0
\end{align}  

Substituting the above in
\eqref{eq:chord-len}, 
%			from \eqref{eq:app-conic_simp_temp_nonparab},  
%					from \eqref{eq:app-F-ell-hyp-parab}
%and 						 \\
the length of the latus rectum  can be expressed as
\begin{align}
 \frac{2\sqrt{
\sbrak{
\vec{e}_2^{\top}\brak{\vec{D}\brak{-\frac{\eta}{4\lambda_2}\vec{e}_1}+\frac{\eta}{2}\vec{e}_1}
}^2
-
\brak
{
\brak{-\frac{\eta}{4\lambda_2}\vec{e}_1}^{\top}\vec{D}\brak{-\frac{\eta}{4\lambda_2}\vec{e}_1} + 2\frac{\eta}{2}\vec{e}_1^{\top}\brak{-\frac{\eta}{4\lambda_2}\vec{e}_1} 
}
\brak{\vec{e}_2^{\top}\vec{D}\vec{e}_2}
}
}
{
\vec{e}_2^{\top}\vec{D}\vec{e}_2
}\norm{\vec{e}_2}
\label{eq:chord-len-sub}
  \end{align}
%\eqref{eq:chord-len-sub}.
  Since 
  \begin{align}
\vec{e}_2^{\top}\vec{D}\vec{e}_1 = 0, 
\vec{e}_2^{\top}\vec{e}_2 = 0,
	  \vec{e}_1^{\top}\vec{D}\vec{e}_1 &= 0,\
	  \\
\vec{e}_1^{\top}\vec{e}_1 = 1,
	  \norm{\vec{e}_1} = 1,
	  \vec{e}_2^{\top}\vec{D}\vec{e}_2 &= \lambda_2,
  \end{align}
\eqref{eq:chord-len-sub} can be expressed as 
  \begin{align}
	  2 \frac{\sqrt{\frac{\eta^2}{4\lambda_2}\lambda_2}}{\lambda_2}
	  = \frac{\eta}{\lambda_2}
  \end{align}
%		See Appendix \ref{app:foc-dir}.
%

		\end{enumerate}
\subsection{Tangent and Normal}
\begin{enumerate}[label=\thesubsection.\arabic*.,ref=\thesubsection.\theenumi]
\item
  If $L$ in \eqref{eq:app-conic_tangent} touches \eqref{eq:app-conic_quad_form} at exactly one point $\vec{q}$, 
  \begin{align}
\label{eq:app-conic_tangent_mq}
  \vec{m}^{\top}\brak{\vec{V}\vec{q}+\vec{u}} = 0
  \end{align}
\begin{proof}
  In this case, \eqref{eq:conic_intercept} has exactly one root.  Hence, 
  in \eqref{eq:app-tangent_roots}
  \begin{align}
  \sbrak{
  \vec{m}^{\top}\brak{\vec{V}\vec{q}+\vec{u}}
  }^2 -\brak{\vec{m}^{\top}\vec{V}\vec{m}}
	  \text{g}\brak
  {
  \vec{q}
%  \vec{q}^{\top}\vec{V}\vec{q} + 2\vec{u}^{\top}\vec{q} +f
  } = 0                                                                                             
  \label{eq:app-conic_tangent_disc}
  \end{align}                    
  $\because \vec{q}$ is the point of contact,
	%$\vec{q}$ satisfies \eqref{eq:app-conic_quad_form}
%  and 
  \begin{align}
	  \text{g}\brak{  \vec{q}} = 0
%  \vec{q}^{\top}\vec{V}\vec{q} + 2\vec{u}^{\top}\vec{q} +f = 0
  \label{eq:app-conic_tangent_qquad}
  \end{align}
  Substituting \eqref{eq:app-conic_tangent_qquad} in \eqref{eq:app-conic_tangent_disc} and simplifying, we obtain \eqref{eq:app-conic_tangent_mq}.
\end{proof}
\item
  Given the point of contact $\vec{q}$, the equation of a tangent to \eqref{eq:app-conic_quad_form} is 
  \begin{align}
  \label{eq:app-conic_tangent_final}
  \brak{\vec{V}\vec{q}+\vec{u}}^{\top}\vec{x}+\vec{u}^{\top}\vec{q}+f = 0
  \end{align}
\begin{proof}
  The normal vector is obtained from \eqref{eq:app-conic_tangent_mq} 
  as
  %
  \begin{align}
  \label{eq:conic_normal_vec}
	  \kappa \vec{n} = \vec{V}\vec{q}+\vec{u}, \kappa \in \mathbb{R}
  \end{align}  
  From \eqref{eq:conic_normal_vec}, the equation of the tangent is\begin{align}
    \brak{\vec{V}\vec{q}+\vec{u}}^{\top}\brak{\vec{x}-\vec{q}} &=0
    \\
    \implies \brak{\vec{V}\vec{q}+\vec{u}}^{\top}\vec{x}-\vec{q}^{\top}\vec{V}\vec{q}-\vec{u}^{\top}\vec{q} &= 0
    \end{align}
    which, upon substituting from \eqref{eq:app-conic_tangent_qquad} and simplifying yields 
  \eqref{eq:app-conic_tangent_final}
%	\eqref{eq:app-conic_tangent}.
\end{proof}
\item
  Given the point of contact $\vec{q}$, the equation of the normal to \eqref{eq:app-conic_quad_form} is 
  \begin{align}
    \brak{\vec{V}\vec{q}+\vec{u}}^{\top}\vec{R}\brak{\vec{x}-\vec{q}} =0
  \end{align}
\begin{proof}
  The direction vector of the tangent is obtained from 
  \eqref{eq:conic_normal_vec} as
  as
  %
  \begin{align}
  \label{eq:app-conic_tangent_vec}
	  \vec{m} = \vec{R}\brak{\vec{V}\vec{q}+\vec{u}}, 
  \end{align}  
  where $\vec{R}$ is the rotation matrix.
  From \eqref{eq:app-conic_tangent_vec}, the equation of the normal is
  given by 
  \eqref{eq:conic_normal_final}
\end{proof}

\item Given the tangent 
\begin{align}
  \label{eq:app-conic_tangent_eq}
\vec{n}^{\top}\vec{x} = c,
\end{align}
the point of  contact to the conic in \eqref{eq:app-conic_quad_form} is given by 
\begin{align}
  \label{eq:app-conic_tangent_contact}
        \myvec{\vec{n}^{\top} \\ \vec{m}^{\top}\vec{V}} \vec{q} = \myvec{c\\ -\vec{m}^{\top}\vec{u}}
\end{align}
		\begin{proof}
			From
  \eqref{eq:app-conic_tangent_mq},
\begin{align}
	\vec{m}^{\top}(\vec{V}\vec{q}+\vec{u})&=0
	\\
	\implies        \vec{m}^{\top}\vec{V}\vec{q} &= -\vec{m}^{\top}\vec{u}
  \label{eq:app-conic_tangent_contact_eq}
\end{align}
Combining 
  \eqref{eq:app-conic_tangent_eq}
  and 
  \eqref{eq:app-conic_tangent_contact_eq}, 
  \eqref{eq:app-conic_tangent_contact} is obtained.

		\end{proof}
\item
  If $\vec{V}^{-1}$ exists, given the normal vector $\vec{n}$, the tangent points of contact to \eqref{eq:app-conic_quad_form} are given by
\begin{align}
  \begin{split}
\vec{q}_i &= \vec{V}^{-1}\brak{\kappa_i \vec{n}-\vec{u}}, i = 1,2
\\
\text{where }\kappa_i &= \pm \sqrt{
\frac{
f_0
%\vec{u}^{\top}\vec{V}^{-1}\vec{u}-f
}
{
\vec{n}^{\top}\vec{V}^{-1}\vec{n}
}
}
  \end{split}
\label{eq:app-conic_tangent_qk}
\end{align}
\begin{proof}
  From \eqref{eq:conic_normal_vec},
\begin{align}
\label{eq:conic_normal_vec_q}
 \vec{q} = \vec{V}^{-1}\brak{\kappa \vec{n}-\vec{u}}, \quad \kappa \in \mathbb{R}
\end{align}
Substituting \eqref{eq:conic_normal_vec_q}
in \eqref{eq:app-conic_tangent_qquad},
\begin{align}
\brak{\kappa \vec{n}-\vec{u}}^{\top}\vec{V}^{-1}\brak{\kappa \vec{n}-\vec{u}} 
%\\
+ 2\vec{u}^{\top}\vec{V}^{-1}\brak{\kappa \vec{n}-\vec{u}} +f &= 0
\\
\implies 
\kappa^2 \vec{n}^{\top}\vec{V}^{-1}\vec{n} - \vec{u}^{\top}\vec{V}^{-1}\vec{u} + f &=0
 \\
 \text{or, } \kappa = \pm \sqrt{\frac{
	 %\vec{u}^{\top}\vec{V}^{-1}\vec{u}-f
	f_0 
 }{\vec{n}^{\top}\vec{V}^{-1}\vec{n}}} &
	\label{eq:conic_normal_k}
\end{align}
%
%yileding 
Substituting \eqref{eq:conic_normal_k} in \eqref{eq:conic_normal_vec_q}
yields \eqref{eq:app-conic_tangent_qk}.
%
\end{proof}
\item For a conic/hyperbola, a line with normal vector $\vec{n}$ cannot be a tangent if 
\begin{align}
\frac{
\vec{u}^{\top}\vec{V}^{-1}\vec{u}-f
}
{
\vec{n}^{\top}\vec{V}^{-1}\vec{n}
} < 0
\end{align}

\item
	\label{eq:conic-p-contact-parab}
  If $\vec{V}$ is not invertible,  given the normal vector $\vec{n}$, the point of contact to \eqref{eq:app-conic_quad_form} is given by the matrix equation
\begin{align}
\label{eq:app-conic_tangent_q_eigen}
\myvec{
\vec{\brak{u+\kappa \vec{n}}}^{\top} \\ \vec{V}
}
\vec{q} &= 
\myvec{
-f
\\
\kappa\vec{n}-\vec{u}
}
\\
\text{where }  \kappa = \frac{\vec{p}_1^{\top}\vec{u}}{\vec{p}_1^{\top}\vec{n}}, \quad \vec{V}\vec{p}_1 &= 0
\label{eq:app-conic_tangent_qk_eigen}
\end{align}


\begin{proof}
  If $\vec{V}$ is non-invertible, it has a zero eigenvalue.  If the corresponding eigenvector is $\vec{p}_1$, then,
\begin{align}
\vec{V}\vec{p}_1 = 0
\label{eq:conic_zero_eigen}
\end{align}
From \eqref{eq:conic_normal_vec},
\begin{align}
\label{eq:conic_zero_eigen_normal}
\kappa \vec{n} &= \vec{V} \vec{q}+\vec{u}, \quad \kappa \in \mathbb{R}
\\
\implies \kappa \vec{p}_1^{\top}\vec{n} &= \vec{p}_1^{\top}\vec{V} \vec{q}+\vec{p}_1^{\top}\vec{u}
\\
\text{or, } \kappa \vec{p}_1^{\top}\vec{n} &= \vec{p}_1^{\top}\vec{u},  \quad \because \vec{p}_1^{\top} \vec{V} = 0, 
%\\
\quad 
\brak{\text{ from } \eqref{eq:conic_zero_eigen}}
%\label{eq:conic_normal_vec_q}
\end{align}
yielding $\kappa$ in \eqref{eq:app-conic_tangent_qk_eigen}. From \eqref{eq:conic_zero_eigen_normal},
\begin{align}
\kappa \vec{q}^{\top}\vec{n} &= \vec{q}^{\top}\vec{V} \vec{q}+\vec{q}^{\top}\vec{u}
\\
\implies \kappa \vec{q}^{\top}\vec{n} &= -f-\vec{q}^{\top}\vec{u} \quad \text{from } \eqref{eq:app-conic_tangent_qquad},
\\
\text{or, } \brak{\kappa \vec{n}+\vec{u}}^{\top}\vec{q} &= -f
\label{eq:conic_zero_eigen_normal_fq}
\end{align}
\eqref{eq:conic_zero_eigen_normal} can be expressed as
\begin{align}
\label{eq:conic_zero_eigen_normal_vq}
\vec{V} \vec{q} = \kappa \vec{n} - \vec{u}.
\end{align}
\eqref{eq:conic_zero_eigen_normal_fq} and \eqref{eq:conic_zero_eigen_normal_vq} clubbed together result in \eqref{eq:app-conic_tangent_q_eigen}.
\end{proof}
\item
	The asymptotes of the hyperbola in 
    \eqref{eq:app-conic_simp_temp_nonparab}, defined to be the lines that do not intersect the hyperbola, are given by 
    \begin{align} 
    \label{eq:app-pair-std}
    \myvec{\sqrt{\abs{\lambda_1}} & \pm \sqrt{\abs{\lambda_2}}}\vec{y} = 0
    \end{align} 
%  \begin{align}
%	  \myvec{\lambda_1 & \pm \lambda_2}\vec{y} = 0   
%  \end{align}
  
  \begin{proof}
	  From 
\eqref{eq:app-conic_simp_temp_nonparab},
it is obvious that 
the pair of lines represented by 
  \begin{align}
	    \vec{y}^{\top}\vec{D}\vec{y} = 0   
      \label{eq:pair-conic}
  \end{align}
  do not intersect the conic 
  \begin{align}
	    \vec{y}^{\top}\vec{D}\vec{y} =  f_0  
  \end{align}
  Thus, 
      \eqref{eq:pair-conic}
      represents the asysmptotes of the hyperbola in 
\eqref{eq:app-conic_simp_temp_nonparab} and can be expressed as 
  \begin{align} 
    \lambda_1y_1^2 +\lambda_2y_1^2 = 0, 
    \label{eq:quad_form_hyper}
    \end{align}
%    \eqref{eq:quad_form_hyper}
which can then be simplified  
using the steps in 
	\eqref{eq:incircle-disc-v}-
	\eqref{eq:incircle-disc-v-lam}
to obtain
    \eqref{eq:app-pair-std}.
  \end{proof}
  \item
\eqref{eq:app-conic_quad_form} represents a pair of straight lines if 
  \begin{align} 
	  \label{eq:pair-cond}
%	  \lambda_1y_1^2 +\lambda_2y_2^2 = 
  \vec{u}^{\top}\vec{V}^{-1}\vec{u} -f  = 0
  \end{align} 
  
  \item
%	  \label{them:pair-mat-sing}
\eqref{eq:app-conic_quad_form} represents a pair of straight lines if 
the matrix 
  \begin{align} 
	  \myvec{\vec{V} & \vec{u}\\ \vec{u}^{\top} & f}  
%	  \label{eq:pair-mat-sing}
  \end{align} 
  is singular.
  
  \begin{proof}
Let 
  \begin{align} 
	  \myvec{\vec{V} & \vec{u}\\ \vec{u}^{\top} & f}  \vec{x} =\vec{0}
  \end{align} 
  Expressing 
  \begin{align} 
	  \vec{x} =\myvec{\vec{y} \\ y_3}, 
  \end{align} 
  \begin{align} 
	  \myvec{\vec{V} & \vec{u}\\ \vec{u}^{\top} & f}   
	  \myvec{\vec{y} \\ y_3} &= \vec{0}
	  \\
	  \implies
	  \label{eq:pair-mat-sing-1}
	  \vec{V} \vec{y} + y_3\vec{u} &= \vec{0} \quad \text{and}
	  \\
	  \vec{u}^{\top}\vec{y} + fy_3 &=0
	  \label{eq:pair-mat-sing-2}
  \end{align} 
  From 
	  \eqref{eq:pair-mat-sing-1} we obtain,
  \begin{align} 
	  \vec{y}^{\top}  \vec{V} \vec{y} + y_3\vec{y}^{\top}\vec{u} &= \vec{0} 
	  \\
	  \implies 
	  \vec{y}^{\top}  \vec{V} \vec{y} + y_3\vec{u}^{\top}\vec{y} &= \vec{0} 
  \end{align} 
  yielding 
	  \eqref{eq:pair-cond} upon substituting from 
	  \eqref{eq:pair-mat-sing-2}.
  \end{proof}
  \item
	  Using the affine transformation, 
    \eqref{eq:app-pair-std}
 can be expressed as the lines 
%
\begin{align} 
\label{eq:quad_form_pair}
\myvec{\sqrt{\abs{\lambda_1}} & \pm \sqrt{\abs{\lambda_2}}}\vec{P}^{\top}\brak{\vec{x}-\vec{c}} = 0
\end{align} 
  
   \item
	   The angle between the asymptotes can be expressed as
\begin{align} 
\label{eq:app-quad_form_pair_ang}
\cos\theta=\frac{\abs{\lambda_1}-\abs{\lambda_2}}
{\abs{\lambda_1}+\abs{\lambda_2}}
\end{align} 
  
  \begin{proof}
The normal vectors of the lines in \eqref{eq:quad_form_pair} are 
  \begin{align} 
  \label{eq:quad_form_pair_normvecs}
  \begin{split}
  \vec{n}_1 &= \vec{P}\myvec{\sqrt{\abs{\lambda_1}} \\[2mm]  \sqrt{\abs{\lambda_2}}}
  \\
  \vec{n}_2 &= \vec{P}\myvec{\sqrt{\abs{\lambda_1}} \\[2mm] - \sqrt{\abs{\lambda_2}}}
  \end{split}
  \end{align} 
  The angle between the asymptotes is given by 
\begin{align} 
\label{eq:app-quad_form_pair_ang_exp}
\cos\theta=\frac{\vec{n_1}^{\top}\vec{n_2}}{\norm{\vec{n_1}}\norm{\vec{n_2}}}
\end{align} 
The orthogonal matrix $\vec{P}$ preserves the norm, i.e.
\begin{align} 
	\norm{\vec{n_1}} &= \norm{\vec{P}\myvec{\sqrt{\abs{\lambda_1}} \\[2mm]  \sqrt{\abs{\lambda_2}}}}
	=\norm{\myvec{\sqrt{\abs{\lambda_1}} \\[2mm]  \sqrt{\abs{\lambda_2}}}}
	\\
	&=\sqrt{\abs{\lambda_1}+\abs{\lambda_2}} = \norm{\vec{n_2}}
\end{align} 
It is easy to verify that 
\begin{align} 
\vec{n_1}^{\top}\vec{n_2} = \abs{\lambda_1}-\abs{\lambda_2}
\end{align} 
%
Thus, the angle between the asymptotes is obtained from \eqref{eq:app-quad_form_pair_ang_exp} as \eqref{eq:app-quad_form_pair_ang}.
  \end{proof}
\item For a circle, the points of contact are
	\begin{align}
	\vec{q}_{ij} &= \brak{\pm r \frac{\vec{n}_j}{\norm{\vec{n}_j}}-\vec{u}}, \quad i,j = 1,2
\end{align}
\begin{proof}
	From 
\eqref{eq:app-conic_tangent_qk},
and 
	\eqref{eq:circ-cr},
\begin{align}
\kappa_{ij} &= \pm 
\frac{r
}
{
	\norm{\vec{n}_j}
}
\end{align}
\end{proof}
\item A point $\vec{h}$ lies on a normal to the conic in \eqref{eq:app-conic_quad_form} 
	if
%\begin{multline}
\begin{equation}
	\label{eq:app-point_of_tangency-m}
	\brak{ {\vec{m}^\top(\vec{Vh}+\vec{u})}}^2\brak{\vec{n}^{\top}\vec{V}\vec{n}} 
%	\\
	- 2\brak{\vec{m}^\top\vec{V}\vec{n}} \brak{ {\vec{m}^\top(\vec{Vh}+\vec{u})}\vec{n}^{\top}\brak{\vec{V}\vec{h}+\vec{u}}} 
%	\\
+  \text{g}\brak{
  \vec{h}
	  }\brak{\vec{m}^\top\vec{V}\vec{n}}^2
%	\vec{h}^{\top}\vec{V}\vec{h} + 2\vec{u}^{\top}\vec{h} +f 
	= 0
\end{equation}
%\end{multline}
\begin{proof}
The point of contact for the normal passing through a point $\vec{h}$ is given by 
\begin{align}
	\label{eq:point_of_tangency}
	\vec{q} = \vec{h} + \mu\vec{n}
	%\vec{q} = \vec{h} + \mu\vec{R}\vec{m}
\end{align}
From 
  \eqref{eq:app-conic_tangent_mq},
	the tangent at $\vec{q}$ satisfies 
\begin{align}
	\label{eq:tangency_condition}
	\vec{m}^\top(\vec{Vq}+\vec{u}) = 0
\end{align}
Substituting \eqref{eq:point_of_tangency} in \eqref{eq:tangency_condition},
\begin{align}
	\label{eq:normal_simp_1}
	\vec{m}^\top(\vec{V}(\vec{h}+\mu\vec{n})+\vec{u}) = 0\\
	%\vec{m}^\top(\vec{V}(\vec{h}+\mu\vec{R}\vec{m})+\vec{u}) = 0\\
	\label{eq:normal_simp_2}
	\implies \mu\vec{m}^\top\vec{V}\vec{n} = -\vec{m}^\top(\vec{Vh}+\vec{u})
	%\implies \mu\vec{m}^\top\vec{V}\vec{R}\vec{m} = -\vec{m}^\top(\vec{Vh}+\vec{u})
\end{align}
yielding 
\begin{align}
	\label{eq:app-point_of_tangency-mu}
	\mu &=- \frac  {\vec{m}^\top(\vec{Vh}+\vec{u})}{\vec{m}^\top\vec{V}\vec{n}},
\end{align}
%	\eqref{eq:app-point_of_tangency-mu}.
	From 
\eqref{eq:conic_intercept},
\begin{align}
	\label{eq:normal_simp_2-quad}
\mu^2\vec{n}^{\top}\vec{V}\vec{n} + 2 \mu\vec{n}^{\top}\brak{\vec{V}\vec{h}+\vec{u}} 
+  \text{g}\brak{
  \vec{h}
	  }
%	\vec{h}^{\top}\vec{V}\vec{h} + 2\vec{u}^{\top}\vec{h} +f 
	&= 0
\end{align}
From 
	\eqref{eq:app-point_of_tangency-mu},
	\eqref{eq:normal_simp_2-quad} can be expressed
	as
\begin{multline}
%	\label{eq:normal_simp_2-quad}
\brak{- \frac{\vec{m}^\top(\vec{Vh}+\vec{u})}{\vec{m}^\top\vec{V}\vec{n}}}^2\vec{n}^{\top}\vec{V}\vec{n} 
	%\\
	+ 2 \brak{- \frac{\vec{m}^\top(\vec{Vh}+\vec{u})}{\vec{m}^\top\vec{V}\vec{n}}}\vec{n}^{\top}\brak{\vec{V}\vec{h}+\vec{u}} 
+  \text{g}\brak{
  \vec{h}
	  }
	= 0
\end{multline}
	yielding
	\eqref{eq:app-point_of_tangency-m}.
\end{proof}
\item  A point $\vec{h}$ lies on a tangent to the conic in \eqref{eq:app-conic_quad_form} if 
\begin{align}
	  \label{eq:app-h-tangents-cond}
  \vec{m}^{\top}  \sbrak{\brak{\vec{V}\vec{h}+\vec{u}}
	  \brak{\vec{V}\vec{h}+\vec{u}}^{\top}
   -\vec{V}
	  \text{g}\brak{
  \vec{h}
	  }
	  }\vec{m} 
	  &= 0                                                                                             
\end{align}
\begin{proof}
 From \eqref{eq:app-tangent_roots}
 and
  \eqref{eq:app-conic_tangent_disc}
  \begin{align}
  \sbrak{
  \vec{m}^{\top}\brak{\vec{V}\vec{h}+\vec{u}}
  }^2 -\brak{\vec{m}^{\top}\vec{V}\vec{m}}
	  \text{g}\brak{
  \vec{h}
	  }
	  &= 0                                                                                             
  \label{eq:app-conic_tangent_disc-h}
  \end{align}                    
  yielding
	  \eqref{eq:app-h-tangents-cond}.
\end{proof}
\item
	The normal vectors of the tangents 
to the conic in \eqref{eq:app-conic_quad_form} 
	from 
	a point $\vec{h}$ 
	are given by 
  \begin{align} 
  \label{eq:app-quad_form_pair_normvecs-sigma}
  \begin{split}
  \vec{n}_1 &= \vec{P}\myvec{\sqrt{\abs{\lambda_1}} \\[2mm]  \sqrt{\abs{\lambda_2}}}
  \\
  \vec{n}_2 &= \vec{P}\myvec{\sqrt{\abs{\lambda_1}} \\[2mm] - \sqrt{\abs{\lambda_2}}}
  \end{split}
  \end{align} 
  where $\lambda_i, \vec{P}$ are the eigenparameters of 
  \begin{align} 
		\bm{\Sigma} &= 
	   \brak{\vec{V}\vec{h}+\vec{u}}
	  \brak{\vec{V}\vec{h}+\vec{u}}^{\top}
   -
  \brak
  {
	  \text{g}\brak{
  \vec{h}
	  }
%  \vec{h}^{\top}\vec{V}\vec{h} + 2\vec{u}^{\top}\vec{h} +f
  }\vec{V}.
	  \label{eq:app-h-tangents-sigma}
  \end{align}                    
  \begin{proof}
	  From 
	  \eqref{eq:app-h-tangents-cond} we obtain
	  \eqref{eq:app-h-tangents-sigma}.  Consequently, from 
  \eqref{eq:quad_form_pair_normvecs}, 
  \eqref{eq:app-quad_form_pair_normvecs-sigma}
  can be obtained.
  \end{proof}

  \item
	  \label{them:pair-mat-sing}
\eqref{eq:app-conic_quad_form} represents a pair of straight lines if 
the matrix 
  \begin{align} 
	  \myvec{\vec{V} & \vec{u}\\ \vec{u}^{\top} & f}  
	  \label{eq:app-pair-mat-sing}
  \end{align} 
  is singular.
\item The intersection of two conics 
with parameters $\vec{V}_i, \vec{u}_i, f_i,\ i = 1,2$
	is defined
as
\begin{align}
	\vec{x}^{\top}\brak{\vec{V}_1 + \mu\vec{V}_2}\vec{x}+2 \brak{\vec{u}_1+\mu \vec{u}_2}^{\top} \vec{x} 
	+ \brak{f_1+\mu f_2}= 0
	  \label{eq:app-pair-mat-sing-conic}
    \end{align}
	  
	  
\item From \eqref{eq:app-pair-mat-sing}, \eqref{eq:app-pair-mat-sing-conic} represents a pair of straight lines if
\begin{align}
	  \label{eq:app-pair-mat-sing-conic-det}
\mydet{\vec{V}_1 + \mu\vec{V}_2 & \vec{u}_1+\mu \vec{u}_2\\ \brak{\vec{u}_1+\mu \vec{u}_2}^{\top} & f_1 + \mu f_2} &= 0
\end{align}
\end{enumerate}


\end{document}


